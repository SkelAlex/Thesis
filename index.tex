% Options for packages loaded elsewhere
\PassOptionsToPackage{unicode}{hyperref}
\PassOptionsToPackage{hyphens}{url}
\PassOptionsToPackage{dvipsnames,svgnames,x11names}{xcolor}
%
\documentclass[
  letterpaper,
  DIV=11,
  numbers=noendperiod]{scrreprt}

\usepackage{amsmath,amssymb}
\usepackage{iftex}
\ifPDFTeX
  \usepackage[T1]{fontenc}
  \usepackage[utf8]{inputenc}
  \usepackage{textcomp} % provide euro and other symbols
\else % if luatex or xetex
  \usepackage{unicode-math}
  \defaultfontfeatures{Scale=MatchLowercase}
  \defaultfontfeatures[\rmfamily]{Ligatures=TeX,Scale=1}
\fi
\usepackage{lmodern}
\ifPDFTeX\else  
    % xetex/luatex font selection
\fi
% Use upquote if available, for straight quotes in verbatim environments
\IfFileExists{upquote.sty}{\usepackage{upquote}}{}
\IfFileExists{microtype.sty}{% use microtype if available
  \usepackage[]{microtype}
  \UseMicrotypeSet[protrusion]{basicmath} % disable protrusion for tt fonts
}{}
\makeatletter
\@ifundefined{KOMAClassName}{% if non-KOMA class
  \IfFileExists{parskip.sty}{%
    \usepackage{parskip}
  }{% else
    \setlength{\parindent}{0pt}
    \setlength{\parskip}{6pt plus 2pt minus 1pt}}
}{% if KOMA class
  \KOMAoptions{parskip=half}}
\makeatother
\usepackage{xcolor}
\setlength{\emergencystretch}{3em} % prevent overfull lines
\setcounter{secnumdepth}{5}
% Make \paragraph and \subparagraph free-standing
\ifx\paragraph\undefined\else
  \let\oldparagraph\paragraph
  \renewcommand{\paragraph}[1]{\oldparagraph{#1}\mbox{}}
\fi
\ifx\subparagraph\undefined\else
  \let\oldsubparagraph\subparagraph
  \renewcommand{\subparagraph}[1]{\oldsubparagraph{#1}\mbox{}}
\fi


\providecommand{\tightlist}{%
  \setlength{\itemsep}{0pt}\setlength{\parskip}{0pt}}\usepackage{longtable,booktabs,array}
\usepackage{calc} % for calculating minipage widths
% Correct order of tables after \paragraph or \subparagraph
\usepackage{etoolbox}
\makeatletter
\patchcmd\longtable{\par}{\if@noskipsec\mbox{}\fi\par}{}{}
\makeatother
% Allow footnotes in longtable head/foot
\IfFileExists{footnotehyper.sty}{\usepackage{footnotehyper}}{\usepackage{footnote}}
\makesavenoteenv{longtable}
\usepackage{graphicx}
\makeatletter
\def\maxwidth{\ifdim\Gin@nat@width>\linewidth\linewidth\else\Gin@nat@width\fi}
\def\maxheight{\ifdim\Gin@nat@height>\textheight\textheight\else\Gin@nat@height\fi}
\makeatother
% Scale images if necessary, so that they will not overflow the page
% margins by default, and it is still possible to overwrite the defaults
% using explicit options in \includegraphics[width, height, ...]{}
\setkeys{Gin}{width=\maxwidth,height=\maxheight,keepaspectratio}
% Set default figure placement to htbp
\makeatletter
\def\fps@figure{htbp}
\makeatother
% definitions for citeproc citations
\NewDocumentCommand\citeproctext{}{}
\NewDocumentCommand\citeproc{mm}{%
  \begingroup\def\citeproctext{#2}\cite{#1}\endgroup}
\makeatletter
 % allow citations to break across lines
 \let\@cite@ofmt\@firstofone
 % avoid brackets around text for \cite:
 \def\@biblabel#1{}
 \def\@cite#1#2{{#1\if@tempswa , #2\fi}}
\makeatother
\newlength{\cslhangindent}
\setlength{\cslhangindent}{1.5em}
\newlength{\csllabelwidth}
\setlength{\csllabelwidth}{3em}
\newenvironment{CSLReferences}[2] % #1 hanging-indent, #2 entry-spacing
 {\begin{list}{}{%
  \setlength{\itemindent}{0pt}
  \setlength{\leftmargin}{0pt}
  \setlength{\parsep}{0pt}
  % turn on hanging indent if param 1 is 1
  \ifodd #1
   \setlength{\leftmargin}{\cslhangindent}
   \setlength{\itemindent}{-1\cslhangindent}
  \fi
  % set entry spacing
  \setlength{\itemsep}{#2\baselineskip}}}
 {\end{list}}
\usepackage{calc}
\newcommand{\CSLBlock}[1]{\hfill\break\parbox[t]{\linewidth}{\strut\ignorespaces#1\strut}}
\newcommand{\CSLLeftMargin}[1]{\parbox[t]{\csllabelwidth}{\strut#1\strut}}
\newcommand{\CSLRightInline}[1]{\parbox[t]{\linewidth - \csllabelwidth}{\strut#1\strut}}
\newcommand{\CSLIndent}[1]{\hspace{\cslhangindent}#1}

\KOMAoption{captions}{tableheading}
\widowpenalty10000
\clubpenalty10000
\makeatletter
\@ifpackageloaded{bookmark}{}{\usepackage{bookmark}}
\makeatother
\makeatletter
\@ifpackageloaded{caption}{}{\usepackage{caption}}
\AtBeginDocument{%
\ifdefined\contentsname
  \renewcommand*\contentsname{Table of contents}
\else
  \newcommand\contentsname{Table of contents}
\fi
\ifdefined\listfigurename
  \renewcommand*\listfigurename{List of Figures}
\else
  \newcommand\listfigurename{List of Figures}
\fi
\ifdefined\listtablename
  \renewcommand*\listtablename{List of Tables}
\else
  \newcommand\listtablename{List of Tables}
\fi
\ifdefined\figurename
  \renewcommand*\figurename{Figure}
\else
  \newcommand\figurename{Figure}
\fi
\ifdefined\tablename
  \renewcommand*\tablename{Table}
\else
  \newcommand\tablename{Table}
\fi
}
\@ifpackageloaded{float}{}{\usepackage{float}}
\floatstyle{ruled}
\@ifundefined{c@chapter}{\newfloat{codelisting}{h}{lop}}{\newfloat{codelisting}{h}{lop}[chapter]}
\floatname{codelisting}{Listing}
\newcommand*\listoflistings{\listof{codelisting}{List of Listings}}
\makeatother
\makeatletter
\makeatother
\makeatletter
\@ifpackageloaded{caption}{}{\usepackage{caption}}
\@ifpackageloaded{subcaption}{}{\usepackage{subcaption}}
\makeatother
\ifLuaTeX
  \usepackage{selnolig}  % disable illegal ligatures
\fi
\usepackage{bookmark}

\IfFileExists{xurl.sty}{\usepackage{xurl}}{} % add URL line breaks if available
\urlstyle{same} % disable monospaced font for URLs
\hypersetup{
  pdftitle={Gendered Political Socialization},
  pdfauthor={Alexandre Fortier-Chouinard},
  colorlinks=true,
  linkcolor={blue},
  filecolor={Maroon},
  citecolor={Blue},
  urlcolor={Blue},
  pdfcreator={LaTeX via pandoc}}

\title{Gendered Political Socialization}
\usepackage{etoolbox}
\makeatletter
\providecommand{\subtitle}[1]{% add subtitle to \maketitle
  \apptocmd{\@title}{\par {\large #1 \par}}{}{}
}
\makeatother
\subtitle{Why Women and Men Still Differ in Political Interest}
\author{Alexandre Fortier-Chouinard}
\date{March 24, 2024}

\begin{document}
\maketitle

\renewcommand*\contentsname{Table of contents}
{
\hypersetup{linkcolor=}
\setcounter{tocdepth}{2}
\tableofcontents
}
\bookmarksetup{startatroot}

\chapter*{Preface}\label{preface}
\addcontentsline{toc}{chapter}{Preface}

\markboth{Preface}{Preface}

\bookmarksetup{startatroot}

\chapter{Introduction: Theories and Explanations for the Gender Gap in
Political Interest}\label{sec-chap1}

Traditional political science studies suggest that men are generally
more interested in politics than women (Burns, Schlozman, and Verba
2001; M. L. Inglehart 1981; R. Inglehart and Norris 2003; Verba, Burns,
and Schlozman 1997). However, several recent studies (R. Campbell and
Winters 2008; Ferrin et al. 2020; Ferrı́n and Garcı́a-Albacete 2023;
Keeling 2023; Kuhn 2004; Sabella 2004a; Tormos and Verge 2022) show that
people do not think about the full range of political actions when they
are asked questions about their political interest and that men are more
interested than women in certain political topics --- notably partisan
politics --- but less interested in others --- like health care and
education politics. In parallel, studies have found parents and peers
play an important role in children's political socialization
(Dostie-Goulet 2009b; Huckfeldt and Sprague 1995; Mayer and Schmidt
2004; Neundorf, Smets, and Garcia-Albacete 2013), especially when they
share the children's gender (Beauregard 2008; Owen and Dennis 1988).
Already at age 6, boys and girls are more likely to draw male than
female politicians when asked to draw a politician (Bos et al. 2022),
suggesting aspects of gendered political socialization happen at a young
age. While structural, institutional, biological, and life-cycle factors
have also been linked to the development of political interest,
childhood socialization is one of its most important determinants
(Jennings and Niemi 1981) --- and political interest remains stable from
an early age (Prior 2010, 2019). However, parental transmission of
political interest has only been studied using the traditional one-item
measure of political interest, while the gendered aspects of peer
transmission of political interest have not been formalized. This
dissertation answers the following questions: \emph{``How do men and
women differ with regard to political interest, and why?''} In response
to these questions, I contend that, as they grow up, most boys develop
more interest in competitive aspects of politics while most girls
develop more interest in cooperative aspects of politics. I further
suggest that these segregated interests mainly emerge as a result of
gendered socialization by their same-gender parent(s) and peers.
Notably, interest in \emph{partisan} politics is distinctly transmitted
from men to boys. Therefore, I argue that gender differences are more
about the type of political interest rather than the level of political
interest.

Among various topics, interest in partisan politics has important
implications for the exercise of citizenship rights. This topic is the
most closely related to Canadian political institutions. Therefore,
interest in partisan politics may be related to greater political
ambition, which leads to men's higher levels of elite representation
(Fox and Lawless 2005). Since policies often have different effects on
men and women, which can be influenced by policymakers' gender
(Chattopadhyay and Duflo 2004; Donato et al. 2008; Rayment 2020), the
fact that women are under-represented in the legislative and executive
spheres has practical consequences for the kinds of legislation adopted.
By studying the origins of gender differences in political interest,
potential solutions can be sought by relevant actors who seek assemblies
that are more gender-balanced. Moreover, political interest has also
been linked with other forms of political participation (Cicognani et
al. 2012). It is therefore possible that interest in various political
topics increases the range of political actions in which men and women
take part.

\section{Conceptual Definitions}\label{conceptual-definitions}

Before defining relevant concepts such as political engagement,
political participation and political interest, it is important to first
define what \emph{politics} is. Politics is a contested concept and has
been for a long time (Gallie 1956). For Weber (1919), ``{[}t{]}he
concept is extremely broad and comprises any kind of leadership in
action'' (1). Heywood (2019) offers a more recent and comprehensive
definition:

\begin{quote}
Politics, in its broadest sense, is the activity through which people
make, preserve and amend the general rules under which they live.
Politics is inextricably linked to the phenomena of conflict and
cooperation. On the one hand, the existence of rival opinions, different
wants, competing needs, and opposing interests guarantees disagreement
about the rules under which people live. On the other hand, people
recognize that, in order to influence these rules or ensure their
enforcement, they must work with others (34).
\end{quote}

Other sources have variously described politics as the art of
government, as public affairs in general, as the non-violent resolution
of disputes, as power and the distribution of resources, or as the
conflictual discussion of controversial topics (Conover, Searing, and
Crewe 2002; Fitzgerald 2013; Heywood 2019; Lane 1962; Sapiro 2013; Walsh
2004).

In this dissertation, for the sake of clarity, \emph{politics} is
defined using Heywood's (2019) definition. Importantly, this definition
emphasizes the notions of cooperation and competition, therefore going
beyond Weber's (1919) notion of leadership. More than a partisan game,
politics includes actions that preserve the policy status quo as well as
actions that aim at disrupting it, including contentious politics (Tilly
and Tarrow 2015) and interest groups that seek to influence the rules
--- from the international level to the local level.

The umbrella term \emph{political engagement} is used throughout this
dissertation to describe forms of commitment to politics through either
\emph{attitudes}, \emph{actions}, or both --- leaving aside the
ideological content of these attitudes and actions. Political engagement
has been used to refer to political interest, political discussion,
political knowledge, voter turnout, political efficacy, and/or party
membership (Coffé and Bolzendahl 2010; Gidengil, O'Neill, and Young
2010; Verba, Burns, and Schlozman 1997). More narrowly, \emph{political
participation} is used to refer to political actions, such as
boycotting, participating in protests, donating money to candidates,
voting, or working with interest groups or voluntary associations that
seek to influence policy at any level --- international, national,
provincial, local, school board, and so on.

\emph{Political interest}, this study's main variable of interest, is
defined as ``the degree to which politics arouses a citizen's
curiosity'' (J. W. Van Deth 1990, 278). It therefore involves both being
aware of politics and caring about it. It is a form of political
engagement through attitudes rather than actions. When used in this
dissertation, the concept is meant to refer to people's self-reported
interest in politics in general. However, studies show that women and
men do not think of politics the same way scholars do, and instead
emphasize partisan politics more specifically when answering survey
questions on political interest (R. Campbell and Winters 2008; Ferrin et
al. 2020; Ferrı́n and Garcı́a-Albacete 2023).\footnote{Women are more
  likely to view gender issues as political than men, but otherwise men
  and women seem to define what counts as political in similar ways
  (Ferrin et al. 2020; Fitzgerald 2013).} The measure of political
interest therefore differs from the theoretical definition of politics
used for this dissertation, as it is biased towards partisan politics.
For similar reasons, when \emph{political discussion} and
\emph{political efficacy} are mentioned in this dissertation, they also
refer to concepts that are likely tilted toward partisan politics.

\emph{Interest in partisan politics} is defined as interest in the
competition between political parties that happens in political
institutions and in election campaigns --- as opposed to interest in
policies, society issues or contentious politics. When studies
explicitly measure interest in partisan politics, this dissertation uses
the expression \emph{interest in partisan politics} rather than
\emph{political interest}.

\emph{Political ambition} refers to the desire to run for political
office at any level (Fox and Lawless 2005). Political ambition is
another attitudinal component of political engagement. This desire can
be short-term or long-term.

Finally, the concept of \emph{gender} is defined here as ``sets of
socially constructed meanings of masculinities and femininities, derived
from context-specific identifications of sex, that is, male and female,
men and women'' (Beckwith 2010, 160). While gender is a social
construct, it is already observable at a young age and further develops
through time due to biological factors --- not environmental ones
(Hatemi et al. 2012). Other than men and women, various other gender
identities have been identified, including transgender, non-binary,
gender-queer, and gender-ambiguous people (Matsuno and Budge 2017).
According to the 2021 Canadian Census, people who identify as
transgender or non-binary made up 0.33\% of all Canadians (R. Easton
2022). Due to concerns with sample size and theoretical grounding, in
this dissertation, only people who identify as men, women, boys, and
girls are studied, as in most studies of gender differences in political
engagement. Further research using purposely selected samples of people
who do not identify as men or women will be needed to get a better
understanding of the determinants of their political interest.

\section{Diagnosis: Gender Differences in Political Interest and
Engagement}\label{diagnosis-gender-differences-in-political-interest-and-engagement}

In Canada and other Western countries, differences in political
engagement are complex and often subject to disagreement between
studies. On some aspects of political engagement, it is hard to tell
whether men or women are more engaged. They seem to be about as likely
to vote in national elections (Kostelka, Blais, and Gidengil 2019),
participate in demonstrations and political rallies (Coffé and
Bolzendahl 2010), donate money to political candidates (conflicting
evidence from Coffé and Bolzendahl (2010) and Tolley, Besco, and Sevi
(2022)), express high levels of political trust (conflicting evidence
from Dassonneville et al. (2012) and Schoon and Cheng (2011)), win
elections when they run for office (Golder et al. 2017; Sevi,
Arel-Bundock, and Blais 2019), and express interest in local politics
(Coffé 2013; Hayes and Bean 1993).\footnote{Controlling for some of the
  other political attitudes mentioned here, women become \emph{more}
  interested in local politics (Coffé 2013) and \emph{more} likely to
  vote in national elections (Coffé and Bolzendahl 2010) than men, but
  the current focus is on raw gender differences.}

However, other aspects of political engagement point in the direction of
men being more politically engaged. Studies have found that they are
more likely to be interested in politics in general, including national
and international politics (Verba, Burns, and Schlozman 1997); to
discuss politics (Beauvais 2020; Rosenthal, Jones, and Rosenthal 2003);
to have an opinion on political issues (Atkeson and Rapoport 2003); to
try to convince other people to vote for a candidate (Beauregard 2014);
to feel a sense of political efficacy, both internal and
external\footnote{\emph{Internal efficacy}: an individual's
  ``self-perception that they are capable of understanding politics and
  competent enough to participate in political acts such as voting.''
  \emph{External efficacy}: an individual's ``perception of being able
  to have an impact on politics'' (Gidengil, Giles, and Thomas 2008,
  538).} (Gidengil, Giles, and Thomas 2008; Verba, Burns, and Schlozman
1997); to give correct answers to knowledge questions on political
institutions (Dolan 2011; Ferrin, Fraile, and García-Albacete 2018;
Fraile 2014; Norris, Lovenduski, and Campbell 2004; Stolle and Gidengil
2010; Verba, Burns, and Schlozman 1997); to vote in second-order
elections\footnote{Second-order elections include European elections,
  municipal elections, and most types of other sub-regional elections.}
(Dassonneville and Kostelka 2021; Kostelka, Blais, and Gidengil 2019);
to be active political party members (Coffé and Bolzendahl 2010; Stolle
and Hooghe 2011); to be politically ambitious; and to run for political
office (Devroe et al. 2023; Fox and Lawless 2004, 2005, 2023; Lawless
and Fox 2010; Tolley 2023).

Finally, research suggests women are more politically engaged in some
aspects. For instance, women are more likely than men to be interested
in health care, education and gender issues (R. Campbell and Winters
2008; Ferrin et al. 2020; Kuhn 2004; Sabella 2004a; Tormos and Verge
2022), give correct answers to knowledge questions on government
services and social issues (Ferrin, Fraile, and García-Albacete 2018;
Keeling 2023; Norris, Lovenduski, and Campbell 2004; Stolle and Gidengil
2010), and engage in private activism such as boycotting and signing
petitions (Coffé and Bolzendahl 2010).

Overall, the types of political engagement in which men appear to be
more involved are more related to political institutions, while those in
which women are more involved are more informal and --- in the case of
boycotts and petitions --- anti-system. Since more power is typically
concentrated in political institutions than in private activism in
Canada, the aspects of politics in which men feel more engaged, such as
interest in partisan and national politics, concern the highest levels
of power in the country. The overall influence of women in Canadian
politics is therefore limited.

\section{Consequences from the Point of View of Democratic
Theory}\label{consequences-from-the-point-of-view-of-democratic-theory}

Different aspects of political engagement are correlated and influence
each other (Bennett and Bennett 1989; Coffé and Bolzendahl 2010; Coffé
2013; Ondercin and Jones-White 2011). This means that a gender gap in
one aspect of political engagement can contribute to a gap in another
aspect. More importantly, when gender differences in \emph{partisan
political interest} emerge at a young age, they can have consequences
for at least two types of substantive citizen participation when
children become adults: discussing partisan politics with other people
and running for elected political office.

\subsection{Political Discussion}\label{political-discussion}

Studies report men are more likely than women to discuss politics, but
men and women tend to think about \emph{partisan politics} more
specifically when they think about the concept of politics (R. Campbell
and Winters 2008; Ferrin et al. 2020). Still, Heywood's (2019)
definition of politics goes beyond partisan competition and elections,
including all the ways through which people make and change the rules
that govern them. Traditional political discussion survey questions
therefore mostly measure discussion of partisan politics, which is one
aspect of politics on which men typically report more interest. As
people usually discuss the topics they are most interested in, it seems
likely that women discuss more often political topics for which they
report more interest, such as health care or gender issues, although
survey questions typically do not ask what kinds of political questions
people discuss the most. Political discussion of various topics is seen
as something desirable in participatory democracy,\footnote{The
  long-standing concept of participatory democracy is also well accepted
  among feminist and intersectional theorists (Collins 2017; Phillips
  1992).} since partisan politics is only a means through which relevant
issues are often addressed.

Discussing with people with different ideas and views also creates a
phenomenon of collective intelligence (Landemore 2013), which is seen as
a desirable outcome from a democratic point of view, since it has been
found both to reduce political polarization and to produce better
reasoning, i.e., a better capacity at finding and evaluating arguments
in deliberative context (Mercier and Landemore 2012). Therefore, it
seems relevant to identify the socialization elements that lead to more
diversity in political discussions --- men and women have different life
experiences but also, often, different ideological viewpoints (Gidengil
et al. 2005).

\subsection{Running for Elected Political
Office}\label{running-for-elected-political-office}

Studies have repeatedly found that men have more political ambition than
women. For instance, there are twice as many men as women who seek to be
nominated as candidates in Canadian elections (Tolley 2023). This might
be because, in other contexts, women ``are more than twice as likely as
men to consider themselves `not at all qualified' to run for office
(29\% of women, compared to 14\% of men) {[}and{]} are roughly 22\% less
likely than men to report parental encouragement'' (Fox and Lawless
2005, 654). These two factors still stand in a 2021 follow-up study (Fox
and Lawless 2023). Political interest, self-perceived qualifications,
and family socialization all predict political ambition --- i.e., having
previously considered the possibility of running for office. Gender
differences in partisan political interest and political efficacy, while
not the only causes,\footnote{Studies have also found that highly
  visible politicians are covered more negatively by the media
  (Fernandez-Garcia 2016; Goodyear-Grant 2013), are more likely to be
  the targets of uncivil tweets (Rheault, Rayment, and Musulan 2019) and
  have a lower income than men --- even in Canada (Thomas 2013).
  Furthermore, when primed about the competitive aspects of politics,
  women's political ambition declines while men's remains stable (Preece
  and Stoddard 2015). These factors could also help to explain women's
  lower political ambition and representation.} might therefore help to
explain why women are under-represented in legislative assemblies in the
vast majority of countries (Inter-Parliamentary Union Parline 2023). In
Canada, too, political issues are most often settled by assemblies and
executives where men are the majority. In 2009, women represented less
than 25\% of Canadian legislators at the federal, provincial, and
municipal levels and only 15\% of mayors (Tolley 2011). At the time of
writing, while the Canadian federal cabinet is gender-balanced, nine out
of ten provincial premiers and most of their ministers are men.

Women's lower level of political ambition is not the only factor
contributing to their legislative under-representation, but other
factors often point in different directions. Discrimination against
women by gatekeepers (Ashe and Stewart 2012) and by parties who make
them candidates in hopeless ridings (Thomas and Bodet 2013) might
explain part of the gender gap in legislative representation in the
country, among other factors. Moreover, female candidates tend to be
more qualified than male candidates, and voters hold them to more
stringent qualification standards compared to male candidates (Bauer
2020). Still, in Canada, women receive as many votes as men on average
(Sevi, Arel-Bundock, and Blais 2019). They are also \emph{more} likely
than men to win elections at the municipal level when they run (Lucas et
al. 2021). An international study also finds female candidates get on
average 2 percentage points \emph{more} of the popular vote than male
candidates (Schwarz and Coppock 2022). Moreover, Canadian elites
discriminate against men when it comes to providing advice to political
aspirants (Dhima 2022).

\section{Potential Explanations: The Emergence of Gender
Differences}\label{potential-explanations-the-emergence-of-gender-differences}

Several theories have been suggested to account for women's lower levels
of self-reported political interest. This includes structural,
institutional, and individual factors. Yet, no convincing unified theory
has been proposed thus far to account for the emergence of this gender
gap.

\subsection{Structural Factors}\label{structural-factors}

\subsubsection{Society Values and
Culture}\label{society-values-and-culture}

Broad cultural differences between countries and through time explain
some of the aggregate variation in gender differences in political
interest. M. L. Inglehart (1981) shows that women are more interested in
politics in traditionally Protestant countries than in traditionally
Catholic countries. More recently, R. Inglehart and Norris (2003)
further suggest that modernization has a positive effect on cultural
attitudes towards gender equality: as societies move from agrarian to
industrial and from industrial to postindustrial, people, especially
younger generations, become more open to gender equality, though
history, religion, and institutions also play a role in shaping country
trajectory. Dassonneville and Kostelka (2021) also demonstrate that
cultural gender differences --- operationalized through differences in
boys' and girls' math scores --- explain countries' gender gap in
political interest. Still, these cross-country variations do not explain
why significant gender gaps in political interest remain in countries
where gender norms are more egalitarian or where women's rights are
stronger.

\subsubsection{Women's Political
Under-Representation}\label{womens-political-under-representation}

Women politicians' relative absence in politics might also explain part
of the gender gap in political interest. Wolbrecht and Campbell (2007)
find that adolescent girls are more likely to discuss politics with
friends in settings where there are more women legislators, while the
increase for adolescent boys is non-significant, suggesting women
politicians indeed can be role models who can increase girls' political
interest. Mayer and Schmidt (2004) and Bos et al. (2022) also find that
adolescent boys and girls in the United States, Mexico, Japan, and China
all think politics is a men's domain, and girls are more likely than
boys to report so. Adolescent boys also report higher political
interest. These two findings might be linked to the fact that citizens
in these four countries, like almost all countries in the world, are
represented by a majority of male legislators (Inter-Parliamentary Union
Parline 2023). Indeed, Bühlmann and Schädel (2012) find political
interest is higher among men than women in 33 European countries, but
this gap is smaller in countries with higher proportions of women in
their countrywide legislative assembly. They suggest the relationship is
not simply an artifact of reverse causality since it holds just as well
in countries with gender quotas as those without gender
quotas.\footnote{There is thus a feedback loop: political interest leads
  to political ambition, which leads to political representation, which
  in turn leads to political interest for the group with a higher level
  of representation.} Still, predicted probabilities show that women
would be significantly less interested in politics even in a country
whose parliament has reached gender parity.

\subsection{Institutional Factors}\label{institutional-factors}

\subsubsection{Political Institutions}\label{political-institutions}

Electoral systems could also influence gender differences in political
interest. Kittilson and Schwindt-Bayer (2010) find that proportional
election systems --- but not federalism and parliamentary systems ---
reduce gender gaps in political interest and political discussion
compared with plurality systems. Beyond this study, there have not been
any academic inquiries into the institutional determinants of the
political interest gender gap. Moreover, this distinction does not
address the root cause of the gender gap in political interest.

\subsection{Individual Factors}\label{individual-factors}

\subsubsection{Life-cycle events}\label{life-cycle-events}

Some individual factors that might contribute to gender differences in
political interest focus on adults and life-cycle events, including
motherhood and employment.

\paragraph{Motherhood}\label{motherhood}

First, parenthood might hurt political interest, especially for women.
In a British study, R. Campbell and Winters (2008) find that being a
mother has a negative effect on political interest similar in size to
being a father. However, women are more likely than men to report
lacking time to keep up with politics due to family time pressures.
Moreover, studies in the United States, Denmark, and Finland have found
that the birth of a first child reduces political discussion and
political participation more among women than men, especially in the
short run, since housework duties and child-rearing are
disproportionately handled by women (Bhatti et al. 2019; Gidengil,
Giles, and Thomas 2008). These findings might also translate to
political interest. Nevertheless, while earlier Canadian studies also
found that childbirth was negatively related to political participation
(Kay et al. 1987), more recent Canadian studies do not find a link
between childbirth, political discussion, and political participation
(Gidengil, Giles, and Thomas 2008; O'Neill et al. 2017).

\paragraph{Employment}\label{employment}

Second, labour force participation could have a different impact on
political interest for women relative to men, as was commonly
hypothesized when women started to enter the labour force (Bashevkin
1993). Schlozman, Burns, and Verba (1999) explain the rationale like
this: ``exposure on the job to a broader array of people and issues
would heighten engagement with politics {[}including political
interest{]}, especially among {[}United States{]} women'' (44). However,
both Schlozman, Burns, and Verba (1999) and Jennings and Niemi (1981)
fail to find strong effects of labour force participation on the gender
gap in political interest in the United States.

Mestre and Marı́n (2012) also find that women work on average three more
hours of paid and unpaid work than men and that the amount of unpaid
domestic work is negatively related to political interest for women,
suggesting women's lower political interest could stem from a lack of
time. However, the same study also finds that the amount of unpaid
domestic work is unrelated to political interest for men, which means
some other factors must be at play to explain gender
differences.\footnote{On the related questions of education and social
  class, Verba, Burns, and Schlozman (1997) find that these variables
  cannot explain the gender gap in self-reported political interest.}

\subsubsection{Biology}\label{biology}

\paragraph{Genetics}\label{genetics}

The impact of biology, and more specifically genetics, on political
interest, has been confirmed by recent studies on twins. Klemmensen et
al. (2012) find that political interest and political efficacy are
heritable and come from the same underlying genetic factor. The
heritability of political interest, i.e.~the proportion of the
variability in political interest due to genetic differences among a
population, varies in important ways from one study to the other and one
national context to the other. It has been estimated that heritability
stands at 24\% (Weinschenk and Dawes (2017), United States), 36\%
(Weinschenk and Dawes (2017), Minnesota), 40\% (Arceneaux, Johnson, and
Maes (2012), United States), 43\% (Klemmensen et al. (2012), United
States), 48\% (Ditmars and Ksiazkiewicz (2023), Germany), 50\% (Dawes et
al. (2014), Sweden and Weinschenk et al. (2019), Germany), 57\%
(Klemmensen et al. (2012), Denmark), and 62\% (Bell, Schermer, and
Vernon (2009), Canada and the United States). Canada therefore seems to
be a country in which genetics explain a lot of the variation in
political interest.

One of these studies, Ditmars and Ksiazkiewicz (2023), assesses
heritability by sex, with estimates of 21.9--50.8\% for women and
39.6--57.2\% for men. Despite these differences not being statistically
significant, they find that the heritability of political interest
remains stable for men across age groups on average, women's political
interest becomes more heritable and less a function of the shared
environment in which they live as they age. They attribute this result
to the fact that when they reach adulthood, ``women can more easily
select into environments in which they can pursue their predisposition
that drives interest in politics, which is supported by our indications
of larger heritability estimates for twins who moved out of the parental
home'' (11).

All these studies emphasize that genetic differences, when they are
found, add to but do not replace differences in political socialization.
While estimates vary between samples, these numbers imply that 38 to
76\% of the variability in political interest within these populations
likely comes from environmental factors. A recent study also finds that
``family socialization can compensate for (genetic) individual
differences and foster increased political engagement,'' including
political interest (Rasmussen et al. 2021, 1). Genetics can also
interact with environmental factors such as socialization (Ditmars and
Ksiazkiewicz 2023).

\subsubsection{Socialization}\label{socialization}

Overall, structural, institutional, life-cycle, and biological factors
provide a partial explanation of the gender gap in political interest,
but socialization seems to be an especially fruitful avenue, as
biological studies themselves admit. Hooghe (2022) defines political
socialization as ``the process where individual actors acquire political
attitudes as a result of outside influences from their direct
environment'' (99).

The idea that gender differences in political interest are rooted in
early childhood socialization has been argued by many for a long time
(Bashevkin 1993; Fraile and Gomez 2017). Mayer and Schmidt (2004) find
that political interest is slightly higher for boys than girls in grades
7--9 in China, Mexico, and the United States, but not in Japan. Bos et
al. (2022) find that girls are slightly more interested in politics at
6--7 years old, but then boys become more interested, and the gap grows
larger until early adolescence.

Moreover, international and longitudinal studies find that political
interest remains remarkably stable throughout life, including for
high-school students, whose political interest is already high (Fraile
and Sánchez-Vítores 2020; Neundorf, Smets, and Garcia-Albacete 2013;
Prior 2010, 2019). Still, studies suggest that political interest
becomes stronger from adolescence through early adult life (Neundorf,
Smets, and Garcia-Albacete 2013), and even more so for men (Jennings and
Niemi 1981, 276). Political socialization keeps happening at the adult
age --- but it does so at a lower rate than among children and
teenagers, which are therefore at the center of this dissertation.

Socialization can be carried out by multiple actors. Studies have found
children's political interest is mostly transmitted or influenced by
four agents of socialization: parents (Dostie-Goulet 2009b; Mayer and
Schmidt 2004; Neundorf, Smets, and Garcia-Albacete 2013; Shehata and
Amnå 2019), peers (Klofstad 2007; Shehata and Amnå 2019; Huckfeldt and
Sprague 1995), media (Holt et al. 2013; Lupia and Philpot 2005; Shehata
and Amnå 2019), and schools (Dassonneville et al. 2012; Mahéo 2019;
Neundorf, Niemi, and Smets 2016). These agents can also influence each
other, with Shehata and Amnå (2019) finding that political news
consumption by parents and peers can influence parents and peers'
political interest and, eventually, their children's. Transmission by
parents and peers can take the form of interactions --- notably
political discussions --- while media and schools can be agents of
socialization through the contents they produce --- notably political
news or citizenship education classes --- or the agents that bring them
--- journalists, teachers, or influencers.

The relative importance of these four agents has been studied by many.
In Quebec, Dostie-Goulet (2009b) finds that 14- to 17-year-old
teenagers' political interest is better predicted by the frequency of
political discussions with their parents than with friends, while
discussions with history class teachers have a significant but lesser
influence on children's political interest. In Sweden, Shehata and Amnå
(2019) finds that 13- to 18-year-old teenagers' political interest is
mostly affected by parents' political interest, with peers also having
an important influence but not long-lasting effects. In Germany, Oswald
and Schmid (1998) find that political discussions and
information-getting among 16-to-18--year-olds come mainly from schools
and media, followed by parents and friends. In Finland, Koskimaa and
Rapeli (2015) find that 16- to 18-year-olds' political interest is
mainly related to the presence of politics in their family and among
their friends, with friends being the most important influence, contrary
to Dostie-Goulet's (2009b) and Shehata and Amnå's (2019) results. The
influence of school is marginally statistically significant. Finally,
Jennings and Niemi (1981) and Neundorf, Smets, and Garcia-Albacete
(2013) find that parents' role in political interest transmission mostly
occurs during teenage years, while other factors explain growth among
adults.

In practice, two main theories may help explain the gendered influence
of these four agents in transmitting political interest to children:
\emph{social learning theory} and \emph{gender homophily theory}. Both
theories point to a broad influence of same-gender role models in
socialization.

\paragraph{Social learning theory}\label{social-learning-theory}

It is not by accident that daughters and sons of famous politicians
often themselves become politicians, as studies of family political
dynasties have found (Geys and Smith 2017). In Canada, in recent memory,
one can think of Justin Trudeau, Caroline Mulroney, Preston Manning or
Ches Crosbie, who were all sons and daughters of prominent politicians
and themselves rose to different levels within the political sphere.
\emph{Social learning theory} suggests that children learn through the
observation of their parents and peers, and model their behaviour,
attitudes, habits and values after them (Gidengil, Wass, and Valaste
2016; Shehata and Amnå 2019). Through this process of observation of
others, ``children gradually learn what is considered appropriate and
socially rewarded, and what is not in different contexts'' (Shehata and
Amnå 2019, 1058). Transmission of political attitudes is deemed to be
more effective when cue-giving and reinforcement from the socializing
agent are strong and consistent (Jennings, Stoker, and Bowers 2009;
Prior 2019).

While research about social learning and politics has most often been
applied to the transmission of political participation (Gidengil, Wass,
and Valaste 2016) or political engagement more generally (Jennings and
Niemi 1981), a few recent studies argue social learning can also apply
to orientations such as interest --- in this case, political interest.
Prior (2019) argues social learning can explain the transmission of
political interest from parents to children. He suggests this
transmission process can only occur at a time when children still live
in the family home and after they start acquiring an understanding of
what politics is --- that is, in their early teens. Shehata and Amnå
(2019) also hypothesize parents and peers who value current news affairs
knowledge can transmit their political interest to children through both
learning opportunities and social pressure. Learning opportunities can
occur through political discussions with parents and peers, or through
children's exposition to their parents watching or reading the news
daily, which can give them a more concrete understanding of how politics
affects their lives, a factor that predicts the development of political
interest. Social pressure can occur when children feel they need to take
an interest in politics to develop a sense of belonging or social
identity. Jennings, Stoker, and Bowers (2009) also highlight parents'
important role in fostering higher political engagement --- including
political interest --- among their children, linking that role to a
social learning process.

Research has found that social learning exhibits gender effects:
\emph{observer--model similarity} leads children to model their
behaviour, values and attitudes based on the behaviour, values and
attitudes of models that resemble them (Bandura 1969). Indeed, past
research has shown that the trickle-down effect of political engagement
from parents to children works in gendered ways, with sons modeling
their behaviour after their fathers and daughters after their mothers
(Gidengil, Wass, and Valaste 2016).\footnote{This dissertation uses the
  words ``mother'' and ``father'' without implying that there is exactly
  one mother and one father per family. Other situations are very
  common.} Political interest is also more strongly correlated between
mothers and daughters and between fathers and sons than any other
combination (Beauregard 2008; Neundorf, Smets, and Garcia-Albacete 2013;
Owen and Dennis 1988).

\paragraph{Gender homophily theory}\label{gender-homophily-theory}

\emph{Homophily} is ``the principle that a contact between similar
people occurs at a higher rate than among dissimilar people''
(McPherson, Smith-Lovin, and Cook 2001, 416). \emph{Gender homophily}
refers to how children of the same gender tend to stick together and
become friends, from at least the beginning of primary school (Stehlé et
al. 2013). Children are even more likely to drop friends who have a
friend from the other gender than to add the other-gender friend to
their own group of friends (McPherson, Smith-Lovin, and Cook 2001).
These patterns increase during primary school but diminish --- without
disappearing --- after puberty (Shrum, Cheek Jr., and Hunter 1988;
Stehlé et al. 2013).

Studies about gender homophily in social networking websites among
teenagers and adults have yielded more nuanced results. Thelwall (2009)
finds that MySpace users have similar levels of interactions regardless
of their gender. However, Laniado et al. (2016) find strong gender
homophily among adolescent girls and boys --- though not always for the
same kinds of online activities --- on a Spanish social networking
service.

\section{Political Interest Dissected: Interest in Different Political
Topics}\label{political-interest-dissected-interest-in-different-political-topics}

Perhaps as a result of social learning processes, research has found
that men and women are often interested in different issues. It is
therefore possible that women are simply interested in aspects of
politics that they do not always see as political. Studies have
suggested the typically reported gender gap in political interest might
be the result of inadequate measurement (Alozie, Simon, and Merrill
2003; Bourque and Grossholtz 1974; Keeling 2023; Tormos and Verge 2022).
While Alozie, Simon, and Merrill (2003) use a very broad concept of
``political orientation'' including discussion, attention, and
participation in various activities, recent studies have found a
creative way of addressing the limitations of the traditional measure of
political interest. On average, women report more interest in topics
such as health care, education and gender issues, while men report more
interest in foreign policy, partisan politics, and law and order, and
topics such as taxes and local politics seem to be equally interesting
to men and women (R. Campbell and Winters 2008; Coffé 2013; Ferrin et
al. 2020; Hayes and Bean 1993; Kuhn 2004; Sabella 2004a; Tormos and
Verge 2022; Verba, Burns, and Schlozman 1997). These studies generally
conclude that men and women are simply interested in different domains
of politics and that politics is mentally associated with elections and
parties, topics in which men have more interest,\footnote{This mental
  association might be culture-specific to some degree. Conceptions of
  politics could be more removed from partisan politics in other
  situations such as Northwest Territories' and Nunavut's non-partisan
  legislative assemblies, but also in authoritarian one-party regimes.}
therefore leading women to report lower levels of political interest
overall. Tormos and Verge (2022) find that prompting for interest in
several topics for which women typically report more interest, including
gender issues and health care, makes women and men rate their political
interest similarly, with an increase in political interest stronger
among women than among men. Conversely, prompting respondents for
partisan politics slightly reduces political interest among both men and
women, but more strongly among women. Keeling (2023) uses a similar
research design and also finds that prompting for interest in several
topics including gender issues reduces the gender gap in self-reported
political interest, although this decline is due to a decline in
self-reported political interest among men rather than an increase among
women.\footnote{Among institutional and structural mentioned in the
  previous section, it is unclear whether these factors explain the
  gender gap in partisan political interest only, or interest in several
  political topics too.}

Why would women be more interested in health care, education and gender
issues specifically? Kuhn (2004) argues that women's ``political
thinking revolves around terms of compassion and cooperation, not around
contest and competition. Females are interested in solving concrete
problems and are driven by social empathy, egalitarian values, and
engagement for other people. {[}\ldots{]} Females prefer unconventional
forms of political participation'' (96). Women's higher interest in
gender issues such as abortion and gender-based violence can be
explained by the fact that these issues are more likely to affect them
directly (Ferrin et al. 2020). For similar reasons, R. Campbell and
Winters (2008) suggest that women's higher interest in health services
matches with statistics suggesting that women are more frequent users of
these services compared to men. Moreover, women's ``continuing
disproportionate share of childcare responsibilities may also lead them
to be more interested in issues relating to education and social
services'' (64). Stolle and Gidengil (2010) also find women give better
answers than men to knowledge questions about the health care system,
and scholars have found strong links between political interest and
political knowledge --- the former often predicting the latter (Elo and
Rapeli 2010; Pettey 1988; Prior 2019).

Why would men be more interested in foreign policy, partisan politics,
and law and order? The competitive nature of partisan politics seems to
tap into men's tendency towards agency, leading them to express more
interest in that aspect of politics. While studies are unclear about why
men report more interest in foreign policy and law and order, it is
possible to argue that men's higher rates of incarceration, stronger
presence in the police force, and overrepresentation among world leaders
could lead them to take more interest in these areas. Stolle and
Gidengil (2010) also find men give better answers than women to
knowledge questions about partisan politics.

The reasonings behind differences in interest for different topics often
center around a dichotomy between competition and cooperation, which may
have to do with different personality traits --- also potentially
resulting from socialization.

\subsection{Personality Traits}\label{personality-traits}

More specifically, gender may affect some personality traits that are
relevant to the development and types of political interest. Rancer and
Dierks-Stewart (1985) show that biological sex does not predict
argumentativeness, while gender identity does, with masculine
individuals more argumentative than feminine individuals. They measure
gender identity using a scale of \emph{expressive} (feminine) and
\emph{instrumental} (masculine) behaviours. In a literature review,
Infante and Rancer (1996) similarly find that men are more likely than
women to value arguing and engage in it, except for workplace-related
arguing. Similarly, Shaw (2002) finds that female MPs in the United
Kingdom are less likely to resort to adversarial language than male MPs.
Experimental studies by Niederle and Vesterlund (2007) and Kanthak and
Woon (2015) also find that men are less risk-averse than women and tend
to value competition more than women do, although Sevi and Blais (2023)
find the opposite to be true, with women showing as much willingness as
men, or perhaps slightly more, to participate in competitive elections.
Individuals also tend to think men are better than women at negotiating,
something that is then internalized by women who behave according to
gender-based expectations, according to a United States study and a
literature review (Eckel, De Oliveira, and Grossman 2008; Kray,
Thompson, and Galinsky 2001; A. K. Schneider 2017).

These differences seem to be politically relevant since a recent British
study by R. Campbell and Winters (2008) shows that men's higher
self-reported political interest derives from the fact that they are
more \emph{agentic}, i.e., focused on self-assertion, while women are
more \emph{communal}, i.e., focused on cooperation. Since the concept of
\emph{politics} is typically seen as more adversarial, it appeals more
to agentic types --- mostly men --- who then develop higher political
efficacy and overall self-reported political interest, a finding also
shared by M. C. Schneider et al. (2016). Yet, our definition of politics
includes both its competitive and cooperative aspects.

\section{Hypotheses: Towards a Unified Socialization
Theory}\label{hypotheses-towards-a-unified-socialization-theory}

This dissertation asks \emph{``How do men and women differ with regard
to political interest, and why?''} It hypothesizes that political
socialization plays a central role in explaining the emergence of gender
differences in interest in political interest, with parents and peers
the most important agents of childhood political socialization.
Furthermore, it emphasizes that gender differences are more about the
type of political interest rather than the level of political interest.

While gender tendencies towards communality and agency respectively
contribute to women's interest in health care and men's interest in
partisan politics,\footnote{These processes might be reinforced by
  societal expectations about men's and women's roles, although this
  explanation is more structural than socialization-driven.} a broader
application of socialization theories that includes parents and peers
would provide a more comprehensive understanding of who influences girls
and boys in how much interest they have in various aspects of politics.
Gender homophily and social learning both suggest that children are
influenced by same-gender role models, but these theories need to be
further specified. A broad application of social learning theory would
predict a parent's political interest influences the development of
their children's political interest more strongly for their same-gender
children than other-gender children, but whether this applies to
interest in political topics other than partisan politics has not been
tested. On the other hand, gender homophily predicts more same-gender
than mixed-gender peer discussions, and this finding applies to
political discussions as well. However, no study has thus far examined
the implications of this theory for gender differences in political
interest.

The study seeks to bridge two kinds of literature on gender,
socialization, and political interest; one emphasizing personality
traits leading to interest in different political topics, and the other
emphasizing the transmission of political interest by same-gender role
models. The goal is to measure interest in various political topics and
link it to parents' and peers' interest in those same topics. This is
something that has not been done before; parents' and children's
political interest is typically compared using a single measure of
political interest, but we do not know if same-gender role models have
the same impact on interest transmission for each political topic.

\emph{\textbf{Hypothesis 1}: Children's interest in specific political
topics is more affected by political discussions with their same-gender
parent(s) than other-gender parent(s).}

\emph{\textbf{Hypothesis 2}: Children's interest in specific political
topics is more affected by political discussions with their same-gender
peers than other-gender peers.}

According to Hypothesis 1, on average, a mother will have more
transmission potential of her interest in specific political topics to
her daughters than sons through political discussion, for example.
Hypothesis 2 uses the same logic for same-gender peers. Hypotheses 1 and
2 are summarized in Figure~\ref{fig-hypotheses}, with a few additional
background details. Put neatly, the general theory is that
\emph{children's interest in specific political topics comes mainly from
socialization by their same-gender parent(s) and peers}.

\begin{figure}

\centering{

\includegraphics{_graphs/DissertationHypotheses.pdf}

}

\caption{\label{fig-hypotheses}Theoretical Framework}

\end{figure}%

\section{Organization of the
Dissertation}\label{organization-of-the-dissertation}

This dissertation mostly relies on survey data collected among Canadian
children and teenagers. Chapter~\ref{sec-chap2} outlines the data and
methods used in all other chapters of this dissertation. It also
provides descriptive statistics about some of the main variables used in
the data analysis, including gender, personality traits, role models,
interest in various political topics, and socio-demographic variables.

The next four chapters are empirical. Chapter~\ref{sec-chap3} studies
the evolution of political interest through life, answering questions
about the stability of political interest and the gender gap in
self-reported political interest. Chapter~\ref{sec-chap4} explores the
role of parents in the transmission of political interest and provides
the answer to Hypothesis 1. Chapter~\ref{sec-chap5} does the same for
the role of peers, answering Hypothesis 2. Finally,
Chapter~\ref{sec-chap6} provides an overall assessment of the influence
of role models on the development of political interest among children
and teenagers, and of the extent to which socialization can explain
political interest gender gaps. Its overarching goals are to provide a
better understanding of how men and women view politics differently, to
raise awareness about this discrepancy and to provide recommendations
for actors who seek to encourage interest in some --- or many ---
political topics among adult citizens and future adult citizens of both
genders.

\bookmarksetup{startatroot}

\chapter{Data, Methods and Descriptive Statistics}\label{sec-chap2}

This chapter seeks to explain the dissertation's datasets and methods,
and analyze descriptive statistics about the main explanatory and
outcome variables used in the next chapters. It seeks to answer the
following question: How can we study the development of political
interest through gendered socialization processes? The types of
analyses, survey weights, and decisions about which variables to include
are provided.

\hfill\break

Chapter~\ref{sec-chap1} has established that political interest is an
important predictor of political discussion and of more concrete forms
of political engagement, which makes this concept worthy of scientific
inquiry. Measuring political interest, its development, and the
socialization processes that lead to it requires a methodological
approach that relies on data collected among both children and adults.
It also requires the data collected among children to inquire about all
potential sources of socialization --- parents, peers, media, and
schools. The measurement of gendered personality traits among children
--- self-assertion for boys and cooperation for girls --- is also an
important factor that may explain how transmission processes work. This
chapter answers the following question: \emph{How can we study the
development of political interest through gendered socialization
processes?} It starts by detailing the datasets and methods that will be
used throughout the chapters of this dissertation. It also provides
descriptive statistics about some of the main variables that will be
used, notably those related to political interest as well as
socio-economic status. Before doing so, it provides a brief explanation
of the case selection.

\section{Why Canada?}\label{why-canada}

The data used for this dissertation comes from Canada, the country I
live in and know best. As Noël (2014) puts it, studying one's own
country produces social scientific knowledge that is directly relevant
to and usable by citizens of that country, and it therefore has a
uniquely important character. Moreover, for the current study, Canada is
a particularly interesting country to study, given that it is often
classified by reports as one of the best for women to live in (Conant
2019; Equal Measures 2030 2020; US News \& World Report 2020), even
though only 30.7\% of elected MPs are women, the 61st highest percentage
among the world's 193 countries (Inter-Parliamentary Union Parline
2023). Canadian Election Study (CES) data since 1997 shows that the
gender gap in political interest has remained fairly stable despite a
recent increase in the percentage of women politicians at the national
and provincial levels (Sevi 2021). These characteristics make Canada a
country worth studying to better understand the underlying reasons
behind the stability of the gender gap in the aggregate measure of
political interest, which mainly taps into interest in partisan
politics. Studies that measure interest in different political topics
have been conducted in Europe and the Middle East, but it is unclear how
the gender differences they find for many topics apply in Canada.
Studying the transmission of political interest to children in Canada
seems like a promising place to look for explanations of these gender
differences that may, or may not, apply elsewhere.

\section{Data}\label{data}

Given the abstract nature of political interest, the observation of
children's behaviour would not provide an accurate answer to this
dissertation's questions. Political interest is different from the
frequency of political discussions and different from political
participation. Surveys, compared with interviews or focus groups, make
it easier to get reliable interest scores that can be compared across
respondents of different backgrounds. Interviews and focus groups are
also more likely to be subject to social desirability bias than surveys
--- students might overestimate their interest in politics and the
frequency of discussions with parents. This dissertation therefore
relies on survey data.

Within Canada, this dissertation relies on five datasets as well as the
2021 Canadian Census. Two of these datasets have been collected for this
dissertation while the other three are large-scale surveys mobilized to
provide extra context and better estimates of political engagement in
Canada --- and its evolution as Canadians age. The two datasets that
were collected for this dissertation share similar questions about
political interest and about interest in five specific political topics:
health care, international affairs, law and crime, education, and
partisan politics. These five topics are those for which past studies
have reported the largest gender differences in interest: women
typically report being more interested in the political aspects of
health care and education, while men typically report being more
interested in the political aspects of the other three topics mentioned
(R. Campbell and Winters 2008; Coffé 2013; Ferrin et al. 2020; Hayes and
Bean 1993; Kuhn 2004; Sabella 2004a; Verba, Burns, and Schlozman 1997).

\subsection{Canadian Children Political Interest Survey
(CCPIS)}\label{canadian-children-political-interest-survey-ccpis}

To create the Canadian Children Political Interest Survey (CCPIS), data
was collected among 698 Canadian children and adolescents aged 9 to 18
from seven elementary and secondary schools and one school board's
Student Senate, all located in various urban areas of Quebec and
Ontario,\footnote{All are considered urban since they are part of a
  census metropolitan area (CMA) as defined by Statistics Canada
  (2023a).} between August 2022 and January 2023. Students had to fill
out a 15-minute online survey questionnaire during classroom time, under
supervision by their teacher, who gave them the link to fill out the
survey. The survey was hosted on the Qualtrics Web platform and was
available in both French and English. This part of the data collection
received clearance from the University of Toronto's Social Sciences,
Humanities and Education Research Ethics Board. It seems worthwhile to
study political socialization in the period of life where political
interest is developed to better understand gender differences since
there seems to be some level of path dependency in individuals'
political interest afterwards (Prior 2010, 2019).

Table~\ref{tbl-descriptive} shows the number of classrooms and students
in each of the eight schools included in the final sample of the CCPIS,
as well as additional information about the schools. Participating
schools include a mix of private and public schools in Quebec and
Ontario, including students from all age groups.

\begin{longtable}[]{@{}
  >{\raggedright\arraybackslash}p{(\columnwidth - 16\tabcolsep) * \real{0.0194}}
  >{\raggedright\arraybackslash}p{(\columnwidth - 16\tabcolsep) * \real{0.0774}}
  >{\raggedright\arraybackslash}p{(\columnwidth - 16\tabcolsep) * \real{0.1290}}
  >{\raggedright\arraybackslash}p{(\columnwidth - 16\tabcolsep) * \real{0.1161}}
  >{\raggedright\arraybackslash}p{(\columnwidth - 16\tabcolsep) * \real{0.0839}}
  >{\raggedright\arraybackslash}p{(\columnwidth - 16\tabcolsep) * \real{0.1290}}
  >{\raggedright\arraybackslash}p{(\columnwidth - 16\tabcolsep) * \real{0.1484}}
  >{\raggedright\arraybackslash}p{(\columnwidth - 16\tabcolsep) * \real{0.1419}}
  >{\raggedright\arraybackslash}p{(\columnwidth - 16\tabcolsep) * \real{0.1548}}@{}}
\caption{Descriptive statistics, CCPIS
data}\label{tbl-descriptive}\tabularnewline
\toprule\noalign{}
\begin{minipage}[b]{\linewidth}\raggedright
\textbf{ID}
\end{minipage} & \begin{minipage}[b]{\linewidth}\raggedright
\textbf{Type}\footnote{The three public bodies that accepted to be part
  of the study are associated with three different school boards.}
\end{minipage} & \begin{minipage}[b]{\linewidth}\raggedright
\textbf{Language}
\end{minipage} & \begin{minipage}[b]{\linewidth}\raggedright
\textbf{Province}
\end{minipage} & \begin{minipage}[b]{\linewidth}\raggedright
\textbf{Ages}\footnote{Age groups are for the school itself, not the
  classrooms selected.}
\end{minipage} & \begin{minipage}[b]{\linewidth}\raggedright
\textbf{Number of students in body}
\end{minipage} & \begin{minipage}[b]{\linewidth}\raggedright
\textbf{Number of students in sample}
\end{minipage} & \begin{minipage}[b]{\linewidth}\raggedright
\textbf{Number of classrooms in sample}
\end{minipage} & \begin{minipage}[b]{\linewidth}\raggedright
\textbf{Number of teachers in sample}
\end{minipage} \\
\midrule\noalign{}
\endfirsthead
\toprule\noalign{}
\begin{minipage}[b]{\linewidth}\raggedright
\textbf{ID}
\end{minipage} & \begin{minipage}[b]{\linewidth}\raggedright
\textbf{Type}{}
\end{minipage} & \begin{minipage}[b]{\linewidth}\raggedright
\textbf{Language}
\end{minipage} & \begin{minipage}[b]{\linewidth}\raggedright
\textbf{Province}
\end{minipage} & \begin{minipage}[b]{\linewidth}\raggedright
\textbf{Ages}{}
\end{minipage} & \begin{minipage}[b]{\linewidth}\raggedright
\textbf{Number of students in body}
\end{minipage} & \begin{minipage}[b]{\linewidth}\raggedright
\textbf{Number of students in sample}
\end{minipage} & \begin{minipage}[b]{\linewidth}\raggedright
\textbf{Number of classrooms in sample}
\end{minipage} & \begin{minipage}[b]{\linewidth}\raggedright
\textbf{Number of teachers in sample}
\end{minipage} \\
\midrule\noalign{}
\endhead
\bottomrule\noalign{}
\endlastfoot
1 & Private & French & Quebec & 12--17 & 450 & 133 & 5 & 2 \\
2 & Public & French & Quebec & 12--17 & 690 & 196 & 10 & 2 \\
3 & Private & French & Quebec & 12--17 & 670 & 78 & 3 & 1 \\
4 & Private & French & Quebec & 12--17 & 900 & 253 & 12 & 3 \\
5\footnote{Mixed on-site/online school.} & Private & English & Ontario &
14--18 & & 5 & 3 & 2 \\
6 & Public & French & Quebec & 5--12 & & 14 & 1 & 1 \\
7 & Private & English & Ontario & 5--14 & & 4 & 3 & 1 \\
8\footnote{Student Senate, a body located at the school board level.} &
Public & English & Ontario & 14--18 & 15 & 15 & 1 & 1 \\
& & & & & \textbf{Total} & \textbf{698} & \textbf{38} & \textbf{13} \\
\end{longtable}

In total, 75 school boards,\footnote{This figure also includes
  institutions known under names such as school service centers, school
  divisions, school districts, school councils, and centers for
  education.} 47 public schools, and 83 private schools across Canada
were originally contacted to take part in the project. Of the 75 school
boards that were contacted, 10 accepted to be part of the study (13\%),
15 refused (20\%), 27 did not reply to emails (36\%), and 23 did not
follow through or required extensive information that was impossible to
provide (31\%). On that last point, several school boards required
extensive documentation about the study or background police checks,
even for virtual data collection, and most of those who were sent that
information still refused to take part in the study after the
documentation was provided to them, often citing a lack of time. Of the
132 schools that were contacted, 10 accepted (8\%), 9 refused (7\%), 2
were deemed ineligible after checking (2\%), 88 did not respond (67\%)
--- in most cases after several attempts --- and 21 stopped replying
after a few email exchanges (16\%). Among the schools that accepted to
take part in the study, 13 teachers made their students fill out the
survey (59\% of all teachers contacted).\footnote{Non-response and
  rejection rates after the first contact are not reported by
  Dostie-Goulet (2009b), Prior (2019), Neundorf, Smets, and
  Garcia-Albacete (2013) or other similar studies about political
  interest among children and teenagers.}

The proportion of schools and school boards who refused to take part in
the CCPIS is admittedly high. Two main factors are likely at play.
First, schools often cited a lack of time or specified they had already
said yes to several other academic projects. The demands on schools and
teachers are high, and a study about socialization may not rank among
their short-term priorities. Second, despite the emphasis in all
documentation and emails about the ease of the process --- with very few
drawbacks on teachers other than coordinating a 15-minute survey period
where students could use either their phones or laptops to fill out the
survey --- there was no direct benefit for schools, financial or other.
Such payments to schools may have raised concerns among schools or
parents.

Within each classroom, 18 students filled the survey on average --- a
number that rises to 22 per classroom when removing school 5, whose
students did not have an allocated classroom period to fill the survey
and therefore did so out of their free time. 22 per classroom is a
number very close to the average 20 to 24 students per classroom in
Ontario and Quebec schools (Bolduc 2023).\footnote{Dostie-Goulet (2009a)
  and Dostie-Goulet (2009b) also reported high within-class response
  rates of 90\% and 80\% respectively.} In total, the 698 students who
took part in the survey, while not as large as the 1,000 generally
reached in regular cross-country opinion polling in Canada, make it
possible to analyze trends that go beyond the local or school level with
sufficient statistical power.

Low response rates for schools and school boards imply that the sample's
descriptive characteristics should not be seen as representative of the
Canadian student population. This sample should be considered a
non-random, convenience sample. Nevertheless, as will become clearer
when examining student-level descriptive statistics, there is no
theoretical reason to believe that the types of correlations studied
here --- transmission of political interest by role models --- should be
affected by the convenience nature of the sample. Given the scarcity of
available data on school-aged children's political predispositions
within Canada, this dataset also provides relevant information about
students' interests, role models, and relation to politics. The core of
the data analysis in each chapter therefore relies on this dataset.

The CCPIS includes information about students' interest in various
political topics, the political topics they say they discuss with their
mothers, fathers and peers, as well as the political topics discussed by
a teacher and an influencer they like. The gender of these role models
is also inquired, to test the implication from social learning theory
that political interest transmission should occur mainly between a child
and role models of their gender, more than role models of other genders
(Prior 2019; Shehata and Amnå 2019).

For questions about students' political interests, the following
questions are asked: ``How interested are you in politics generally? Set
the slider to a number from 0 to 10, where 0 means no interest at all,
and 10 means a great deal of interest'', and ``If you were to open a
news website and see the following articles how interested would you be
in reading each article? Set the slider to a number from 0 to 10, where
0 means `Not at all interested, I would not read it,' and 10 means `Very
interested, I would most likely read it.' (a) Health care (i.e.,
pandemic restrictions, working conditions of nurses); (b) International
affairs (i.e., diplomatic disputes between Canada and China, Ukrainian
war); (c) Law and crime (i.e., police funding, sentences for violent
crimes); (d) Education (i.e., university tuition, funding of public and
private schools); (e) Partisan politics (i.e., federal elections,
political parties)''. These questions specify concrete examples of
political issues related to each topic, in order to make the political
aspects of these topics more salient and avoid students answering while
thinking about their health or the classes they are taking. The
question's phrasing is meant to be easily understood by children and
teenagers.

The questionnaire was pre-tested on two teenagers, a boy and a girl aged
12. Minor adjustments were made to the questionnaire to ensure children
of all ages would understand the questions asked of them, notably with
international relations topics. This involved selecting more recent
events (i.e., the Ukrainian War and diplomatic disputes between Canada
and China) rather than events that occurred a longer time ago. Since
topics with some degree of importance in recent news were selected, both
topics chosen also embody competitive rather than cooperative aspects of
international relations, which may reinforce agentic children --- more
often boys according to past research --- to report more interest in
them. The questionnaire's length was also deemed to be reasonable during
these pre-tests.

The main socialization-related explanatory variables for children's
political interest identified in the literature concern the role of four
agents of socialization: parents, peers, schools, and media. In order to
determine the importance of these role models' gender, the following
questions were asked to children: ``Which parent do you discuss most
often with? (a) Mother; (b) Father; (c) Both equally''; ``What is the
gender of most of your friends? (a) Girls; (b) Boys; (c) About the same
for both genders''; ``Think about a teacher that you like(d). Is that
teacher\ldots{} (a) A woman; (b) A man; (c) Other (e.g.~Trans,
non-binary, two-spirit, gender-queer)''; ``Think about someone that you
like and sometimes read or watch on social media --- including YouTube.
Is that person\ldots{} (a) A man; (b) A woman; (c) Other (e.g.~Trans,
non-binary, two-spirit, gender-queer).'' These questions make it
possible to assess the influence of role models and to test an
interaction effect with role models' gender. Questions are then asked
about the topics most discussed by parents, peers, teachers and
influencers among the five political topics highlighted beforehand.

The CCPIS also uses the Personal Attributes Questionnaire (Spence and
Helmreich 1978; Ward et al. 2006) to assess students' degrees of agency
and communality and to assess whether interest correlations between
children and their same-gender role models change when controlling for
the child's degree of agency or communality. Men are typically more
assertive, which brings them closer to partisan politics, while women
are typically more cooperative, which leads them to be more interested
in topics such as health care and education. Given the importance of the
agency/communality distinction in explaining gender differences in
political interest (R. Campbell and Winters 2008), this hypothesis is
tested alongside hypotheses about the role of parents, peers and other
role models.

Finally, the CCPIS asks socio-demographic questions about children's
gender, language spoken at home, immigrant status, age, and ethnicity.
These questions are mostly inspired by the Canadian Election Study
(Stephenson et al. 2020), except for the ethnicity question, which is
based on the 2016 Census (Statistics Canada 2017). No question is asked
about students' socio-economic background or their parents' level of
education since this information may be difficult for them to reliably
assess.\footnote{Data about schools' socio-economic level is available
  for public schools in Quebec, but only one public Quebec school took
  part in the CCPIS.} The English and French CCPIS questionnaires are
available in Appendix~\ref{sec-appendix1} and
Appendix~\ref{sec-appendix2}.

\subsection{Datagotchi Post-Election Survey (Datagotchi
PES)}\label{datagotchi-post-election-survey-datagotchi-pes}

The second dataset collected throughout this dissertation is the 2022
Quebec Datagotchi Post-Election Survey, which includes data on 2,228
Quebec adult respondents (Leadership Chair in the Teaching of Digital
Social Sciences 2023). This dataset was collected in February and March
2023 using a panel of respondents who agreed to be contacted by email
during the 2022 Quebec general election through the Datagotchi Web app
(Leadership Chair in the Teaching of Digital Social Sciences 2022).
Opt-in surveys like this one are typically more reliable but less
externally valid than traditional surveys (Herrick et al. 2019; Thielo,
Graham, and Cullen 2021). For this dissertation, the questions of
interest that were asked to panel respondents inquired about their level
of interest in the five aforementioned political topics and general
socio-economic status. The same question wording was used among
students, with the same five political topics. The Datagotchi
Post-Election Survey French questionnaire is available in
Appendix~\ref{sec-appendix3}.

\subsection{Cross-Country Datasets}\label{cross-country-datasets}

Other than these two main datasets, three often-used datasets including
Canadian respondents are mobilized to give a more general empirical
overlook of political interest in Canada. These datasets include only
one question about political interest --- rather than interest in five
topics --- but each of them asks about other forms of political
engagement --- which will further be analyzed in Chapter~\ref{sec-chap3}
--- and two of them provide time-series data about political interest
and engagement, while the third provides a less political questionnaire
and therefore provides data about a different crowd of Canadian
respondents which may be more representative of Canadians' overall level
of interest in politics.

First, Canadian Election Study (CES) data on various aspects of
political engagement is analyzed (Stephenson et al. 2022). Other than
typical socio-economic status questions, the CES includes data about
political interest, political knowledge, internal efficacy, external
efficacy, political discussion, political debating, participation in
protests, participation in boycotts, participation in petitions, party
membership, donations to parties, and voting. CES data allows time
series analyses of the evolution of gender gaps, with some political
engagement questions having been asked since 1965. There were 20,968
respondents to the 2021 CES.

Second, time-series World Values Survey (WVS) data is also used. These
data, like CES data, make it possible to visually represent descriptive
statistics by age and gender on political engagement. Moreover, the
World Values Survey (WVS) Wave 7 (2017--22) includes data from Canada
(2020) and 56 other countries, therefore providing the opportunity for
cross-country comparisons on political interest and political engagement
(Haerpfer et al. 2022). Other than typical socio-economic status
questions, respondents were asked a different political interest
question than in other surveys: ``How interested would you say you are
in politics? (a) Very interested; (b) Somewhat interested; (c) Not very
interested; (d) Not at all interested'' (Haerpfer et al.
2022).\footnote{This 4-point scale is transformed into a 0--10 scale
  comparable with the CES's political interest scale, using the same
  method as Prior (2010).} Time-series data about political interest are
also used in some analyses, with political interest questions being
asked as early as 1982 in some countries. There were 4,018 Canadian
respondents to the 2020 WVS and 84,638 for wave 7 worldwide.

Third, the 2013 and 2020 Canadian General Social Surveys (GSS), cycles
27 and 35 (Social Identity), are also used to get data about Canadians'
political interest and engagement coming from a less political survey.
The GSS's political interest questions ask ``Generally speaking, how
interested are you in politics? (e.g., international, national,
provincial or municipal) (a) Very interested; (b) Somewhat interested;
(c) Not very interested; (d) Not at all interested'' (Statistics Canada
2023b). There were 34,044 respondents to the 2020 GSS.

\subsection{Canadian Census}\label{canadian-census}

Finally, data from the 2021 Canadian Census is used to compare data from
all other sources to Canadian population figures (Statistics Canada
2022) --- or Quebec population figures, in the case of the Datagotchi
PES. More specifically, the Individual Public Use Microdata Files (PUMF)
are used to cross-reference data. The PUMF includes data collected among
980,868 Canadians or a representative segment of 2.7\% of the Canadian
population. These data are specifically meant to be used to create
survey weights and calculate aggregate statistics about the Canadian
population, which I do with the Datagotchi PES data.

\section{Methods}\label{methods}

\subsection{Avoiding Social Desirability
Bias}\label{avoiding-social-desirability-bias}

Questions about political interest can be subject to some degree of
social desirability bias, which is ``the difference between an
individual's own intention and his/her perception of his/her peers'
intention'' (Chung and Monroe 2003, 296). Respondents may want to
portray themselves as ``good citizens'' who are interested in public
affairs, regardless of their real interest in politics. However, it is
not entirely clear whether girls' scores or boys' scores would be more
inflated as a result of social desirability bias. Studies have found
that in Canada, Australia, and the United States, \emph{women} generally
exhibit higher social desirability bias when answering survey questions
(Chung and Monroe 2003; Cohen, Pant, and Sharp 1998, 2001). However,
studies in these same countries and more instead show that \emph{men}
are more likely to over-report having voted in the previous elections
(Belli, Traugott, and Beckmann 2001; Herrick and Pryor 2020; Stockemer
and Sundstrom 2023). For political interest, it is unclear which gender
--- if any --- tends to over-report it, but disaggregating the concept
into five topics might be the best approach to avoid systematic social
desirability bias associated with one particular question. This is
therefore the approach taken throughout this dissertation.

\subsection{Controlling for Classroom-Level
Effects}\label{controlling-for-classroom-level-effects}

In the CCPIS dataset, students are clustered within classrooms, which
are themselves nested within schools and provinces. There may be some
schools or classrooms in which relationships between socialization and
political interest develop differently. Multilevel regressions with two
levels are therefore used to help disentangle classroom-level effects
from individual-level effects. Since the numbers of schools (8) and
provinces (2) in the final sample are too low for multilevel regressions
to be conducted, only classroom-level fixed effects are used in the
analyses.

Empty models are calculated to assess what percentage of the total
variance in political interest --- and each of the five topics --- is
located at the classroom level. Results show that 6.2\% of the variance
in political interest is located at the classroom level, as well as
4.6\% of the variance in interest in health care, 3.6\% of the variance
in interest in international affairs, 1.4\% of the variance in interest
in law and crime, 8.1\% of the variance in interest in education, and
2.6\% of the variance in interest in partisan politics. In all cases,
the bulk of variance therefore seems to be located at another level than
the classroom --- presumably the individual level.

\subsection{Assessing Multicollinearity, Heteroskedasticity and
Autocorrelation}\label{assessing-multicollinearity-heteroskedasticity-and-autocorrelation}

The four datasets collected among adults do not involve individuals
clustered within classrooms or other broader groupings of interest for
this study. Instead of multilevel regressions, simple and multiple
ordinary least squares (OLS) and weighted least squares (WLS)
regressions are instead performed. All models are tested for
heteroskedasticity using the Breusch-Pagan test (Halunga, Orme, and
Yamagata 2017) and for autocorrelation using the Durbin--Watson test
(Uyanto 2020). OLS is used when both the Durbin--Watson and
Breusch--Pagan tests indicate values above 0.05 since there is no
evidence of autocorrelation or heteroskedasticity. When either of the
two tests indicates a value below 0.05 for an OLS model, meaning either
autocorrelation or heteroskedasticity might be an issue, WLS is instead
used.

Models using all datasets are also tested for multicollinearity. When
the values of two variables vary together systematically, one of these
variables is kept to avoid any variance inflation factor (VIF) above 5
--- except regressions which include a squared term for age and
interaction terms for gender with age and ethnicity, where
multicollinearity is to be expected.

\subsection{What are Potential Confounding Variables to Links Between
Socialization and Political
Interest?}\label{what-are-potential-confounding-variables-to-links-between-socialization-and-political-interest}

Other than Chapter~\ref{sec-chap3}, which takes a time-series approach
to studying political interest, most analyses of the coming chapters
have political interest as the outcome variable and rely on CCPIS data.
Typically, a simple regression model with one explanatory variable ---
often a given role model's interest in a specific topic --- is followed
by multiple regression models with control variables for three blocs of
variables: (1) personality traits, (2) role models' gender, and (3)
socio-economic status variables. Personality traits include the agency
and communality scales described in the following section, while
socioeconomic status variables include confounding variables that have
been linked with political interest and could therefore mediate any
given relationship found between two persons' political interest. These
include gender, age --- Chapter~\ref{sec-chap3} explores the
relationships between gender, age and political interest in more detail
--- as well as language, ethnicity, and immigrant status. For analyses
involving adult data, income and level of education, both of which can
also be linked to political interest, are also added as control
variables.

\section{Descriptive Statistics}\label{descriptive-statistics}

\subsection{General}\label{general}

Before analyzing average levels of interest in politics in each of the
datasets, it is first important to assess the extent to which each of
them is (un)representative of the Canadian population on some of the
main socio-economic status variables that may relate to political
interest, including age, gender, education, income, and so on. In cases
where the data is deemed unrepresentative, a weighting method is applied
and described.

Figure~\ref{fig-ccpis1} shows the distribution of CCPIS students by age,
gender, language, ethnicity, immigrant status, degree of agency and
communality, and family situation. Most respondents are aged 12 to 17,
with a large proportion (41\%) being exactly 16 years old. This roughly
corresponds to the moment when adolescents' political interest starts
increasing alongside a gender gap in which boys report being more
interested in politics (Fraile and Sánchez-Vítores 2020; Janmaat,
Hoskins, and Pensiero 2022; Prior 2019). This age distribution should
make it possible to assess the influence of all types of role models at
the moment when they are the most likely to be influential for students'
political interest. The sample is therefore compared with 2021 Census
microdata on 12--17-year-old Canadian teenagers. The gender composition
of the sample is roughly balanced, with 50.4\% of girls and 47.1\% of
boys while, among all Canadians, 51.5\% of teenagers are boys and 48.5\%
are girls.\footnote{The number of transgender and non-binary people for
  this age group is not disclosed in the microdata.} Respondents are
mainly white --- 57.9\%, which is close to the Canadian average of
59.8\%. They are also overwhelmingly Francophone due to the
over-representation of Quebec within sampled schools --- 66.5\% mostly
speak French at home, which is much higher than the Canadian average of
18.3\% --- but several students who mainly speak another language than
English or French at home have also filled the survey --- 25\%, which is
higher than the Canadian average of 13.2\%. 84.4\% of students in the
sample were born in Canada --- close to the 86.8\% countrywide average.
The distributions of agency and communality among students both follow a
bell-shaped distribution centered around the averages of 0.62 and 0.68
respectively. Most students live with one mother and one father
(71.8\%), with a substantial minority (18.3\%) also having stepparents.
Other family arrangements are less common. Other than language and
province, these students do not appear to be markedly different from the
Canadian teenager population on the main socio-demographic variables.
Descriptive statistics about role models' gender and interest in
political topics are analyzed in Chapter~\ref{sec-chap4} and
Chapter~\ref{sec-chap5}, alongside further analysis related to these
role models' influence on children's political interest.

\begin{figure}

\centering{

\includegraphics{_graphs/CCPISDescriptive.pdf}

}

\caption{\label{fig-ccpis1}CCPIS Descriptive Statistics --- General}

\end{figure}%

For agency and communality, scales are replicated from Ward et al.
(2006). Factor analysis in Figure~\ref{fig-factor1} and
Figure~\ref{fig-factor2} show that, with one exception on the agency
list --- an item that was reverse-coded\footnote{This could be a data
  quality issue, since this is the only question for which the agentic
  pole is reverse-coded among the first eight questions asked to
  children, and none of the communality scale questions are
  reverse-coded. This means some children might have paid less attention
  to the actual question and responded quickly. However, given the fact
  that the non-reverse-coded measure is also unrelated to other elements
  of the scale, and since children responded differently to the agency
  and communality scales, this seems to be a minor concern. The scale is
  an exact replication of the one used by Ward et al. (2006).} --- all
items scale well together, with factor loadings at least medium ---
above 0.3 (Shevlin et al. 2000) --- and first eigenvalues larger than
the conventionally accepted value of 1 (Williams, Onsman, and Brown
2010). The Cronbach's alpha for the agency scale (0.63) is slightly
under the 0.7 to 0.9 range suggested by scholars (Tavakol and Dennick
2011), while it is within that range for the communality scale (0.78),
therefore meeting the standard benchmarks for valid and reliable scales
established in peer-reviewed studies about measurement scales.

\begin{figure}

\centering{

\includegraphics{_graphs/AgencyScale.pdf}

}

\caption{\label{fig-factor1}CCPIS Factor Analysis: Agency Scale}

\end{figure}%

\begin{figure}

\centering{

\includegraphics{_graphs/CommunalityScale.pdf}

}

\caption{\label{fig-factor2}CCPIS Factor Analysis: Communality Scale}

\end{figure}%

Figure~\ref{fig-dg} shows descriptive statistics for the Datagotchi PES,
which is not a representative sample before a form of weighting is
applied. The unweighted sample includes an over-representation of men
(56.8\%), white people (97.2\%), Francophones (94\% responded to the
French version of the survey), university-educated people (71\%), people
born in Canada (86\%), and those with a yearly household income over
\$110,000 (45\%). By comparison, Census numbers among adult Quebeckers
are 49.3\% of men, 82.1\% of white people, 82.1\% of people who speak
French more than English, 28.2\% university-educated people, 80.7\%
people born in Canada, and 34.3\% with a yearly household income above
\$110,000 (Statistics Canada 2022). The median age is 46, with a bimodal
distribution centered around 35--40 and 65, compared with a median age
within the 50--54 years old range among adult Quebeckers.

\begin{figure}

\centering{

\includegraphics{_graphs/DGDescriptive.pdf}

}

\caption{\label{fig-dg}Datagotchi PES Descriptive Statistics ---
General}

\end{figure}%

Given the mismatch between many of these characteristics --- notably
gender, level of education and ethnicity --- and the Quebec population,
survey weights are created and used to make the the Datagotchi PES
sample more representative of the Quebec population as per the 2021
Census (Statistics Canada 2022) on these two aspects, as well as income
and age. Using the \texttt{anesrake} package which is used to weight
results from the American National Election Studies (Pasek 2018), a
raking procedure is used to produce these weights, which range from 0 to
5. Default settings of the \texttt{anesrake} function are used. After
using this procedure, the population percentages match with sample
percentages within 10 percentage points --- and typically less than 5
--- for each category of the five variables.

Figure~\ref{fig-ces} shows descriptive statistics for the 2021 CES.
Compared to 2021 Canadian Census data collected among adults (Statistics
Canada 2022), the unweighted sample includes a slight
over-representation of women --- 54.5\% vs.~51.1\% --- and a larger
over-representation of Allophones --- 33.1\% vs.~16.5\% --- of white
people --- 89.1\% vs.~70.2\% --- of university-educated people ---
54.4\% vs.~30.7\% --- and of people born in Canada --- 84.4\%
vs.~70.5\%. The median yearly household income is within the
\$60,001--\$90,000 range, which includes the narrower \$85,000 to
\$89,999 median range among adult Canadians. The median age is 53,
slightly above the median 45--49 age range for the adult Canadian
population, and there is again a bimodal distribution centered, this
time centered around 30--35 and 60--70. Respondents from Ontario are
slightly under-represented --- 35.0\% vs.~38.6\% --- while those from
Quebec are over-represented --- 29.6\% vs.~22.9\%. Given the mismatch
between many of these characteristics --- notably province, education
and age --- and the Canadian population, the CES's survey weights based
on the 2016 Census, which account for province, gender, age, and
education, are used. These weights' values vary between 0.2 and 5. After
using this procedure, the population percentages match with sample
percentages on these aspects.

\begin{figure}

\centering{

\includegraphics{_graphs/CESDescriptive.pdf}

}

\caption{\label{fig-ces}2021 CES Descriptive Statistics - General}

\end{figure}%

Figure~\ref{fig-wvs} shows descriptive statistics for the 2020 WVS in
Canada. Compared to 2021 Canadian Census data collected among adults,
the unweighted sample includes a similar proportion of women --- 48.8\%
vs.~51.1\% --- an under-representation of Allophones --- 6.0\%
vs.~16.5\% --- and an over-representation of university-educated people
--- 55.5\% vs.~30.7\% --- of white people --- 80.5\% vs.~70.2\% --- and
of people born in Canada --- 82.1\% vs.~70.5\%. The median age is 45,
which broadly matches the median 45--49 age range for the adult Canadian
population. Respondents from Ontario are under-represented --- 25.0\%
vs.~38.6\%. Given the mismatch between many of these characteristics ---
notably province and education --- and the Canadian population, the
WVS's survey weights for Canada based on the 2016 Census, which account
for age, gender, education, and region, are used. These weights' values
vary between 0.12 and 3.69. After using this procedure, the population
percentages match with sample percentages on these aspects.

\begin{figure}

\centering{

\includegraphics{_graphs/WVSDescriptive.pdf}

}

\caption{\label{fig-wvs}2020 WVS Descriptive Statistics --- General}

\end{figure}%

Figure~\ref{fig-gss} shows descriptive statistics for the 2020 GSS.
Compared to 2021 Canadian Census data collected among adults (Statistics
Canada 2022), the unweighted sample includes a similar proportion of
women --- 51\% vs.~51.1\% --- an over-representation of Allophones ---
23.1\% vs.~16.5\% --- of university-educated people --- 40.8\%
vs.~30.7\% --- of visible minorities --- 42\% vs.~29.8\% --- and of
immigrants --- 41.1\% vs.~29.5\%. The median yearly household income is
within the \$50,000--\$74,999 range, compared with the higher \$85,000
to \$89,999 median range among adult Canadians. The median age is within
the 45--54 age range, which broadly matches the median 45--49 age range
for the adult Canadian population. Respondents from Ontario are slightly
under-represented --- 34.4\% vs.~38.6\%. Given the mismatch between many
of these characteristics and the Canadian population, the GSS's survey
weights based on the 2016 Census, which account for gender, age,
province, CMA, and visible minority status, are used. These weights'
values vary between 1 and 32,631. After using this procedure, the
population percentages match with sample percentages on these aspects.

\begin{figure}

\centering{

\includegraphics{_graphs/GSSDescriptive.pdf}

}

\caption{\label{fig-gss}2020 GSS Descriptive Statistics (Cycle 35 -
Social Identity) --- General}

\end{figure}%

Overall, the four studies conducted among adults have samples that vary
from each other. Some of them over-sample immigrants while others
under-sample them; some over-sample older citizens while others
under-sample them. A common theme in the three studies of Canadian
adults --- CES, WVS and GSS --- is the under-sampling of Ontarians.
However, all these variations are corrected using raking or
post-stratification weights.

\subsection{Political Interest}\label{political-interest}

After survey weights are applied, what is the average level of political
interest in each of the datasets? What is the distribution of political
interest among respondents, and which political topics are Canadians
most interested in? By looking at the distribution of political interest
among respondents topic by topic, this section lays the ground for
future chapters that will try to explain these political interest
scores.

Figure~\ref{fig-ccpis2} shows the distribution of CCPIS students by
interest in politics by topic. 0 is associated with the lowest level of
interest and 10 with the highest. For all topics, 9 and 10 are among the
least common answers. International relations as well as law and crime
are generally well-balanced, with students' mean interest above 5/10
(5.2--5.3). On the other hand, fewer students report high interest in
health care (4.1), education (4.1), or partisan politics (3.5). Partisan
politics in particular has a relatively steady positive skew, where the
mode is a 0/10 level of interest. For other topics, the distributions
are relatively wide, often taking a flattened bell shape.

\begin{figure}

\centering{

\includegraphics{_graphs/CCPISInterest.pdf}

}

\caption{\label{fig-ccpis2}CCPIS Descriptive Statistics --- Political
Interest}

\end{figure}%

Figure~\ref{fig-dginterest} and Figure~\ref{fig-ceswvsgssinterest} show
the level of interest in politics among the four adult Canadian surveys,
including interest in the five topics in the case of the Datagotchi PES,
after survey weights are applied. While adult Quebeckers (Datagotchi
PES) report being more interested in each of the five topics compared
with Canadian students (CCPIS), interest in law and crime (5.5/10) is
lower than for other topics, followed by partisan politics (5.9/10).
General political interest (7.1/10) is higher on average than interest
in any of the five topics. Interest in the five political topics also
follows a more bell-shaped, normal distribution than what CCPIS found
among Canadian youth, although a negative skew can be found for all
topics. Canadian WVS respondents are slightly less interested in
politics (5.6/10) than respondents to the 2020 GSS and 2021 CES (6.0/10
and 6.1/10 respectively). Datagotchi PES respondents score an even
higher average general political interest score of 7.1/10, which might
not be surprising given the fact that these respondents are former users
of the Datagotchi Web app. Analyses using the Datagotchi PES throughout
this dissertation can therefore be considered to be a high bar for the
level of interest in each of the political topics.

\begin{figure}

\centering{

\includegraphics{_graphs/DGInterest.pdf}

}

\caption{\label{fig-dginterest}Datagotchi PES Descriptive Statistics ---
Political Interest}

\end{figure}%

\begin{figure}

\centering{

\includegraphics{_graphs/CESWVSGSSInterest.pdf}

}

\caption{\label{fig-ceswvsgssinterest}CES, WVS and GSS Descriptive
Statistics --- Political Interest}

\end{figure}%

How can one make sense of these numbers? Are they relatively low or high
compared with what other studies have found in various countries among
respondents of different age groups? The next chapter will provide more
thorough answers to these questions while studying what period of life
is associated with the development of political interest.

\bookmarksetup{startatroot}

\chapter{Gender and Political Interest Development: Canadian Trends from
Childhood to Adulthood}\label{sec-chap3}

As Chapter~\ref{sec-chap1} showed, studies have long shown a gender gap
in political interest among adults in Western countries, in which men
report being more interested in politics than women, and one of the main
explanations for that gap is socialization. This process can be
influenced by four main agents: parents, peers, media and schools.
Moreover, when dissecting interests by topic, women typically develop
stronger interests in cooperation-focused political topics, while men
typically develop stronger interests in self-assertion-focused political
topics. Socialization can happen at various stages throughout the life
course. It can start from an early age but does not stop completely
after someone reaches adulthood. The timing of gendered political
socialization processes can be puzzling. This chapter therefore seeks to
provide an overview of how political interest evolves while Canadians
age. Identifying the critical periods of life in which political
interest can fluctuate will lay the groundwork for
Chapter~\ref{sec-chap4} and Chapter~\ref{sec-chap5}, which will focus
their attention on explaining how various socialization agents can
influence the development of political interest, especially among
children and teenagers.

This chapter addresses two related questions: \emph{Throughout the
average person's life, when does political interest increase, decrease,
or remain stable? How does the evolution of political interest over the
life course differ between girls and boys, and later between women and
men?} After exploring what studies have found about these two questions
in various countries, the chapter uses a variety of Canadian and
international datasets to provide context-specific answers to these two
questions and break results down by age, gender, other socio-demographic
characteristics, other forms of political engagement and, most
importantly, interest in various political topics. This chapter
hypothesizes that for both boys and girls, adolescence is the moment in
time when gender differences in interest in various political topics
emerge. Notably, it is argued that boys develop more interest in
partisan politics, which is a contributing factor to their higher
interest in running for elected office when they reach adulthood.

\section{Political Interest Evolution Over the Life
Course}\label{political-interest-evolution-over-the-life-course}

Political interest is remarkably stable over the life course, as
longitudinal studies conducted in European countries have shown (Fraile
and Sánchez-Vítores 2020; Neundorf, Smets, and Garcia-Albacete 2013;
Prior 2010, 2019). This finding is robust to changes in survey question
wording. However, there is an important exception to this rule: children
start with a lower and less stable level of political interest than
adults.\footnote{Children also seem to discuss politics more with
  parents and peers as they age (Rebenstorf 2004a).}

Before the age of 15, it is unclear if children and teenagers experience
rising, falling or stable levels of interest in politics. In the United
States context, Hess and Torney (1967) find that children aged 7 to 14
report less and less political interest as they grow older, and Bos et
al. (2022) show the same for children aged 6 to 12. The explanation for
this decline is not provided, but it seems to affect mostly girls, a
result attributed to internalizing gender roles and the idea that
politics is a men's domain, as exemplified by girls becoming more likely
as they age to draw a man when asked to draw a politician. On the other
hand, more recent peer-reviewed studies by Russo and Stattin (2017) and
Shehata and Amnå (2019) both find a slight increase in political
interest between 13 years old and 15 years old. In the Canadian context,
Dostie-Goulet (2009a) also finds children's political interest falls
between ages 14 and 15, before increasing between 15 and 16 years old.

Around the age of 15, numerous European studies find children start
experiencing an important uptick in political interest, which keeps
increasing until they reach 25 years old approximately, after which it
stabilizes (Neundorf, Smets, and Garcia-Albacete 2013; Prior 2019;
Quintelier and Van Deth 2014; Russo and Stattin 2017; Shehata and Amnå
2019). Russo and Stattin (2017) and Prior (2019) suggest the rise in
political interest among adolescents could be due to an increasingly
clear sense of what politics is during those formative years (15--25),
although it has also been found that by age 10, children have already
gained an understanding of what politics means (Hess and Torney 1967).
Russo and Stattin (2017) put it in these terms:

\begin{quote}
{[}W{]}e observed a general increase in interest in politics, which is
much steeper between 16 and 18 years of age than in the 13--15 age
range. One interpretation of this finding lies in the ideas that --- at
these ages --- adolescents obtain cognitive abilities that allow
abstract thinking and reasoning {[}\ldots{]} and they learn more about
society and the wider world. Another possibility is that youths become
more interested in political issues because they are approaching voting
age, and their `social environment (as parents and teachers) anticipate
a 'life event' in becoming an enfranchised voter' {[}\ldots{]} Even if
our results are consistent with both these ideas, it is worth noting
that there was no national election in Sweden in 2012 (when we collected
data from the 18 year-olds). Hence, cognitive maturation is a more
plausible explanation for the increase in political interest that we
observed between 16 and 18 years of age (654).
\end{quote}

Adolescence and early adulthood are important, as Russo and Stattin
(2017) suggest that it ``is during this period that parents, teachers,
and role models in general can potentially raise youths' interest in
political and societal issues'' (655). Learning when exactly political
interest increases during one's lifetime can help in understanding the
political socialization processes at play.

How large is this increase in political interest between 15 and 25 years
old? In panel data collected among British, Swiss and German
respondents, Prior (2019) estimates there is a 10--15 percentage point
increase. Political interest keeps increasing after 25 years old, but at
a slower pace, and almost entirely due to cohort effects: older cohorts
of voters, especially those born in the 1940s, are particularly
interested in politics. However, within each cohort of people, after
reaching 25 years old, political interest remains very stable until
death. Prior's (2019) findings are similar in all three countries
studied.

Political interest also becomes more \emph{stable} at the individual
level over the teenage years. Prior (2019) finds an increase in the
stability of political interest between the ages of 11 and 20. Russo and
Stattin (2017) also show that the stability of political interest
increases drastically from 13 years old to 20 years old, after which it
remains high. Using a 5-category response scale to measure political
interest, they find that 21.8\% of adolescents aged 13--15 changed their
answer by two or more response categories over two years, compared with
only 4.5\% of those aged 26 to 28.

\section{Gender Differences in Political Interest
Evolution}\label{gender-differences-in-political-interest-evolution}

\subsection{Size of the Gender Gap}\label{size-of-the-gender-gap}

Among scientific studies on the gender gap in political interest among
children and teenagers, a first group of studies finds that boys already
report higher levels of political interest. Among children aged 7 to 14,
Hess and Torney (1967) find a gender gap varying between two and five
percentage points, with boys being more interested in United States
government and current events. Owen and Dennis (1988) also find that
boys aged 10--13 and 14--17 in the United States are more interested in
politics than girls of that age. More recently, Arens and Watermann
(2017) find a gender gap of 7.4 points for 12-year-old Germans, which
increases to 10.8 points for 15-year-olds, where boys report being more
interested in politics.

However, the findings are not all consistent across studies and
contexts. Dowse and Hughes (1971) find no statistically significant
gender gap (-0.3 to +1.4 percentage point, where positive numbers are
associated with boys) in political interest for children aged 11--17.
Mayer and Schmidt (2004) also find that no strong gender differences in
political interest exist between boys and girls aged 12 to 15 in China,
Mexico, the United States, and Japan. Similar results are found among
Quebec students aged 14 and 15 (Beauregard 2008; Dostie-Goulet 2009a).
However, Bos et al. (2022) find that \emph{girls} are slightly but
significantly (3.3 points) more interested in politics at 6--7 years
old, but then the gap quickly reverses and grows larger until early
adolescence.\footnote{Political interest is measured using
  age-appropriate questions adapted from the Noyce Enthusiasm for
  Science scale. ``Interest in political activities is an index of
  agree/disagree responses to the following sentiments: (1) politics,
  government, and history are exciting topics; (2) curiosity to learn
  about politics, government, and history; (3) desire to have a
  political job; and (4) learning about government is boring {[}reverse
  coded{]}'' (Bos et al. 2022, 488).} By age 12, \emph{boys} are 6.7
points more interested in politics. Overall, US boys are found to be 2.4
points more interested in politics than girls (Bos et al. 2020, 2022).

For older teenagers and young adults, starting at age 15 --- the time
when political interest starts increasing markedly --- the literature is
clearer: studies conducted in various contexts generally show an
important gender gap in self-reported political interest, where men are
the most interested, and the gap is growing through time for those who
report longitudinal data. Using a -100--+100 scale where positive
numbers are associated with men, the gender gap in political interest
has been measured at +2 (Koskimaa and Rapeli (2015), Finland, 16--18
years old and Dostie-Goulet (2009a), Canada, 14--16 years old); +5
(Janmaat, Hoskins, and Pensiero (2022), UK, 16 years old); +11
(Cicognani et al. (2012), Belgium, 15--19 years old and Lawless and Fox
(2013), USA, 18--25 years old); +15 (Burns, Schlozman, and Verba (2001),
United States, 18 years old and Muxel (2002), France, 18--25 years old);
+20 (Fraile and Sánchez-Vítores (2020), UK, 15 years old); +22 (Janmaat,
Hoskins, and Pensiero (2022), UK, 30 years old); +27 (Hyman (1959),
Germany, 15--24 years old); and +30 (Fraile and Sánchez-Vítores (2020),
UK, 25 years old). Among these results, only Koskimaa and Rapeli's
(2015) are not statistically significant.

The increases in the gender gap between late adolescence and early
adulthood reported by the recent studies of Fraile and Sánchez-Vítores
(2020) and Janmaat, Hoskins, and Pensiero (2022) are substantially
large, although both rely on the same panel dataset. Fraile and
Sánchez-Vítores (2020) also test for sub-periods within the 1991--2009
time frame and find similar results.

Among the general adult population, studies also point to a greater
interest in politics by men compared with women. J. Van Deth (2000)
shows a gender gap in political interest in the Netherlands, which has
remained steady or increased through time. Among adults, Prior (2019)
finds that men are 10--15 percentage points more interested in politics
than women, a wider gap than the one found among younger respondents.
Sánchez-Vítores (2019) finds a statistically significant gender gap in
13 countries. Fraile and Sánchez-Vítores (2020) suggest that after the
formative years of 15 to 25 years old, ``attitudes crystallize and so
does the gender gap, remaining at the same size (around 30 percentage
points of difference between women and men) over the life course'' (89).
Using 2002 European Social Survey data, they find a gender gap across 15
European countries, varying between 4 and 13 percentage points.
Similarly, Kittilson and Schwindt-Bayer (2012) use 2000 World Values
Survey data among 29 countries and measure gender gaps of +1 (Argentina)
to +28 (India), with an average gender gap of approximately +15.

\subsection{Gaps in Interest for Certain
Topics}\label{gaps-in-interest-for-certain-topics}

Some studies conducted among teenagers report the types of interests
girls and boys have, but do not always find significant differences
there. Beauregard (2008) finds no gender gap in reported interest in
domestic and international politics. Burns, Schlozman, and Verba (2001)
also find null results with regards to interest in community and social
issues, which contrasts with the gender gap in political interest they
report. Finally, Oswald and Schmid (1998) find that ``girls are more
interested than boys in topics like peace, ecology and problems of the
Third World, whereas boys are more interested in governmental and
international affairs than girls'' (153). The authors reason that girls
might not be ``interested as much in the institutions of politics and in
the everyday business of negotiation in government and parliament and
that the single question measures mainly this sphere of front-page
politics'' (153). All three studies focus on a relatively limited number
of topics.

Among adults, on average, studies have found that women report more
interest in topics such as health care, education and gender issues,
while men report more interest in foreign policy, partisan politics, and
law and order (R. Campbell and Winters 2008; Coffé 2013; Ferrin et al.
2020; Hayes and Bean 1993; Kuhn 2004; Sabella 2004a; Verba, Burns, and
Schlozman 1997). Topics such as taxes and local politics seem to be
equally interesting to men and women.

\subsection{Political Interest Evolution in
Canada}\label{political-interest-evolution-in-canada}

Political interest has generally been found to be higher in Canada than
in other Western countries. Howe (2010) finds that 59\% of Canadian
citizens report being very or somewhat interested in politics, a higher
percentage than in most European countries. Similarly, using World
Values Survey data, Gidengil et al. (2004) show that Canada ranks fourth
among seventeen democratic countries when it comes to the average level
of political interest.\footnote{The timing of the World Values Survey
  might be a confounding factor since Canadian data was collected
  shortly before the Meech Lake Accord failed --- a time of intense
  political discussion (Howe 2010).} On political interest evolution,
mirroring trends in other Western countries, Canadians aged 21--29 have
long been less interested in politics than those aged 50--65, and this
age gap seems to be growing (Gidengil et al. 2004; Howe 2010). Political
interest also varies by province: ``Residents of Quebec and Saskatchewan
are typically less interested (5.0) in politics in general than
Canadians at large. Meanwhile, interest is highest in British Columbia
(5.8), followed by Manitoba, Ontario, Newfoundland and Labrador (5.7),
and Alberta (5.6)'' (Gidengil et al. 2004, 24).

Again using 2000 WVS data, Kittilson and Schwindt-Bayer (2012) find that
Canadians stand close to the middle of the pack in terms of countries'
gender gap in political interest, with a 13-point gap in which men
report being more interested. For their part, Gidengil et al. (2004)
evaluate the gender gap in political interest between Canadian women and
men to be worth 5 percentage points, relying on older 1990 WVS data.

\section{Theorizing Change Over Time}\label{theorizing-change-over-time}

It seems worthwhile to study political socialization in the period of
life where political interest is developed --- childhood to early
adulthood --- since there seems to be some level of path dependency in
individuals' political interest afterwards. Gender differences also seem
to become starker at that moment: the early increase in self-reported
political interest seems to be stronger for men (Jennings and Niemi
1981, 276), and political socialization seems to be faster-paced during
the teenage years. Fraile and Sánchez-Vítores (2020) suggest that ``the
development of gender roles during early childhood is a crucial phase in
the source of the gender gap, deserving further attention from
scholars'' (89). Moreover, no study has measured the evolution of
interest in various political topics with age among children or adults.

\emph{\textbf{Hypothesis 0a}: Interest in specific political topics
starts rising around age 15 and increases until age 25, after which it
stabilizes.}

\emph{\textbf{Hypothesis 0b}: Gender differences in interest in specific
political topics start rising around age 15 and increase until age 25,
after which they stabilize.}

Given findings among adults that women and men are interested in
different topics depending on these topics being assertion-focused or
cooperation-focused (see Chapter~\ref{sec-chap1}), Hypotheses 0c and 0d
provide two alternative views about when these differences should start
to occur. Hypothesis 0c suggests such differences in interest exist
prior to the increase in self-reported political interest around age 15,
perhaps as a result of tendencies towards agency and communality which
start developing early.\footnote{Gender differences in assertive speech
  use may develop between 1 and 2 years old, as Fagot et al. (1985)
  suggest, or may not have already developed at that age, as Brownell,
  Ramani, and Zerwas (2006) suggest. A meta-analysis of studies done by
  Leaper and Smith (2004) including children from various age groups
  more generally finds that boys use more assertive speech than girls,
  although this is not the case in mixed-gender interactions. Noakes and
  Rinaldi (2006) also find that, among boys and girls aged 9--14, boys
  reported having more interpersonal disagreements related to status and
  hierarchy, and girls were more likely to use cooperative conflict
  resolution strategies. Finally, Caravita and Cillessen (2012) find
  that, among 9--15-year-old girls and boys, boys appear to have more
  agentic goals and girls appear to have more communal goals, although
  aging makes the gender gap in agency appear and the gap in communality
  disappear, which could be due to a relatively small sample size.}
Hypothesis 0d instead suggests that --- likely as a result of cognitive
maturation --- personality traits only start fostering interest in
assertion-focused or cooperation-focused political topics between ages
15 and 25.

\emph{\textbf{Hypothesis 0c}: Gender differences in interest in specific
political topics already exist prior to age 15, with boys already more
interested in assertion-focused political topics such as law and crime,
international affairs and partisan politics, and girls more interested
in cooperation-focused political topics such as health care and
education politics at 10--15 years old.}

\emph{\textbf{Hypothesis 0d}: Between ages 15 and 25, boys develop more
interest in assertion-focused political topics, while girls develop more
interest in cooperation-focused political topics. These differences then
carry on at the adult age.}

Hypothesis 0a, 0b and 0c should apply to interest in the five topics
used throughout this book: health care, education, partisan politics,
law and crime, and international affairs. Since past studies have found
the self-reported measure of political interest is mostly associated
with interest in partisan politics (R. Campbell and Winters 2008), it
should be treated as an assertion-focused political topic.

Hypothesis 0e suggests, again as a result of the cognitive maturation
theory laid out by Russo and Stattin (2017), that issues related to
partisan politics --- such as elections and parties --- are more
political than other not explicitly partisan issues such as the working
conditions of nurses or tuition fees. This would be congruent with
Hypothesis 0d.

\emph{\textbf{Hypothesis 0e}: Boys and girls both see issues related to
partisan politics as more political than other political issues starting
at age 15.}

Studies have found strong causal or correlational associations between
political interest, political knowledge, political efficacy, and other
indicators of political engagement (Bennett and Bennett 1989; Coffé and
Bolzendahl 2010; Coffé 2013; Ondercin and Jones-White 2011; Prior 2019)
--- who are sometimes bundled together to create general political
engagement scales. Prior (2019) notably finds modest positive
correlations between political interest and a measure of political
efficacy in three countries, and positive bidirectional causal links
between efficacy and interest in one of these countries --- Switzerland.
Therefore, Hypotheses 0f and 0g suggest that the value of these
indicators tends to covary with age and that gender differences in these
indicators should also emerge within similar time frames.

\emph{\textbf{Hypothesis 0f}: Various indicators of political
engagement, such as political interest, political knowledge and
political efficacy, mostly increase at the same time --- under age 25.}

\emph{\textbf{Hypothesis 0g}: Gender differences in various indicators
of political engagement, such as political interest, political knowledge
and political efficacy, when they exist, mostly arise at the same time
--- under age 25.}

\section{Data and Methods}\label{data-and-methods}

This chapter wants to portray the evolution of political interest by age
and through time, and therefore relies on five datasets across several
survey years: the Canadian Children Political Interest Survey (CCPIS),
the Datagotchi Post-Election Survey (Datagotchi PES), the Canadian
Election Study (CES), the World Values Survey (WVS) and the Canadian
General Social Survey (GSS). The CCPIS includes survey data collected
among 9-to-18--year-olds in school settings in 2022, while the
Datagotchi PES includes data collected among Quebec adults in 2023. More
about these datasets is explained in Chapter~\ref{sec-chap2}. Both the
CCPIS and Datagotchi PES include questions about interests in specific
political topics, which can better measure the concept. These datasets
are therefore used to assess the relationship between gender and
interest in five topics: health care, international affairs, law and
crime, education, and partisan politics. OLS and WLS regressions are
used with Datagotchi PES data, while multilevel regressions with
classroom fixed effects are used for CCPIS data. While the CES, WVS and
GSS do not ask these specific political topics, they each have unique
advantages. The CES includes the most survey years --- with a question
on political interest being asked in surveys from 1997 to 2021 ---
making it possible to measure other indicators of political engagement
than political interest, including political knowledge, efficacy, and
participation. The WVS makes it possible to compare Canadians' level of
interest with other countries. Finally, two GSS surveys --- 2013 and
2020 --- provide the largest datasets and, since it is not designed as a
political survey per survey, may provide more accurate measures of
Canadians' level of interest in politics --- despite the limitations of
a single-item measure.

Apart from questions about political interest, this chapter also relies
on four other measures of political engagement taken from the Canadian
Election Studies: internal political efficacy, external political
efficacy, knowledge of political figures' names, and political
participation.

\emph{Internal political efficacy} is defined as ``individuals'
self-perceptions that they are capable of understanding politics and
competent enough to participate in political acts such as voting'',
while \emph{external political efficacy} is individuals' belief that the
public can influence political outcomes because government leaders and
institutions are responsive to their needs (Craig and Maggiotto (1982),
86). Internal efficacy is measured by the degree of agreement with the
following statement: ``Sometimes, politics and government seem so
complicated that a person like me can't really understand what's going
on.'' (Stephenson et al. 2022). External efficacy is measured by the
degree of agreement with the following statement: ``People like me don't
have any say about what the government does.'' (Stephenson et al. 2022).
Both questions are reverse-coded, with more agreement associated with
higher political efficacy.

Knowledge of political figures' names is measured by correct answers to
questions about who the minister of Finance, governor general of Canada,
and their provincial premier are. Political participation is measured by
a scale including 13 items: volunteering for groups and organizations,
attending political meetings and speeches, attending protests,
boycotting products, signing petitions, following politicians on social
media, volunteering for politicians, contacting elected officials,
donating to candidates, donating to causes, being a group's active
member, commenting political content, and discussing politics on social
media. CES respondents were asked how often they had done these
activities in the past 12 months.

Factor analysis for knowledge of political figures' names
(Figure~\ref{fig-factor3}) and for the political participation scale
(Figure~\ref{fig-factor4}) show that items generally scale well
together, with factor loadings at least medium --- above 0.3 (Shevlin et
al. 2000) --- for all elements of the political participation scale and
two out of three of the knowledge scale, and first eigenvalues larger
than the conventionally accepted value of 1 (Williams, Onsman, and Brown
2010). The Cronbach's alpha for the knowledge scale (0.39) is under the
0.7 to 0.9 range suggested by scholars (Tavakol and Dennick 2011), while
it is within that range for the political participation scale (0.84),
therefore meeting the standard benchmarks for valid and reliable scales
established in peer-reviewed studies about measurement scales.

\begin{figure}

\centering{

\includegraphics{_graphs/KnowScale.pdf}

}

\caption{\label{fig-factor3}CCPIS Factor Analysis: Knowledge Scale}

\end{figure}%

\begin{figure}

\centering{

\includegraphics{_graphs/ParticScale.pdf}

}

\caption{\label{fig-factor4}CCPIS Factor Analysis: Political
Participation Scale}

\end{figure}%

\section{Results}\label{results}

\subsection{Evolution of Political Interest Among
Adults}\label{evolution-of-political-interest-among-adults}

In order to explore what level of general (self-reported) political
interest Canadian men and women report, three datasets are mobilized:
the 2021 CES, the WVS's Wave 7 (2017--22) and the 2020 GSS.
Figure~\ref{fig-yeargender} shows the evolution in Canadians' average
level of political interest by gender from 1990 to 2021. In general, CES
respondents are the most interested in politics according to these data
--- 6.3/10 --- which is unsurprising given these surveys are conducted
during election campaigns, and ask questions of a more clearly political
nature. WVS respondents are the least interested --- 4.5/10 --- with GSS
respondents sitting somewhere in between --- 5.8/10. In general,
political interest seems to have increased between 2000 and 2021, with
an average interest of 4.9/10 in the 2000 CES and WVS and an average
interest of 6.1/10 in the 2019--20 CES, WVS and GSS.

\begin{figure}

\centering{

\includegraphics{_graphs/InterestYearGender.pdf}

}

\caption{\label{fig-yeargender}General Political Interest by Year and
Gender Among Canadian Adults, CES, WVS (Canada) and GSS
\newline \textit{Notes}: On the \textit{y} axis, 0 = no interest at all,
and 10 = a great deal of interest. 95\% confidence intervals represented
by shaded areas. CES, WVS and GSS weights are applied.}

\end{figure}%

Despite fluctuations in the level of political interest between surveys
\emph{and} within surveys taken at different times, the gender gap in
self-reported political interest remains relatively constant through
time, with a notable exception in the 2021 CES, where a sharp drop in
women's political interest occurs while men's political interest remains
largely stable, bringing the largest gender gap of these studies ---
1.3/10 points. The size of the gender gap slightly varies between
studies, standing on average at 0.6 points in the GSS, 0.7 points in the
CES and 0.9 points in the WVS. While differences between the three
survey organizations are twice as large as gender differences in
political interest, men are significantly more likely to report being
interested in politics in every survey and every year.

Again relying on data from the CES, WVS and GSS,
Figure~\ref{fig-timeceswvsgss} shows average self-reported political
interest by age and gender among Canadians, using a local regression
model (LOESS). The trends among all three surveys are very similar and
match to some degree with cross-country WVS results as well.

\begin{figure}

\centering{

\includegraphics{_graphs/TimeCESWVSGSS.pdf}

}

\caption{\label{fig-timeceswvsgss}Self-Reported Level of General
Political Interest by Age Among Canadian Adults, 2021 CES, WVS Wave 7
and 2020 Canadian GSS \newline \textit{Notes}: On the \textit{y} axis, 0
= no interest at all, and 10 = a great deal of interest. Dots represent
average interest by age and gender. 95\% confidence intervals
represented by shaded areas. CES, WVS and GSS weights are applied.}

\end{figure}%

In all three datasets, interest in politics increases as people age, but
it does so relatively slowly from 18 to 50 years old --- or not at all,
according to CES data. Findings from the GSS seem partially consistent
with Hypothesis 0a, with a slightly larger increase in interest between
ages 15 and 24 than afterwards, but CES data and WVS data from Canada
are both inconsistent with Hypothesis 0a, with the WVS even showing a
decrease in political interest between ages 18 and 25.

A large gender gap also appears for those within that age group in all
three datasets. For instance, CES data shows that on average, women aged
18 to 50 report being neither interested nor disinterested in politics
--- 5/10; men, on the other hand, report being somewhat interested in
politics --- 6.5/10. This 1.5-point gap is statistically significant
(\emph{p}\textless0.001), larger than the 1-point increase in political
interest between ages 50 and 75 and almost as large as the 1.8-point gap
in average political interest between CES respondents and Canadian WVS
respondents over the years. The gender gap seems to increase mostly
between ages 15 and 25 in the GSS and WVS data, corroborating Hypothesis
0b, although CES data rather shows stability within that age range.

After age 50, both men and women start reporting higher levels of
interest in politics, and this interest keeps increasing through their
60s, 70s, 80s and 90s. Moreover, again starting at age 50, the gender
gap progressively reduces, as women's political interest increases more
quickly than men's. Around age 75--80, the gap becomes statistically
non-significant, although confidence intervals also become wider due to
smaller sample sizes. For instance, in their early 90s, CES data shows
that the average interest in politics stands at 8/10 for both men and
women. Overall, throughout people's life course, CES data shows that
women's political interest averages 5.4, while men's averages 6.8
(\emph{p}\textless0.001).

How does Canada compare with other countries? Across all 57 countries
surveyed during wave 7 of the WVS, the average political interest is
4.8/10 for men and 4/10 for women (\emph{p}\textless0.001), and while
political interest increases with people's age, the size of the gender
gap remains relatively stable. Among Canadian WVS respondents, the
average political interest is 6.2/10 for men and 5/10 for women: this
gender gap is 0.35 points \emph{larger} than the WVS average and is
statistically significant (\emph{p}\textless0.001). Moreover, the age
pattern is somewhat different in the Canadian WVS compared with the
WVS's cross-country results, becoming statistically insignificant around
age 75 while a large and significant gap remains in other countries.
Despite smaller sample sizes for people aged 75 and over in all Canadian
surveys, all of them show a shrinking of the gap at that age in Canada,
which reinforces confidence in that finding.

Within Canada, across all three studies, average political interest is
higher in larger provinces, with an average of 6.1/10 in Alberta, 6/10
in Quebec and Ontario, and 5.9/10 in British Columbia. New Brunswick
stands last at 5.5/10. Notably, the gap between the province with the
highest and lowest scores is only 0.6 points, smaller than the gender
gap in any of the three survey organizations over time. However,
between-study differences in provinces' average political interest are
large, suggesting these results need to be interpreted with caution.

Figure~\ref{fig-cesgapyearage} uses a LOESS model to look at the size of
the gender gap by age for each CES, making it possible to assess the
presence of generational effects. Positive values indicate women report
being more interested, while negative values indicate men report being
more interested. The 2021 CES stands out as having the largest gender
gap in political interest on average for respondents aged 20 to 80. All
other Canadian Election Studies are associated with similar trends in
the evolution of the gender gap with age: the gap slowly reduces with
time. Values tend to be less consistent among both older and younger
voters, as a result of smaller sample sizes at the extremes of the age
distribution. When respondents aged 18--20 and those aged 85 and over
are removed from the sample, differences between surveys become smaller.

\begin{figure}

\centering{

\includegraphics{_graphs/CESGapYearAge.pdf}

}

\caption{\label{fig-cesgapyearage}General Political Interest by Year and
Gender Among Canadian Adults, CES \newline \textit{Notes}: On the
\textit{y} axis, 0 = no gender difference in interest at all, positive
values (up to +10) = women more interested, negative values (down to
-10) = men more interested. CES weights are applied.}

\end{figure}%

Figure~\ref{fig-timepoliticalengagement} shows the level of political
engagement among 2021 CES respondents. Levels of external political
efficacy and political participation remain low and stable for
respondents aged 18 to 90, and no statistically significant gender gap
is found. Conversely, large 2-point gender gaps in internal political
efficacy and knowledge of political figures' names are found among young
respondents. These gaps become smaller, as respondents' age increases.

\begin{figure}

\centering{

\includegraphics{_graphs/TimePoliticalEngagement.pdf}

}

\caption{\label{fig-timepoliticalengagement}Level of Political
Engagement Across Several Measures Among Canadian Adults, 2021 CES
\newline \textit{Notes}: On the \textit{y} axis, 0 = no engagement at
all, and 10 = a great deal of engagement. Dots represent average
interest by age and gender. 95\% confidence intervals represented by
shaded areas. CES weights are applied.}

\end{figure}%

Some of these trends mirror quite closely what is found for political
interest among these same CES respondents. Notably, large gender gaps in
knowledge of political figures' names and internal political efficacy
appear among younger respondents. These gaps are larger than the
13-point gap in political interest. Similar to political interest, these
gaps also seem to progressively fade among older respondents, and after
age 80--85, the level of political engagement seems to be the same for
women and men across all indicators. With regards to knowledge, the gap
was to be expected and mirrors findings from past research --- notably
Stolle and Gidengil (2010) --- as recalling the names of political
figures is closely tied to partisan politics and assertion as opposed to
cooperation. Unfortunately, the data does not make it possible to assess
respondents' knowledge about health care or education policy, but Stolle
and Gidengil (2010) suggest that the gender gap should be expected to be
in the opposite direction, with women more knowledgeable about these
topics on average.

Hypothesis 0f suggested that the various indicators of political
engagement were likely to vary simultaneously, with a notable increase
between ages 18 and 25. Figure~\ref{fig-timepoliticalengagement} does
not corroborate this claim: internal political efficacy increases at a
relatively even pace until age 40, the knowledge scale also increases
constantly until age 75, while external efficacy and political
participation both slightly \emph{decline} between ages 18 and 25. These
findings, while inconsistent with Hypothesis 0f, are consistent with the
finding that political interest is also stable, not increasing, until
age 25 --- in the CES at least. However, these results do not indicate
that various indicators of political engagement are unrelated to each
other. Indeed, correlations of political interest with internal
political efficacy (0.42), external political efficacy (0.15), knowledge
of political figures' names (0.34) and political participation (0.39)
are all positive and significant at the 99.9\% confidence level.
Correlations between political efficacy and interest are somewhat higher
than the correlations of 0.09 to 0.27 found by Prior (2019) in three
countries while aggregating both internal and external dimensions.
Correlations between political participation and interest are also
higher than the correlations of 0.07 to 0.31 found by Grechyna (2023) in
the EU and the UK.

Hypothesis 0g is also not corroborated by the data: the gender gap in
internal political efficacy is stable, not increasing, among
18-to-25--year-olds, while for knowledge of political figures' names,
the gap starts shrinking within that age group. External political
efficacy and political participation do not display any notable gender
gap for that age group or other age groups.

Is the gender gap in political interest concentrated among certain
demographic groups only? Figure~\ref{fig-interestwavegroup} provides a
robustness check to the existence of a gender gap in self-reported
political interest by looking at gender differences through an
intersectional lens, with gender--ethnicity pairs and gender--immigrant
status pairs included across several surveys and years for which the
information was available. Within each survey--year, self-reported
political interest is higher among white men than white women, and
non-white men than non-white women. The same applies to the gender gap
among immigrants and non-immigrants. By comparison, differences in
interest between Caucasians and non-Caucasians are found to be smaller,
and there are no differences in interest between people born in Canada
and those born abroad. The gender gap in interest therefore seems to
trump other socio-demographic characteristics such as ethnicity and
immigrant status, which reinforces the need to understand how gendered
socialization shapes this gap.\footnote{With regard to immigration,
  these results are consistent with the findings of previous studies,
  which found no significant difference between levels of political
  interest of immigrants and non-immigrants (Hochman and Garcı́a-Albacete
  2019). Prior and Bougher (2018) found higher levels of interest in
  politics among white than black Americans. In the Dominican Republic,
  Spierings (2012) found that Protestant women report levels of
  political interest similar to men and higher than Catholic women,
  while Mestizo women report lower levels of political interest than
  Indigenous women.}

\begin{figure}

\centering{

\includegraphics{_graphs/InterestWaveGroup.pdf}

}

\caption{\label{fig-interestwavegroup}General Political Interest by
Year, Gender, Ethnicity and Immigrant Status Among Canadian Adults, CES,
WVS (Canada) and GSS \newline \textit{Notes}: On the \textit{y} axis, 0
= no interest at all, and 10 = a great deal of interest. 95\% confidence
intervals shown. CES, WVS and GSS weights are applied. Not all surveys
included here asked questions about both ethnicity and immigrant
status.}

\end{figure}%

Now, how do adults' interest in various political topics change
depending on age and gender? This question is best answered using
Datagotchi PES data. As Table \ref{tab:olsInterestDG} shows, in models
without controls, Quebec men generally report being more interested in
politics than women, with a 0.9-point gap, similar to the other three
studies. Men's interest in international affairs and partisan politics
is also higher than women's (0.7-point and 0.6-point difference
respectively; both \emph{p}\textless0.001). However, women are
significantly more interested in health care and education (0.7-point
and 0.5-point difference respectively, both \emph{p}\textless0.001) than
men are. Interest in law and crime is about the same for both genders.
Interestingly, the largest coefficient found is for interest in politics
in general --- not any of specific topic.\footnote{Figure~\ref{fig-interestdg}
  in Appendix~\ref{sec-appendix4} shows the gender coefficient for an
  intermediate model with SES (socio-economic status) controls but no
  interactions. Before adding interactions between (i) gender and age,
  (ii) gender and ethnicity, and (iii) age squared, coefficients from
  the simple regression model remain significant.}

\begin{table}
\centering\centering
\caption{Interest in Topic by Gender, Datagotchi PES \label{tab:olsInterestDG}}
\centering
\fontsize{6}{8}\selectfont
\begin{tabular}[t]{lcccccc}
\toprule
  & Politics (general) & Health care & International affairs & Law and crime & Education & Partisan politics\\
\midrule
\addlinespace[0.5em]
\multicolumn{7}{l}{\textit{Without Controls}}\\
\midrule \hspace{1em}(Intercept) & 7.724*** & 6.449*** & 7.453*** & 5.421*** & 6.586*** & 6.334***\\
\hspace{1em} & (0.064) & (0.075) & (0.071) & (0.079) & (0.075) & (0.084)\\
\hspace{1em}Gender (1 = women) & -0.877*** & 0.690*** & -0.592*** & 0.104 & 0.517*** & -0.645***\\
\hspace{1em} & (0.097) & (0.116) & (0.109) & (0.121) & (0.116) & (0.128)\\
\hspace{1em}Num.Obs. & 1575 & 1575 & 1575 & 1575 & 1575 & \vphantom{1} 1575\\
\hspace{1em}R2 & 0.049 & 0.022 & 0.018 & 0.000 & 0.013 & 0.016\\
\hspace{1em}R2 Adj. & 0.048 & 0.022 & 0.018 & 0.000 & 0.012 & 0.015\\
\hspace{1em}Log.Lik. & -3254.524 & -3523.301 & -3437.633 & -3591.294 & -3524.218 & -3687.310\\
\addlinespace[0.5em]
\multicolumn{7}{l}{\textit{With Controls}}\\
\midrule \hspace{1em}(Intercept) & 7.751*** & 4.166*** & 7.652*** & 6.469*** & 5.947*** & 8.463***\\
\hspace{1em} & (0.819) & (0.815) & (0.834) & (1.020) & (0.841) & (0.941)\\
\hspace{1em}Gender (1 = women) & -0.580 & 1.733* & -2.020* & -0.251 & -1.744* & -3.951***\\
\hspace{1em} & (0.694) & (0.836) & (0.856) & (0.865) & (0.864) & (0.966)\\
\hspace{1em}Age & -0.030 & -0.005 & -0.042+ & 0.013 & -0.055* & -0.144***\\
\hspace{1em} & (0.020) & (0.023) & (0.024) & (0.025) & (0.024) & (0.027)\\
\hspace{1em}Age squared & 0.000+ & 0.000 & 0.001* & 0.000 & 0.001* & 0.001***\\
\hspace{1em} & (0.000) & (0.000) & (0.000) & (0.000) & (0.000) & (0.000)\\
\hspace{1em}Ethnicity (1 = white) & 0.503 & -0.957* & -0.833+ & -1.305* & -0.722 & 0.056\\
\hspace{1em} & (0.504) & (0.477) & (0.488) & (0.628) & (0.492) & (0.551)\\
\hspace{1em}Immigrant & -0.443+ & 1.361*** & 0.057 & -0.428 & 1.035*** & 0.045\\
\hspace{1em} & (0.229) & (0.298) & (0.305) & (0.285) & (0.308) & (0.344)\\
\hspace{1em}French spoken at home & 0.106 & 0.659** & 0.324 & -0.243 & 0.307 & 0.237\\
\hspace{1em} & (0.340) & (0.224) & (0.229) & (0.423) & (0.231) & (0.259)\\
\hspace{1em}Income between \$60,000 and \$150,000 & 0.040 & 0.371+ & 0.182 & 0.257 & 0.360+ & 0.987***\\
\hspace{1em} & (0.125) & (0.190) & (0.194) & (0.156) & (0.196) & (0.219)\\
\hspace{1em}Income above \$150,000 & 0.122 & 0.172 & 0.433* & 0.439* & 0.395* & 0.622**\\
\hspace{1em} & (0.163) & (0.182) & (0.186) & (0.203) & (0.188) & (0.210)\\
\hspace{1em}Education: college & 0.118 & 0.335+ & 0.273 & 0.180 & 0.472** & 0.254\\
\hspace{1em} & (0.206) & (0.175) & (0.179) & (0.256) & (0.181) & (0.203)\\
\hspace{1em}Education: university & 0.466* & 0.735*** & 0.529* & -0.124 & 1.396*** & 0.802***\\
\hspace{1em} & (0.192) & (0.204) & (0.209) & (0.239) & (0.211) & (0.236)\\
\hspace{1em}Gender (1 = women):Age & 0.018** & 0.021** & 0.020** & -0.009 & 0.017* & 0.039***\\
\hspace{1em} & (0.006) & (0.007) & (0.007) & (0.007) & (0.007) & (0.008)\\
\hspace{1em}Gender (1 = women):Ethnicity (1 = white) & -1.181+ & -1.872* & 0.775 & 0.863 & 1.764* & 1.551+\\
\hspace{1em} & (0.657) & (0.802) & (0.821) & (0.819) & (0.828) & (0.926)\\
\hspace{1em}Num.Obs. & 1575 & 1575 & 1575 & 1575 & 1575 & 1575\\
\hspace{1em}R2 & 0.076 & 0.129 & 0.053 & 0.017 & 0.128 & 0.065\\
\hspace{1em}R2 Adj. & 0.069 & 0.122 & 0.046 & 0.010 & 0.121 & 0.057\\
\hspace{1em}Log.Lik. & -3232.057 & -39061.721 & -39098.369 & -3578.081 & -39112.301 & -39288.383\\
\bottomrule
\multicolumn{7}{l}{\rule{0pt}{1em}+ p $<$ 0.1, * p $<$ 0.05, ** p $<$ 0.01, *** p $<$ 0.001}\\
\multicolumn{7}{l}{\rule{0pt}{1em}Without controls: Ordinary least squares (OLS) regressions}\\
\multicolumn{7}{l}{\rule{0pt}{1em}With controls: OLS for Politics (general) and Law and Crime; Weighted least squares (WLS) for other regressions}\\
\end{tabular}
\end{table}

In models with controls for socio-demographic variables and interaction
terms for gender and ethnicity as well as gender and age, Table
\ref{tab:olsInterestDG} shows some of the relationships found in models
without controls remain significant while others lose significance.
Self-reported political interest is not associated with being a man, but
it is positively associated with being an older woman and marginally
negatively associated with being a white woman. Women remain more likely
to be interested in health care politics under these specifications, and
again, there is a positive association with being an older woman and a
negative association with being a white woman. International affairs are
positively associated with being a man in general, but there is again a
positive association with being an older woman. Interest in law and
crime remains unlinked with gender, while interest in partisan politics
is again strongly associated with being a man --- also with a
significant impact of being an older woman. The coefficients for
interest in education politics are notable, as being a white woman and
being an older woman are both positively related to the outcome, but the
coefficient for gender is now negative, suggesting that men are expected
to be more interested in education politics after controlling for
interactions between gender, ethnicity and age.

Figure~\ref{fig-datagotchi} inquires further into age trends in the
evolution of interest in each of the topics, including gendered aspects
of these age trends. Similar to Figure~\ref{fig-ceswvsgssinterest}, the
gender gap in self-reported political interest diminishes among older
respondents, as older women tend to become more interested. This trend
of increased interest in politics among older women compared with
younger ones can also be observed in interest in health care,
international affairs, law and crime, and partisan politics. After age
50, men are no more likely than women to report being interested in
partisan politics or international affairs, despite a significant gender
gap among younger respondents.

\begin{figure}

\centering{

\includegraphics{_graphs/InterestAgeGenderDG.pdf}

}

\caption{\label{fig-datagotchi}Self-Reported Level of Interest in
Various Topics by Age Among Canadian Adults, 2022 Datagotchi PES
\newline \textit{Notes}: On the \textit{y} axis, 0 = no interest at all,
and 10 = a great deal of interest. Dots represent average interest by
age and gender.}

\end{figure}%

Among Quebec men, there seems to be a general increase in the level of
political interest among older respondents for general political
interest, interest in health care and international affairs. Interest in
law and crime seems to be mostly stable, while a downward trend is found
for partisan politics. Large variations among older respondents can be
due to smaller sample sizes. This decline in interest in partisan
politics is difficult to explain. One possibility is that this is the
result of lower levels of college attendance among men than women.

Interest in the politics of education shows a unique trajectory, with a
shape that resembles a cubic equation function for both women and men.
One potential explanation for this trend is that education increases
quickly between ages 25--40 as women and men become parents and care
about the potential effects of education policy on their children. As
children grow older, parents lose interest in education policy, but
after retiring around age 65, women and men often become grandparents
and start caring about the education of their grandchildren. While R.
Campbell and Winters (2008) find that parenthood is associated with a
lower general political interest in the UK context, Grechyna (2023) find
no statistically significant relationship. Yet, none of these other
studies test specifically for interest in education politics.

Overall, Figure~\ref{fig-datagotchi} provides little evidence to support
either Hypothesis 0a, that interest in politics increases until age 25,
Hypothesis 0b, that gender differences increase mostly until age 25.
Interest in all topics generally increases with age among women and men,
while adult men's interest in partisan politics declines with age. But
most of the movement in average political interest and in gender gaps
happens after age 25. Hypothesis 0d, that men develop more interest in
assertion-focused topics in their young adult years while women develop
in cooperation-focused topics, is also not corroborated by the data. For
partisan politics and international affairs, the gender gap is
concentrated among younger respondents and disappears among older ones;
for health care and education, the gender gap becomes significant only
among respondents aged 35 and over; finally, no gender gap is found for
interest in law and crime.

\subsection{Evolution of Political Interest Among
Children}\label{evolution-of-political-interest-among-children}

What kinds of political topics are children and adolescents interested
in? Are there gender differences in how interested they are in these
topics? Does gender predict political interest in itself, or is it only
a proxy for other variables? Data from the CCPIS is used to answer these
questions. Table \ref{tab:lmeInterestCCPIS} shows the link between
gender and interest in each of the topics among elementary and high
school students. The upper part includes gender as the only predictor,
while the lower part includes controls for socio-demographic variables
and personality traits.

\begin{table}
\centering\centering
\caption{Interest in Topic by Gender, CCPIS \label{tab:lmeInterestCCPIS}}
\centering
\fontsize{6}{8}\selectfont
\begin{tabular}[t]{lcccccc}
\toprule
  & Politics (general) & Health care & International affairs & Law and crime & Education & Partisan politics\\
\midrule
\addlinespace[0.5em]
\multicolumn{7}{l}{\textit{Without Controls}}\\
\midrule \hspace{1em}(Intercept) & 4.579*** & 4.041*** & 5.724*** & 4.956*** & 4.219*** & 4.007***\\
\hspace{1em} & (0.184) & (0.167) & (0.180) & (0.173) & (0.206) & (0.171)\\
\hspace{1em}Gender (1 = girl) & -0.434* & 0.128 & -0.980*** & 0.488* & -0.103 & -0.854***\\
\hspace{1em} & (0.207) & (0.197) & (0.229) & (0.231) & (0.223) & (0.232)\\
\hspace{1em}SD (Intercept Class) & 0.651 & 0.542 & 0.473 & 0.377 & 0.771 & 0.344\\
\hspace{1em}SD (Observations) & 2.499 & 2.397 & 2.802 & 2.837 & 2.701 & 2.855\\
\hspace{1em}Num.Obs. & 617 & 623 & 620 & 619 & 623 & 620\\
\hspace{1em}R2 Marg. & 0.007 & 0.001 & 0.029 & 0.007 & 0.000 & 0.022\\
\addlinespace[0.5em]
\multicolumn{7}{l}{\textit{With Controls}}\\
\midrule \hspace{1em}(Intercept) & -2.429 & 10.710 & 13.284 & 7.365 & 19.424* & 21.801**\\
\hspace{1em} & (8.113) & (7.495) & (8.444) & (8.799) & (8.916) & (8.210)\\
\hspace{1em}Gender (1 = girl) & 1.497 & 0.074 & -2.296 & 0.903 & -0.121 & -0.640\\
\hspace{1em} & (2.046) & (1.984) & (2.271) & (2.326) & (2.233) & (2.275)\\
\hspace{1em}Age & 0.293 & -1.304 & -1.420 & -0.601 & -2.627* & -2.814*\\
\hspace{1em} & (1.111) & (1.031) & (1.159) & (1.211) & (1.221) & (1.127)\\
\hspace{1em}Age squared & -0.004 & 0.049 & 0.047 & 0.026 & 0.098* & 0.092*\\
\hspace{1em} & (0.038) & (0.035) & (0.040) & (0.042) & (0.042) & (0.039)\\
\hspace{1em}Ethnicity (1 = white) & 0.630* & 0.214 & 0.886* & -0.053 & 0.050 & 0.662+\\
\hspace{1em} & (0.320) & (0.315) & (0.354) & (0.365) & (0.352) & (0.357)\\
\hspace{1em}Immigrant & -0.163 & 0.167 & -0.713* & -0.690+ & 0.212 & -0.413\\
\hspace{1em} & (0.321) & (0.317) & (0.357) & (0.371) & (0.357) & (0.359)\\
\hspace{1em}English spoken at home & -0.445 & -0.218 & -1.644*** & -0.514 & 0.253 & -0.106\\
\hspace{1em} & (0.451) & (0.436) & (0.493) & (0.507) & (0.499) & (0.491)\\
\hspace{1em}French spoken at home & -0.035 & 0.093 & -0.358 & -0.245 & 0.422 & 0.550+\\
\hspace{1em} & (0.279) & (0.275) & (0.308) & (0.319) & (0.307) & (0.308)\\
\hspace{1em}Agency & 3.494*** & 1.319* & 2.544*** & 2.077** & 1.321+ & 3.335***\\
\hspace{1em} & (0.629) & (0.614) & (0.694) & (0.719) & (0.690) & (0.702)\\
\hspace{1em}Communality & 1.662* & 1.066 & 2.253** & -0.174 & 1.283+ & 1.243\\
\hspace{1em} & (0.658) & (0.656) & (0.739) & (0.760) & (0.733) & (0.756)\\
\hspace{1em}Gender (1 = girl):Age & -0.092 & 0.025 & 0.115 & -0.010 & 0.018 & 0.030\\
\hspace{1em} & (0.130) & (0.126) & (0.145) & (0.148) & (0.142) & (0.145)\\
\hspace{1em}Gender (1 = girl):Ethnicity (1 = white) & -0.764+ & -0.500 & -0.635 & -0.218 & -0.505 & -0.929+\\
\hspace{1em} & (0.428) & (0.420) & (0.473) & (0.490) & (0.470) & (0.477)\\
\hspace{1em}SD (Intercept Class) & 0.515 & 0.401 & 0.335 & 0.366 & 0.642 & 0.203\\
\hspace{1em}SD (Observations) & 2.385 & 2.371 & 2.686 & 2.774 & 2.636 & 2.717\\
\hspace{1em}Num.Obs. & 558 & 563 & 559 & 559 & 561 & 561\\
\hspace{1em}R2 Marg. & 0.109 & 0.041 & 0.119 & 0.039 & 0.057 & 0.109\\
\bottomrule
\multicolumn{7}{l}{\rule{0pt}{1em}+ p $<$ 0.1, * p $<$ 0.05, ** p $<$ 0.01, *** p $<$ 0.001}\\
\multicolumn{7}{l}{\rule{0pt}{1em}Method: Multilevel linear regression}\\
\multicolumn{7}{l}{\rule{0pt}{1em}Fixed Effects: Classroom}\\
\multicolumn{7}{l}{\rule{0pt}{1em}Reference Category for Language: Other languages spoken at home}\\
\end{tabular}
\end{table}

In the upper half of the table, taking into account classroom fixed
effects, boys generally report being more interested in politics than
girls, but the gender gap is relatively minimal, standing at 0.4 for the
11-point political interest scale (\emph{p}\textless0.05). Boys'
interest in international affairs and partisan politics is higher than
girls' (1-point and 0.9-point difference respectively; both
\emph{p}\textless0.001). This seems to be in line with the results among
adult respondents. However, girls' interest in law and crime is also
slightly higher (0.5 point, \emph{p}\textless0.05). This result is more
surprising given previous literature showing the contrary. Yet, the
direction of the gender gap for these three topics matches findings
among Datagotchi PES adult respondents. Interest in the other two
topics, health care and education, are almost even between the genders.
Coupled with Datagotchi PES data showing significant gaps for both of
these topics only emerge after age 35, when women become more interested
than men in both cases, this suggests socialization for these two topics
may happen well into the adult age rather than during the formative
years, while gender patterns for interest in partisan politics and
international affairs emerge early and may fade after age 50. The
experiences of several women as mothers and caregivers later in life
might be the elements that shape the importance they start giving to
both issues.

However, when controlling for socio-demographic factors and personality
traits and adding interactions for gender and age as well as gender and
ethnicity, all relationships between gender and interest disappear, as
the lower part of Table \ref{tab:lmeInterestCCPIS} shows. Moreover, none
of the interaction terms are significant at the \emph{p}\textless0.05
level. Yet, when interaction terms are removed but other control
variables are kept, coefficients from the simple regression model remain
significant except for general political interest, which marginally
loses statistical significance.\footnote{Figure~\ref{fig-interestccpis}
  in Appendix~\ref{sec-appendix4} shows the gender coefficient for two
  intermediate models, one with SES controls and one with SES and
  personality traits but no interactions.} These results may therefore
be an artifact of high multicollinearity.

Figure~\ref{fig-interestyoungold} re-analyzes those results for children
aged 9--15 (\emph{n}=277) and teenagers aged 16--18 (\emph{n}=365).
Among younger students, the gender gap in self-reported political
interest is 0.1/10 (not statistically significant). Among those aged
16--18, this gap grows to 0.5/10 (\emph{p}\textless0.05). Notably, among
both groups, both interest in international affairs and interest in
partisan politics are higher among boys than girls, suggesting some
gender differences in interests may already exist before adolescence.

\begin{figure}

\centering{

\includegraphics{_graphs/GenderCCPISYO.pdf}

}

\caption{\label{fig-interestyoungold}Gender Differences in Interest for
Specific Political Topics by Age Group Among Canadian Children, 2022
CCPIS \newline \textit{Notes}: No controls are added.}

\end{figure}%

Moreover, interest in each of the five topics is higher among children
aged 16--18 than those aged 10--15, confirming insights from past
literature. Gender gaps amount to 0.8 points for general political
interest (\emph{p}\textless0.001), 0.8 points for health care
(\emph{p}\textless0.001), 0.6 points for international affairs
(\emph{p}\textless0.05), 0.7 points for law and crime
(\emph{p}\textless0.01), 1.1 points for education
(\emph{p}\textless0.001), and 0.2 points for partisan politics
(N.S.).\footnote{The full regression table for
  Figure~\ref{fig-interestyoungold}, Table
  \ref{tab:lmeInterestYoungOldCCPIS}, can be found in
  Appendix~\ref{sec-appendix4}.}

These results, coupled with those among young adults, support the idea
that, while a gender gap in self-reported political interest emerges
between ages 15--25, gender gaps in the other five topics develop in
different ways, contrary to the expectations of Hypotheses 0a, 0b and
0d. For education politics, significant fluctuations in the overall
level of interest seem to be related to life-cycle effects including the
birth of children and grandchildren. For health care politics, the
gender gap becomes significant only around age 35, perhaps as care
experiences among women have become more common. For partisan politics
and international affairs, young boys already express significantly more
interest in these topics than young girls, which lends partial support
for Hypothesis 0c, that gender differences in assertion-focused topics
would emerge at 10--15 years old or earlier. The gender gaps for these
two topics then disappear around age 40--50 --- the same moment when the
gender gap in general political interest starts narrowing in Canadian
election studies.

Figure~\ref{fig-political} shows how students view each of the 10
concrete issues associated with the 5 topics as political or
non-political depending on their age. As expected by Hypothesis 0e, the
two issues related to partisan politics are almost universally seen as
political, followed by issues related to international affairs. These
are the two topics for which boys report being more interested. Issues
related to the other three topics are perceived by 25\% to 60\% of
students as being non-political --- substantially large proportions. R.
Campbell and Winters (2008) and Ferrin et al. (2020) had similarly found
that issues related to partisan politics were seen as more political
than those related to other topics. Visibly, this trend also applies to
Canadian children, which suggests that topics that are seen as political
are also those for which men typically report a higher degree of
interest.

\begin{figure}

\centering{

\includegraphics{_graphs/CCPISPolitical.pdf}

}

\caption{\label{fig-political}Views of Topics as Political or
Non-Political By Canadian Students By Age Group, 2022 CCPIS}

\end{figure}%

Contrary to Hypothesis 0e, age does not seem to matter in defining which
types of issues students see as political. Older students are more
likely to define issues as political, but the ordering of issues as
political does not change much between both groups --- partisan politics
and international affairs always are at the top of the list. It seems
that ideas about what politics is and is not are already well-defined in
children's minds at a young age. This is particularly striking as
interest in partisan politics is not significantly higher among older
children than younger ones, suggesting ideas about partisan politics ---
both what it is and how interesting it is --- are already ingrained in
the early teens.

\section{Discussion}\label{discussion}

This chapter highlights gender differences in interest in various
political topics, but also shows differences between children and
adults. For international relations and partisan politics, male students
report more interest in these topics, just like adults in previous
studies. However, girls' higher interest in law and crime and similar
levels of interest in health care and education compared to boys
contrast with data previously found among adults (R. Campbell and
Winters 2008; Coffé 2013; Ferrin et al. 2020; Hayes and Bean 1993; Kuhn
2004; Sabella 2004a; Verba, Burns, and Schlozman 1997).

Russo and Stattin (2017) suggested that the 13--15 age range is a period
when parents, teachers, and other role models have the highest potential
for transmission of interest in politics to children. Applying these
results to the Canadian context, it appears that the 15--25-year-old
period, similar to what previous studies have found, is also the point
in time where a gender gap in interest for politics more generally
emerges, with boys/men reporting higher interest in politics in general
than girls/women.

However, these findings do not seem to apply to any political topic
individually. Given the results of the Datagotchi PES, which shows
higher levels of interest in the political aspects of health care and
education among adult women than men, it seems that the gender gap in
interest in these topics emerges around 30--40 years old, perhaps as a
result of life events that happen at that stage, such as women becoming
more likely to care for children and relatives. Moreover, with the help
of CCPIS data, it is possible to note, for the first time, that some
gender differences in interests are not necessarily pre-existent among
adolescents. On the other hand, reported interest in politics in
general, partisan politics and international affairs is already higher
among boys aged 10 to 18, showing for the first time that gender
differences in interest for certain aspects of politics seem to emerge
before the emergence of a gap in interest in politics in general.
Moreover, young students already think of issues related to these two
topics as being more political than other issues related to health care
or education politics, highlighting the fact that political thinking may
be starting early.

Overall, measuring political interest using a single-item may give the
impression that political interest emerges between ages 15 and 25,
alongside a gender gap where men start reporting higher interest. Yet,
looking at different data sources, it seems that the evolution of
interest in politics varies not only by gender, but also by topic.

\bookmarksetup{startatroot}

\chapter{Parent--Child Political Interest Transmission: Do Moms
Influence their Daughters and Dads Influence Their
Sons?}\label{sec-chap4}

Parents play a prominent role in raising their children and therefore
are among the main actors who can transmit political interest to them.
Political scientists have long studied the extent of parents' role in
that regard and the mechanisms through which political interest can be
transmitted. A. Campbell et al. (1960) already suggested that ``interest
in politics, like partisanship, is readily transmitted within the family
from generation to generation'' (413). Indeed, recent research has
highlighted inequalities in the extent to which boys and girls match
their parents' level of political interest. This chapter aims to uncover
the complex role gender can play in mediating parent--child political
interest transmission. When and how does political interest transmission
occur? To what extent do mothers influence political interest
development in their children? Do they have a greater influence on their
sons or daughters? What about fathers? Do these results hold for
different political topics, when political interest is measured by
sector? The chapter relies on social learning theory and seeks to test
this dissertation's first hypothesis highlighted in
Chapter~\ref{sec-chap1}: \emph{Children's interest in specific political
topics is more related to political discussions with their same-gender
parent(s) than their other-gender parent(s).}

\section{Parental Political Interest
Transmission}\label{parental-political-interest-transmission}

The transmission of interest in politics between parents and children
has typically been measured by using methods to statistically compare
their answers to the same political interest questions. Studies have
generally shown a significant relationship between parents' political
interest and their children's political interest (Beauregard 2008;
Janmaat, Hoskins, and Pensiero 2022; Neundorf, Smets, and
Garcia-Albacete 2013; Prior 2019; Shehata and Amnå 2019). While
Jennings, Stoker, and Bowers (2009) find no statistically significant
relationship, their results show that successful parent--child political
interest transmission occurs when the family environment is more
politicized. Prior (2019) estimates moderately strong Pearson
correlation coefficients of 0.3 to 0.4 for parent--child political
interest scores across three countries. By comparison, correlation
coefficients of different indicators of political interest for the same
individual vary between 0.6 and 0.7.

Parent--child political interest correlations vary by age. Prior (2019)
finds very weak (0.05) correlations at age 11, followed by a steady
growth until age 15. Parent--child correlations then remain stronger
when both parents share a similar level of political interest, when this
parental political interest is stable through time, and when children
move out late from their parents' place. Children who move out early of
their parents' place tend to see a quick drop in the extent to which
their political interest matches their parents'. Janmaat, Hoskins, and
Pensiero (2022) also find that, between ages 11 and 15, the gap in
political interest between children whose parents are not interested in
politics and children whose parents are interested in politics grows
every year. Shehata and Amnå (2019), for their part, find the strength
of the relationship between parents' news media use and children's
political interest remains stable between ages 13 and 18.

While parents' political interest often seems to match their children's
political interest, most studies lack the kind of data needed to
establish a causal link between both. However, Prior (2019) uses panel
data collected among parents and children and finds a weak but
noticeable causal link in parental transmission of political interest.
In time series, an increase in mothers' political interest is often
accompanied by an increase in their children's political interest,
something Dostie-Goulet (2009b) also finds. The same goes for fathers as
well as for decreases rather than increases. These trends are clearer in
the United Kingdom and Germany, despite weaker evidence in Switzerland.
In the first two countries, it seems reasonable to assume that a change
in parents' political interest could cause a similar change in their
children's political interest (Prior 2019).

The main causal mechanism for this transmission process seems to be
parent--child political discussions. D. E. Campbell and Wolbrecht (2006)
find children are much more likely to have political discussions at home
than with peers or teachers. Scholars have long suggested that the
development of political interest can happen through increasingly
complex discussions about political topics between parents and children
at home (D. Easton, Dennis, and Easton 1969; Greenstein 1965). These
discussions can be initiated either by the parent or by the child, and
the more they occur, the likelier the child is to be interested in
politics. This relationship between the frequency of political
discussions with parents and the child's interest in politics is
statistically significant in the United States (Shehata and Amnå 2019),
China, Mexico, Japan (Mayer and Schmidt 2004), and Canada (Dostie-Goulet
2009b). Similarly, in Poland, Furman, Szczepańska, and Maison (2022)
find that an aggregate of discussions with parents and parents'
political interest is significantly related to children's political
interest, controlling for other factors, a finding that is confirmed
with in-depth qualitative interview data, while Levinsen and Yndigegn
(2015) find the same for political discussions with young people's best
friends in Denmark. Furthermore, Shehata and Amnå (2019) find that
changes in the frequency of political discussions with parents
positively predict changes in adolescents' level of political interest.

It is also important to recognize that children can play a role in
shaping political discussions with their parents --- and potentially
their parents' interest in politics. McDevitt and Chaffee (2002) find
evidence of trickle-up political socialization, in which adolescent
children initiate political discussions with parents, who react by
increasing their news consumption or finding other ways to gain
knowledge about politics in order to maintain a leadership role in the
family. In this study, children's interest is first triggered through a
civics curriculum. York (2019) even finds that adolescents' news use and
political discussions with peers have a positive influence on their
political discussions with parents, while the opposite effects ---
political discussions with parents influencing adolescents' news use or
political discussions with peers --- are not found to hold. In a similar
vein, Stattin and Russo (2022) show that adolescents' initial level of
political interest can predict changes in their perceptions of their
parents' political interest, while their parents' initial political
interest (as perceived by the child) does not predict changes in their
own future political interest.

Yet, most other studies about the relative influence of various
socialization agents on the development of political interest in
children and adolescents mostly find that parents play a larger role in
transmitting political interest than any other socialization agent ---
including friends, media and schools (Dostie-Goulet (2009b) and Shehata
and Amnå (2019) but not Koskimaa and Rapeli (2015)). Trickle-down
political socialization --- from parents to children --- therefore
remains an important explanation of children's development of political
interest.

\section{Gender Differences in Parental
Transmission}\label{gender-differences-in-parental-transmission}

Research has found that the trickle-down effect of political interest
from parents to children works in gendered ways, and much of that
evidence suggests transmission works better for parent--child pairs of
the same gender. From the perspective of parents, mothers' political
interest has a stronger effect on their daughters than sons' political
interest, while fathers' political interest has a stronger effect on
their sons' political interest (Beauregard 2008; Owen and Dennis 1988;
Prior 2019). Disregarding parents' gender, Sabella (2004b) finds that
parents' potential to transmit political interest to their sons is
stronger than to their daughters. From the perspective of children,
daughters' political interest seems to be influenced mostly by their
mothers, with the mother--daughter political interest link stronger than
all other combinations (Beauregard 2008; Owen and Dennis 1988; Prior
2019), although Rebenstorf (2004b) strikes a discordant note, finding
that fathers' potential to transmit political interest to their children
is stronger than mothers'. Finally, it is not clear whether sons'
political interest is influenced mostly by their father (Beauregard
2008; Owen and Dennis 1988) or mother (Prior 2019).

The strength of parent--child political interest correlations by gender
is inquired by Prior (2019), whose findings are more recent and rely on
panel data. In the British dataset, they find mother--daughter political
interest correlations are the strongest (0.43), while mother--son,
father--daughter and father--son correlations all sit between 0.31 and
0.34. In their German dataset, all four pairs are relatively close
together, but exact numbers are not provided by the author. Finally, in
the Swiss dataset, mother--daughter correlations are the strongest
(0.36), followed by mother--son (0.31), father--son (0.21) and
father--daughter (0.2). There is overall a stronger causal effect of
mothers' political interest on their children's political interest,
compared with fathers.\footnote{Kestilä-Kekkonen et al. (2023) also find
  stronger effects of mothers on children when it comes to the
  transmission of political self-efficacy, but do not find a stronger
  relationship between parents and children of the same gender.} Oswald
and Schmid (1998) find that the gap in the frequency of political
discussions with mothers and fathers remains stable between ages 16 and
18.

Studies have investigated the gender patterns in parent--child political
discussions, but the amount of political discussions does not seem to
vary based on parents' and children's gender. While earlier studies
found that children discuss politics more often with their fathers than
mothers (Levinsen and Yndigegn 2015; Noller and Bagi 1985; Oswald and
Schmid 1998), most of the recent research has found no significant
difference between fathers and mothers (Hooghe and Boonen 2015; Mayer
and Schmidt 2004; Shulman and DeAndrea 2014). Noller and Bagi (1985)
also find that parents discuss politics more often with their sons.
However, most studies again find a different pattern in which parents
discuss politics with their daughters as much as with their sons (Dowse
and Hughes 1971; Lawless and Fox 2015; Mayer and Schmidt 2004). Overall,
it seems reasonable to assume, as Hooghe and Boonen (2015) do, that
political discussions involving the father tend to revolve mostly around
partisan politics, while political discussions with mothers might center
on other topics --- presumably health care, education, gender issues,
and so on.

The nature of discussions about politics could also vary by gender, with
more conflictual discussions happening between sons and fathers, or
between mothers and daughters, than between other-gender pairs.
Vuchinich (1987) shows the opposite to be true: parents are more likely
to begin conflictual interactions with their other-gender children than
same-gender children, while children initiate conflictual interactions
with their mother more than with their father. It is not clear whether
these patterns also apply to political discussions.

While past studies have found no differences in the frequency of
political discussions by child gender or parent gender, these studies
have not studied whether the specific topics of political discussions in
the family could vary by parent gender or child gender. Are fathers more
likely than mothers to discuss agency-related political topics with
their children? Are parents more likely to discuss communality-related
political topics with their daughters than sons? Hypotheses 1a--1f
provide hypotheses about the frequency of political discussions by
gender in the family unit.

\begin{itemize}
\item
  \emph{\textbf{Hypothesis 1a}: Mothers are as likely as fathers to
  discuss any specific political topic with their children.}
\item
  \emph{\textbf{Hypothesis 1b}: Mothers are more likely than fathers to
  discuss the politics of health care and education with their
  children.}
\item
  \emph{\textbf{Hypothesis 1c}: Fathers are more likely than mothers to
  discuss law and crime, international affairs, and partisan politics
  with their children.}
\item
  \emph{\textbf{Hypothesis 1d}: Parents are as likely to discuss any
  specific political topic with their sons as with their daughters.}
\item
  \emph{\textbf{Hypothesis 1e}: Parents are more likely to discuss the
  politics of health care and education with their daughters than sons.}
\item
  \emph{\textbf{Hypothesis 1f}: Parents are more likely to discuss law
  and crime, international affairs, and partisan politics with their
  sons than daughters.}
\end{itemize}

Hypotheses 1a, 1b and 1c focus their attention on discrepancies between
mothers' and fathers' likelihood of discussing certain topics with their
children, while Hypotheses 1d, 1e and 1f focus their attention on
discrepancies between daughters and sons in their likelihood of
discussing politics with --- or being told about politics by --- their
parents. Hypotheses 1a and 1d seek to corroborate previous findings by
Mayer and Schmidt (2004) and others that political discussion frequency
in the family does not vary by gender. On the contrary, Hypotheses 1b
and 1e suggest that parent gender is a significant factor that explains
which topics will be discussed with children, and Hypotheses 1c and 1f
suggest that child gender is a key variable that explains which topics
they will discuss with their parents.

Parents' education can also affect their children's interest in
politics. Status transmission theory suggests that ``well-educated
parents are more likely to provide a politically stimulating home
environment'' and therefore have politically engaged children (Gidengil,
Wass, and Valaste 2016, 373). It is not clear if status transmission
theory works in gendered ways. Beauregard (2008) and Janmaat, Hoskins,
and Pensiero (2022) find that parents' education has a positive
relationship with their adolescents' political interest regardless of
parents' gender, while other studies find a stronger effect of one
parent. Neundorf, Smets, and Garcia-Albacete (2013) and Koskimaa and
Rapeli (2015) find a positive relationship between fathers' education
and children's interest in politics, Sanjuan and Mantas (2022) instead
find a positive relationship for mothers, Jennings, Stoker, and Bowers
(2009) find no relationship for both fathers and mothers, and Koskimaa
and Rapeli (2015) even find a \emph{negative} relationship between
mothers' education and children's political interest.

Other parental characteristics can also influence the development of
children's political interest in gendered or non-gendered ways.
Gidengil, O'Neill, and Young (2010) and Cicognani et al. (2012) find
that a mother's level of political participation has a positive link
with her daughter's political interest. Jennings, Stoker, and Bowers
(2009) do not find a relationship between parents' income and children's
political interest, but Neundorf, Smets, and Garcia-Albacete (2013) find
a positive relationship between fathers' income and children's political
interest, which they suggest may be due to an indirect process in which
fathers' income increases children's level of education and development
of higher class civic attitudes, which then increases their level of
interest in politics. Borkowska and Luthra (2024) find that political
interest transmission patterns vary between immigrant and non-immigrant
families, with intergenerational transmission somewhat weaker in
immigrant families than other families, notably because other factors
--- such as socialization in a good-performing democracy or not --- have
a more significant influence. The authors find no moderating effect for
naturalization.

\section{Social Learning Theory}\label{social-learning-theory-1}

As highlighted in Chapter~\ref{sec-chap1}, social learning theory
suggests that children learn by observing their parents' behaviour,
attitudes, habits and values, and model their behaviour, attitudes,
habits and values after them (Gidengil, Wass, and Valaste 2016; Shehata
and Amnå 2019). This process applies to the transmission of political
interest (Jennings, Stoker, and Bowers 2009; Prior 2019; Shehata and
Amnå 2019). Prior (2019) suggests the transmission process starts in
early adolescence when children still live in the family home but start
acquiring an understanding of what politics is. Its effectiveness is
higher when parents give consistent and strong cues about their own
level of political interest (Jennings, Stoker, and Bowers 2009; Prior
2019). Parents' political opinions and leanings are also more likely to
be known by their children when the family environment is politicized
since this environment can foster social learning (Jennings, Stoker, and
Bowers 2009; Neundorf, Smets, and Garcia-Albacete 2013). In a political
home environment, children can either feel social pressure to become
interested in politics to create or maintain a sense of social belonging
in the family unit, be exposed to more news media content, listen to or
participate in more political discussions at home, or all of these
(Shehata and Amnå 2019). Bandura (1969) further explains that
observer--model similarity leads children to model their behaviour,
values, attitudes and habits on models that resemble them, notably their
parent(s) of the same gender.

Social learning theory and observer--model similarity both assume that
political interest should be more strongly correlated between mothers
and daughters and between fathers and sons than any other combination.
While this assumption has somewhat effectively been tested in past
research, mothers seem to have an overall stronger influence on the
development of political interest in their children. Moreover, it
remains unclear if this finding applies similarly across various
political topics, or if the cooperation-oriented topics in which women
typically report being more interested --- health care and education ---
can better be transmitted by mothers while those assertion-focused
topics in which men typically report being more interested --- law and
crime, international affairs, and partisan politics --- can better be
transmitted by fathers. Four hypotheses are therefore tested alongside
each other:

\begin{itemize}
\item
  \emph{\textbf{Hypothesis 1g}: Children's interest in specific
  political topics is more affected by political discussions with their
  same-gender parent(s) than their other-gender parent(s).}
\item
  \emph{\textbf{Hypothesis 1h}: Children's interest in the politics of
  health care and education is more affected by political discussions
  with their mother(s) than their father(s).}
\item
  \emph{\textbf{Hypothesis 1i}: Children's interest in law and crime,
  international affairs, and partisan politics is more affected by
  political discussions with their father(s) than their mother(s).}
\item
  \emph{\textbf{Hypothesis 1j}: Children's interest in specific
  political topics is more affected by political discussions with their
  mother(s) than their father(s).}
\end{itemize}

According to Hypothesis 1g, which is the one most consistent with social
learning theory, a parent's interest in a specific political topic
should influence interest in that topic more strongly for their
same-gender children than other-gender children. On average, a mother is
expected to have more transmission potential of her interest in health
care issues to her daughters than sons through political discussion, for
example. Given the importance of parents' role in socialization in
childhood and adolescence, this transmission process would explain part
of the gender gap in interest in topics such as health care and partisan
politics that exists among adults.

While Hypothesis 1g supposes similar effects of parental discussions for
different topics and suggests the key factor explaining interest
transmission is parent--child gender congruence, the other three
hypotheses provide different answers to the question of how political
interest gender gaps could be transmitted by parents. Hypotheses 1h and
1i suppose the key factor explaining interest transmission is the nature
of the topics themselves --- with mothers having more transmission
potential of interest in topics for which the average woman reports more
interest, and the same for fathers and men's political interests.
Finally, Hypothesis 1j supposes that mothers have more transmission
potential than fathers for all topics. Hypotheses 1g and 1j provide
further tests of the findings of Beauregard (2008), Owen and Dennis
(1988), and Prior (2019), while Hypotheses 1h and 1i are alternate
hypotheses. Since R. Campbell and Winters (2008) suggest that women tend
to see political issues through the lens of cooperation while men tend
to see them through a lens of self-assertion, Hypotheses 1c and 1d test
the suggestion that parental socialization of boys favours
agency-related political topics while parental socialization of girls
favours communality-related political topics.

\begin{itemize}
\tightlist
\item
  \emph{\textbf{Hypothesis 1k}: Children's interest in specific
  political topics becomes more and more affected by political
  discussions with their parent(s) as they age.}
\end{itemize}

Finally, Hypothesis 1k tests the time trends in parental influence over
children's political interest, testing results put forward by Janmaat,
Hoskins, and Pensiero (2022), Prior (2019) and Shehata and Amnå (2019),
but relying on measures of parent--child political discussions rather
than parental political interest. The hypothesis implies that social
learning and observer--model similarity keep affecting children more and
more as they age, with girls developing more interest in topics they
discuss with their mothers and boys developing more interest in topics
they discuss with their fathers. For the sake of simplicity, the effect
is presumed to be more or less constant across political topics and all
parent--child pairs --- mother--son, mother--daughter, father--son,
father--daughter.

\section{Data and Methods}\label{data-and-methods-1}

The 2022--23 Canadian Children Political Interest Survey (CCPIS) is used
to study relationships between students' interest in certain topics and
parents' discussions of these same topics. This web-collected bilingual
dataset includes survey responses from 698 Canadian children and
adolescents aged 9 to 18. The CCPIS includes information about students'
interest in five political topics: health care, international affairs,
law and crime, education, and partisan politics. Further information
about the dataset and question wording for interest questions are found
in Chapter~\ref{sec-chap2}.

In order to determine the importance of parents depending on their
gender, the following question was asked to children: ``Which parent do
you discuss most often with? (a) Mother; (b) Father; (c) Both equally''.
Several questions are then used to assess the role of parents in
transmitting interest to their children. Given the important role of
political discussion in the transmission of political interest, this
variable is used to measure parents' transmission potential: ``For each
of the following topics, which parent do you discuss most often with?''
Answers are either ``My mother'', ``My father'' or ``Don't know/Prefer
not to answer''.\footnote{Students who do not have one parent of either
  gender are removed from the analysis.} Second, students are asked
``Among these five topics, which one do you discuss most often with your
mother(s)?'' Each of the five topics is listed, and the same question is
then asked about the father(s). While topics such as health care and
education can be spoken about without referring to their political
aspects, children were given examples of political issues related to
each of these topics shortly before in the same survey when they were
asked which topic they were most interested in.

All multilevel regression results presented in this chapter include
controls for socio-economic status, agency, communality, and classroom
fixed effects, except if otherwise specified. A simple multilevel
regression model with one explanatory variable --- often a given role
model's interest in a specific topic --- is followed by multiple
regression models with control variables for three blocs of variables:
(1) socio-economic status variables, (2) personality traits, and (3) all
other role models (peers, teachers, and influencers). Socio-economic
status variables include confounding variables that have been linked
with political interest and could therefore mediate any given
relationship found between two persons' political interest. These
include gender, age, language, ethnicity, and immigrant status.
Personality traits include the agency and communality scales developed
in Chapter~\ref{sec-chap2}.

\section{Results}\label{results-1}

\subsection{Parent--Child Political Discussions by
Gender}\label{parentchild-political-discussions-by-gender}

Figure~\ref{fig-parents} shows, for each topic, which parent students
report discussing the most often with. Out of 698 students, after
removing non-answers and missing data, 82\% of students report
discussing health care more often with their mother than their father,
and 74\% say the same for education. On the contrary, 64\% of students
say they discuss law and crime more often with their father, as well as
68\% for partisan politics and 71\% for international affairs. There are
more non-answers and missing data for partisan politics, which
presumably means no parent discusses the topic at home with their
children. When these results are broken down by students' gender (``All
discussions'' panel on the left), very similar results are found for
boys and girls, with one exception: in general, for all types of
discussions --- regardless of their political nature --- 62\% of boys
say they have more discussions with their mother, compared with 82\% of
girls who report the same.

\begin{figure}

\centering{

\includegraphics{_graphs/ParentTopicsGrey.pdf}

}

\caption{\label{fig-parents}Topic most often discussed with parents by
child gender, 2023 CCPIS data}

\end{figure}%

These results match with what previous literature has found about women
reporting more interest in education and health care, and men reporting
more interest in partisan politics, law and crime, and international
affairs. Hypotheses 1b and 1c are therefore corroborated, as opposed to
Hypothesis 1a. Also in line with the literature, parents seem to talk
about various political topics just as much with their sons as with
their daughters. Hypothesis 1d is therefore corroborated, as opposed to
Hypotheses 1e and 1f, which hypothesized agency-related topics would be
more prevalent in discussions with sons while communality-related topics
would be more prevalent in discussions with daughters.

Figure~\ref{fig-parentsmomdad} shows the extent to which mothers discuss
each of the five topics and the same for fathers. Mothers overwhelmingly
discuss education and health care according to their children, while
fathers discuss international affairs more than other topics but seem
like a more heterogeneous group. Despite partisan politics being the
least discussed topic by both parents, fathers are much more likely to
discuss it than mothers according to this metric. For law and crime too,
fathers are more likely to discuss it. Again, when these results are
broken down by students' gender, very similar results are found for boys
and girls.

\begin{figure}

\centering{

\includegraphics{_graphs/ParentTopicsMomDadGrey.pdf}

}

\caption{\label{fig-parentsmomdad}Topic Most Often Discussed by Mothers
and Fathers, 2022 CCPIS}

\end{figure}%

The gender gap in political topics discussed is pretty stark between
mothers and fathers, according to their children's assessments. It is
important to specify that given the question's phrasing, it is not clear
if these interactions are initiated by parents or by their children, and
if it is reflective of top-down or trickle-up political socialization
(McDevitt and Chaffee 2002). Moreover, the question does not ask
children to specify a percentage of interactions about each topic
started by the mother or father; it simply asks them to pick the parent
most likely to have these discussions with them. There is no assumption
that mothers are the only ones talking about health care and fathers the
only ones to talk about international affairs, but on average, they are
more likely to raise this topic than the other-gender parent --- with
the caveat that children may have been thinking about health care and
education in a broader sense than simply their political aspects,
despite prior prompting that should encourage them to think about
political aspects.

Regardless, the fact that results among boys and girls both strongly
point in the same direction is revealing. It makes little doubt that
mothers and fathers speak differently about politics when they raise the
topic with their children. Both questions were formulated in concrete
ways, asking children the extent to which they discuss these topics with
their parents. There is no Hawthorne effect among kids; they would not
have an expected answer. Parents speak in starkly different ways about
politics to their children, and this confirms that political
socialization is a deeply gendered process.

\subsection{Topics Parents Discuss the
Most}\label{topics-parents-discuss-the-most}

When it comes to parents' role in political interest transmission, Table
\ref{tab:lmeParentCtrl} shows the relationship between students'
interest in each of the five topics and the gender of the parent who
discusses the topic with them the most. The top part of the table shows
determinants of interest among boys, while the bottom part does the same
for girls. Column ``All'' is an index that aggregates all parent--child
linkages matched by topic. Children's interest in either of the five
topics does not seem to be related to the gender of the parent who
discusses the topic the most; the topic-by-topic analysis shows all
relationships are statistically insignificant. However, when aggregated,
the fact that their father discusses a topic more than their mother
increases sons' interest in that topic by an average of 0.6 points on an
11-point scale (\emph{p}\textless0.001). This effect size represents
more than half of the 0.9-point gender gap in interest in partisan
politics and 1-point gender gap in interest in international affairs
seen in Chapter~\ref{sec-chap3}. The gender of the parent who discusses
more a topic does not have a significant effect on daughters' interest
in that topic..\footnote{Table \ref{tab:lmeParent} in
  Appendix~\ref{sec-appendix6} shows very similar findings when control
  variables are removed.} No variables among those tested significantly
predict girls' political interest, but agency has a strong positive
relationship with boys' interest, with strong individual coefficients
for partisan politics, health care and international affairs. Boys'
interest in any topic seems to be associated four times more with their
competitive and assertive traits than with their father discussing the
topic with them more than their mother. Table
\ref{tab:lmeParentCtrlInterac} in Appendix~\ref{sec-appendix5}
replicates the analysis while putting boys and girls in the same model.
A significant positive effect of fathers on their children's political
interest is found, supplemented by a negative term associated with being
a white girl.

\begin{table}
\centering\centering
\caption{Interest in Topic by Gender of Parent who Discusses that Topic the Most \label{tab:lmeParentCtrl}}
\centering
\fontsize{6}{8}\selectfont
\begin{tabular}[t]{lcccccc}
\toprule
  & All & Health care & International affairs & Law and crime & Education & Partisan politics\\
\midrule
\addlinespace[0.5em]
\multicolumn{7}{l}{\textit{Boys}}\\
\midrule \hspace{1em}(Intercept) & 0.767 & -0.406 & 4.571+ & 1.583 & -1.822 & -0.390\\
\hspace{1em} & (1.413) & (2.067) & (2.393) & (2.207) & (2.037) & (2.547)\\
\hspace{1em}Mother discusses topic more than father & -0.616*** & -0.506 & -0.216 & -0.396 & 0.266 & 0.190\\
\hspace{1em} & (0.163) & (0.404) & (0.388) & (0.399) & (0.389) & (0.472)\\
\hspace{1em}Age & 0.135 & 0.198 & -0.111 & 0.193 & 0.215+ & -0.026\\
\hspace{1em} & (0.090) & (0.133) & (0.148) & (0.140) & (0.129) & (0.158)\\
\hspace{1em}Ethnicity (1 = white) & 0.421* & 0.236 & 1.198** & 0.033 & -0.254 & 0.685\\
\hspace{1em} & (0.195) & (0.380) & (0.386) & (0.420) & (0.405) & (0.503)\\
\hspace{1em}Immigrant & -0.336 & 0.129 & -1.140* & -0.012 & -0.693 & -0.708\\
\hspace{1em} & (0.269) & (0.515) & (0.532) & (0.565) & (0.550) & (0.719)\\
\hspace{1em}English spoken at home & -0.278 & -0.362 & -1.169 & -0.411 & -0.507 & 0.674\\
\hspace{1em} & (0.414) & (0.796) & (0.812) & (0.857) & (0.831) & (1.140)\\
\hspace{1em}French spoken at home & -0.006 & 0.228 & -0.940* & -0.038 & 0.034 & 0.593\\
\hspace{1em} & (0.233) & (0.452) & (0.454) & (0.492) & (0.473) & (0.590)\\
\hspace{1em}Agency & 2.655*** & 3.022** & 2.612* & 1.390 & 1.503 & 5.376***\\
\hspace{1em} & (0.513) & (0.960) & (1.050) & (1.106) & (1.038) & (1.346)\\
\hspace{1em}Communality & 1.021 & -0.068 & 3.060* & 0.260 & 3.289* & 1.661\\
\hspace{1em} & (0.622) & (1.177) & (1.268) & (1.303) & (1.299) & (1.600)\\
\hspace{1em}SD (Intercept Class) & 0.667 & 0.570 & 0.701 & 0.353 & 0.338 & 0.002\\
\hspace{1em}SD (Observations) & 2.603 & 2.315 & 2.353 & 2.632 & 2.649 & 2.821\\
\hspace{1em}Num.Obs. & 1058 & 223 & 211 & 212 & 236 & 176\\
\hspace{1em}R2 Marg. & 0.059 & 0.071 & 0.136 & 0.028 & 0.081 & 0.135\\
\addlinespace[0.5em]
\multicolumn{7}{l}{\textit{Girls}}\\
\midrule \hspace{1em}(Intercept) & 3.092* & 0.684 & 2.805 & 5.115* & 1.322 & 2.077\\
\hspace{1em} & (1.397) & (2.026) & (2.253) & (2.319) & (2.228) & (2.772)\\
\hspace{1em}Mother discusses topic more than father & -0.211 & 0.154 & 0.009 & 0.153 & -0.421 & -0.036\\
\hspace{1em} & (0.178) & (0.453) & (0.439) & (0.426) & (0.452) & (0.500)\\
\hspace{1em}Age & 0.124 & 0.205 & 0.076 & 0.124 & 0.212 & 0.025\\
\hspace{1em} & (0.087) & (0.125) & (0.134) & (0.139) & (0.135) & (0.161)\\
\hspace{1em}Ethnicity (1 = white) & -0.348 & -0.290 & 0.148 & -0.259 & -0.435 & -0.128\\
\hspace{1em} & (0.221) & (0.399) & (0.492) & (0.468) & (0.429) & (0.579)\\
\hspace{1em}Immigrant & -0.171 & 0.234 & -0.439 & -1.151+ & 0.767 & -0.112\\
\hspace{1em} & (0.293) & (0.528) & (0.667) & (0.669) & (0.565) & (0.715)\\
\hspace{1em}English spoken at home & -0.520 & -0.061 & -0.311 & -1.217 & 0.184 & 0.090\\
\hspace{1em} & (0.383) & (0.647) & (0.831) & (0.759) & (0.746) & (0.897)\\
\hspace{1em}French spoken at home & 0.073 & 0.229 & -0.070 & -0.791 & 0.405 & 0.705\\
\hspace{1em} & (0.246) & (0.445) & (0.539) & (0.531) & (0.479) & (0.630)\\
\hspace{1em}Agency & 0.094 & -0.393 & 0.164 & 1.422 & -0.028 & 1.866\\
\hspace{1em} & (0.607) & (1.100) & (1.343) & (1.370) & (1.186) & (1.556)\\
\hspace{1em}Communality & 0.027 & 0.994 & 1.471 & -1.837 & 0.038 & -0.483\\
\hspace{1em} & (0.583) & (1.036) & (1.249) & (1.276) & (1.140) & (1.567)\\
\hspace{1em}SD (Intercept Class) & 0.734 & 0.715 & 0.392 & 0.534 & 0.820 & 0.514\\
\hspace{1em}SD (Observations) & 2.686 & 2.367 & 2.742 & 2.797 & 2.587 & 2.727\\
\hspace{1em}Num.Obs. & 978 & 222 & 189 & 200 & 218 & 149\\
\hspace{1em}R2 Marg. & 0.014 & 0.032 & 0.015 & 0.044 & 0.042 & 0.019\\
\bottomrule
\multicolumn{7}{l}{\rule{0pt}{1em}+ p $<$ 0.1, * p $<$ 0.05, ** p $<$ 0.01, *** p $<$ 0.001}\\
\multicolumn{7}{l}{\rule{0pt}{1em}Method: Multilevel linear regression}\\
\multicolumn{7}{l}{\rule{0pt}{1em}Fixed Effects: Classroom}\\
\multicolumn{7}{l}{\rule{0pt}{1em}Reference Category for Language: Other languages spoken at home}\\
\end{tabular}
\end{table}

\subsection{Topics Most Often Discussed with
Mothers}\label{topics-most-often-discussed-with-mothers}

Table \ref{tab:lmeAgentsCtrl} shows students' interest in each of the
five topics depending on whether this topic is the one they most often
discuss with various role models, starting with mothers. The top part of
the table shows determinants of interest among boys, while the bottom
part does the same for girls. Again, column ``All'' is an index that
aggregates all role model--child linkages matched by topic. For boys,
interest in either of the five topics or their aggregate is unrelated to
their discussion of these topics with their mothers. For girls, interest
in education is related to their mothers discussing education
(\emph{p}\textless0.01). For other topics, no statistically significant
effect is found\footnote{The coefficient for partisan politics was
  dropped from the model for rank deficiency.} --- but all effect sizes
are still largely positive. In the aggregate, if a girl's mother
discusses mostly one of the five topics, her interest in that topic is
expected to increase by 0.7 points on the 11-point scale
(\emph{p}\textless0.05), almost as large as the gender gaps in interest
in international affairs (1) or partisan politics (0.9).

\begin{table}
\centering\centering
\caption{Interest in Topic Most Often Discussed with Role Models \label{tab:lmeAgentsCtrl}}
\centering
\fontsize{6}{8}\selectfont
\begin{tabular}[t]{lcccccc}
\toprule
  & All & Health care & International affairs & Law and crime & Education & Partisan politics\\
\midrule
\addlinespace[0.5em]
\multicolumn{7}{l}{\textit{Boys}}\\
\midrule \hspace{1em}(Intercept) & -2.542 & -16.247 & -18.190 & 17.323 & 9.570 & 8.688\\
\hspace{1em} & (12.427) & (24.337) & (17.889) & (21.442) & (23.386) & (24.865)\\
\hspace{1em}Topic most discussed with mother? & -0.219 & 0.992 & 0.321 & -0.185 & -0.809 & 0.626\\
\hspace{1em} & (0.287) & (0.590) & (0.817) & (1.289) & (0.557) & (1.602)\\
\hspace{1em}Topic most discussed with father? & 0.809** & 1.206 & 0.065 & 1.178+ & 0.187 & 2.159*\\
\hspace{1em} & (0.289) & (1.331) & (0.453) & (0.590) & (0.713) & (1.050)\\
\hspace{1em}Topic most discussed with female friends? & 0.694* & -0.742 & -0.607 & 2.815** & 1.972** & 0.750\\
\hspace{1em} & (0.295) & (0.787) & (0.498) & (0.840) & (0.560) & (2.164)\\
\hspace{1em}Topic most discussed with male friends? & 0.614* & -2.506* & 0.457 & 0.201 & 0.325 & 2.620+\\
\hspace{1em} & (0.310) & (1.219) & (0.482) & (0.663) & (0.721) & (1.470)\\
\hspace{1em}Topic most discussed by teacher? & 0.356 & 1.201 & 1.517** & -0.733 & 0.220 & -0.311\\
\hspace{1em} & (0.297) & (1.042) & (0.477) & (1.185) & (0.640) & (1.372)\\
\hspace{1em}Topic most discussed by social media influencer? & 0.751* & 0.760 & 0.247 & 0.351 & 1.776 & -1.281\\
\hspace{1em} & (0.292) & (0.690) & (0.451) & (0.692) & (1.108) & (2.173)\\
\hspace{1em}Age & 0.361 & 2.390 & 3.033 & -2.646 & -1.452 & -1.147\\
\hspace{1em} & (1.741) & (3.427) & (2.510) & (3.010) & (3.238) & (3.494)\\
\hspace{1em}Age squared & -0.006 & -0.081 & -0.107 & 0.100 & 0.059 & 0.043\\
\hspace{1em} & (0.060) & (0.119) & (0.087) & (0.104) & (0.112) & (0.121)\\
\hspace{1em}Ethnicity (1 = white) & 0.112 & -0.778 & 0.539 & 0.441 & -0.352 & 0.544\\
\hspace{1em} & (0.284) & (0.644) & (0.530) & (0.552) & (0.636) & (0.693)\\
\hspace{1em}Immigrant & -0.328 & 0.350 & -1.476+ & 0.978 & -1.529 & -0.698\\
\hspace{1em} & (0.422) & (0.912) & (0.752) & (0.817) & (0.927) & (0.988)\\
\hspace{1em}English spoken at home & 0.266 & -0.438 & -0.819 & 0.992 & -0.105 & 0.403\\
\hspace{1em} & (0.743) & (1.575) & (1.202) & (1.423) & (1.524) & (1.608)\\
\hspace{1em}French spoken at home & 0.405 & 0.694 & -0.826 & 1.158 & -1.373 & 0.834\\
\hspace{1em} & (0.404) & (0.915) & (0.680) & (0.786) & (0.857) & (0.907)\\
\hspace{1em}Agency & 1.140 & 1.603 & 2.458+ & 2.961+ & -0.851 & 2.368\\
\hspace{1em} & (0.853) & (1.888) & (1.390) & (1.594) & (1.752) & (1.831)\\
\hspace{1em}Communality & 3.560*** & 3.189 & 3.062+ & 1.648 & 7.827*** & 1.784\\
\hspace{1em} & (0.921) & (1.945) & (1.528) & (1.713) & (1.980) & (2.063)\\
\hspace{1em}SD (Intercept Class) & 0.853 & 1.085 & 0.000 & 0.918 & 0.653 & 1.017\\
\hspace{1em}SD (Observations) & 2.317 & 2.288 & 1.912 & 2.043 & 2.364 & 2.440\\
\hspace{1em}Num.Obs. & 444 & 88 & 89 & 89 & 89 & 89\\
\hspace{1em}R2 Marg. & 0.170 & 0.203 & 0.298 & 0.263 & 0.285 & 0.201\\
\addlinespace[0.5em]
\multicolumn{7}{l}{\textit{Girls}}\\
\midrule \hspace{1em}(Intercept) & -31.417+ & -7.555 & -44.190 & 2.501 & -32.938 & -12.741\\
\hspace{1em} & (16.033) & (24.727) & (32.490) & (28.593) & (26.734) & (28.197)\\
\hspace{1em}Topic most discussed with mother? & 0.730* & 0.359 & 1.057 & 1.719+ & 2.038** & \\
\hspace{1em} & (0.317) & (0.568) & (1.818) & (0.976) & (0.632) & \\
\hspace{1em}Topic most discussed with father? & 0.455 & 0.243 & -0.158 & 2.013* & 0.355 & 0.114\\
\hspace{1em} & (0.311) & (0.834) & (0.729) & (0.744) & (0.715) & (0.984)\\
\hspace{1em}Topic most discussed with female friends? & 0.467 & -0.119 & 0.884 & -0.420 & 0.639 & 2.020\\
\hspace{1em} & (0.320) & (0.645) & (0.869) & (0.776) & (0.606) & (2.549)\\
\hspace{1em}Topic most discussed with male friends? & 0.647* & -0.718 & -0.160 & -0.212 & 0.325 & 3.073**\\
\hspace{1em} & (0.306) & (0.844) & (0.766) & (0.620) & (0.754) & (1.089)\\
\hspace{1em}Topic most discussed by teacher? & 0.086 & -0.749 & 0.102 & -0.729 & 0.149 & 3.032***\\
\hspace{1em} & (0.307) & (1.217) & (0.804) & (0.763) & (0.617) & (0.759)\\
\hspace{1em}Topic most discussed by social media influencer? & 0.812** & 1.581* & -0.325 & 1.504* & -1.612 & 1.577\\
\hspace{1em} & (0.309) & (0.683) & (0.716) & (0.713) & (0.977) & (1.505)\\
\hspace{1em}Age & 4.591* & 1.251 & 6.523 & 0.339 & 4.015 & 2.223\\
\hspace{1em} & (2.176) & (3.393) & (4.472) & (3.950) & (3.666) & (3.896)\\
\hspace{1em}Age squared & -0.148* & -0.025 & -0.217 & -0.011 & -0.120 & -0.073\\
\hspace{1em} & (0.073) & (0.116) & (0.153) & (0.135) & (0.125) & (0.133)\\
\hspace{1em}Ethnicity (1 = white) & -0.157 & 0.551 & -0.235 & -0.488 & 0.390 & -0.359\\
\hspace{1em} & (0.346) & (0.648) & (0.783) & (0.729) & (0.699) & (0.639)\\
\hspace{1em}Immigrant & -0.381 & -0.384 & -1.110 & -2.435* & 0.843 & -0.526\\
\hspace{1em} & (0.440) & (0.845) & (1.058) & (0.972) & (0.867) & (0.848)\\
\hspace{1em}English spoken at home & -0.435 & -0.306 & -0.954 & -0.702 & -0.086 & -0.717\\
\hspace{1em} & (0.498) & (0.841) & (1.058) & (0.935) & (0.938) & (0.886)\\
\hspace{1em}French spoken at home & 0.183 & -0.371 & -0.118 & -0.405 & 0.626 & -0.440\\
\hspace{1em} & (0.414) & (0.729) & (0.910) & (0.801) & (0.771) & (0.764)\\
\hspace{1em}Agency & 1.609 & -1.131 & 2.586 & 1.309 & 3.809+ & 2.930\\
\hspace{1em} & (1.012) & (1.839) & (2.348) & (2.036) & (2.080) & (1.886)\\
\hspace{1em}Communality & -0.997 & -1.252 & -0.452 & 0.439 & 0.017 & -3.603+\\
\hspace{1em} & (1.087) & (1.886) & (2.229) & (2.039) & (2.069) & (1.868)\\
\hspace{1em}SD (Intercept Class) & 1.051 & 0.000 & 0.482 & 0.223 & 0.905 & 0.784\\
\hspace{1em}SD (Observations) & 2.370 & 2.203 & 2.618 & 2.345 & 2.223 & 2.095\\
\hspace{1em}Num.Obs. & 387 & 78 & 77 & 76 & 78 & 78\\
\hspace{1em}R2 Marg. & 0.103 & 0.202 & 0.100 & 0.299 & 0.257 & 0.307\\
\bottomrule
\multicolumn{7}{l}{\rule{0pt}{1em}+ p $<$ 0.1, * p $<$ 0.05, ** p $<$ 0.01, *** p $<$ 0.001}\\
\multicolumn{7}{l}{\rule{0pt}{1em}Method: Multilevel linear regression}\\
\multicolumn{7}{l}{\rule{0pt}{1em}Fixed Effects: Classroom}\\
\multicolumn{7}{l}{\rule{0pt}{1em}Reference Category for Language: Other languages spoken at home}\\
\end{tabular}
\end{table}

Among significant predictors of girls' interest in all political topics,
only social media influencers have a marginally larger substantial
(0.8-point) effect, which highlights the important role mothers can have
among broader socialization influences. Among significant predictors of
boys' interest in all political topics, several other role models seem
to play an important role, just like communality: more cooperative boys
tend to be more interested in all topics, largest driven by a strong
relationship between cooperativeness and interest in education politics,
with extremes of the communality scale associated with a 7.8-point
difference on the 11-point interest scale.

\subsection{Topics Most Often Discussed with
Fathers}\label{topics-most-often-discussed-with-fathers}

Table \ref{tab:lmeAgentsCtrl} then shows students' interest in a
specific topic depending on it being the topic they most often discuss
with their father(s). For boys, interest in partisan politics is related
to their fathers discussing partisan politics (\emph{p}\textless0.05).
At the aggregate level, if a boy's father discusses mostly one of the
five topics, their interest in that topic is expected to increase by 0.8
points on an 11-point interest scale (\emph{p}\textless0.01), almost as
large as the gender gap in interest in partisan politics. For girls,
interest in law and crime is related to their fathers discussing law and
crime (\emph{p}\textless0.05). At the aggregate level, however, no
significant effect is found. None of the other topic-specific
father--son and father--daughter relationships are statistically
significant at the 95\% confidence level.\footnote{Coefficients for
  peers are analyzed in Chapter~\ref{sec-chap5}.}

Overall, results from Table \ref{tab:lmeAgentsCtrl} lend support to
Hypothesis 1g, which suggests that children's interest in a given
political topic should be more related to their same-gender parent
discussing that topic with them than their other-gender parent
discussing it with them. Fathers' influence on sons seems to be similar
to mothers' influence on daughters, contrary to some past research in
which the influence of mothers on political interest development was
deemed to be generally stronger than fathers. Table
\ref{tab:lmeParentCtrl} also shows that when a topic is discussed by
their father rather than their mother, their sons' interest in that
topic is more likely to increase. For girls, whichever parent discusses
a topic more does not seem to significantly predict their interest in
that topic when the question is asked this way --- in particular for
white girls. Overall, none of the relationships tested lend support to
Hypothesis 1j, that mothers' influence in the transmission of interest
in political topics would be greater than fathers.

Two robustness checks attempt to uncover whether the relationship
between parental discussion and child interest is robust to the
inclusion and removal of confounding variables. First, Tables
\ref{tab:lmeMother} and \ref{tab:lmeFather} in
Appendix~\ref{sec-appendix6} reproduce coefficients for mothers and
fathers from Table \ref{tab:lmeAgentsCtrl} without controls. Generally,
the same relationships are found, although the influence of mothers on
their daughters and of fathers on their sons increases by 1 point and
becomes significant at a higher level. Father--daughter interest
transmission at the aggregate level becomes significant at
\emph{p}\textless0.01 but remains substantively smaller than both
father--son and mother--daughter transmission. Mother--son,
mother--daughter and father--daughter law and crime interest
transmission also become marginally significant at least at the
\emph{p}\textless0.05 level. Figure~\ref{fig-mother} and
Figure~\ref{fig-father} in Appendix~\ref{sec-appendix5} also include
models where only SES is controlled for, and where SES and personality
traits are controlled for. Results are similar to these models without
controls.

Second, Table \ref{tab:lmeAgentsCtrlInterac} in
Appendix~\ref{sec-appendix5} replicates the Table
\ref{tab:lmeAgentsCtrl} analysis while putting boys and girls in the
same model. Here, the influence of fathers remains strong, positive and
significant at the 99\% confidence level, but the coefficient for
mothers becomes \emph{negative} --- discussion with the mother is not
associated with more interest in any of the five topics or the aggregate
index. However, the interaction term between mother discussion and child
interest is significant and even larger than the coefficient for the
influence of fathers (0.9 vs.~0.8), again confirming mothers do play a
role in the interests of their daughters.

Overall, the topic-by-topic analysis does not show a consistent pattern,
with relatively few significant coefficients and a lower sample size.
However, Table \ref{tab:lmeAgentsCtrl} shows that interest in
\emph{some} cooperation-oriented topics such as education is more likely
to be transmitted by mothers --- at least for girls --- while interest
in \emph{some} assertion-oriented topics such as partisan politics and
law and crime is more likely to be transmitted by fathers --- at least
for sons. Consequently, Hypotheses 1h and 1i are only tentatively
corroborated.

\subsection{Age Trends}\label{age-trends}

Figure~\ref{fig-discussparentyoctrl} shows relationships between
parents' discussion of certain topics and children's interest in those
topics broken down by age group. For the sake of simplicity and to avoid
small sample sizes, topic-by-topic analyses are excluded, with all
analyses aggregating each parent discussion--child interest pair.
Results are mixed but seem to show that in some respects, older
teenagers (ages 16--18) are more influenced by their parents' discussion
of certain topics than 10--15-year-olds. Notably, for girls, the
influence of both their father and mother is only statistically
significant among 16--18 year-olds, and it grows in size compared with
girls aged 10--15. No such difference between age groups is found among
boys. However, for boys aged 16--18, when their parent most interested
in a specific topic is their father, they become more interested in that
topic, while this relationship is weaker and non-significant for boys
aged 10--15. These results provide mixed evidence for Hypothesis 1k:
aging makes more statistically significant relationships emerge, but the
effect sizes only become somewhat wider between age groups. Yet, among
10--15-year-olds, only father--son political interest transmission is
statistically significant, which could indicate that early parental
socialization is more limited than what was previously thought. It could
be the case that due to a more limited sample size, smaller effect sizes
are present but do not reach statistical significance. This would be
consistent with the finding that, for all significant parent--child
interest relationships found among 16--18-year-olds, the direction of
the relationship among 10--15-year-olds is the same --- but the
coefficient is larger. Moreover, again confirming Hypothesis 1g, the
largest effect sizes among 16--18-year-olds are for mother--daughter and
father--son transmission. Substantively, when a topic is the one their
mother discusses the most, 16--18-year-old daughters' interest in that
topic increases on average by 1.2 points. Similarly, when a topic is the
one their father discusses the most, 16--18-year-old sons' interest in
that topic increases on average by 1 point.\footnote{Figure~\ref{fig-discussparentyo}
  in Appendix~\ref{sec-appendix6} finds similar results when controls
  are removed.}

\begin{figure}

\centering{

\includegraphics{_graphs/DiscussParentYOCtrl.pdf}

}

\caption{\label{fig-discussparentyoctrl}Interest in Topics by Gender,
Age and Discussion with Parents, 2022 CCPIS}

\end{figure}%

\section{Discussion}\label{discussion-1}

Overall, the analysis generally confirms the main hypotheses put
forward, with a few minor caveats. Broadly, it seems that social
learning theory applies just as well when political interest
transmission is evaluated across a range of political topics --- but not
necessarily for each parent--child pair and topic measured individually.

Descriptive results show a clear trend: mothers are much more likely to
speak about education and health care, while fathers are more likely to
speak about the other three topics. These trends mirror previous
literature on interest in these topics.

Moreover, fathers' discussion of various political topics is related to
their sons' interest in these topics, while mothers' discussion of
various political topics is related to their daughters' interest in
these topics. Mothers also seem to have a null impact on sons' political
interests, while fathers' influence over their daughters' political
interests disappears when measured in a model alongside other
socialization agents. Political interest seems to trickle down from
parents to children in gendered ways.

Results by age groups seem to indicate political interest grows with
time, but these results also seem to apply to girls more than to boys
--- and the increase seems rather limited. Future studies conducted
among a larger number of children may be able to see more clearly time
trends for the growth of parent--child political interest transmission
potential.

Results highlight the transmission potential of parents for interest in
various political topics to their children and they reinforce the idea
that, in some respects, there are different socialization routes for
girls and boys.

\bookmarksetup{startatroot}

\chapter{Gender Homophily and Social Learning: How Can Friends Shape
Political Interests?}\label{sec-chap5}

Peers play a crucial role in children's socialization, especially during
adolescence. This chapter aims to decipher peer-to-peer political
interest transmission and the role gender can play in that transmission.
To what extent do female peers and friends influence political interest
development? Does that influence grow as children age? Do female friends
have a greater influence in transmitting political interest to boys or
girls? What about male peers? Do these results hold for different
political topics, when political interest is measured by sector? The
chapter relies on gender homophily theory and social learning. It seeks
to test this dissertation's second hypothesis highlighted in
Chapter~\ref{sec-chap1}: \emph{Children's interest in specific political
topics is more related to political discussions with their same-gender
peer(s) than other-gender peer(s).}

As Chapter~\ref{sec-chap1} lays it out, theories of gender homophily and
social learning suggest that children are influenced by role models,
especially those who share their gender, and this finding applies to
attitudes such as political interest. Gender homophily can lead to
social learning with observer--model similarity: as children and
teenagers stick with other children who resemble them --- mostly of
their gender --- they can start imitating their habits, attitudes and
behaviours, including interests in various political topics.

\section{Peer Political Interest
Transmission}\label{peer-political-interest-transmission}

Friends and acquaintances sometimes discuss political issues. This is
especially true starting in adolescence since this period of life is
associated with increased social interactions between peers (Berndt
1982; Hunter 1985). While children generally report having more
political discussions with parents and teachers than with their friends
(Dostie-Goulet 2009b; Hess and Torney 1967), Oswald and Schmid (1998)
find that teenagers become more likely to discuss politics with best
friends, other friends, and classmates between ages 16 and 18, while
increases in political discussions with other socialization agents
remain more modest. These results are confirmed by Shehata and Amnå
(2019), who show that political discussions with parents and friends
increase with age, but discussions with friends increase the most
between ages 13 and 18. Both of these studies still find more political
discussions happen within the family than within friend groups.

Political discussions between friends and acquaintances matter since
they have been associated with higher political interest among child and
adult respondents. Those who find the greatest influence for friends are
probably Koskimaa and Rapeli (2015), whose results suggest that friends'
interest in politics and political discussions with these friends
influence 16-to-18--year-old Finnish teenagers' political interest to a
greater extent than family and school. Using the same two indicators to
determine friends' influence, Furman, Szczepańska, and Maison (2022)
find Polish young adults are significantly influenced by their friends,
controlling for the influence of other sources of socialization. They
also find similar results in dozens of in-depth qualitative interviews.
Among Danish teenagers and young adults, political discussion with their
best friends is also associated with political interest (Levinsen and
Yndigegn 2015). Janmaat, Hoskins, and Pensiero (2022) also find that
having politically interested friends reduces the influence of parental
education on young teenagers' political interest evolution.
Dostie-Goulet (2009b) also finds 14-to-17--year-old children's political
interest is explained both by their parents' political interest and by
their friends' political interest, but contrary to Koskimaa and Rapeli
(2015), she finds parents are more significant in that respect.
Interestingly, her results suggest that the influence of friends'
political interest on personal political interest does not increase and
may even decrease as children age. Another study by Klofstad (2007)
shows that, among first-year college students sharing their dormitory
with a randomly assigned roommate, there is a strong correlation (0.46,
\emph{p}\textless0.001) between discussing politics and current events
often with that roommate and saying discussions with that roommate have
increased their political interest.

Beyond associations between peers' discussion of political issues and
political interest, studies inquiring about longitudinal or causal
effects have gotten more mixed results. Dostie-Goulet (2009b) finds
evidence that \emph{changes} in political interest from one year to the
next are linked to changes in the number of political discussions with
friends just as much as changes in the number of such discussions with
parents. Shehata and Amnå (2019) also show that, using a two-way fixed
effects panel model, changes in children's frequency of political
discussions with peers are related to their changes in political
interest just as much as the frequency of political discussions with
parents. However, another test they make finds that political
discussions with peers at ages 14--15 do not lead to more political
interest two years later, after controlling for initial levels of
political interest and frequency of political discussions with parents,
suggesting parents play a more long-lasting role in the socialization
process over the long run. Stattin and Russo (2022) further temper the
results of this literature by showing that adolescents' initial level of
political interest can predict changes in their perceptions of their
peers' political interest, while their peers' initial political interest
(as perceived by the child) does not predict changes in their own future
political interest.

Other than political discussions, friends can also contribute to
children's political interest development in other ways. Shehata and
Amnå (2019) find that variations in the number of friends who are
interested in what happens in the world and consume news are related to
variations in political interest: the more news-watching peers, the
higher the political interest. They find that this consumption of news
by friends encourages children to watch more political news, but a
direct long-term impact of friends' news consumption on adolescent
political interest several years later is not found. Peer-group
diversity can also contribute to the development of political interest.
The size of one's social network has been found to have a positive
relationship with political interest, while political disagreement is
not related to it (Pattie and Johnston 2009).

\section{Gender Differences in Peer
Transmission}\label{gender-differences-in-peer-transmission}

Political discussions between friends typically exhibit gender effects.
Among adults, leaving aside relatives, 84\% of men report discussing
politics only with men, while 64\% of women report discussing politics
only with women (Huckfeldt and Sprague 1995). This could be a by-product
of gender homophily since people generally spend more time with other
people of the same gender. Lawless and Fox (2013) find that, among
United States college students aged 18--25, 20\% of women and 27\% of
men report discussing politics frequently with friends, and a similar
significant 6-point gap is found for discussion of current events with
friends. Kittilson and Schwindt-Bayer (2012) concur, showing that, among
29 countries located on different continents, there is a gender gap in
how often women and men discuss politics with their friends, with men
being on average 7 percentage points more likely to do so than women.
Among Canadian respondents, they estimate a 9-point gap. In Denmark,
young men have also been found to have more political discussions with
their friends than young women (Levinsen and Yndigegn 2015).

Interestingly, despite gender homophily studies finding children are
more likely to discuss with other children who share their gender, older
literature has not always found this to be the case for political
discussions. Hess and Torney (1967) find children are as likely to
discuss politics with their peers regardless of their gender, while
Dowse and Hughes (1971) find boys and girls are just as likely to have
heated political arguments with their friends.

There are four main limitations to the scientific literature about
peer-to-peer political discussions and political interest transmission.
Each of these limitations is associated with distinct hypotheses that
are tested in this chapter.

First, studies about political interest and socialization by peers have
not tested the effects of peer discussions or peer political interest on
chidlren's interest in political sub-topics, despite recent political
interest literature often measuring interest by sub-topic given the
important gender differences in interest found between issues (R.
Campbell and Winters 2008; Coffé 2013; Ferrin et al. 2020; Hayes and
Bean 1993; Kuhn 2004; Sabella 2004a; Tormos and Verge 2022; Verba,
Burns, and Schlozman 1997). This chapter's hypotheses integrate this
concern by inquiring about discussions of politics with friends by topic
instead of suggesting a single-item measure of political discussion and
political interest.

Second, no recent study has reported the relative frequency of
children's political discussions with female and male friends in any
country. This chapter seeks to provide that information in the Canadian
context. While research has historically not found large gender gaps in
children's political discussions with friends, recent study results
among adult populations have found men tend to discuss politics more
often with their friends than women do. Moreover, studies about gender
homophily suggest children mostly discuss with their same-gender peers
in general (Shrum, Cheek Jr., and Hunter 1988; Stehlé et al. 2013).
Hypotheses 2a and 2b suggest children are more likely to discuss
agency-focused topics with their male friends and communality-focused
topics with their female friends.

\begin{itemize}
\item
  \emph{\textbf{Hypothesis 2a}: Children are more likely to discuss the
  politics of health care and education with their female friends than
  male friends.}
\item
  \emph{\textbf{Hypothesis 2b}: Children are more likely to discuss law
  and crime, international affairs, and partisan politics with their
  male friends than female friends.}
\end{itemize}

Third, social learning theory and observer--model similarity suggest
that attitudes such as political interest should be transmitted by
same-gender peers more easily than by other-gender peers, an assumption
yet untested in any study. Studies about social learning theory have
instead mostly focused their attention on gender differences in parental
transmission of political interest. As Chapter~\ref{sec-chap4} reports,
gender congruence between a parent and their children plays an important
role in explaining parental transmission of political interest through
discussions. Most of the literature has found friends play an important
role in transmitting political interest, although some recent studies
such as Shehata and Amnå (2019) and Stattin and Russo (2022) suggest the
role of friends may be limited or nonexistent. Delving into the gendered
aspects of the transmission process may help clarify the role friends
can play in children's political interest development. Hypothesis 2c
suggests that discussion of a political topic with same-gender friends
should be associated with higher interest in that topic compared with
discussion with other-gender friends.

\begin{itemize}
\tightlist
\item
  \emph{\textbf{Hypothesis 2c}: Children's interest in specific
  political topics is more affected by political discussions with their
  same-gender friends than their other-gender friends.}
\end{itemize}

Fourth, past studies have found mixed results about whether political
interest transmission by friends increases as children get older, and
they have not inquired about the role of gender in that process. Since
gender homophily starts at a very young age, it is not clear whether
interest transmission by same-gender vs.~other-gender friends becomes
more prevalent as children get older. Hypothesis 2d will test the theory
that political discussions with friends become more important predictors
of political interest in that topic as children grow.

\begin{itemize}
\tightlist
\item
  \emph{\textbf{Hypothesis 2d}: Children's interest in specific
  political topics becomes more and more affected by political
  discussions with their friends as they age.}
\end{itemize}

\section{Data and Methods}\label{data-and-methods-2}

The 2022--23 Canadian Children Political Interest Survey (CCPIS) is used
to study relationships between students' interest in certain topics and
their friends' discussions of these same topics. This web-collected
bilingual dataset includes survey responses from 698 Canadian children
and adolescents aged 9 to 18. The CCPIS includes information about
students' interest in five political topics: health care, international
affairs, law and crime, education, and partisan politics. Further
information about the dataset and question wording for interest
questions are found in Chapter~\ref{sec-chap2}.

In order to determine the importance of peers depending on their gender,
the following question was asked to children: ``What is the gender of
most of your friends? (a) Girls; (b) Boys; (c) About the same for both
genders''. Several questions are then used to assess the role of peers
in transmitting interest through discussions: ``Among these five topics,
which one do you discuss most often with your male friends?'' Each of
the five topics is listed, and the same question is then asked about
female friends. While topics such as health care and education can be
spoken about without referring to their political aspects, children were
given examples of political issues related to each of these topics
shortly before in the same survey when they were asked which topic they
were most interested in.

All multilevel regression results presented in this chapter include
controls for socio-economic status, agency, communality, and classroom
fixed effects, except if otherwise specified. A simple multilevel
regression model with one explanatory variable --- often a given role
model's interest in a specific topic --- is followed by multiple
regression models with control variables for three blocs of variables:
(1) socio-economic status variables, (2) personality traits, and (3) all
other role models (parents, teachers, and influencers). Socio-economic
status variables include confounding variables that have been linked
with political interest and could therefore mediate any given
relationship found between two persons' political interest. These
include gender, age, language, ethnicity, and immigrant status.
Personality traits include the agency and communality scales developed
in Chapter~\ref{sec-chap2}.

\section{Results}\label{results-2}

\subsection{Peer-to-Peer Political Discussions by
Gender}\label{peer-to-peer-political-discussions-by-gender}

Figure~\ref{fig-peers-genderage} shows the percentage of boys and girls
who report having more friends of their gender, more of the other gender
or about the same for both genders. Boys and girls aged 9--15 and 16--18
all say they have more friends of their gender, with percentages of 67\%
and 65\% for younger boys and girls, respectively, against 68\% and 46\%
of older boys and girls. On the contrary, only 3 to 8\% of boys and
girls of either age group report having more friends of the other gender
than friends of their own gender. Finally, 28--30\% of boys and girls
aged 9--15 and 31\% of boys aged 16--18 say they have as many friends of
both genders. The percentage is higher for girls aged 16--18, at 46\%.

\begin{figure}

\centering{

\includegraphics{_graphs/PeersGenderAge.pdf}

}

\caption{\label{fig-peers-genderage}Children's Friends by Gender and Age
Group, 2022 CCPIS}

\end{figure}%

Gender homophily therefore holds among younger and older students, who
mostly bond with students of their gender. When excluding students who
say they have as many friends of both genders, no less than 95\% of boys
and 91\% of girls report that most of their friends are of the same
gender as theirs. The high proportion of girls aged 16--18 reporting
having as many friends of either gender may be due to the diminishing
effects gender homophily has over time, which has been reported by
Shrum, Cheek Jr., and Hunter (1988), Stehlé et al. (2013) and others,
although the fact that boys do not see this trend could mean gender
homophily remains present for a longer time frame among them.

Figure~\ref{fig-peers-topics} shows which of the five topics is most
discussed with friends of each gender among boys and girls. In a similar
vein to Figure~\ref{fig-parentsmomdad} in Chapter~\ref{sec-chap4}, boys
and girls alike are more likely to report discussing the
cooperation-focused topics of health care and education with female role
models --- mothers there, female friends here. They are also more likely
to report discussing the other three more assertion-focused topics of
law and crime, international affairs, and partisan politics with male
role models --- fathers in Chapter~\ref{sec-chap4}, male friends here.
These results provide evidence to support Hypotheses 2a and 2b. For each
of the five topics, expectations derived from studies among adults are
corroborated: children discuss education politics with female friends,
international affairs with male friends, etc. A simple t-test shows that
these differences are all statistically significant at the
\emph{p}\textless0.05 level, except partisan politics since few male and
female friends discuss this topic more than all others.

\begin{figure}

\centering{

\includegraphics{_graphs/PeersTopicsGrey.pdf}

}

\caption{\label{fig-peers-topics}Topic most often discussed with friends
by child gender and friends' gender, 2023 CCPIS data}

\end{figure}%

Education is the most commonly discussed topic with female friends for
both boys and girls, while the most commonly discussed political topics
with male friends are international affairs --- especially among boys,
highest percentage at 30\% --- and law and crime --- especially among
girls. Partisan politics is infrequently mentioned as the most discussed
topic by either boys or girls.

\subsection{Topics Most Often Discussed with Female
Friends}\label{topics-most-often-discussed-with-female-friends}

When it comes to female friends' role in political interest
transmission, Figure~\ref{fig-femalefriends} shows students' interest in
a specific topic depending on it being the topic they most often discuss
with their female friends. The left part of the figure shows
coefficients among boys and the right part shows coefficients among
girls. ``All topics'' is an index that aggregates all peer-to-peer
linkages matched by topic. Analysis centers on the first bloc at the
top, which includes no controls, as well as the fourth bloc at the
bottom of the figure, which includes all controls. In the first bloc,
for boys, interest in law and crime is related to their female friends
discussing law and crime (\emph{p}\textless0.01). At the aggregate
level, interest in any of the five topics is related to discussing this
topic with female friends (\emph{p}\textless0.05). If a boy's female
friends discuss mostly one of the five topics, his interest in that
topic is expected to increase by 0.5 points on the 11-point interest
scale, roughly half the size of the gender gaps in interest in
international affairs (1) or partisan politics (0.9) among children. For
girls, interest in law and crime (\emph{p}\textless0.05), but also
international affairs (\emph{p}\textless0.05) and partisan politics
(\emph{p}\textless0.01), is associated with their interest in those
matters. At the aggregate level, interest in any of the five topics is
related to discussing this topic with female friends
(\emph{p}\textless0.001). If a girl's female friends discuss mostly one
of the five topics, her interest in that topic is expected to increase
by 0.9 points on the 11-point scale, similar to the gender gaps in
interest in international affairs or partisan politics and higher than
the 0.5-point coefficient found among boys. So far, Hypothesis 2c
tentatively seems to be corroborated, with a larger effect size for
girls' political discussions than girls with boys, although both effect
sizes are within each other's confidence intervals.

\begin{figure}

\centering{

\includegraphics{_graphs/FemaleFriendsDiscuss.pdf}

}

\caption{\label{fig-femalefriends}Interest in topic most often discussed
with one's female friends}

\end{figure}%

In the fourth bloc, after adding controls, for boys, interest in law and
crime is related to their female friends discussing law and crime, and
the same now goes for education (both \emph{p}\textless0.01). At the
aggregate level, interest in any of the five topics is related to
discussing this topic with female friends (\emph{p}\textless0.05). If a
boy's female friends discuss mostly one of the five topics, his interest
in that topic is expected to increase by 0.7 points on the 11-point
scale. For girls, however, none of the relationships are statistically
significant at \emph{p}\textless0.05 in this model, and the coefficients
for health care and law and crime are even negative. These results are
opposite to what Hypothesis 2c predicts: other-gender influence on
political interests is marginally significant, but not same-gender
influence.

\subsection{Topics Most Often Discussed with Male
Friends}\label{topics-most-often-discussed-with-male-friends}

Figure~\ref{fig-malefriends} shows students' interest in a specific
topic depending on it being the topic they most often discuss with their
male friends. The left part of the figure again shows coefficients among
boys and the right part shows coefficients among girls. ``All topics''
is an index that aggregates all peer-to-peer linkages matched by topic.
Analysis again centers on the first and fourth blocs. In the first bloc,
without controls, for boys, interest in international affairs and law
and crime is associated with discussions of these topics with male
friends (both \emph{p}\textless0.001). For girls, interest in partisan
politics is associated with discussions of this topic with male friends
(\emph{p}\textless0.05). At the aggregate level, more discussions of any
of the five topics with male friends are associated with a higher
interest in that topic among both boys and girls
(\emph{p}\textless0.001). If a boy's male friends discuss one topic
more, his interest in that topic is expected to increase by 1.5 points
on the 11-point scale, larger than any gender gaps in interest reported
in Chapter~\ref{sec-chap3}. If a girl's male friends discuss one topic
more, her interest in that topic is also expected to increase by 0.9
points on the 11-point scale. In this model, the effect of male friends
seems to be twice as large for boys than girls, which is in line with
Hypothesis 2c.

\begin{figure}

\centering{

\includegraphics{_graphs/MaleFriendsDiscuss.pdf}

}

\caption{\label{fig-malefriends}Interest in Topic Most Often Discussed
with One's Male Friends, 2022 CCPIS}

\end{figure}%

In the fourth bloc, when adding controls for both boys and girls, the
coefficient for interest in partisan politics is the highest for both
genders. It is significant at the 99\% confidence level among girls,
similar to the model without controls, but does not quite reach
statistical significance among boys. Other than partisan politics,
discussions of political topics with male friends are not associated
with higher sectoral political interest for the child. Discussion of
health care with male friends is even associated with lower interest in
health care among boys (\emph{p}\textless0.05). Yet, at the aggregate
level, more discussions of any of the five topics with male friends
remain marginally associated with a higher interest in that topic among
both boys and girls (\emph{p}\textless0.05). Both coefficients are only
marginally statistically significant. If a boy's male friends discuss
one topic more, his interest in that topic is expected to increase by
0.6 points on the 11-point scale. If a girl's male friends discuss one
topic more, her interest in that topic is also expected to increase by
0.6 points on the 11-point scale (\emph{p}\textless0.05). Contrary to
the model with no controls, where male friends' influence on boys was
higher, in this case, both coefficients have the same size, which goes
against Hypothesis 2c.\footnote{The complete regression table for models
  with all controls in Figure~\ref{fig-femalefriends} and
  Figure~\ref{fig-malefriends} is found in Chapter~\ref{sec-chap4},
  Table \ref{tab:lmeAgentsCtrl}, while complete regression tables
  without controls can be found in Tables \ref{tab:lmeFemaleFriends} and
  \{tab:lmeMaleFriends\} in Appendix~\ref{sec-appendix7}.}

Patterns from Figure~\ref{fig-femalefriends} and
Figure~\ref{fig-malefriends} are somewhat complex to understand and do
not seem consistent with gender homophily theory. Discussion of a
political topic with peers has a significant and positive relationship
with interest in that topic in most models, confirming the influence of
peers in the socialization process overall, but gender patterns are not
those expected. If anything, other-gender peers seem to have a larger
influence than same-gender peers. Hypothesis 2c is therefore not
corroborated.

\subsection{Age Trends}\label{age-trends-1}

Figure~\ref{fig-discusspeersyoctrl} shows age patterns in peer-to-peer
political interest transmission. Topic-by-topic analyses are excluded,
with all analyses using the aggregate index. Interest in a political
topic is related to female friends' discussion of that topic for boys
aged 9--15, but not for boys aged 16--18. Similarly for the other
different-gender pair, interest in a political topic is related to male
friends' discussion of that same topic for girls aged 9--15, but not for
boys aged 16--18. The only significant relationship among children aged
16--18 is for male friends with boys, and it barely reaches statistical
significance at \emph{p}\textless0.05.

\begin{figure}

\centering{

\includegraphics{_graphs/DiscussPeersYOCtrl.pdf}

}

\caption{\label{fig-discusspeersyoctrl}Interest in Topics by Gender, Age
and Discussion with Peers, 2022 CCPIS}

\end{figure}%

Again, these results seem to be highly dependent on the inclusion or
exclusion of certain control variables. Figure~\ref{fig-discusspeersyo}
in Appendix~\ref{sec-appendix6} finds that, when controls are removed,
discussions of political topics with male friends have a positive and
significant influence on political interest in that topic among children
of both genders and age groups. Coefficients for female friends are also
all positive, but are weaker and only reach statistical significance in
the case of girls aged 16 to 18. Since more coefficients are significant
among older adolescents only, Hypothesis 2d is not corroborated.

\section{Discussion}\label{discussion-2}

This chapter provides mixed results regarding gendered peer influences
on political interest development. On the one hand, some results confirm
the presumption that friends matter and that gendered patterns appear in
children's discussions of political topics. Among children who say their
friends are mostly of one gender, more than 90\%, both girls and boys,
report they have more friends of their own gender. Moreover, female
friends tend to discuss topics that are more cooperation-focused, while
male friends tend to discuss topics that are more assertion-focused. In
all models but one, in the aggregate, discussions with male friends or
female friends of a certain topic are associated with a higher interest
in that topic among children. These include models who control for
background characteristics, personality traits and even the influence of
other role models. Clearly, peers seem to matter in the socialization
process, like past research has found. Multilevel regression models with
no controls also seem to indicate children's discussion of political
topics is mostly related to discussions of these topics with same-gender
friends as opposed to other-gender friends, although effect sizes'
confidence intervals do overlap.

On the other hand, in multilevel regression models with all controls,
discussions with other-gender friends seem to have a similarly-sized
influence on children's political interests than those with same-gender
friends. Moreover, the influence of discussions with friends on
political interescts does not seem to grow as children age.

Overall, it seems that social learning theory may not apply in the same
gendered way for friends than it does for parents, despite results
confirming the presence of gender homophily and gendered patterns in
political discussions. It could be the case that much of the gendered
transmission of interests is done by parents, while friends' influence
is more limited --- and less gendered, after taking into account
influences from other role models.

\bookmarksetup{startatroot}

\chapter{Conclusion}\label{sec-chap6}

This dissertation started by asking one question: \emph{How do men and
women differ with regard to political interest, and why?} For the
``how'' question, boys and girls, and later women and men, tend to be
interested in different political topics. These gender differences
intersect with age: older women are particularly interested in the
politics of education and health care, while younger men and boys tend
to report high interest in partisan politics and international
relations. For the \emph{why} question, socialization by parents and
peers has often been found to be the most important predictor of
political interest in countries where norms about gender equality are
more widely shared. This dissertation has clarified the role of parents
and peers in children's political interest development: parents
successfully transmit their political interests to their children who
share their gender, while peers' role seems to be as important
regardless of gender congruence.

\section{Contributions to the Scholarly
Literature}\label{contributions-to-the-scholarly-literature}

What this dissertation makes clear is that developing interest in health
care politics, law and crime, or another political topic, is not
something that happens \emph{solely} in the formative years of 15--25.
Although interest in most of these topics does increase through
adolescence, a general increase throughout adulthood can also be seen
for all topics except partisan politics. Some later developments in life
seem to be associated with the appearance of gender gaps in interest in
health care and education politics, which only appear in the mid-30s.
Women's and men's different experiences in caring for children and
relatives, notably, remain different in countries like Canada and seem
to be the most plausible explanations for the emergence of gender gaps
in interest in health care and education.

This dissertation contributes to the literature on social learning
theory. One of the implications of social learning theory is that
children's attitudes and interests should be influenced by role models
who share their gender. This assumption was tested among parents and
confirmed for political interests more generally in
Chapter~\ref{sec-chap4} --- although not individually for each of the
topics tested. The assumption had, however, never been tested among
friends. Chapter~\ref{sec-chap5} shows that friends' gender does not
matter as much as parents' gender in predicting children's political
interests. This result is surprising given that, as predicted by gender
homophily theory, children overwhelmingly tend to stick with other
children who share their gender. They also tend to discuss different
topics depending on other children's gender, but this does not translate
into higher political interest transmission of friends from one gender
compared with friends of another gender.

The analysis presented here also contributes to the literature about
political interest by clarifying, in the vein of recent studies by R.
Campbell and Winters (2008) and Ferrin et al. (2020), that gender
differences in political interest are not only about the \emph{level} of
political interest but also about the \emph{type} of political interest.
The level of political interest as well as the gender gaps vary between
each of the five topics tested --- health care, international affairs,
law and crime, education, and partisan politics. While all interest
associations between topics are positive, most associations between
interest in one political topic and interest in another topic are
relatively weak among children and adults,\footnote{The majority of
  correlation coefficients are under 0.4, moderately weak according to
  most scales (Akoglu 2018). See Figure~\ref{fig-cordg} and
  Figure~\ref{fig-corccpis} in Appendix~\ref{sec-appendix8}.}. Canadians
who are interested in the politics of health care may not be interested
in international relations or partisan politics, and so on. Political
interest can be divided into several components, and despite its close
association with partisan politics, both measures remain different. The
determinants of interest in each of the five topics would be worthy of
future scientific investigation.

This dissertation also emphasizes gender differences in politics that do
not have to do with ideology. Interest in a certain topic is not simply
a measure of ideological support for extra government funding in that
area --- if anything, the opposite might be true, as voters are more
likely to start supporting increased funding after their awareness and
interest in certain issues has been aroused. A recent study by Motta
(2019) finds that increased interest in science over time leads to more
support for government funding of science, even after controlling for
ideology, for instance. This research is therefore an invitation for
more research about the determinants of political interests --- as these
interests can lead to more discussion of various topics, something seen
as desirable in participatory democracy.

For parents, what these findings suggest is that parental education
matters. Political topics discussed at home have a significant influence
over children's interests, and this is particularly the case for
mother--daughter and father--son discussions. However, parents are only
one source of socialization for children among many others. Other than
parents and peers, two other socialization influences were tested in
full models used in Chapter~\ref{sec-chap4} and Chapter~\ref{sec-chap5}:
teachers and social media influencers. These findings deserve further
interest, given the importance of these two influences have sometimes
been found to be as important as parents and peers (Oswald and Schmid
1998).

\section{Media}\label{media}

Tested alongside other influences, influencers' discussion of a specific
topic is actually the strongest overall predictor of children's interest
in that same topic in \ref{tab:lmeAgentsCtrl}. Other research about the
role of media has generally found they play an important role in
socialization.

An emerging literature has found influencers can increase youth
political interest. Schmuck et al. (2022) hypothesize and show evidence
that, among adolescents and young adults, political interest is
negatively related to exposure to influencer content. Moreover, the
relationship is stronger when there were interactions between influencer
and follower in which the influencer engaged in perceived simplification
of politics. Harff (2022) similarly find among young women that
interactions with an influencer increase the strength of the
relationship between following an influencer and being interested in
politics. Harff and Schmuck (2024) find no relationship between
adolescents and young adults' use of social media influencers as a
\emph{primary} source of political information and their level of
political interest. Among older adults, Wasike (2023) finds that
following a social media influencer is positively related to political
interest, which suggests it is possible that this type of influence can
extend later in life.

With regards to media more generally, research has been done on the
influence of political news on the development of political interest
among adolescents. Although political interest can influence news media
consumption, studies have shown the relationship also works in the other
direction: watching more political news on traditional media and social
media has a positive effect on political interest (Holt et al. 2013;
Shehata and Amnå 2019). Websites can also have a positive influence on
people's political interest. Lupia and Philpot (2005) find young adults
can gain political interest by visiting informative websites under
certain conditions. More recently, Pap, Ham, and Bilandžić (2018) find
adolescents and young adults' political discussions on Facebook are
related to their interest in politics, but not those on Twitter.

Given the emerging interest in the role of influencers in political
socialization, it would be worth conducting a study to determine the
extent to which transmission of political interests is easier when the
influencer and follower share the same gender. A preliminary analysis in
Appendix~\ref{sec-appendix9}'s Table \ref{tab:lmeInfluencer} shows few
differences between the influence of same-gender influencers or
other-gender influencers on political interests, although the
coefficient for influencers who share the child's gender is larger (1.3
points agsinst 1.1 point). A closer examination of the role of
influencers might be warranted.

\section{Schools}\label{schools}

Tested alongside other influences, teachers' discussion of a specific
topic is not significantly related to children's interest in that same
topic in \ref{tab:lmeAgentsCtrl}. While teachers do not have a distinct
influence on interest into particular topics in this study, the broader
influence of schools on political interest development has been assessed
time and again in various studies. Schools can influence childhood
socialization in many ways, including citizenship education classes,
classroom political discussions, extracurricular activities, and active
learning strategies.

While earlier studies found only weak links between citizenship
education classes and political interest (Langton and Jennings 1968),
more recent studies find a positive link (Dassonneville et al. 2012;
Mahéo 2019; Neundorf, Niemi, and Smets 2016). \emph{Citizenship
education} and \emph{civic education} classes include classes teaching
facts about government and politics but also promoting political
engagement (Althof and Berkowitz 2006; Themistokleous and Avraamidou
2016).

Extracurricular and active learning activities can also affect youth
political interest. The number of class group projects and membership in
the school council are positively related to political interest, but not
participation in voluntary associations nor parliamentary visits
(Dassonneville et al. 2012).

Classroom political discussions in general can also increase students'
political interest. For instance, students' perceptions of an \emph{open
classroom climate} marginally increase their political interest
(Dassonneville et al. 2012), as well as several other aspects of their
political engagement. An \emph{open classroom climate} is one in which
``students experience the discussion of social and political issues
while in class and {[}in which{]} they feel comfortable contributing
their own opinions during such discussions'' (D. E. Campbell 2007, 62).

The role of classroom political discussions in political socialization
might be gendered. For instance, Mahony (1985) finds that girls are less
likely to participate in classroom discussions of politics because boys
make the classroom climate aggressive. However, Rosenthal, Jones, and
Rosenthal (2003) show that, while girls' presence has a slightly
positive impact on girls' speaking time, interruptions occur as
frequently between adolescents whatever their gender and studies find
that 8th- to 12th-grade girls are \emph{more} likely than boys to report
an open classroom climate (Blankenship 1990; D. E. Campbell 2007;
Maurissen, Claes, and Barber 2018).\footnote{Among adults, Karpowitz and
  Mendelberg (2014) and Beauvais (2020) also show that women's and men's
  relative speaking time in a deliberative and decision-making setting
  depends on the number of women. When decisions are made by a majority,
  the presence of more women leads to more speaking time for each woman.}

Encouraging a variety of perspectives in political debates is something
in participatory democracy (Collins 2017; Phillips 1992). For political
actors who seek a more gender-balanced set of political interests,
having a variety of political topics spoken about in class may be a good
way of getting different children's attention. Discussing health care
politics and education politics in an open classroom climate could be a
productive way for educators, organizations and political actors to
foster long-lasting interest in health care and education politics. The
extent to which the effects would be long-lasting may depend on several
factors and would be worthy of future scientific inquiry. On the other
hand, discussing partisan politics and international affairs in class
may also increase girls' interest in these topics, since they tend to be
less interested in that topic from a young age. In Quebec, these kinds
of open classroom discussions about politics would be well-suited for
classes such as the new \emph{culture and Quebec citizenship} class
(elementary and high school) as well as the \emph{contemporary world}
class (end of high school). In Ontario, they would be well-suited for
grade 10 Civics classes but also for grades 1--6 Social Studies classes.
More research about the extent to which this approach is currently used
in Ontario and Quebec would be warranted.

With regards to interest in partisan politics, the fact that younger men
report more interest may partly explain their overall higher intention
to run for political office later in life. Discussing this topic more
specifically in elementary school citizenship education classes could
help even out girls and boys' interest in partisan politics and
encourage young girls to see this as an option that is available to them
when they reach adulthood. For scholars studying women's descriptive
representation in politics, this link between interest in \emph{partisan
politics} specifically and political ambition would be a fruitful area
where more studies could be conducted.

\section{Limitations}\label{limitations}

Some limitations about this dissertation are worth keeping in mind.
Notably, age trends, especially in Chapter~\ref{sec-chap3}, should be
interpreted with caution. None of the data used is panel data, making it
difficult to distinguish age effects from generational effects and
life-cycle effects. While it is possible to say the gender gap in
interest in health care seems to increase with age among adults, it is
not possible to tell whether this is the result of older generations
having a larger gender gap or whether similar patterns will happen when
today's youth gets older.

Relatedly, CCPIS students from different age groups are not drawn in
similar proportions from the same schools, and some schools only include
classes from a certain age range, while others include students from
several school years. Since schools may differ from each other on a
variety of aspects, analyses comparing both age groups do not compare
groups who are similar in every respect other than age. Comparisons of
adults by age do not suffer from this issue since respondents are all
drawn from the same pool.

More generally, CCPIS student data is a non-random, convenience sample.
Its findings may not be generalizable to the broader population of
Canadian adolescents or to adolescents worldwide. Representative samples
of children and adolescents are difficult to access. These data remain
an indication of what relationships may be in the broader population
between the relevant variables, and there are no specific reasons to
expect different correlations in the broader population. Percentages of
girls, immigrants and ethnic minorities are broadly similar in the
sample and general Canadian population of adolescents, despite
geographic concentration in Ontario and --- mostly --- Quebec. Data
analysis relying on the CES, WVS and GSS should be broadly
representative of the Canadian population, especially after the use of
survey weights, and the same goes for the Datagotchi PES and the Quebec
population. Similar studies done elsewhere could provide further
evidence about how important gender congruence is in the parental
transmission of political interests in other contexts.

Finally, Stattin and Russo (2022) suggest that personal political
interest can influence perceptions of others' political interest or
perceptions of the frequency of discussions with others. This study does
not rely on panel data and cannot assess whether this is the case in
Canada. However, authors rely on a single-item measure of political
interest that may make the concept more blurry in children's minds than
specifying specific political topics linked with even more specific
political issues, as was done here. It seems unlikely that children's
interest in diplomatic disputes between Canada and China would influence
their perceptions of the frequency of their discussions with parents and
peers about such disputes in the same way that general political
interest could affect perceptions of the frequency of political
discussions with adults. However, to formally test this hypothesis,
future longitudinal studies should assess the causal influence of role
models on children's political interests.

\section{Grammar Check}\label{grammar-check}

\bookmarksetup{startatroot}

\chapter*{References}\label{references}
\addcontentsline{toc}{chapter}{References}

\markboth{References}{References}

\phantomsection\label{refs}
\begin{CSLReferences}{1}{0}
\bibitem[\citeproctext]{ref-akoglu2018}
Akoglu, Haldun. 2018. {``{User's Guide to Correlation Coefficients}.''}
\emph{{Turkish Journal of Emergency Medicine}} 18 (3): 91--93.

\bibitem[\citeproctext]{ref-alozie2003}
Alozie, Nicholas O., James Simon, and Bruce D. Merrill. 2003. {``{Gender
and Political Orientation in Childhood}.''} \emph{The Social Science
Journal} 40 (1): 1--18.

\bibitem[\citeproctext]{ref-althof2006}
Althof, Wolfgang, and Marvin W. Berkowitz. 2006. {``{Moral Education and
Character Education: Their Relationship and Roles in Citizenship
Education}.''} \emph{Journal of Moral Education} 35 (4): 495--518.

\bibitem[\citeproctext]{ref-arceneaux2012}
Arceneaux, Kevin, Martine Johnson, and Hermine H. Maes. 2012. {``{The
Genetic Basis of Political Sophistication}.''} \emph{Twin Research and
Human Genetics} 15 (1): 34--41.

\bibitem[\citeproctext]{ref-arens2017}
Arens, A. Katrin, and Rainer Watermann. 2017. {``{Political Efficacy in
Adolescence: Development, Gender Differences, and Outcome Relations}.''}
\emph{Developmental Psychology} 53 (5): 933.

\bibitem[\citeproctext]{ref-ashe2012}
Ashe, Jeanette, and Kennedy Stewart. 2012. {``{Legislative Recruitment:
Using Diagnostic Testing to Explain Underrepresentation}.''} \emph{Party
Politics} 18 (5): 687--707.

\bibitem[\citeproctext]{ref-atkesonrapoport2003}
Atkeson, Lonna Rae, and Ronald B. Rapoport. 2003. {``{The More Things
Change the More they Stay the Same: Examining Gender Differences in
Political Attitude Expression, 1952--2000}.''} \emph{Public Opinion
Quarterly} 67 (4): 495--521.

\bibitem[\citeproctext]{ref-bandura1969}
Bandura, Albert. 1969. {``{Social-Learning Theory of Identificatory
Processes}.''} In \emph{{Handbook of Socialization Theory and
Research}}, edited by David A. Goslin, 213--62. New York: Rand McNally.

\bibitem[\citeproctext]{ref-bashevkin1993}
Bashevkin, Sylvia. 1993. \emph{{Toeing the Lines: Women and Party
Politics in English Canada}}. Oxford University Press.

\bibitem[\citeproctext]{ref-bauer2020}
Bauer, Nichole M. 2020. {``{Shifting Standards: How Voters Evaluate the
Qualifications of Female and Male Candidates}.''} \emph{The Journal of
Politics} 82 (1): 1--12.

\bibitem[\citeproctext]{ref-beauregard2008}
Beauregard, Katrine. 2008. {``{L'intérêt politique chez les adolescents
selon les sexes}.''} Université de Montréal.

\bibitem[\citeproctext]{ref-beauregard2014}
---------. 2014. {``{Gender, Political Participation and Electoral
Systems: A Cross-National Analysis}.''} \emph{European Journal of
Political Research} 53 (3): 617--34.

\bibitem[\citeproctext]{ref-beauvais2020}
Beauvais, Edana. 2020. {``{The Gender Gap in Political Discussion Group
Attendance}.''} \emph{Politics \& Gender} 16 (2): 315--38.

\bibitem[\citeproctext]{ref-beckwith2010}
Beckwith, Karen. 2010. {``{A Comparative Politics of Gender Symposium
Introduction: Comparative Politics and the Logics of a Comparative
Politics of Gender}.''} \emph{Perspectives on Politics} 8 (1): 159--68.

\bibitem[\citeproctext]{ref-bell2009}
Bell, Edward, Julie Aitken Schermer, and Philip A Vernon. 2009. {``{The
Origins of Political Attitudes and Behaviours: An Analysis Using
Twins}.''} \emph{Canadian Journal of Political Science/Revue Canadienne
de Science Politique} 42 (4): 855--79.

\bibitem[\citeproctext]{ref-belli2001}
Belli, Robert F., Michael W. Traugott, and Matthew N. Beckmann. 2001.
{``{What Leads to Voting Overreports? Contrasts of Overreporters to
Validated Voters and Admitted Nonvoters in the American National
Election Studies}.''} \emph{Journal of Official Statistics} 17 (4): 479.

\bibitem[\citeproctext]{ref-bennett1989}
Bennett, Linda L. M., and Stephen Earl Bennett. 1989. {``{Enduring
Gender Differences in Political Interest: The Impact of Socialization
and Political Dispositions}.''} \emph{American Politics Quarterly} 17
(1): 105--22.

\bibitem[\citeproctext]{ref-berndt1982}
Berndt, Thomas J. 1982. {``{The Features and Effects of Friendship in
Early Adolescence}.''} \emph{Child Development} 53 (6): 1447--60.

\bibitem[\citeproctext]{ref-bhatti2019}
Bhatti, Yosef, Kasper M. Hansen, Elin Naurin, Dietlind Stolle, and Hanna
Wass. 2019. {``{Can you Deliver a Baby and Vote? The Effect of the First
Stages of Parenthood on Voter Turnout}.''} \emph{Journal of Elections,
Public Opinion and Parties} 29 (1): 61--81.

\bibitem[\citeproctext]{ref-blankenship1990}
Blankenship, Glen. 1990. {``{Classroom Climate, Global Knowledge, Global
Attitudes, Political Attitudes}.''} \emph{Theory \& Research in Social
Education} 18 (4): 363--86.

\bibitem[\citeproctext]{ref-bolduc2019}
Bolduc, Michel. 2023. {``Taille Des Classes : Quel Est Le Nombre
d'élèves Idéal?''} \emph{Radio-Canada}, October.
\url{https://ici.radio-canada.ca/nouvelle/1158160/education-politique-eleves-doug-ford-budget-compressions}.

\bibitem[\citeproctext]{ref-borkowska2024}
Borkowska, Magda, and Renee Luthra. 2024. {``{Socialization Disrupted:
The Intergenerational Transmission of Political Engagement in Immigrant
Families}.''} \emph{{International Migration Review}} 58 (1): 238--65.

\bibitem[\citeproctext]{ref-bos2022}
Bos, Angela L., Jill S. Greenlee, Mirya R. Holman, Zoe M. Oxley, and J.
Celeste Lay. 2022. {``This One's for the Boys: How Gendered Political
Socialization Limits Girls' Political Ambition and Interest.''}
\emph{American Political Science Review} 116 (2): 484--501.

\bibitem[\citeproctext]{ref-bos2020}
Bos, Angela L., Mirya R. Holman, Jill S. Greenlee, Zoe M. Oxley, and J.
Celeste Lay. 2020. {``{100 Years of Suffrage and Girls Still Struggle to
Find their "Fit"" in Politics}.''} \emph{PS: Political Science \&
Politics} 53 (3): 474--78.

\bibitem[\citeproctext]{ref-bourque1974}
Bourque, Susan C., and Jean Grossholtz. 1974. {``{Politics an Unnatural
Practice: Political Science Looks at Female Participation}.''}
\emph{Politics \& Society} 4 (2): 225--66.

\bibitem[\citeproctext]{ref-brownell2006}
Brownell, Celia A., Geetha B. Ramani, and Stephanie Zerwas. 2006.
{``{Becoming a Social Partner with Peers: Cooperation and Social
Understanding in One-and Two-Year-Olds}.''} \emph{Child Development} 77
(4): 803--21.

\bibitem[\citeproctext]{ref-buhlmann2012}
Bühlmann, Marc, and Lisa Schädel. 2012. {``{Representation Matters: The
Impact of Descriptive Women's Representation on the Political
Involvement of Women}.''} \emph{Representation} 48 (1): 101--14.

\bibitem[\citeproctext]{ref-burns2001}
Burns, Nancy, Kay Lehman Schlozman, and Sidney Verba. 2001. \emph{{The
Private Roots of Public Action}}. Harvard University Press.

\bibitem[\citeproctext]{ref-campbell1960}
Campbell, Angus, Philip Converse, Warren E. Miller, and Donald Stokes.
1960. \emph{{The American Voter}}. Chicago: The University of Chicago
Press.

\bibitem[\citeproctext]{ref-campbell2007}
Campbell, David E. 2007. {``{Sticking Together: Classroom Diversity and
Civic Education}.''} \emph{American Politics Research} 35 (1): 57--78.

\bibitem[\citeproctext]{ref-campbellwolbrecht2006}
Campbell, David E., and Christina Wolbrecht. 2006. {``{See Jane Run:
Women Politicians as Role Models for Adolescents}.''} \emph{The Journal
of Politics} 68 (2): 233--47.

\bibitem[\citeproctext]{ref-campbellrosie2008}
Campbell, Rosie, and Kristi Winters. 2008. {``{Understanding Men's and
Women's Political Interests: Evidence from a Study of Gendered Political
Attitudes}.''} \emph{Journal of Elections, Public Opinion and Parties}
18 (1): 53--74.

\bibitem[\citeproctext]{ref-caravita2012}
Caravita, Simona C. S., and Antonius H. N. Cillessen. 2012. {``{Agentic
or Communal? Associations between Interpersonal Goals, Popularity, and
Bullying in Middle Childhood and Early Adolescence}.''} \emph{Social
Development} 21 (2): 376--95.

\bibitem[\citeproctext]{ref-chattopadhyay2004}
Chattopadhyay, Raghabendra, and Esther Duflo. 2004. {``{Women as Policy
Makers: Evidence from a Randomized Policy Experiment in India}.''}
\emph{Econometrica} 72 (5): 1409--43.

\bibitem[\citeproctext]{ref-chung2003}
Chung, Janne, and Gary S. Monroe. 2003. {``{Exploring Social
Desirability Bias}.''} \emph{Journal of Business Ethics} 44 (4):
291--302.

\bibitem[\citeproctext]{ref-cicognani2012}
Cicognani, Elvira, Bruna Zani, Bernard Fournier, Claire Gavray, and
Michel Born. 2012. {``{Gender Differences in Youths' Political
Engagement and Participation. The Role of Parents and of Adolescents'
Social and Civic Participation}.''} \emph{Journal of Adolescence} 35
(3): 561--76.

\bibitem[\citeproctext]{ref-coffe2013}
Coffé, Hilde. 2013. {``{Women Stay Local, Men Go National and Global?
Gender Differences in Political Interest}.''} \emph{Sex Roles} 69 (5-6):
323--38.

\bibitem[\citeproctext]{ref-coffe2010}
Coffé, Hilde, and Catherine Bolzendahl. 2010. {``{Same Game, Different
Rules? Gender Differences in Political Participation}.''} \emph{Sex
Roles} 62 (5-6): 318--33.

\bibitem[\citeproctext]{ref-cohen1998}
Cohen, Jeffrey R., Laurie W. Pant, and David J. Sharp. 1998. {``{The
Effect of Gender and Academic Discipline Diversity on the Ethical
Evaluations, Ethical Intentions and Ethical Orientation of Potential
Public Accounting Recruits}.''} \emph{Accounting Horizons} 12 (3): 250.

\bibitem[\citeproctext]{ref-cohen2001}
---------. 2001. {``{An Examination of Differences in Ethical
Decision-Making Between Canadian Business Students and Accounting
Professionals}.''} \emph{Journal of Business Ethics} 30 (4): 319--36.

\bibitem[\citeproctext]{ref-collins2017}
Collins, Patricia Hill. 2017. {``{The Difference that Power Makes:
Intersectionality and Participatory Democracy}.''} \emph{Investigaciones
Feministas} 8 (1): 19--39.

\bibitem[\citeproctext]{ref-conant2019}
Conant, Eve. 2019. {``{The Best and Worst Countries to be a Woman}.''}
\emph{National Geographic}.
\url{/url\%7Bhttps://www.nationalgeographic.com/culture/2019/10/peril-progress-prosperity-womens-well-being-around-the-world-feature/\%7D}.

\bibitem[\citeproctext]{ref-conover2002}
Conover, Pamela Johnston, Donald D. Searing, and Ivor M. Crewe. 2002.
{``{The Deliberative Potential of Political Discussion}.''}
\emph{British Journal of Political Science} 32 (1): 21--62.

\bibitem[\citeproctext]{ref-craig1982}
Craig, Stephen C., and Michael A. Maggiotto. 1982. {``{Measuring
Political Efficacy}.''} \emph{Political Methodology} 8 (2): 85--109.

\bibitem[\citeproctext]{ref-dassonneville2021}
Dassonneville, Ruth, and Filip Kostelka. 2021. {``{The Cultural Sources
of the Gender Gap in Voter Turnout}.''} \emph{{British Journal of
Political Science}} 51 (3): 1040--61.

\bibitem[\citeproctext]{ref-dassonneville2012}
Dassonneville, Ruth, Ellen Quintelier, Marc Hooghe, and Ellen Claes.
2012. {``{The Relation Between Civic Education and Political Attitudes
and Behavior: A Two-Year Panel Study Among Belgian Late Adolescents}.''}
\emph{Applied Developmental Science} 16 (3): 140--50.

\bibitem[\citeproctext]{ref-dawes2014}
Dawes, Christopher, David Cesarini, James H. Fowler, Magnus Johannesson,
Patrik K. E. Magnusson, and Sven Oskarsson. 2014. {``{The Relationship
Between Genes, Psychological Traits, and Political Participation}.''}
\emph{American Journal of Political Science} 58 (4): 888--903.

\bibitem[\citeproctext]{ref-devroe2023}
Devroe, Robin, Hilde Coffé, Audrey Vandeleene, and Bram Wauters. 2023.
{``{Gender Gaps in Political Ambition on Different Levels of
Policy-Making}.''} \emph{Parliamentary Affairs} 76 (4): 924--46.

\bibitem[\citeproctext]{ref-dhima2022}
Dhima, Kostanca. 2022. {``{Do Elites Discriminate against Female
Political Aspirants? Evidence from a Field Experiment}.''}
\emph{Politics \& Gender} 18 (1): 126--57.

\bibitem[\citeproctext]{ref-vanditmars2023}
Ditmars, Mathilde M. van, and Aleksander Ksiazkiewicz. 2023. {``{The
Gender Gap in Political Interest: Heritability, Gendered Political
Socialization, and the Enriched Environment Hypothesis}.''}
\emph{Politics and the Life Sciences}, 1--15.

\bibitem[\citeproctext]{ref-dolan2011}
Dolan, Kathleen. 2011. {``{Do Women and Men Know Different Things?
Measuring Gender Differences in Political Knowledge}.''} \emph{The
Journal of Politics} 73 (1): 97--107.

\bibitem[\citeproctext]{ref-donato2008}
Donato, Katharine M., Chizuko Wakabayashi, Shirin Hakimzadeh, and Amada
Armenta. 2008. {``{Shifts in the Employment Conditions of Mexican
Migrant Men and Women: The Effect of US Immigration Policy}.''}
\emph{Work and Occupations} 35 (4): 462--95.

\bibitem[\citeproctext]{ref-dostiegoulet2009fr}
Dostie-Goulet, Eugénie. 2009a. {``{Le d{é}veloppement de l'int{é}r{ê}t
pour la politique chez les adolescents}.''} PhD thesis.

\bibitem[\citeproctext]{ref-dostiegoulet2009en}
---------. 2009b. {``{Social Networks and the Development of Political
Interest}.''} \emph{Journal of Youth Studies} 12 (4): 405--21.

\bibitem[\citeproctext]{ref-dowse1971}
Dowse, Robert E., and John A. Hughes. 1971. {``{Girls, Boys and
Politics}.''} \emph{The British Journal of Sociology} 22 (1): 53--67.

\bibitem[\citeproctext]{ref-easton1969}
Easton, David, Jack Dennis, and Sylvia Easton. 1969. \emph{{Children in
the Political System: Origins of Political Legitimacy}}. New York:
McGraw-Hill.

\bibitem[\citeproctext]{ref-cbc2022}
Easton, Rob. 2022. {``'Historic' Census Data Sheds Light on Number of
Trans and Non-Binary People for First Time.''} \emph{CBC}, July.
\url{https://www.cbc.ca/news/canada/calgary/census-data-trans-non-binary-statscan-1.6431928}.

\bibitem[\citeproctext]{ref-eckel2008}
Eckel, Catherine, Angela C. M. De Oliveira, and Philip J. Grossman.
2008. {``{Gender and Negotiation in the Small: Are Women (Perceived to
Be) More Cooperative than Men?}''} \emph{Negotiation Journal} 24 (4):
429--45.

\bibitem[\citeproctext]{ref-elo2010}
Elo, Kimmo, and Lauri Rapeli. 2010. {``{Determinants of Political
Knowledge: The Effects of the Media on Knowledge and Information}.''}
\emph{Journal of Elections, Public Opinion and Parties} 20 (1): 133--46.

\bibitem[\citeproctext]{ref-em22020}
Equal Measures 2030. 2020. {``{Harnessing the Power of Data for Gender
Quality: Introducing the 2019 EM2030 SDG Gender Index}.''}
\url{https://data.em2030.org/wp-content/uploads/2019/05/EM2030_2019_Global_Report_ENG.pdf}.

\bibitem[\citeproctext]{ref-fagot1985}
Fagot, Beverly I., Richard Hagan, Mary Driver Leinbach, and Sandra
Kronsberg. 1985. {``{Differential Reactions to Assertive and
Communicative Acts of Toddler Boys and Girls}.''} \emph{Child
Development} 56 (6): 1499--1505.

\bibitem[\citeproctext]{ref-fernandezgarcia2016}
Fernandez-Garcia, Núria. 2016. {``{Framing Gender and Women Politicians
Representation: Print Media Coverage of Spanish Women Ministers}.''} In
\emph{{Gender in Focus: (New) Trends in Media}}, edited by Carl
Cerqueira, Rosa Cabecinhas, and Sara Isabel Magalhaes, 141--60.
CECS-Publica{ç}{õ}es/eBooks.

\bibitem[\citeproctext]{ref-ferrin2018}
Ferrin, Monica, Marta Fraile, and Gema García-Albacete. 2018. {``{Is It
Simply Gender?: Content, Format, and Time in Political Knowledge
Measures}.''} \emph{Politics \& Gender} 14: 162--85.

\bibitem[\citeproctext]{ref-ferrin2020}
Ferrin, Monica, Marta Fraile, Gema M. Garcıa-Albacete, and Raul Gomez.
2020. {``{The Gender Gap in Political Interest Revisited}.''}
\emph{International Political Science Review} 41 (4): 473--89.

\bibitem[\citeproctext]{ref-ferrin2023}
Ferrı́n, Mónica, and Gema Garcı́a-Albacete. 2023. {``{Disinterested or
Enraged? Understanding People's Political Interest}.''} \emph{Acta
Politica}.

\bibitem[\citeproctext]{ref-fitzgerald2013}
Fitzgerald, Jennifer. 2013. {``{What Does "Political" Mean to You?}''}
\emph{Political Behavior} 35: 453--79.

\bibitem[\citeproctext]{ref-fox2004}
Fox, Richard L., and Jennifer L. Lawless. 2004. {``{Entering the Arena?
Gender and the Decision to Run for Office}.''} \emph{American Journal of
Political Science} 48 (2): 264--80.

\bibitem[\citeproctext]{ref-fox2005}
---------. 2005. {``{To Run or Not to Run for Office: Explaining Nascent
Political Ambition}.''} \emph{American Journal of Political Science} 49
(3): 642--59.

\bibitem[\citeproctext]{ref-fox2023}
---------. 2023. {``{The Invincible Gender Gap in Political
Ambition}.''} \emph{PS: Political Science \& Politics}, 1--7.

\bibitem[\citeproctext]{ref-fraile2014}
Fraile, Marta. 2014. {``{Do Women Know Less About Politics than Men? The
Gender Gap in Political Knowledge in Europe}.''} \emph{Social Politics}
21 (2): 261--89.

\bibitem[\citeproctext]{ref-fraile2017}
Fraile, Marta, and Raul Gomez. 2017. {``{Bridging the Enduring Gender
Gap in Political Interest in Europe: The Relevance of Promoting Gender
Equality}.''} \emph{European Journal of Political Research} 56 (3):
601--18.

\bibitem[\citeproctext]{ref-fraile2020}
Fraile, Marta, and Irene Sánchez-Vítores. 2020. {``{Tracing the Gender
Gap in Political Interest over the Life Span: A Panel Analysis}.''}
\emph{Political Psychology} 41 (1): 89--106.

\bibitem[\citeproctext]{ref-furman2022}
Furman, Aleksandra, Dagmara Szczepańska, and Dominika Maison. 2022.
{``{The Role of Family, Peers and School in Political Socialization:
Quantitative and Qualitative Study of Polish Young Adults'
Experiences}.''} \emph{Citizenship Teaching \& Learning} 17 (1):
123--43.

\bibitem[\citeproctext]{ref-gallie1956}
Gallie, Walter Bryce. 1956. {``{Essentially Contested Concepts}.''} In
\emph{{The Importance of Language}}, 167--98. Cornell University Press.

\bibitem[\citeproctext]{ref-geys2017}
Geys, Benny, and Daniel M. Smith. 2017. {``{Political Dynasties in
Democracies: Causes, Consequences and Remaining Puzzles}.''} \emph{The
Economic Journal} 127 (605): F446--54.

\bibitem[\citeproctext]{ref-gidengil2004}
Gidengil, Elisabeth, André Blais, Neil Nevitte, and Richard Nadeau.
2004. \emph{Citizens}. Vancouver: UBC Press.

\bibitem[\citeproctext]{ref-gidengil2008}
Gidengil, Elisabeth, Janine Giles, and Melanee Thomas. 2008. {``{The
Gender Gap in Self-Perceived Understanding of Politics in Canada and the
United States}.''} \emph{Politics \& Gender} 4 (4): 535--61.

\bibitem[\citeproctext]{ref-gidengil2005}
Gidengil, Elisabeth, Matthew Hennigar, André Blais, and Neil Nevitte.
2005. {``{Explaining the Gender Gap in Support for the New Right: The
Case of Canada}.''} \emph{Comparative Political Studies} 38 (10):
1171--95.

\bibitem[\citeproctext]{ref-gidengil2010}
Gidengil, Elisabeth, Brenda O'Neill, and Lisa Young. 2010. {``{Her
Mother's Daughter? The Influence of Childhood Socialization on Women's
Political Engagement}.''} \emph{Journal of Women, Politics \& Policy} 31
(4): 334--55.

\bibitem[\citeproctext]{ref-gidengil2016}
Gidengil, Elisabeth, Hanna Wass, and Maria Valaste. 2016. {``{Political
Socialization and Voting: The Parent--Child Link in Turnout}.''}
\emph{Political Research Quarterly} 69 (2): 373--83.

\bibitem[\citeproctext]{ref-golder2017}
Golder, Sona N., Laura B. Stephenson, Karine van der Straeten, André
Blais, Damien Bol, Philipp Harfst, and Jean-François Laslier. 2017.
{``{Votes for Women: Electoral Systems and Support for Female
Candidates}.''} \emph{Politics \& Gender} 13 (1): 107--31.

\bibitem[\citeproctext]{ref-goodyeargrant2013}
Goodyear-Grant, Elizabeth. 2013. \emph{{Gendered News: Media Coverage
and Electoral Politics in Canada}}. UBC Press.

\bibitem[\citeproctext]{ref-grechyna2023}
Grechyna, Daryna. 2023. {``{Parenthood and Political Engagement}.''}
\emph{European Journal of Political Economy} 76: 102238.

\bibitem[\citeproctext]{ref-greenstein1965}
Greenstein, Fred I. 1965. \emph{{Children and Politics}}. New Haven:
Yale University Press.

\bibitem[\citeproctext]{ref-wvs2022}
Haerpfer, Christian W., Ronald Inglehart, Alejandro Moreno, Christian
Welzel, Kseniya Kizilova, Jaime Diez-Medrano, M. Lagos, Pippa Norris,
Eduard Ponarin, and Bi Puranen. 2022. {``{World Values Survey Wave 7
(2017--2022) Cross-National Data-Set. Version: 4.0.0}.''} World Values
Survey Association.
\href{https://doi.org/10.14281/18241.18}{doi.org/10.14281/18241.18}.

\bibitem[\citeproctext]{ref-halunga2017}
Halunga, Andreea G., Chris D. Orme, and Takashi Yamagata. 2017. {``{A
Heteroskedasticity Robust Breusch--Pagan Test for Contemporaneous
Correlation in Dynamic Panel Data Models}.''} \emph{Journal of
Econometrics} 198 (2): 209--30.

\bibitem[\citeproctext]{ref-harff2022}
Harff, Darian. 2022. {``{Political Content from Virtual 'Friends': How
Influencers Arouse Young Women's Political Interest via Parasocial
Relationships}.''} \emph{The Journal of Social Media in Society} 11 (2):
97--121.

\bibitem[\citeproctext]{ref-harff2024}
Harff, Darian, and Desiree Schmuck. 2024. {``{Who Relies on Social Media
Influencers for Political Information? A Cross-Country Study Among
Youth}.''} \emph{The International Journal of Press/Politics},
19401612241234898.

\bibitem[\citeproctext]{ref-hatemi2012}
Hatemi, Peter K., Rose McDermott, J. Michael Bailey, and Nicholas G.
Martin. 2012. {``{The Different Effects of Gender and Sex on Vote
Choice}.''} \emph{Political Research Quarterly} 65 (1): 76--92.

\bibitem[\citeproctext]{ref-hayes1993}
Hayes, Bernadette C., and Clive S. Bean. 1993. {``{Gender and Local
Political Interest: Some International Comparisons}.''} \emph{Political
Studies} 41 (4): 672--82.

\bibitem[\citeproctext]{ref-herrick2019}
Herrick, Rebekah, Jeanette Morehouse Mendez, Ben Pryor, and James A.
Davis. 2019. {``{Surveying American Indians with Opt-In Internet
Surveys}.''} \emph{The American Indian Quarterly} 43 (3): 281--305.

\bibitem[\citeproctext]{ref-herrick2020}
Herrick, Rebekah, and Ben Pryor. 2020. {``{Gender and Race Gaps in
Voting and Over-Reporting: An Intersectional Comparison of CCES with
ANES Data}.''} \emph{The Social Science Journal}, 1--14.

\bibitem[\citeproctext]{ref-hess1967}
Hess, Robert D., and Judith V. Torney. 1967. \emph{{The Development of
Political Attitudes in Children}}. Routledge.

\bibitem[\citeproctext]{ref-heywood2019}
Heywood, Andrew. 2019. \emph{{Politics. Fifth Edition}}. Red Globe
Press.

\bibitem[\citeproctext]{ref-hochman2019}
Hochman, Oshrat, and Gema Garcı́a-Albacete. 2019. {``{Political Interest
Among European Youth With and Without an Immigrant Background}.''}
\emph{Social Inclusion} 7 (4): 257--78.

\bibitem[\citeproctext]{ref-holt2013}
Holt, Kristoffer, Adam Shehata, Jesper Strömbäck, and Elisabet
Ljungberg. 2013. {``{Age and the Effects of News Media Attention and
Social Media use on Political Interest and Participation: Do Social
Media Function as Leveller?}''} \emph{European Journal of Communication}
28 (1): 19--34.

\bibitem[\citeproctext]{ref-hooghe2022}
Hooghe, Marc. 2022. {``{Political Socialization}.''} In \emph{Handbook
on Politics and Public Opinion}, edited by Thomas Rudolph, 99--110.
Edward Elgar Publishing.

\bibitem[\citeproctext]{ref-hooghe2015}
Hooghe, Marc, and Joris Boonen. 2015. {``{The Intergenerational
Transmission of Voting Intentions in a Multiparty Setting: An Analysis
of Voting Intentions and Political Discussion Among 15-Year-Old
Adolescents and their Parents in Belgium}.''} \emph{Youth \& Society} 47
(1): 125--47.

\bibitem[\citeproctext]{ref-howe2010}
Howe, Paul. 2010. \emph{{Citizens Adrift: The Democratic Disengagement
of Young Canadians}}. Vancouver: UBC Press.

\bibitem[\citeproctext]{ref-huckfeldt1995}
Huckfeldt, Robert, and John Sprague. 1995. {``{Gender Effects on
Political Discussion: The Political Networks of Men and Women}.''} In
\emph{{Citizens, Politics and Social Communication: Information and
Influence in an Election Campaign}}. Cambridge University Press.

\bibitem[\citeproctext]{ref-hunter1985}
Hunter, Fumiyo T. 1985. {``{Adolescents' Perception of Discussions with
Parents and Friends}.''} \emph{Developmental Psychology} 21 (3):
433--40.

\bibitem[\citeproctext]{ref-hyman1959}
Hyman, Herbert H. 1959. \emph{{Political Socialization: A Study in the
Psychology of Political Behavior}}. Free Press.

\bibitem[\citeproctext]{ref-infante1996}
Infante, Dominic A., and Andrew S. Rancer. 1996. {``{Argumentativeness
and Verbal Aggressiveness: A Review of Recent Theory and Research}.''}
\emph{Annals of the International Communication Association} 19 (1):
319--52.

\bibitem[\citeproctext]{ref-inglehart1981}
Inglehart, Margaret L. 1981. {``{Political Interest in West European
Women: An Historical and Empirical Comparative Analysis}.''}
\emph{Comparative Political Studies} 14 (3): 299--326.

\bibitem[\citeproctext]{ref-inglehart2003}
Inglehart, Ronald, and Pippa Norris. 2003. \emph{{Rising Tide: Gender
Equality and Cultural Change Around the World}}. Cambridge University
Press.

\bibitem[\citeproctext]{ref-ipu2023}
Inter-Parliamentary Union Parline. 2023. {``{Monthly Ranking of Women in
National Parliaments}.''}
\url{https://data.ipu.org/women-ranking?month=4&year=2023}.

\bibitem[\citeproctext]{ref-janmaat2022}
Janmaat, Jan Germen, Bryony Hoskins, and Nicola Pensiero. 2022. {``{The
Development of Social and Gender Disparities in Political Engagement
During Adolescence and Early Adulthood: What Role Does Education
Play?}''}

\bibitem[\citeproctext]{ref-jennings1981}
Jennings, M. Kent, and Richard G. Niemi. 1981. \emph{{Generations and
Politics: A Panel Study of Young Adults and their Parents}}. Princeton
University Press.

\bibitem[\citeproctext]{ref-jennings2009}
Jennings, M. Kent, Laura Stoker, and Jake Bowers. 2009. {``{Politics
Across Generations: Family Transmission Reexamined}.''} \emph{The
Journal of Politics} 71 (3): 782--99.

\bibitem[\citeproctext]{ref-kanthak2015}
Kanthak, Kristin, and Jonathan Woon. 2015. {``{Women Don't Run? Election
Aversion and Candidate Entry}.''} \emph{American Journal of Political
Science} 59 (3): 595--612.

\bibitem[\citeproctext]{ref-karpowitz2014}
Karpowitz, Christopher F., and Tali Mendelberg. 2014. \emph{{The Silent
Sex}}. Princeton University Press.

\bibitem[\citeproctext]{ref-kay1987}
Kay, Barry J., Ronald D. Lambert, Steven D. Brown, and James E. Curtis.
1987. {``{Gender and Political Activity in Canada, 1965--1984}.''}
\emph{Canadian Journal of Political Science/Revue Canadienne de Science
Politique} 20 (4): 851--63.

\bibitem[\citeproctext]{ref-keeling2023}
Keeling, Silvia. 2023. {``{A Matter of Content: Overcoming the Gender
Gap in Political Knowledge, Expression of Knowledge, and Interest}.''}
\emph{Italian Political Science Review/Rivista Italiana Di Scienza
Politica} 53 (3): 384--98.

\bibitem[\citeproctext]{ref-kestilakekkonen2023}
Kestilä-Kekkonen, Elina, Josefina Sipinen, Lauri Rapeli, and Salla
Vadén. 2023. {``{The Role of Political Engagement of Parents in Reducing
the Gender Gap in Political Self-Efficacy}.''} \emph{European Journal of
Politics and Gender}, 1--23.

\bibitem[\citeproctext]{ref-kittilson2010}
Kittilson, Miki Caul, and Leslie Schwindt-Bayer. 2010. {``Engaging
Citizens: The Role of Power-Sharing Institutions.''} \emph{The Journal
of Politics} 72 (4): 990--1002.

\bibitem[\citeproctext]{ref-kittilson2012}
Kittilson, Miki Caul, and Leslie A. Schwindt-Bayer. 2012. \emph{{The
Gendered Effects of Electoral Institutions: Political Engagement and
Participation}}. Oxford University Press.

\bibitem[\citeproctext]{ref-klemmensen2012}
Klemmensen, Robert, Peter K. Hatemi, Sara B. Hobolt, Axel Skytthe, and
Asbjørn S. Nørgaard. 2012. {``{Heritability in Political Interest and
Efficacy Across Cultures: Denmark and the United States}.''} \emph{Twin
Research and Human Genetics} 15 (1): 15--20.

\bibitem[\citeproctext]{ref-klofstad2007}
Klofstad, Casey A. 2007. {``{Talk Leads to Recruitment: How Discussions
about Politics and Current Events Increase Civic Participation}.''}
\emph{Political Research Quarterly} 60 (2): 180--91.

\bibitem[\citeproctext]{ref-koskimaa2015}
Koskimaa, Vesa, and Lauri Rapeli. 2015. {``{Political Socialization and
Political Interest: The Role of School Reassessed}.''} \emph{Journal of
Political Science Education} 11 (2): 141--56.

\bibitem[\citeproctext]{ref-kostelka2019}
Kostelka, Filip, André Blais, and Elisabeth Gidengil. 2019. {``{Has the
Gender Gap in Voter Turnout Really Disappeared?}''} \emph{West European
Politics} 42 (3): 437--63.

\bibitem[\citeproctext]{ref-kray2001}
Kray, Laura J., Leigh Thompson, and Adam Galinsky. 2001. {``{Battle of
the Sexes: Gender Stereotype Confirmation and Reactance in
Negotiations}.''} \emph{Journal of Personality and Social Psychology} 80
(6): 942--58.

\bibitem[\citeproctext]{ref-kuhn2004}
Kuhn, Hans Peter. 2004. {``{Gender Differences in Political Interest
Among Adolescents from Brandenburg}.''} In \emph{{Democratic
Development?: East German, Israeli and Palestinian Adolescents}}, edited
by Hilke Rebenstorf. Springer Science \& Business Media.

\bibitem[\citeproctext]{ref-landemore2013}
Landemore, Hélène. 2013. \emph{{Democratic Reason: Politics, Collective
Intelligence, and the Rule of the Many}}. Princeton University Press.

\bibitem[\citeproctext]{ref-lane1962}
Lane, Robert Edwards. 1962. \emph{{Political Ideology: Why the American
Common Man Believes What He Does}}. New York: Free Press of Glencoe.

\bibitem[\citeproctext]{ref-langton1968}
Langton, Kenneth P., and M. Kent Jennings. 1968. {``{Political
Socialization and the High School Civics Curriculum in the United
States}.''} \emph{American Political Science Review} 62 (3): 852--67.

\bibitem[\citeproctext]{ref-laniado2016}
Laniado, David, Yana Volkovich, Karolin Kappler, and Andreas
Kaltenbrunner. 2016. {``{Gender Homophily in Online Dyadic and Triadic
Relationships}.''} \emph{EPJ Data Science} 5 (19): 1--23.

\bibitem[\citeproctext]{ref-lawless2010}
Lawless, Jennifer L., and Richard L. Fox. 2010. \emph{{It Still Takes a
Candidate: Why Women Don't Run for Office}}. Cambridge University Press.

\bibitem[\citeproctext]{ref-lawless2013}
---------. 2013. \emph{{Girls Just Wanna not Run: The Gender Gap in
Young Americans' Political Ambition}}. Washington, DC: Women \& Politics
Institute.

\bibitem[\citeproctext]{ref-lawless2015}
Lawless, Jennifer L., and Richard Logan Fox. 2015. \emph{{Running from
Office: Why Young Americans are Turned Off to Politics}}. Oxford
University Press, USA.

\bibitem[\citeproctext]{ref-datagotchi2022}
Leadership Chair in the Teaching of Digital Social Sciences. 2022.
{``{Datagotchi}.''} \url{https://datagotchi.com/}.

\bibitem[\citeproctext]{ref-datagotchi2023}
---------. 2023. {``{Datagotchi Post-Electoral Survey}.''}

\bibitem[\citeproctext]{ref-leaper2004}
Leaper, Campbell, and Tara E. Smith. 2004. {``{A Meta-Analytic Review of
Gender Variations in Children's Language Use: Talkativeness, Affiliative
Speech, and Assertive Speech}.''} \emph{Developmental Psychology} 40
(6): 993.

\bibitem[\citeproctext]{ref-levinsen2015}
Levinsen, Klaus, and Carsten Yndigegn. 2015. {``{Political Discussions
with Family and Friends: Exploring the Impact of Political Distance}.''}
\emph{The Sociological Review} 63 (S2): 72--91.

\bibitem[\citeproctext]{ref-lucas2021}
Lucas, Jack, Reed Merrill, Kelly Blidook, Sandra Breux, Laura Conrad,
Gabriel Eidelman, Royce Koop, Daniella Marciano, Zack Taylor, and Salomé
Vallette. 2021. {``{Women's Municipal Electoral Performance: An
Introduction to the Canadian Municipal Elections Database}.''}
\emph{Canadian Journal of Political Science/Revue Canadienne de Science
Politique} 54 (1): 125--33.

\bibitem[\citeproctext]{ref-lupia2005}
Lupia, Arthur, and Tasha S. Philpot. 2005. {``{Views from Inside the
Net: How Websites Affect Young Adults' Political Interest}.''} \emph{The
Journal of Politics} 67 (4): 1122--42.

\bibitem[\citeproctext]{ref-maheo2019}
Mahéo, Valérie-Anne. 2019. {``{Get-Out-The-Children's-Vote: A Field
Experiment On Families' Mobilization and Participation in the
Election}.''}

\bibitem[\citeproctext]{ref-mahony1985}
Mahony, Pat. 1985. \emph{{Schools for the Boys?: Co-Education
Reassessed}}. London: Hutchinson Publishing Group.

\bibitem[\citeproctext]{ref-matsuno2017}
Matsuno, Emmie, and Stephanie L. Budge. 2017. {``{Non-Binary/Genderqueer
Identities: A Critical Review of the Literature}.''} \emph{Current
Sexual Health Reports} 9 (3): 116--20.

\bibitem[\citeproctext]{ref-maurissen2018}
Maurissen, Lies, Ellen Claes, and Carolyn Barber. 2018. {``{Deliberation
in Citizenship Education: How the School Context Contributes to the
Development of an Open Classroom Climate}.''} \emph{Social Psychology of
Education} 21 (4): 951--72.

\bibitem[\citeproctext]{ref-mayer2004}
Mayer, Jeremy D., and Heather M. Schmidt. 2004. {``{Gendered Political
Socialization in Four Contexts: Political Interest and Values Among
Junior High School Students in China, Japan, Mexico, and the United
States}.''} \emph{The Social Science Journal} 41 (3): 393--407.

\bibitem[\citeproctext]{ref-mcdevitt2002}
McDevitt, Michael, and Steven Chaffee. 2002. {``{From Top-Down to
Trickle-Up Influence: Revisiting Assumptions about the Family in
Political Socialization}.''} \emph{Political Communication} 19 (3):
281--301.

\bibitem[\citeproctext]{ref-mcpherson2001}
McPherson, Miller, Lynn Smith-Lovin, and James M. Cook. 2001. {``{Birds
of a Feather: Homophily in Social Networks}.''} \emph{Annual Review of
Sociology} 27 (1): 415--44.

\bibitem[\citeproctext]{ref-mercier2012}
Mercier, Hugo, and Hélène Landemore. 2012. {``{Reasoning is for Arguing:
Understanding the Successes and Failures of Deliberation}.''}
\emph{Political Psychology} 33 (2): 243--58.

\bibitem[\citeproctext]{ref-mestre2012}
Mestre, Tània Verge, and Raül Tormos Marı́n. 2012. {``{The Persistence of
Gender Differences in Political Interest}.''} \emph{Revista Espa{ñ}ola
de Investigaciones Sociol{ó}gicas (REIS)} 138 (1): 1--19.

\bibitem[\citeproctext]{ref-motta2019}
Motta, Matthew. 2019. {``{Explaining Science Funding Attitudes in the
United States: The Case for Science Interest}.''} \emph{Public
Understanding of Science} 28 (2): 161--76.

\bibitem[\citeproctext]{ref-muxel2002}
Muxel, Anne. 2002. {``{La participation politique des jeunes:
soubresauts, fractures et ajustements}.''} \emph{Revue Fran{ç}aise de
Science Politique} 52 (5): 521--44.

\bibitem[\citeproctext]{ref-neundorf2016}
Neundorf, Anja, Richard G. Niemi, and Kaat Smets. 2016. {``{The
Compensation Effect of Civic Education on Political Engagement: How
Civics Classes Make Up for Missing Parental Socialization}.''}
\emph{Political Behavior} 38 (4): 921--49.

\bibitem[\citeproctext]{ref-neundorf2013}
Neundorf, Anja, Kaat Smets, and Gema M. Garcia-Albacete. 2013.
{``{Homemade Citizens: The Development of Political Interest During
Adolescence and Young Adulthood}.''} \emph{Acta Politica} 48 (1):
92--116.

\bibitem[\citeproctext]{ref-niederle2007}
Niederle, Muriel, and Lise Vesterlund. 2007. {``{Do Women Shy Away from
Competition? Do Men Compete too Much?}''} \emph{The Quarterly Journal of
Economics} 122 (3): 1067--1101.

\bibitem[\citeproctext]{ref-noakes2006}
Noakes, Melanie A., and Christina M. Rinaldi. 2006. {``{Age and Gender
Differences in Peer Conflict}.''} \emph{Journal of Youth and
Adolescence} 35: 881--91.

\bibitem[\citeproctext]{ref-noel2014}
Noël, Alain. 2014. {``{Studying Your Own Country: Social Scientific
Knowledge for Our Times and Places: Presidential Address to the Canadian
Political Science Association, St Catharines, May 28, 2014}.''}
\emph{Canadian Journal of Political Science/Revue Canadienne de Science
Politique} 47 (4): 647--66.

\bibitem[\citeproctext]{ref-noller1985}
Noller, Patricia, and Stephen Bagi. 1985. {``{Parent-Adolescent
Communication}.''} \emph{Journal of Adolescence} 8 (2): 125--44.

\bibitem[\citeproctext]{ref-norris2004}
Norris, Pippa, Joni Lovenduski, and Rosie Campbell. 2004. {``{Gender and
Political Participation}.''} London: The Electoral Commission.

\bibitem[\citeproctext]{ref-oneill2017}
O'Neill, Brenda, Elisabeth Gidengil, Melanee Thomas, and Amanda Bittner.
2017. {``{Motherhood's Role in Shaping Political and Civic
Participation}.''} In \emph{{Mothers and Others. The Role of Parenthood
in Politics}}, edited by Melanee Thomas and Amanda Bittner, 268--87. UBS
Press Vanconver.

\bibitem[\citeproctext]{ref-ondercin2011}
Ondercin, Heather L., and Daniel Jones-White. 2011. {``{Gender Jeopardy:
What is the Impact of Gender Differences in Political Knowledge on
Political Participation?}''} \emph{Social Science Quarterly} 92 (3):
675--94.

\bibitem[\citeproctext]{ref-oswald1998}
Oswald, Hans, and Christine Schmid. 1998. {``{Political Participation of
Young People in East Germany}.''} \emph{German Politics} 7 (3): 147--64.

\bibitem[\citeproctext]{ref-owen1988}
Owen, Diana, and Jack Dennis. 1988. {``{Gender Differences in the
Politicization of American Children}.''} \emph{Women \& Politics} 8 (2):
23--43.

\bibitem[\citeproctext]{ref-pap2018}
Pap, Ana, Marija Ham, and Karla Bilandžić. 2018. {``{Does Social Media
Usage Influence Youth's Interest in Politics?}''} \emph{International
Journal of Multidisciplinarity in Business and Science} 4 (5): 84--90.

\bibitem[\citeproctext]{ref-pasek2018}
Pasek, Josh. 2018. \emph{{anesrake: ANES Raking Implementation}}.
\url{https://CRAN.R-project.org/package=anesrake}.

\bibitem[\citeproctext]{ref-pattie2009}
Pattie, Charles J., and Ronald J. Johnston. 2009. {``{Conversation,
Disagreement and Political Participation}.''} \emph{Political Behavior}
31 (2): 261--85.

\bibitem[\citeproctext]{ref-pettey1988}
Pettey, Gary R. 1988. {``{The Interaction of the Individual's Social
Environment, Attention and Interest, and Public Affairs Media Use on
Political Knowledge Holding}.''} \emph{Communication Research} 15 (3):
265--81.

\bibitem[\citeproctext]{ref-phillips1992}
Phillips, Anne. 1992. {``{Must Feminists Give Up on Liberal
Democracy?}''} \emph{Political Studies} 40: 68--82.

\bibitem[\citeproctext]{ref-preece2015}
Preece, Jessica, and Olga Stoddard. 2015. {``{Why Women Don't Run:
Experimental Evidence on Gender Differences in Political Competition
Aversion}.''} \emph{Journal of Economic Behavior \& Organization} 117:
296--308.

\bibitem[\citeproctext]{ref-prior2010}
Prior, Markus. 2010. {``{You've Either Got It or you Don't? The
Stability of Political Interest over the Life Cycle}.''} \emph{The
Journal of Politics} 72 (3): 747--66.

\bibitem[\citeproctext]{ref-prior2019}
---------. 2019. \emph{{Hooked: How Politics Captures People's
Interest}}. Cambridge University Press.

\bibitem[\citeproctext]{ref-prior2018}
Prior, Markus, and Lori D. Bougher. 2018. {``{"Like They've Never, Ever
Seen in this Country"? Political Interest and Voter Engagement in
2016}.''} \emph{Public Opinion Quarterly} 82 (S1): 822--42.

\bibitem[\citeproctext]{ref-quintelier2014}
Quintelier, Ellen, and Jan W. Van Deth. 2014. {``{Supporting Democracy:
Political Participation and Political Attitudes. Exploring Causality
Using Panel Data}.''} \emph{Political Studies} 62 (1\_suppl): 153--71.

\bibitem[\citeproctext]{ref-rancer1985}
Rancer, Andrew S., and Kathi J. Dierks-Stewart. 1985. {``{The Influence
of Sex and Sex-Role Orientation on Trait Argumentativeness}.''}
\emph{Journal of Personality Assessment} 49 (1): 69--70.

\bibitem[\citeproctext]{ref-rasmussen2021}
Rasmussen, Stig Hebbelstrup Rye, Aaron Weinschenk, Chris Dawes, Jacob vB
Hjelmborg, and Robert Klemmensen. 2021. {``{Parental Transmission and
the Importance of the (Non-Causal) Effects of Education on Political
Engagement: Missing the Forest for the Trees}.''} PsyArXiv.

\bibitem[\citeproctext]{ref-rayment2020}
Rayment, Erica Jane. 2020. {``{Women in the House: The Impact of Elected
Women on Parliamentary Debate and Policymaking in Canada}.''} PhD
thesis, University of Toronto (Canada).

\bibitem[\citeproctext]{ref-rebenstorf2004chap9}
Rebenstorf, Hilke. 2004a. {``{Political Participation --- Its Meaning
and General Development}.''} In \emph{{Democratic Development?: East
German, Israeli and Palestinian Adolescents}}, edited by Hilke
Rebenstorf. Springer Science \& Business Media.

\bibitem[\citeproctext]{ref-rebenstorf2004political}
---------. 2004b. {``{Political Participation of Adolescents in
Brandenburg: The Significance of the Family Context}.''} In
\emph{{Democratic Development?: East German, Israeli and Palestinian
Adolescents}}, edited by Hilke Rebenstorf. Springer Science \& Business
Media.

\bibitem[\citeproctext]{ref-rheault2019}
Rheault, Ludovic, Erica Rayment, and Andreea Musulan. 2019.
{``{Politicians in the Line of Fire: Incivility and the Treatment of
Women on Social Media}.''} \emph{Research \& Politics} 6 (1): 1--7.

\bibitem[\citeproctext]{ref-rosenthal2003}
Rosenthal, Cindy Simon, Jocelyn Jones, and James A Rosenthal. 2003.
{``{Gendered Discourse in the Political Behavior of Adolescents}.''}
\emph{Political Research Quarterly} 56 (1): 97--104.

\bibitem[\citeproctext]{ref-russo2017}
Russo, Silvia, and Håkan Stattin. 2017. {``Stability and Change in
Youths' Political Interest.''} \emph{Social Indicators Research} 132:
643--58.

\bibitem[\citeproctext]{ref-sabella2004}
Sabella, Bernard. 2004a. {``{Gender Differences in Political Interest
Among Palestinian Youngsters in the West Bank}.''} In \emph{{Democratic
Development?: East German, Israeli and Palestinian Adolescents}}, edited
by Hilke Rebenstorf. Springer Science \& Business Media.

\bibitem[\citeproctext]{ref-sabella2004political}
---------. 2004b. {``{Political Participation of Adolescents in the West
Bank of the Palestinian Territories: The Significance of the Family
Context}.''} In \emph{{Democratic Development?: East German, Israeli and
Palestinian Adolescents}}, edited by Hilke Rebenstorf. Springer Science
\& Business Media.

\bibitem[\citeproctext]{ref-sanchezvitores2019}
Sánchez-Vítores, Irene. 2019. {``{Different Governments, Different
Interests: The Gender Gap in Political Interest}.''} \emph{{Social
Politics: International Studies in Gender, State \& Society}} 26 (3):
348--69.

\bibitem[\citeproctext]{ref-sanjuan2022}
Sanjuan, Renee, and Eleni M. Mantas. 2022. {``{The Effects of
Controversial Classroom Debates on Political Interest: An Experimental
Approach}.''} \emph{Journal of Political Science Education} 18 (3):
343--61.

\bibitem[\citeproctext]{ref-sapiro2013}
Sapiro, Virginia. 2013. {``{Gender, Social Capital, and Politics}.''} In
\emph{{Gender and Social Capital}}, edited by Brenda O'Neill and
Elisabeth Gidengil, 151--83. Routledge.

\bibitem[\citeproctext]{ref-schlozman1999}
Schlozman, Kay Lehman, Nancy Burns, and Sidney Verba. 1999. {``{"What
Happened at Work Today?": A Multistage Model of Gender, Employment, and
Political Participation}.''} \emph{The Journal of Politics} 61 (1):
29--53.

\bibitem[\citeproctext]{ref-schmuck2022}
Schmuck, Desirée, Melanie Hirsch, Anja Stevic, and Jörg Matthes. 2022.
{``{Politics -- Simply Explained? How Influencers Affect Youth's
Perceived Simplification of Politics, Political Cynicism, and Political
Interest}.''} \emph{The International Journal of Press/Politics} 27 (3):
738--62.

\bibitem[\citeproctext]{ref-schneider2017}
Schneider, Andrea Kupfer. 2017. {``{Negotiating While Female}.''}
\emph{SMU Law Review} 70 (3): 695--719.

\bibitem[\citeproctext]{ref-schneider2016}
Schneider, Monica C., Mirya R. Holman, Amanda B. Diekman, and Thomas
McAndrew. 2016. {``{Power, Conflict, and Community: How Gendered Views
of Political Power Influence Women's Political Ambition}.''}
\emph{Political Psychology} 37 (4): 515--31.

\bibitem[\citeproctext]{ref-schoon2011}
Schoon, Ingrid, and Helen Cheng. 2011. {``{Determinants of Political
Trust: A Lifetime Learning Model}.''} \emph{Developmental Psychology} 47
(3): 619--31.

\bibitem[\citeproctext]{ref-schwarz2022}
Schwarz, Susanne, and Alexander Coppock. 2022. {``{What Have We Learned
about Gender from Candidate Choice Experiments? A Meta-Analysis of
Sixty-Seven Factorial Survey Experiments}.''} \emph{The Journal of
Politics} 84 (2): 655--68.

\bibitem[\citeproctext]{ref-sevi2021}
Sevi, Semra. 2021. {``{Who Runs? Canadian Federal and Ontario Provincial
Candidates from 1867 to 2019}.''} \emph{Canadian Journal of Political
Science/Revue Canadienne de Science Politique} 54 (2): 471--76.

\bibitem[\citeproctext]{ref-sevi2019}
Sevi, Semra, Vincent Arel-Bundock, and André Blais. 2019. {``{Do Women
Get Fewer Votes? No.}''} \emph{Canadian Journal of Political
Science/Revue Canadienne de Science Politique} 52 (1): 201--10.

\bibitem[\citeproctext]{ref-sevi2023}
Sevi, Semra, and André Blais. 2023. {``{Are Women Election Averse?}''}
\emph{Electoral Studies} 86: 102712.

\bibitem[\citeproctext]{ref-shaw2002}
Shaw, Sylvia. 2002. {``{Language and Eender in Political Debates in the
House of Commons}.''} PhD thesis, Institute of Education, University of
London.

\bibitem[\citeproctext]{ref-shehata2019}
Shehata, Adam, and Erik Amnå. 2019. {``{The Development of Political
Interest Among Adolescents: A Communication Mediation Approach using
Five Waves of Panel Data}.''} \emph{Communication Research} 46 (8):
1055--77.

\bibitem[\citeproctext]{ref-shevlin2000}
Shevlin, Mark, JNV Miles, MNO Davies, and Stephanie Walker. 2000.
{``{Coefficient Alpha: A Useful Indicator of Reliability?}''}
\emph{Personality and Individual Differences} 28 (2): 229--37.

\bibitem[\citeproctext]{ref-shrum1988}
Shrum, Wesley, Neil H. Cheek Jr., and Saundra MacD. Hunter. 1988.
{``{Friendship in School: Gender and Racial Homophily}.''}
\emph{Sociology of Education} 61 (4): 227--39.

\bibitem[\citeproctext]{ref-shulman2014}
Shulman, Hillary C., and David C. DeAndrea. 2014. {``{Predicting
Success: Revisiting Assumptions About Family Political
Socialization}.''} \emph{Communication Monographs} 81 (3): 386--406.

\bibitem[\citeproctext]{ref-spence1978}
Spence, Janet T., and Robert L. Helmreich. 1978. \emph{{Masculinity and
Femininity: Their Psychological Dimensions, Correlates, and
Antecedents}}. University of Texas Press.

\bibitem[\citeproctext]{ref-spierings2012}
Spierings, Niels. 2012. {``{The Inclusion of Quantitative Techniques and
Diversity in the Mainstream of Feminist Research}.''} \emph{European
Journal of Women's Studies} 19 (3): 331--47.

\bibitem[\citeproctext]{ref-census2017}
Statistics Canada. 2017. {``{2016 Canadian Census of Population}.''}

\bibitem[\citeproctext]{ref-census2022}
---------. 2022. {``{2021 Canadian Census of Population}.''}

\bibitem[\citeproctext]{ref-statcan2023}
---------. 2023a. {``{Census Profile Downloads}.''}
\url{https://www12.statcan.gc.ca/census-recensement/2021/dp-pd/prof/details/download-telecharger.cfm?Lang=E}.

\bibitem[\citeproctext]{ref-gss2020}
---------. 2023b. {``{General Social Survey Cycle 35: Social Identity,
2020}.''} Abacus Data Network.
\url{https://hdl.handle.net/11272.1/AB2/R7HAAF}.

\bibitem[\citeproctext]{ref-stattin2022}
Stattin, Håkan, and Silvia Russo. 2022. {``{Youth's Own Political
Interest Can Explain their Political Interactions with Important
Others}.''} \emph{International Journal of Behavioral Development} 46
(4): 297--307.

\bibitem[\citeproctext]{ref-stehle2013}
Stehlé, Juliette, François Charbonnier, Tristan Picard, Ciro Cattuto,
and Alain Barrat. 2013. {``{Gender Homophily from Spatial Behavior in a
Primary School: A Sociometric Study}.''} \emph{Social Networks} 35 (4):
604--13.

\bibitem[\citeproctext]{ref-ces2019}
Stephenson, Laura B., Allison Harell, Daniel Rubenson, and Peter John
Loewen. 2020. {``{2019 Canadian Election Study --- Online Survey}.''}
Harvard Dataverse, V1. \url{https://doi.org/10.7910/DVN/DUS88V}.

\bibitem[\citeproctext]{ref-ces2021}
---------. 2022. {``{2021 Canadian Election Study (CES)}.''} Harvard
Dataverse, V1. \url{https://doi.org/10.7910/DVN/XBZHKC}.

\bibitem[\citeproctext]{ref-stockemer2023}
Stockemer, Daniel, and Aksel Sundstrom. 2023. {``{The Gender Gap in
Voter Turnout: An Artefact of Men's Over-Reporting in Survey
Research?}''} \emph{The British Journal of Politics and International
Relations} 25 (1): 21--41.

\bibitem[\citeproctext]{ref-stolle2010}
Stolle, Dietlind, and Elisabeth Gidengil. 2010. {``{What Do Women Really
Know? A Gendered Analysis of Varieties of Political Knowledge}.''}
\emph{Perspectives on Politics} 8 (1): 93--109.

\bibitem[\citeproctext]{ref-stolle2011}
Stolle, Dietlind, and Marc Hooghe. 2011. {``{Shifting Inequalities:
Patterns of Exclusion and Inclusion in Emerging Forms of Political
Participation}.''} \emph{European Societies} 13 (1): 119--42.

\bibitem[\citeproctext]{ref-tavakol2011}
Tavakol, Mohsen, and Reg Dennick. 2011. {``{Making Sense of Cronbach's
Alpha}.''} \emph{International Journal of Medical Education} 2: 53--55.

\bibitem[\citeproctext]{ref-thelwall2009}
Thelwall, Mike. 2009. {``{Homophily in MySpace}.''} \emph{Journal of the
American Society for Information Science and Technology} 60 (2):
219--31.

\bibitem[\citeproctext]{ref-themistokleous2016}
Themistokleous, Sotiris, and Lucy Avraamidou. 2016. {``{The Role of
Online Games in Promoting Young Adults' Civic Engagement}.''}
\emph{Educational Media International} 53 (1): 53--67.

\bibitem[\citeproctext]{ref-thielo2021}
Thielo, Angela J., Amanda Graham, and Francis T. Cullen. 2021. {``{The
Opt-In Internet Survey}.''} \emph{The Encyclopedia of Research Methods
in Criminology and Criminal Justice} 1: 274--79.

\bibitem[\citeproctext]{ref-thomas2013}
Thomas, Melanee. 2013. {``{Barriers to Women's Political Participation
in Canada}.''} \emph{University of New Brunswick Law Journal} 64 (1):
218--33.

\bibitem[\citeproctext]{ref-thomasbodet2013}
Thomas, Melanee, and Marc André Bodet. 2013. {``{Sacrificial Lambs,
Women Candidates, and District Competitiveness in Canada}.''}
\emph{Electoral Studies} 32 (1): 153--66.

\bibitem[\citeproctext]{ref-tilly2015}
Tilly, Charles, and Sidney G. Tarrow. 2015. \emph{{Contentious Politics.
Second Edition}}. Oxford University Press.

\bibitem[\citeproctext]{ref-tolley2011}
Tolley, Erin. 2011. {``{Do Women "Do Better" in Municipal Politics?
Electoral Representation Across Three Levels of Government}.''}
\emph{Canadian Journal of Political Science/Revue Canadienne de Science
Politique} 44 (3): 573--94.

\bibitem[\citeproctext]{ref-tolley2023}
---------. 2023. {``{Gender Is Not a Proxy: Race and Intersectionality
in Legislative Recruitment}.''} \emph{Politics \& Gender} 19 (2):
373--400.

\bibitem[\citeproctext]{ref-tolleybescosevi2022}
Tolley, Erin, Randy Besco, and Semra Sevi. 2022. {``{Who Controls the
Purse Strings? A Longitudinal Study of Gender and Donations in Canadian
Politics}.''} \emph{Politics \& Gender} 18 (1): 244--72.

\bibitem[\citeproctext]{ref-tormos2022}
Tormos, Raül, and Tània Verge. 2022. {``{Challenging the Gender Gap in
Political Interest: A By-Product of Survey Specification Error}.''}
\emph{Public Opinion Quarterly} 86 (1): 107--33.

\bibitem[\citeproctext]{ref-usnews2020}
US News \& World Report. 2020. {``{Monthly Ranking of Women in National
Parliaments}.''}
\url{https://www.usnews.com/news/best-countries/best-women}.

\bibitem[\citeproctext]{ref-uyanto2020}
Uyanto, Stanislaus S. 2020. {``{Power Comparisons of Five Most Commonly
Used Autocorrelation Tests}.''} \emph{Pakistan Journal of Statistics and
Operation Research} 16 (1): 119--30.

\bibitem[\citeproctext]{ref-vandeth2000}
Van Deth, J. 2000. {``{Political Interest and Apathy: The Decline of a
Gender Gap?}''} \emph{Acta Politica} 35 (3): 247--74.

\bibitem[\citeproctext]{ref-vandeth1990}
Van Deth, Jan W. 1990. \emph{Interest in Politics}. Edited by M. Kent
Jennings, Jan W. Van Deth, Samuel H. Barnes, Dieter Fuchs, Felix J.
Heunks, Ronald Inglehart, Max Kaase, Hans-Dieter Klingemann, and Jacques
J. A. Thomassen. Walter de Gruyter GmbH \& Co KG.

\bibitem[\citeproctext]{ref-verba1997}
Verba, Sidney, Nancy Burns, and Kay Lehman Schlozman. 1997. {``{Knowing
and Caring About Politics: Gender and Political Engagement}.''}
\emph{The Journal of Politics} 59 (4): 1051--72.

\bibitem[\citeproctext]{ref-vuchinich1987}
Vuchinich, Samuel. 1987. {``{Starting and Stopping Spontaneous Family
Conflicts}.''} \emph{Journal of Marriage and the Family} 49 (3):
591--601.

\bibitem[\citeproctext]{ref-walsh2004}
Walsh, Katherine Cramer. 2004. \emph{{Talking about Politics: Informal
Groups and Social Identity in American Life}}. Chicago: University of
Chicago Press.

\bibitem[\citeproctext]{ref-ward2006}
Ward, L. Charles, Beverly E. Thorn, Kristi L. Clements, Kim E. Dixon,
and Stacy D. Sanford. 2006. {``{Measurement of Agency, Communion, and
Emotional Vulnerability with the Personal Attributes Questionnaire}.''}
\emph{Journal of Personality Assessment} 86 (2): 206--16.

\bibitem[\citeproctext]{ref-wasike2023}
Wasike, Ben. 2023. {``{I Am an Influencer and I Approve This Message!
Examining How Political Social Media Influencers Affect Political
Interest, Political Trust, Political Efficacy, and Political
Participation}.''} \emph{International Journal of Communication} 17:
3110--32.

\bibitem[\citeproctext]{ref-weber1919}
Weber, Max. 1919. \emph{{Politics as a Vocation}}. Munich: Duncker \&
Humblot.

\bibitem[\citeproctext]{ref-weinschenk2017}
Weinschenk, Aaron C., and Christopher T. Dawes. 2017. {``{The
Relationship Between Genes, Personality Traits, and Political
Interest}.''} \emph{Political Research Quarterly} 70 (3): 467--79.

\bibitem[\citeproctext]{ref-weinschenk2019}
Weinschenk, Aaron C., Christopher T. Dawes, Christian Kandler, Edward
Bell, and Rainer Riemann. 2019. {``{New Evidence on the Link Between
Genes, Psychological Traits, and Political Engagement}.''}
\emph{Politics and the Life Sciences} 38 (1): 1--13.

\bibitem[\citeproctext]{ref-williams2010}
Williams, Brett, Andrys Onsman, and Ted Brown. 2010. {``{Exploratory
Factor Analysis: A Five-Step Guide for Novices}.''} \emph{Journal of
Emergency Primary Health Care} 8 (3): 1--13.

\bibitem[\citeproctext]{ref-wolbrecht2007}
Wolbrecht, Christina, and David E. Campbell. 2007. {``{Leading by
Example: Female Members of Parliament as Political Role Models}.''}
\emph{American Journal of Political Science} 51 (4): 921--39.

\bibitem[\citeproctext]{ref-york2019}
York, Chance. 2019. {``{Is it Top-Down, Trickle-Up, or Reciprocal?:
Testing Longitudinal Relationships Between Youth News Use and Parent and
Peer Political Discussion}.''} \emph{Communication Studies} 70 (4):
377--93.

\end{CSLReferences}

\cleardoublepage
\phantomsection
\addcontentsline{toc}{part}{Appendices}
\appendix

\chapter{CCPIS English Questionnaire}\label{sec-appendix1}

TITLE OF THE RESEARCH PROJECT: Gendered Political Socialization: Why
Women and Men Still Differ on Political Interest

MAIN RESEARCHER: Alexandre Fortier-Chouinard, PhD student in political
science at the University of Toronto

PROJECT BACKGROUND: PhD dissertation supervised by Professor Christopher
Brian Cochrane

INFORMATION ON THE PROJECT: This research project aims to answer the
following question: How do gender differences in interest in different
political topics emerge?

YOUR PARTICIPATION: Your participation in this research consists in
completing this questionnaire comprising 22 questions on political
interest, on yourself and on your social network. Although answering
each question is important for this research project, you are free to
leave some or all questions unanswered or to terminate your
participation at any time. Participation in this study will not be
evaluated by your teacher. However, data obtained from a participant who
chooses to withdraw from the project after submitting their
questionnaire cannot be destroyed. One optional question will ask for
your email address. If you decide to answer this question, which will
only be used to contact you for optional follow-up surveys in 5 and 10
years, you can send me an email at
\texttt{alexandre.fortier.chouinard@mail.utoronto.ca} to remove your
email address from the list at any moment.

DATA CONSERVATION: The research ethics program may have confidential
access to data to help ensure participant protection procedures are
followed. The data from your answers will be destroyed in August 2037.

THANKS: Your collaboration is central to this study. Therefore, we would
like to thank you for the time and attention you are willing to devote
by participating.

CERTIFICATION OF CONSENT: Simply sending the completed online survey
will be considered as an implied expression of your consent to
participate in the project.

ADDITIONAL INFORMATION: If you have questions about the research or the
implications of your participation, please contact Alexandre
Fortier-Chouinard (email:
\texttt{alexandre.fortier.chouinard@mail.utoronto.ca}) or the project's
faculty supervisor, Prof.~Christopher Brian Cochrane (email:
\texttt{christopher.cochrane@utoronto.ca}). If you have questions about
your rights as research participants, you can also contact the
University of Toronto's office of research ethics
(\texttt{ethics.review@utoronto.ca}, 416-946-3273). We are taking all
safety precautions to reduce the risk of spread of COVID-19.

\begin{verbatim}
- I accept to participate in this survey
- I do not accept to participate in this survey
\end{verbatim}

\begin{enumerate}
\def\labelenumi{\arabic{enumi}.}
\setcounter{enumi}{-1}
\item
  What 8-digit number have you been assigned?
\item
  When you think about politics, what kinds of things come to your mind?
  What does politics mean for you?

  \begin{itemize}
  \tightlist
  \item
    (Open field)
  \end{itemize}
\item
  How interested are you in politics generally? Set the slider to a
  number from 0 to 10, where 0 means no interest at all, and 10 means a
  great deal of interest.

  \begin{itemize}
  \tightlist
  \item
    (0--10 slider)
  \item
    Don't know/Prefer not to answer
  \end{itemize}
\item
  Among the following topics, please indicate which ones you think are
  political and which you think are not political.

  \begin{itemize}
  \tightlist
  \item
    3.1 Pandemic restrictions

    \begin{itemize}
    \tightlist
    \item
      Political
    \item
      Not political
    \item
      Don't know/Prefer not to answer
    \end{itemize}
  \item
    3.2 Working conditions of nurses

    \begin{itemize}
    \tightlist
    \item
      Political
    \item
      Not political
    \item
      Don't know/Prefer not to answer
    \end{itemize}
  \item
    3.3 Diplomatic disputes between Canada and China

    \begin{itemize}
    \tightlist
    \item
      Political
    \item
      Not political
    \item
      Don't know/Prefer not to answer
    \end{itemize}
  \item
    3.4 Ukrainian war

    \begin{itemize}
    \tightlist
    \item
      Political
    \item
      Not political
    \item
      Don't know/Prefer not to answer
    \end{itemize}
  \item
    3.5 Police funding

    \begin{itemize}
    \tightlist
    \item
      Political
    \item
      Not political
    \item
      Don't know/Prefer not to answer
    \end{itemize}
  \item
    3.6 Sentences for violent crimes

    \begin{itemize}
    \tightlist
    \item
      Political
    \item
      Not political
    \item
      Don't know/Prefer not to answer
    \end{itemize}
  \item
    3.7 University tuition

    \begin{itemize}
    \tightlist
    \item
      Political
    \item
      Not political
    \item
      Don't know/Prefer not to answer
    \end{itemize}
  \item
    3.8 Funding of public and private schools

    \begin{itemize}
    \tightlist
    \item
      Political
    \item
      Not political
    \item
      Don't know/Prefer not to answer
    \end{itemize}
  \item
    3.9 Federal elections

    \begin{itemize}
    \tightlist
    \item
      Political
    \item
      Not political
    \item
      Don't know/Prefer not to answer
    \end{itemize}
  \item
    3.10 Political parties

    \begin{itemize}
    \tightlist
    \item
      Political
    \item
      Not political
    \item
      Don't know/Prefer not to answer
    \end{itemize}
  \end{itemize}
\item
  If you were to open a news website and see the following articles how
  interested would you be in reading each article? Set the slider to a
  number from 0 to 10, where 0 means ``Not at all interested, I would
  not read it,'' and 10 means ``Very interested, I would most likely
  read it.''

  \begin{itemize}
  \tightlist
  \item
    4.1 Health care (i.e., pandemic restrictions, working conditions of
    nurses)

    \begin{itemize}
    \tightlist
    \item
      (0--10 slider)
    \item
      Don't know/Prefer not to answer
    \end{itemize}
  \item
    4.2 International affairs (i.e., diplomatic disputes between Canada
    and China, Ukrainian war)

    \begin{itemize}
    \tightlist
    \item
      (0--10 slider)
    \item
      Don't know/Prefer not to answer
    \end{itemize}
  \item
    4.3 Law and crime (i.e., police funding, sentences for violent
    crimes)

    \begin{itemize}
    \tightlist
    \item
      (0--10 slider)
    \item
      Don't know/Prefer not to answer
    \end{itemize}
  \item
    4.4 Education (i.e., university tuition, funding of public and
    private schools)

    \begin{itemize}
    \tightlist
    \item
      (0--10 slider)
    \item
      Don't know/Prefer not to answer
    \end{itemize}
  \item
    4.5 Partisan politics (i.e., federal elections, political parties)

    \begin{itemize}
    \tightlist
    \item
      (0--10 slider)
    \item
      Don't know/Prefer not to answer
    \end{itemize}
  \end{itemize}
\item
  Which of the following best describes your family situation,
  regardless of whether your biological parents live together or not?
\end{enumerate}

\begin{itemize}
\tightlist
\item
  One mother, one father and no stepparents
\item
  One mother, one father and at least one stepparent
\item
  One mother only {[}skip questions 6, 7 and 9{]}
\item
  One father only {[}skip questions 6--8{]}
\item
  Two mothers {[}skip questions 6, 7 and 9{]}
\item
  Two fathers {[}skip questions 6--8{]}
\item
  Other {[}skip questions 6--9{]}
\end{itemize}

\begin{enumerate}
\def\labelenumi{\arabic{enumi}.}
\setcounter{enumi}{5}
\tightlist
\item
  Which parent do you discuss most often with?
\end{enumerate}

\begin{itemize}
\tightlist
\item
  Mother
\item
  Father
\item
  Both equally
\item
  Don't know/Prefer not to answer
\end{itemize}

\begin{enumerate}
\def\labelenumi{\arabic{enumi}.}
\setcounter{enumi}{6}
\item
  For each of the following topics, which parent do you discuss most
  often with?

  \begin{itemize}
  \tightlist
  \item
    7.1 Health care

    \begin{itemize}
    \tightlist
    \item
      My mother
    \item
      My father
    \item
      Don't know/Prefer not to answer
    \end{itemize}
  \item
    7.2 International affairs

    \begin{itemize}
    \tightlist
    \item
      My mother
    \item
      My father
    \item
      Don't know/Prefer not to answer
    \end{itemize}
  \item
    7.3 Law and crime

    \begin{itemize}
    \tightlist
    \item
      My mother
    \item
      My father
    \item
      Don't know/Prefer not to answer
    \end{itemize}
  \item
    7.4 Education

    \begin{itemize}
    \tightlist
    \item
      My mother
    \item
      My father
    \item
      Don't know/Prefer not to answer
    \end{itemize}
  \item
    7.5 Partisan politics

    \begin{itemize}
    \tightlist
    \item
      My mother
    \item
      My father
    \item
      Don't know/Prefer not to answer
    \end{itemize}
  \end{itemize}
\item
  Among these five topics, which one do you discuss most often with your
  mother(s)?
\end{enumerate}

\begin{itemize}
\tightlist
\item
  Health care
\item
  International affairs
\item
  Law and crime
\item
  Education
\item
  Partisan politics
\item
  Don't know/Prefer not to answer
\end{itemize}

\begin{enumerate}
\def\labelenumi{\arabic{enumi}.}
\setcounter{enumi}{8}
\tightlist
\item
  Among these five topics, which one do you discuss most often with your
  father(s)?
\end{enumerate}

\begin{itemize}
\tightlist
\item
  Health care
\item
  International affairs
\item
  Law and crime
\item
  Education
\item
  Partisan politics
\item
  Don't know/Prefer not to answer
\end{itemize}

\begin{enumerate}
\def\labelenumi{\arabic{enumi}.}
\setcounter{enumi}{9}
\tightlist
\item
  What is the gender of most of your friends?
\end{enumerate}

\begin{itemize}
\tightlist
\item
  Girls
\item
  Boys
\item
  About the same for both genders
\item
  Don't know/Prefer not to answer
\end{itemize}

\begin{enumerate}
\def\labelenumi{\arabic{enumi}.}
\setcounter{enumi}{10}
\tightlist
\item
  Among these five topics, which one do you discuss most often with your
  male friends?
\end{enumerate}

\begin{itemize}
\tightlist
\item
  Health care
\item
  International affairs
\item
  Law and crime
\item
  Education
\item
  Partisan politics
\item
  Don't know/Prefer not to answer
\end{itemize}

\begin{enumerate}
\def\labelenumi{\arabic{enumi}.}
\setcounter{enumi}{11}
\tightlist
\item
  Among these five topics, which one do you discuss most often with your
  female friends?
\end{enumerate}

\begin{itemize}
\tightlist
\item
  Health care
\item
  International affairs
\item
  Law and crime
\item
  Education
\item
  Partisan politics
\item
  Don't know/Prefer not to answer
\end{itemize}

\begin{enumerate}
\def\labelenumi{\arabic{enumi}.}
\setcounter{enumi}{12}
\item
  Think about a teacher that you like(d).

  \begin{itemize}
  \tightlist
  \item
    13.1 Among these five topics, which one is (was) mentioned most
    often by this teacher?

    \begin{itemize}
    \tightlist
    \item
      Health care
    \item
      International affairs
    \item
      Law and crime
    \item
      Education
    \item
      Partisan politics
    \item
      Don't know/Prefer not to answer
    \end{itemize}
  \item
    13.2 Is that teacher\ldots{}

    \begin{itemize}
    \tightlist
    \item
      A woman
    \item
      A man
    \item
      Other (e.g.~Trans, non-binary, two-spirit, gender-queer)
    \end{itemize}
  \end{itemize}
\item
  Think about someone that you like and sometimes read or watch on
  social media --- including YouTube.

  \begin{itemize}
  \tightlist
  \item
    14.1 Among these five topics, which one is mentioned most often by
    this person?

    \begin{itemize}
    \tightlist
    \item
      Health care
    \item
      International affairs
    \item
      Law and crime
    \item
      Education
    \item
      Partisan politics
    \item
      Don't know/Prefer not to answer
    \end{itemize}
  \item
    14.2 Is that person\ldots{}

    \begin{itemize}
    \tightlist
    \item
      A man
    \item
      A woman
    \item
      Other (e.g.~Trans, non-binary, two-spirit, gender-queer)
    \end{itemize}
  \end{itemize}
\item
  Are you\ldots{}
\end{enumerate}

\begin{itemize}
\tightlist
\item
  A girl
\item
  A boy
\item
  Other (e.g.~Trans, non-binary, two-spirit, gender-queer)
\end{itemize}

\begin{enumerate}
\def\labelenumi{\arabic{enumi}.}
\setcounter{enumi}{15}
\tightlist
\item
  In what year were you born?
\end{enumerate}

\begin{itemize}
\tightlist
\item
  (All years from 1990 to 2021)
\end{itemize}

\begin{enumerate}
\def\labelenumi{\arabic{enumi}.}
\setcounter{enumi}{16}
\tightlist
\item
  Which language do you usually speak at home?
\end{enumerate}

\begin{itemize}
\tightlist
\item
  English
\item
  French
\item
  Aboriginal language (please specify \_\_\_\_)
\item
  Arabic
\item
  Chinese, Cantonese, Mandarin
\item
  Filipino / Tagalog
\item
  German
\item
  Indian, Hindi, Gujarati
\item
  Italian
\item
  Korean
\item
  Pakistani, Punjabi, Urdu
\item
  Persian, Farsi
\item
  Russian
\item
  Spanish
\item
  Tamil
\item
  Vietnamese
\item
  Other (please specify \_\_\_\_)
\item
  Don't know/Prefer not to answer
\end{itemize}

\begin{enumerate}
\def\labelenumi{\arabic{enumi}.}
\setcounter{enumi}{17}
\tightlist
\item
  Are you\ldots{}
\end{enumerate}

\begin{itemize}
\tightlist
\item
  First Nations (North American Indian), Métis or Inuk (Inuit)
\item
  White
\item
  South Asian (e.g., East Indian, Pakistani, Sri Lankan, etc.)
\item
  Chinese
\item
  Black
\item
  Filipino
\item
  Latin American
\item
  Arab
\item
  Southeast Asian (e.g., Vietnamese, Cambodian, Laotian, Thai, etc.)
\item
  West Asian (e.g., Iranian, Afghan, etc.)
\item
  Korean
\item
  Japanese
\item
  Other (please specify \_\_\_\_)
\item
  Don't know/Prefer not to answer
\end{itemize}

\begin{enumerate}
\def\labelenumi{\arabic{enumi}.}
\setcounter{enumi}{18}
\tightlist
\item
  Were you born in Canada?
\end{enumerate}

\begin{itemize}
\tightlist
\item
  Yes
\item
  No
\item
  Don't know/Prefer not to answer
\end{itemize}

\begin{enumerate}
\def\labelenumi{\arabic{enumi}.}
\setcounter{enumi}{19}
\item
  For each of these pairs of characteristics, indicate where you fall on
  a scale between both extremes.

  \begin{itemize}
  \tightlist
  \item
    20.1 Not at all independent - Very independent

    \begin{itemize}
    \tightlist
    \item
      (1--5 slider)
    \item
      Don't know/Prefer not to answer
    \end{itemize}
  \item
    20.2 Very passive - Very active

    \begin{itemize}
    \tightlist
    \item
      (1--5 slider)
    \item
      Don't know/Prefer not to answer
    \end{itemize}
  \item
    20.3 Not at all competitive - Very competitive

    \begin{itemize}
    \tightlist
    \item
      (1--5 slider)
    \item
      Don't know/Prefer not to answer
    \end{itemize}
  \item
    20.4 Can make decisions easily - Have difficulty making decisions

    \begin{itemize}
    \tightlist
    \item
      (1--5 slider)
    \item
      Don't know/Prefer not to answer
    \end{itemize}
  \item
    20.5 Give up very easily - Never give up easily

    \begin{itemize}
    \tightlist
    \item
      (1--5 slider)
    \item
      Don't know/Prefer not to answer
    \end{itemize}
  \item
    20.6 Not at all self-confident - Very self confident

    \begin{itemize}
    \tightlist
    \item
      (1--5 slider)
    \item
      Don't know/Prefer not to answer
    \end{itemize}
  \item
    20.7 Feel very inferior - Feel very superior

    \begin{itemize}
    \tightlist
    \item
      (1--5 slider)
    \item
      Don't know/Prefer not to answer
    \end{itemize}
  \item
    20.8 Go to pieces under pressure - Stand up well under pressure

    \begin{itemize}
    \tightlist
    \item
      (1--5 slider)
    \item
      Don't know/Prefer not to answer
    \end{itemize}
  \item
    20.9 Not at all emotional - Very emotional

    \begin{itemize}
    \tightlist
    \item
      (1--5 slider)
    \item
      Don't know/Prefer not to answer
    \end{itemize}
  \item
    20.10 Not at all able to devote self to others - Able to devote self
    completely to others

    \begin{itemize}
    \tightlist
    \item
      (1--5 slider)
    \item
      Don't know/Prefer not to answer
    \end{itemize}
  \item
    20.11 Very rough - Very gentle

    \begin{itemize}
    \tightlist
    \item
      (1--5 slider)
    \item
      Don't know/Prefer not to answer
    \end{itemize}
  \item
    20.12 Not at all helpful to others - Very helpful to others

    \begin{itemize}
    \tightlist
    \item
      (1--5 slider)
    \item
      Don't know/Prefer not to answer
    \end{itemize}
  \item
    20.13 Not at all kind - Very kind

    \begin{itemize}
    \tightlist
    \item
      (1--5 slider)
    \item
      Don't know/Prefer not to answer
    \end{itemize}
  \item
    20.14 Not at all aware of feelings of others - Very aware of
    feelings of others

    \begin{itemize}
    \tightlist
    \item
      (1--5 slider)
    \item
      Don't know/Prefer not to answer
    \end{itemize}
  \item
    20.15 Not at all understanding of others - Very understanding of
    others

    \begin{itemize}
    \tightlist
    \item
      (1--5 slider)
    \item
      Don't know/Prefer not to answer
    \end{itemize}
  \item
    20.16 Very cold in relations with others - Very warm in relations
    with others

    \begin{itemize}
    \tightlist
    \item
      (1--5 slider)
    \item
      Don't know/Prefer not to answer
    \end{itemize}
  \end{itemize}
\item
  We might do a follow-up survey in a few years. If you accept to be
  contacted by email for this survey, what is your current email
  address? Leave the field blank if you do not wish to be recontacted
  for this.
\end{enumerate}

\begin{itemize}
\tightlist
\item
  (Open field)
\end{itemize}

\chapter{CCPIS French Questionnaire}\label{sec-appendix2}

TITRE DE LA RECHERCHE : Socialisation politique genrée : Pourquoi les
femmes et les hommes diffèrent encore en matière d'intérêt politique

CHERCHEUR PRINCIPAL : Alexandre Fortier-Chouinard, étudiant au doctorat
en science politique à l'University of Toronto

CONTEXTE DU PROJET : Thèse de doctorat dirigée le professeur Christopher
Brian Cochrane

RENSEIGNEMENTS SUR LE PROJET : Ma thèse vise à répondre à la question
suivante : Comment émergent les différences de genre dans l'intérêt pour
différents sujets politiques?

VOTRE PARTICIPATION : Votre participation à cette recherche consistera à
remplir le présent questionnaire comprenant 22 questions portant sur
l'intérêt politique, sur vous-mêmes et sur votre entourage. Bien que les
réponses à chacune des questions soient importantes pour la recherche,
vous demeurez libre de choisir de ne pas répondre à l'une ou l'autre
d'entre elles ou encore de mettre fin à votre participation à tout
moment. La participation à cette étude ne sera pas évaluée par votre
enseignant. Toutefois, les données obtenues d'un(e) participant(e) qui
choisirait de se retirer du projet après avoir soumis son questionnaire
ne pourront être détruites. Une question facultative vous demandera
votre adresse courriel. Si vous décidez de répondre à cette question,
qui sera seulement utilisée pour vous contacter pour des sondages de
suivi facultatifs dans 5 et 10 ans, vous pouvez m'envoyer un courriel à
\texttt{alexandre.fortier.chouinard@mail.utoronto.ca} pour retirer votre
adresse courriel de la liste à n'importe quel moment. Une question vous
demandera votre numéro d'élève. Nous ne pouvons pas lier cette
information avec aucune information personnelle au sujet de l'élève,
incluant son nom, et cette information peut seulement être utilisée pour
lier des questionnaires soumis à différents moments dans le temps par le
même élève.

CONSERVATION DES DONNÉES : Le programme d'éthique de la recherche peut
avoir un accès confidentiel aux données pour aider à garantir le respect
des procédures de protection des participants. Les données issues de vos
réponses seront détruites en août 2037.

REMERCIEMENTS : Votre collaboration est précieuse pour nous permettre de
réaliser cette étude. C'est pourquoi nous tenons à vous remercier pour
le temps et l'attention que vous acceptez de consacrer à votre
participation.

ATTESTATION DU CONSENTEMENT : Le simple retour du sondage en ligne
rempli sera considéré comme l'expression implicite de votre consentement
à participer au projet.

RENSEIGNEMENTS SUPPLÉMENTAIRES: Si vous avez des questions sur la
recherche ou sur les implications de votre participation, veuillez
communiquer avec Alexandre Fortier-Chouinard (courriel :
\texttt{alexandre.fortier.chouinard@mail.utoronto.ca}) ou le superviseur
du projet, le professeur Christopher Brian Cochrane (courriel :
\texttt{christopher.cochrane@utoronto.ca}). Si vous avez des questions
sur vos droits en tant que participant(e) à la recherche, vous pouvez
également contacter le bureau d'éthique de la recherche de l'University
of Toronto (\texttt{ethics.review@utoronto.ca}, 416-946-3273). Nous
prenons toutes les précautions de sécurité pour réduire le risque de
propagation de la COVID-19.

\begin{verbatim}
- J'accepte de participer à cette recherche
- Je n'accepte pas de participer à cette recherche
\end{verbatim}

0 Quel numéro à 8 chiffres t'a-t-on attribué?

\begin{enumerate}
\def\labelenumi{\arabic{enumi}.}
\item
  Quand tu penses à la politique, quelles sont les choses qui te
  viennent à l'esprit? Que signifie la politique pour toi?

  \begin{itemize}
  \tightlist
  \item
    (Champ libre)
  \end{itemize}
\item
  Quel est ton intérêt pour la politique en général? Glisse la barre sur
  un chiffre de 0 à 10, où 0 indique aucun intérêt du tout et 10 indique
  beaucoup d'intérêt.

  \begin{itemize}
  \tightlist
  \item
    (barre de 0--10)
  \item
    Je ne sais pas/Préfère ne pas répondre
  \end{itemize}
\item
  Parmi les sujets suivants, indique ceux qui, selon toi, sont
  politiques et ceux qui ne le sont pas.

  \begin{itemize}
  \tightlist
  \item
    3.1 Restrictions en cas de pandémie

    \begin{itemize}
    \tightlist
    \item
      Politique
    \item
      Pas politique
    \item
      Je ne sais pas/Préfère ne pas répondre
    \end{itemize}
  \item
    3.2 Conditions de travail des infirmiers(ères)

    \begin{itemize}
    \tightlist
    \item
      Politique
    \item
      Pas politique
    \item
      Je ne sais pas/Préfère ne pas répondre
    \end{itemize}
  \item
    3.3 Conflits diplomatiques entre le Canada et la Chine

    \begin{itemize}
    \tightlist
    \item
      Politique
    \item
      Pas politique
    \item
      Je ne sais pas/Préfère ne pas répondre
    \end{itemize}
  \item
    3.4 Guerre en Ukraine

    \begin{itemize}
    \tightlist
    \item
      Politique
    \item
      Pas politique
    \item
      Je ne sais pas/Préfère ne pas répondre
    \end{itemize}
  \item
    3.5 Financement de la police

    \begin{itemize}
    \tightlist
    \item
      Politique
    \item
      Pas politique
    \item
      Je ne sais pas/Préfère ne pas répondre
    \end{itemize}
  \item
    3.6 Peines pour des crimes violents

    \begin{itemize}
    \tightlist
    \item
      Politique
    \item
      Pas politique
    \item
      Je ne sais pas/Préfère ne pas répondre
    \end{itemize}
  \item
    3.7 Frais de scolarité universitaires

    \begin{itemize}
    \tightlist
    \item
      Politique
    \item
      Pas politique
    \item
      Je ne sais pas/Préfère ne pas répondre
    \end{itemize}
  \item
    3.8 Financement des écoles publiques et privées

    \begin{itemize}
    \tightlist
    \item
      Politique
    \item
      Pas politique
    \item
      Je ne sais pas/Préfère ne pas répondre
    \end{itemize}
  \item
    3.9 Élections fédérales

    \begin{itemize}
    \tightlist
    \item
      Politique
    \item
      Pas politique
    \item
      Je ne sais pas/Préfère ne pas répondre
    \end{itemize}
  \item
    3.10 Partis politiques

    \begin{itemize}
    \tightlist
    \item
      Politique
    \item
      Pas politique
    \item
      Je ne sais pas/Préfère ne pas répondre
    \end{itemize}
  \end{itemize}
\item
  Si tu ouvrais un site Web d'information et que tu voyais les articles
  suivants, dans quelle mesure serais-tu intéressé par la lecture de
  chaque article? Déplace la barre vis-à-vis un nombre compris entre 0
  et 10, où 0 signifie \textless\textless~Pas du tout intéressé, je ne
  le lirai pas~\textgreater\textgreater{} et 10 signifie
  \textless\textless~Très intéressé, je le lirai très
  probablement~\textgreater\textgreater.

  \begin{itemize}
  \tightlist
  \item
    4.1 Santé (ex.: restrictions en cas de pandémie, conditions de
    travail des infirmiers(ères))

    \begin{itemize}
    \tightlist
    \item
      (barre de 0--10)
    \item
      Je ne sais pas/Préfère ne pas répondre
    \end{itemize}
  \item
    4.2 Affaires internationales (ex.: conflits diplomatiques entre le
    Canada et la Chine, guerre en Ukraine)

    \begin{itemize}
    \tightlist
    \item
      (barre de 0--10)
    \item
      Je ne sais pas/Préfère ne pas répondre
    \end{itemize}
  \item
    4.3 Loi et crime (ex.: financement de la police, peines pour des
    crimes violents)

    \begin{itemize}
    \tightlist
    \item
      (barre de 0--10)
    \item
      Je ne sais pas/Préfère ne pas répondre
    \end{itemize}
  \item
    4.4 Éducation (ex.: frais de scolarité universitaires, financement
    des écoles publiques et privées)

    \begin{itemize}
    \tightlist
    \item
      (barre de 0--10)
    \item
      Je ne sais pas/Préfère ne pas répondre
    \end{itemize}
  \item
    4.5 Politique partisane (ex.: élections fédérales, partis
    politiques)

    \begin{itemize}
    \tightlist
    \item
      (barre de 0--10)
    \item
      Je ne sais pas/Préfère ne pas répondre
    \end{itemize}
  \end{itemize}
\item
  Lequel des choix suivants décrit le mieux ta situation familiale, peu
  importe que tes parents biologiques vivent ensemble ou non?
\end{enumerate}

\begin{itemize}
\tightlist
\item
  Une mère, un père et aucun beau-parent
\item
  Une mère, un père et au moins un beau-parent
\item
  Une mère uniquement {[}sauter les questions 6, 7 et 9{]}
\item
  Un père uniquement {[}sauter les questions 6--8{]}
\item
  Deux mères {[}sauter les questions 6, 7 et 9{]}
\item
  Deux pères {[}sauter les questions 6--8{]}
\item
  Autre {[}sauter les questions 6--9{]}
\end{itemize}

\begin{enumerate}
\def\labelenumi{\arabic{enumi}.}
\setcounter{enumi}{5}
\tightlist
\item
  Avec quel parent discutes-tu le plus souvent?
\end{enumerate}

\begin{itemize}
\tightlist
\item
  Ma mère
\item
  Mon père
\item
  Les deux autant
\item
  Je ne sais pas/Préfère ne pas répondre
\end{itemize}

\begin{enumerate}
\def\labelenumi{\arabic{enumi}.}
\setcounter{enumi}{6}
\item
  Pour chacun des sujets suivants, avec quel parent discutes-tu le plus
  souvent?

  \begin{itemize}
  \tightlist
  \item
    7.1 Santé

    \begin{itemize}
    \tightlist
    \item
      Ma mère
    \item
      Mon père
    \item
      Je ne sais pas/Préfère ne pas répondre
    \end{itemize}
  \item
    7.2 Affaires internationales

    \begin{itemize}
    \tightlist
    \item
      Ma mère
    \item
      Mon père
    \item
      Je ne sais pas/Préfère ne pas répondre
    \end{itemize}
  \item
    7.3 Loi et crime

    \begin{itemize}
    \tightlist
    \item
      Ma mère
    \item
      Mon père
    \item
      Je ne sais pas/Préfère ne pas répondre
    \end{itemize}
  \item
    7.4 Éducation

    \begin{itemize}
    \tightlist
    \item
      Ma mère
    \item
      Mon père
    \item
      Je ne sais pas/Préfère ne pas répondre
    \end{itemize}
  \item
    7.5 Politique partisane

    \begin{itemize}
    \tightlist
    \item
      Ma mère
    \item
      Mon père
    \item
      Je ne sais pas/Préfère ne pas répondre
    \end{itemize}
  \end{itemize}
\item
  Parmi ces cinq sujets, lequel discutes-tu le plus souvent avec ta mère
  (tes mères)?
\end{enumerate}

\begin{itemize}
\tightlist
\item
  Santé
\item
  Affaires internationales
\item
  Loi et crime
\item
  Éducation
\item
  Politique partisane
\item
  Je ne sais pas/Préfère ne pas répondre
\end{itemize}

\begin{enumerate}
\def\labelenumi{\arabic{enumi}.}
\setcounter{enumi}{8}
\tightlist
\item
  Parmi ces cinq sujets, lequel discutes-tu le plus souvent avec ton
  père (tes pères)?
\end{enumerate}

\begin{itemize}
\tightlist
\item
  Santé
\item
  Affaires internationales
\item
  Loi et crime
\item
  Éducation
\item
  Politique partisane
\item
  Je ne sais pas/Préfère ne pas répondre
\end{itemize}

\begin{enumerate}
\def\labelenumi{\arabic{enumi}.}
\setcounter{enumi}{9}
\tightlist
\item
  Quel est le genre de la plupart de tes amis?
\end{enumerate}

\begin{itemize}
\tightlist
\item
  Filles
\item
  Garçons
\item
  Environ autant des deux genres
\item
  Je ne sais pas/Préfère ne pas répondre
\end{itemize}

\begin{enumerate}
\def\labelenumi{\arabic{enumi}.}
\setcounter{enumi}{10}
\tightlist
\item
  Parmi ces cinq sujets, lequel discutes-tu le plus souvent avec tes
  amis garçons?
\end{enumerate}

\begin{itemize}
\tightlist
\item
  Santé
\item
  Affaires internationales
\item
  Loi et crime
\item
  Éducation
\item
  Politique partisane
\item
  Je ne sais pas/Préfère ne pas répondre
\end{itemize}

\begin{enumerate}
\def\labelenumi{\arabic{enumi}.}
\setcounter{enumi}{11}
\tightlist
\item
  Parmi ces cinq sujets, lequel discutes-tu le plus souvent avec tes
  amies filles?
\end{enumerate}

\begin{itemize}
\tightlist
\item
  Santé
\item
  Affaires internationales
\item
  Loi et crime
\item
  Éducation
\item
  Politique partisane
\item
  Je ne sais pas/Préfère ne pas répondre
\end{itemize}

\begin{enumerate}
\def\labelenumi{\arabic{enumi}.}
\setcounter{enumi}{12}
\item
  Pense à un(e) enseignant(e) que tu apprécies (ou as apprécié).

  \begin{itemize}
  \tightlist
  \item
    13.1 Parmi ces cinq sujets, lequel est (était) mentionné le plus
    souvent par cet(te) enseignant(e)?

    \begin{itemize}
    \tightlist
    \item
      Santé
    \item
      Affaires internationales
    \item
      Loi et crime
    \item
      Éducation
    \item
      Politique partisane
    \item
      Je ne sais pas/Préfère ne pas répondre
    \end{itemize}
  \item
    13.2 Cet enseignant(e) est-il(elle)\ldots{}

    \begin{itemize}
    \tightlist
    \item
      Une femme
    \item
      Un homme
    \item
      Autre (ex.: trans, non-binaire, bispirituel, gender-queer)
    \end{itemize}
  \end{itemize}
\item
  Pense à une personne que tu apprécies et que tu lis ou regardes
  parfois sur les médias sociaux --- y compris YouTube.

  \begin{itemize}
  \tightlist
  \item
    14.1 Parmi ces cinq sujets, lequel est mentionné le plus souvent par
    cette personne?

    \begin{itemize}
    \tightlist
    \item
      Santé
    \item
      Affaires internationales
    \item
      Loi et crime
    \item
      Éducation
    \item
      Politique partisane
    \item
      Je ne sais pas/Préfère ne pas répondre
    \end{itemize}
  \item
    14.2 Cette personne est-elle\ldots{}

    \begin{itemize}
    \tightlist
    \item
      Une femme
    \item
      Un homme
    \item
      Autre (ex.: trans, non-binaire, bispirituel, gender-queer)
    \end{itemize}
  \end{itemize}
\item
  Es-tu\ldots{}
\end{enumerate}

\begin{itemize}
\tightlist
\item
  Une fille
\item
  Un garçon
\item
  Autre (ex.: trans, non-binaire, bispirituel, gender-queer)
\end{itemize}

\begin{enumerate}
\def\labelenumi{\arabic{enumi}.}
\setcounter{enumi}{15}
\tightlist
\item
  En quelle année es-tu né(e)?
\end{enumerate}

\begin{itemize}
\tightlist
\item
  (Toutes les années entre 1990 et 2021)
\end{itemize}

\begin{enumerate}
\def\labelenumi{\arabic{enumi}.}
\setcounter{enumi}{16}
\tightlist
\item
  Quelle langue parles-tu à la maison d'habitude?
\end{enumerate}

\begin{itemize}
\tightlist
\item
  Anglais
\item
  Français
\item
  Langue autochtone (veuillez préciser \_\_\_\_)
\item
  Arabe
\item
  Chinois, cantonais, mandarin
\item
  Philippin / tagalog
\item
  Allemand
\item
  Indien, Hindi, Gujarati
\item
  Italien
\item
  Coréen
\item
  Pakistanais, Pendjabi, Ourdou
\item
  Persan, farsi
\item
  Russe
\item
  Espagnol
\item
  Tamil
\item
  Vietnamien
\item
  Autre (veuillez spécifier \_\_\_\_)
\item
  Je ne sais pas/Préfère ne pas répondre
\end{itemize}

\begin{enumerate}
\def\labelenumi{\arabic{enumi}.}
\setcounter{enumi}{17}
\tightlist
\item
  Es-tu\ldots{}
\end{enumerate}

\begin{itemize}
\tightlist
\item
  Première Nation (Indien(ne) de l'Amérique du Nord), Métis(se) ou Inuk
  (Inuit)
\item
  Blanc(he)
\item
  Sud-Asiatique (ex.: Indien(ne) de l'Inde, Pakistanais(e),
  Sri-Lankais(e), etc.)
\item
  Chinois(e)
\item
  Noir(e)
\item
  Philippin(e)
\item
  Latino-Américain(e)
\item
  Arabe
\item
  Asiatique du Sud-Est (ex.: Vietnamien(ne), Cambodgien(ne),
  Laotien(ne), Thaïlandais(e), etc.)
\item
  Asiatique occidental(e) (e.g., Iranien(ne), Afghan(e), etc.)
\item
  Coréen(ne)
\item
  Japonais(e)
\item
  Autre (veuillez spécifier \_\_\_\_)
\item
  Je ne sais pas/Préfère ne pas répondre
\end{itemize}

\begin{enumerate}
\def\labelenumi{\arabic{enumi}.}
\setcounter{enumi}{18}
\tightlist
\item
  Es-tu né(e) au Canada?
\end{enumerate}

\begin{itemize}
\tightlist
\item
  Oui
\item
  Non
\item
  Je ne sais pas/Préfère ne pas répondre
\end{itemize}

\begin{enumerate}
\def\labelenumi{\arabic{enumi}.}
\setcounter{enumi}{19}
\item
  Pour chacune de ces paires de caractéristiques, indique où tu te
  situes sur une échelle entre les deux extrêmes.

  \begin{itemize}
  \tightlist
  \item
    20.1 Pas du tout indépendant(e) - Très indépendant(e)

    \begin{itemize}
    \tightlist
    \item
      (barre de 1--5)
    \item
      Je ne sais pas/Préfère ne pas répondre
    \end{itemize}
  \item
    20.2 Très passif(ive) - Très actif(ive)

    \begin{itemize}
    \tightlist
    \item
      (barre de 1--5)
    \item
      Je ne sais pas/Préfère ne pas répondre
    \end{itemize}
  \item
    20.3 Pas du tout compétitif(ive) - Très compétitif(ive)

    \begin{itemize}
    \tightlist
    \item
      (barre de 1--5)
    \item
      Je ne sais pas/Préfère ne pas répondre
    \end{itemize}
  \item
    20.4 Peut prendre des décisions facilement - A de la difficulté à
    prendre des décisions

    \begin{itemize}
    \tightlist
    \item
      (barre de 1--5)
    \item
      Je ne sais pas/Préfère ne pas répondre
    \end{itemize}
  \item
    20.5 Abandonne très facilement - N'abandonne jamais facilement

    \begin{itemize}
    \tightlist
    \item
      (barre de 1--5)
    \item
      Je ne sais pas/Préfère ne pas répondre
    \end{itemize}
  \item
    20.6 Pas du tout confiant(e) en soi - Très confiant(e) en soi

    \begin{itemize}
    \tightlist
    \item
      (barre de 1--5)
    \item
      Je ne sais pas/Préfère ne pas répondre
    \end{itemize}
  \item
    20.7 Se sent très inférieur(e) - Se sent très supérieur(e)

    \begin{itemize}
    \tightlist
    \item
      (barre de 1--5)
    \item
      Je ne sais pas/Préfère ne pas répondre
    \end{itemize}
  \item
    20.8 S'effondre sous la pression - Résiste bien à la pression

    \begin{itemize}
    \tightlist
    \item
      (barre de 1--5)
    \item
      Je ne sais pas/Préfère ne pas répondre
    \end{itemize}
  \item
    20.9 Pas du tout émotionnel(le) - Très émotionnel(le)

    \begin{itemize}
    \tightlist
    \item
      (barre de 1--5)
    \item
      Je ne sais pas/Préfère ne pas répondre
    \end{itemize}
  \item
    20.10 Pas du tout capable de se dévouer aux autres - Capable de se
    dévouer complètement aux autres

    \begin{itemize}
    \tightlist
    \item
      (barre de 1--5)
    \item
      Je ne sais pas/Préfère ne pas répondre
    \end{itemize}
  \item
    20.11 Très rude - Très doux (douce)

    \begin{itemize}
    \tightlist
    \item
      (barre de 1--5)
    \item
      Je ne sais pas/Préfère ne pas répondre
    \end{itemize}
  \item
    20.12 Pas très aidant(e) avec les autres - Très aidant(e) avec les
    autres

    \begin{itemize}
    \tightlist
    \item
      (barre de 1--5)
    \item
      Je ne sais pas/Préfère ne pas répondre
    \end{itemize}
  \item
    20.13 Pas très gentil(le) - Très gentil(le)

    \begin{itemize}
    \tightlist
    \item
      (barre de 1--5)
    \item
      Je ne sais pas/Préfère ne pas répondre
    \end{itemize}
  \item
    20.14 Pas du tout conscient(e) des sentiments des autres - Très
    conscient(e) des sentiments des autres

    \begin{itemize}
    \tightlist
    \item
      (barre de 1--5)
    \item
      Je ne sais pas/Préfère ne pas répondre
    \end{itemize}
  \item
    20.15 Pas du tout compréhensif(ive) des autres - Très
    compréhensif(ive) des autres

    \begin{itemize}
    \tightlist
    \item
      (barre de 1--5)
    \item
      Je ne sais pas/Préfère ne pas répondre
    \end{itemize}
  \item
    20.16 Très froid(e) dans les relations avec les autres - Très
    chaud(e) dans les relations avec les autres

    \begin{itemize}
    \tightlist
    \item
      (barre de 1--5)
    \item
      Je ne sais pas/Préfère ne pas répondre
    \end{itemize}
  \end{itemize}
\item
  Il est possible que nous faisions une enquête de suivi dans quelques
  années. Si tu acceptes d'être contacté(e) par courriel pour cette
  enquête, quelle est ton adresse courriel actuelle? Garde le champ vide
  si tu ne veux pas être recontacté(e) pour cela.
\end{enumerate}

\begin{itemize}
\tightlist
\item
  (Champ libre)
\end{itemize}

\chapter{2022 Quebec Datagotchi Post-Election Survey French
Questionnaire}\label{sec-appendix3}

\section{Socio-Economic Status}\label{socio-economic-status}

\begin{enumerate}
\def\labelenumi{\arabic{enumi}.}
\tightlist
\item
  Quel est votre genre?
\end{enumerate}

\begin{itemize}
\tightlist
\item
  Masculin
\item
  Féminin
\item
  Masculin (homme trans)
\item
  Féminin (femme trans)
\item
  Non-binaire
\item
  Queer
\item
  Agenre
\item
  Préfère ne pas répondre
\end{itemize}

\begin{enumerate}
\def\labelenumi{\arabic{enumi}.}
\setcounter{enumi}{1}
\tightlist
\item
  Parmi les appellations suivantes, laquelle décrit le mieux votre
  orientation sexuelle?
\end{enumerate}

\begin{itemize}
\tightlist
\item
  Hétérosexuel(le)
\item
  Gai ou lesbienne
\item
  Bisexuel(le)
\item
  Autre
\item
  Préfère ne pas répondre
\end{itemize}

\begin{enumerate}
\def\labelenumi{\arabic{enumi}.}
\setcounter{enumi}{2}
\tightlist
\item
  Quel âge avez-vous?
\end{enumerate}

Veuillez indiquer votre âge

\begin{enumerate}
\def\labelenumi{\arabic{enumi}.}
\setcounter{enumi}{3}
\tightlist
\item
  Laquelle des catégories suivantes vous décrit le mieux?
\end{enumerate}

\begin{itemize}
\tightlist
\item
  Blanc
\item
  Noir
\item
  Autochtone
\item
  Asiatique
\item
  Hispanique
\item
  Arabe
\item
  Autre
\item
  Préfère ne pas répondre
\end{itemize}

\begin{enumerate}
\def\labelenumi{\arabic{enumi}.}
\setcounter{enumi}{4}
\tightlist
\item
  Parmi les catégories suivantes, laquelle décrit le mieux votre domaine
  d'emploi?
\end{enumerate}

\begin{itemize}
\tightlist
\item
  Agriculteurs, bûcherons et pêcheurs
\item
  Propriétaires de magasins et d'usines
\item
  Professions libérales
\item
  Cadres et fonctionnaires
\item
  Cols blancs
\item
  Ouvriers
\item
  Vente et services
\item
  Au foyer
\item
  Étudiants et sans profession
\end{itemize}

\begin{enumerate}
\def\labelenumi{\arabic{enumi}.}
\setcounter{enumi}{5}
\tightlist
\item
  Approximativement, dans laquelle des catégories suivantes le revenu
  total de votre ménage, avant impôts, se situe-t-il?
\end{enumerate}

\begin{itemize}
\tightlist
\item
  Aucun revenu
\item
  1\$ à 30 000\$
\item
  30 001\$ à 60 000\$
\item
  60 001\$ à 90 000\$
\item
  90 001 à 110 000\$
\item
  110 001\$ à 150 000\$
\item
  150 001\$ à 200 000\$
\item
  Plus de 200 000\$
\end{itemize}

\begin{enumerate}
\def\labelenumi{\arabic{enumi}.}
\setcounter{enumi}{6}
\tightlist
\item
  Quel est votre plus haut niveau de scolarité complété?
\end{enumerate}

\begin{itemize}
\tightlist
\item
  Aucune scolarité
\item
  École primaire
\item
  École secondaire
\item
  Collège, CÉGEP ou Collège classique
\item
  Baccalauréat
\item
  Maîtrise
\item
  Doctorat
\end{itemize}

\begin{enumerate}
\def\labelenumi{\arabic{enumi}.}
\setcounter{enumi}{7}
\tightlist
\item
  Quelle est votre religion, si vous en avez une?
\end{enumerate}

\begin{itemize}
\tightlist
\item
  Aucune/Athée
\item
  Agnostique
\item
  Bouddhisme
\item
  Hindou
\item
  Judaïsme
\item
  Musulman
\item
  Sikhisme
\item
  Catholique
\item
  Protestantisme
\item
  Orthodoxe
\item
  Autre (veuillez préciser)
  \_\_\_\_\_\_\_\_\_\_\_\_\_\_\_\_\_\_\_\_\_\_\_\_\_\_\_\_\_\_\_\_\_\_\_\_\_\_\_\_\_\_\_\_\_\_\_\_\_\_
\item
  Préfère ne pas répondre
\end{itemize}

\begin{enumerate}
\def\labelenumi{\arabic{enumi}.}
\setcounter{enumi}{8}
\tightlist
\item
  Quel est votre degré d'attachement à cette église, dénomination
  religieuse ou communauté religieuse ?
\end{enumerate}

\begin{itemize}
\tightlist
\item
  Très faible
\item
  Plutôt faible
\item
  Modéré
\item
  Plutôt fort
\item
  Très fort
\end{itemize}

\begin{enumerate}
\def\labelenumi{\arabic{enumi}.}
\setcounter{enumi}{9}
\tightlist
\item
  Combien d'enfant ayant moins de 18 ans vivent avec vous?
\end{enumerate}

\begin{itemize}
\tightlist
\item
  0
\item
  1
\item
  2
\item
  3
\item
  4
\item
  5
\item
  Plus de 5
\end{itemize}

\begin{enumerate}
\def\labelenumi{\arabic{enumi}.}
\setcounter{enumi}{10}
\tightlist
\item
  Quel est votre statut matrimonial?
\end{enumerate}

\begin{itemize}
\tightlist
\item
  Célibataire
\item
  Marié(e)
\item
  Conjoint de fait
\item
  Veuf/veuve
\item
  Divorcé/séparé
\end{itemize}

\begin{longtable}[]{@{}l@{}}
\toprule\noalign{}
\endhead
\bottomrule\noalign{}
\endlastfoot
\end{longtable}

\begin{enumerate}
\def\labelenumi{\arabic{enumi}.}
\setcounter{enumi}{11}
\tightlist
\item
  Dans quel pays êtes-vous né(e)?
\end{enumerate}

\_\_\_\_\_\_\_\_\_\_\_\_\_\_\_\_\_\_\_\_\_\_\_\_\_\_\_\_\_\_\_\_\_\_\_\_\_\_\_\_\_\_\_\_\_\_\_\_\_\_\_\_\_\_\_\_\_\_\_\_\_\_\_\_

\begin{enumerate}
\def\labelenumi{\arabic{enumi}.}
\setcounter{enumi}{12}
\tightlist
\item
  Êtes-vous\ldots{}
\end{enumerate}

\begin{itemize}
\tightlist
\item
  En emploi à temps plein
\item
  En emploi à temps partiel
\item
  Étudiant
\item
  Retraité
\item
  Sans emploi
\end{itemize}

\begin{enumerate}
\def\labelenumi{\arabic{enumi}.}
\setcounter{enumi}{13}
\tightlist
\item
  Comment décririez-vous l'endroit où vous vivez?
\end{enumerate}

\begin{itemize}
\tightlist
\item
  Ville
\item
  Banlieu
\item
  Petite ville
\item
  Campagne/village
\end{itemize}

\begin{enumerate}
\def\labelenumi{\arabic{enumi}.}
\setcounter{enumi}{14}
\tightlist
\item
  Quel est votre secteur d'emploi
\end{enumerate}

\begin{itemize}
\tightlist
\item
  Secteur public
\item
  Secteur privé
\item
  Secteur associatif
\item
  Ne travaille pas dans un emploi formel
\end{itemize}

\section{Attitudes}\label{attitudes}

\begin{enumerate}
\def\labelenumi{\arabic{enumi}.}
\setcounter{enumi}{15}
\tightlist
\item
  Quel est l'enjeu le plus important pour vous, personnellement?
\end{enumerate}

\_\_\_\_\_\_\_\_\_\_\_\_\_\_\_\_\_\_\_\_\_\_\_\_\_\_\_\_\_\_\_\_\_\_\_\_\_\_\_\_\_\_\_\_\_\_\_\_\_\_\_\_\_\_\_\_\_\_\_\_\_\_\_\_

\begin{enumerate}
\def\labelenumi{\arabic{enumi}.}
\setcounter{enumi}{16}
\tightlist
\item
  Quel que soit le parti pour lequel vous avez l'intention de voter à
  l'occasion de la prochaine élection provinciale québécoise, en
  général, quelle est la probabilité que vous appuyiez {[}Sur une
  échelle de 0 à 10, où 0 signifie très peu probable, et 10 très
  probable{]} :
\end{enumerate}

\begin{longtable}[]{@{}ll@{}}
\toprule\noalign{}
Préfère ne pas répondre & \\
\midrule\noalign{}
\endhead
\bottomrule\noalign{}
\endlastfoot
\end{longtable}

\begin{longtable}[]{@{}llllllllllll@{}}
\toprule\noalign{}
0 & 1 & 2 & 3 & 4 & 5 & 6 & 7 & 8 & 9 & 10 & \\
\midrule\noalign{}
\endhead
\bottomrule\noalign{}
\endlastfoot
\end{longtable}

\begin{longtable}[]{@{}ll@{}}
\toprule\noalign{}
Coalition avenir Québec (CAQ) & \\
\midrule\noalign{}
\endhead
\bottomrule\noalign{}
\endlastfoot
Parti libéral du Québec (PLQ) & \\
Parti Québécois (PQ) & \\
Québec solidaire (QS) & \\
Parti conservateur du Québec (PCQ) & \\
\end{longtable}

\begin{enumerate}
\def\labelenumi{\arabic{enumi}.}
\setcounter{enumi}{17}
\tightlist
\item
  Quel que soit le parti pour lequel vous avez l'intention de voter à
  l'occasion de la prochaine élection fédérale canadienne, en général,
  quelle est la probabilité que vous appuyiez {[}Sur une échelle de 0 à
  10, où 0 signifie très peu probable, et 10 très probable{]} :
\end{enumerate}

\begin{longtable}[]{@{}ll@{}}
\toprule\noalign{}
Préfère ne pas répondre & \\
\midrule\noalign{}
\endhead
\bottomrule\noalign{}
\endlastfoot
\end{longtable}

\begin{longtable}[]{@{}llllllllllll@{}}
\toprule\noalign{}
0 & 1 & 2 & 3 & 4 & 5 & 6 & 7 & 8 & 9 & 10 & \\
\midrule\noalign{}
\endhead
\bottomrule\noalign{}
\endlastfoot
\end{longtable}

\begin{longtable}[]{@{}ll@{}}
\toprule\noalign{}
Parti libéral du Canada (PLC) & \\
\midrule\noalign{}
\endhead
\bottomrule\noalign{}
\endlastfoot
Parti conservateur du Canada (PCC) & \\
Nouveau Parti démocratique (NPD) & \\
Bloc Québécois (BQ) & \\
Parti vert du Canada (PV) & \\
\end{longtable}

\begin{enumerate}
\def\labelenumi{\arabic{enumi}.}
\setcounter{enumi}{18}
\tightlist
\item
  Lors d'une élection, certaines personnes ne peuvent pas voter parce
  qu'elles sont malades ou occupées, ou pour une autre raison. Avez-vous
  voté aux élections provinciales québécoises de 2022 ?
\end{enumerate}

\begin{itemize}
\tightlist
\item
  Oui
\item
  Non
\end{itemize}

\begin{enumerate}
\def\labelenumi{\arabic{enumi}.}
\setcounter{enumi}{19}
\tightlist
\item
  Pour quel parti avez-vous voté lors des élections provinciales
  québécoises de 2022?
\end{enumerate}

\begin{itemize}
\tightlist
\item
  Coalition Avenir Québec
\item
  Parti libéral du Québec
\item
  Parti Québécois
\item
  Québec Solidaire
\item
  Parti conservateur du Québec
\item
  Autre parti (veuillez spécifier)
  \_\_\_\_\_\_\_\_\_\_\_\_\_\_\_\_\_\_\_\_\_\_\_\_\_\_\_\_\_\_\_\_\_\_\_\_\_\_\_\_\_\_\_\_\_\_\_\_\_\_
\item
  Préfère ne pas répondre
\end{itemize}

\begin{enumerate}
\def\labelenumi{\arabic{enumi}.}
\setcounter{enumi}{20}
\tightlist
\item
  En politique, nous discutons parfois de ``gauche'' et ``droite''. Où
  est-ce que vous vous situerez sur l'échelle du placement
  gauche-droite? L'échelle va de 0 à 10 : 0 signifie que vous vous
  situez très à gauche, 10 signifie que vous vous situez très à droite
  (5 est le centre).
\end{enumerate}

\begin{longtable}[]{@{}llllllllllll@{}}
\toprule\noalign{}
0 & 1 & 2 & 3 & 4 & 5 & 6 & 7 & 8 & 9 & 10 & \\
\midrule\noalign{}
\endhead
\bottomrule\noalign{}
\endlastfoot
\end{longtable}

\begin{longtable}[]{@{}ll@{}}
\toprule\noalign{}
Veuillez déplacer le curseur & \\
\midrule\noalign{}
\endhead
\bottomrule\noalign{}
\endlastfoot
\end{longtable}

\begin{enumerate}
\def\labelenumi{\arabic{enumi}.}
\setcounter{enumi}{21}
\tightlist
\item
  En politique provinciale québécoise, vous considérez-vous
  habituellement comme étant\ldots{}
\end{enumerate}

\begin{itemize}
\tightlist
\item
  Caquiste (CAQ)
\item
  Libéral (PLQ)
\item
  Péquiste (PQ)
\item
  Solidaire (QS)
\item
  Conservateur (PCQ)
\item
  Vert (PVQ)
\item
  Un autre parti
\item
  Aucun de ces partis
\item
  Je ne sais pas/Préfère ne pas répondre
\end{itemize}

\begin{enumerate}
\def\labelenumi{\arabic{enumi}.}
\setcounter{enumi}{22}
\tightlist
\item
  En utilisant l'échelle ci-dessous, comment évaluez-vous votre niveau
  de connaissances politiques?
\end{enumerate}

\begin{itemize}
\tightlist
\item
  Très élevé
\item
  Élevé
\item
  Ni élevé ni faible
\item
  Faible
\item
  Très faible
\end{itemize}

\begin{enumerate}
\def\labelenumi{\arabic{enumi}.}
\setcounter{enumi}{23}
\tightlist
\item
  Quel est votre degré d'accord avec les énoncés suivants?
\end{enumerate}

\begin{longtable}[]{@{}
  >{\raggedright\arraybackslash}p{(\columnwidth - 10\tabcolsep) * \real{0.1667}}
  >{\raggedright\arraybackslash}p{(\columnwidth - 10\tabcolsep) * \real{0.1667}}
  >{\raggedright\arraybackslash}p{(\columnwidth - 10\tabcolsep) * \real{0.1667}}
  >{\raggedright\arraybackslash}p{(\columnwidth - 10\tabcolsep) * \real{0.1667}}
  >{\raggedright\arraybackslash}p{(\columnwidth - 10\tabcolsep) * \real{0.1667}}
  >{\raggedright\arraybackslash}p{(\columnwidth - 10\tabcolsep) * \real{0.1667}}@{}}
\toprule\noalign{}
\begin{minipage}[b]{\linewidth}\raggedright
Fortement en accord
\end{minipage} & \begin{minipage}[b]{\linewidth}\raggedright
Plutôt en accord
\end{minipage} & \begin{minipage}[b]{\linewidth}\raggedright
Ni en accord ni en désaccord
\end{minipage} & \begin{minipage}[b]{\linewidth}\raggedright
Plutôt en désaccord
\end{minipage} & \begin{minipage}[b]{\linewidth}\raggedright
Fortement en désaccord
\end{minipage} & \begin{minipage}[b]{\linewidth}\raggedright
\end{minipage} \\
\midrule\noalign{}
\endhead
\bottomrule\noalign{}
\endlastfoot
Est-ce que vous accepteriez que la Taxe de Vente du Québec augmente et
passe de 9.975\% à 10.575\% pour avoir accès à des soins dentaires et
psychologiques gratuits? & & & & & \\
\end{longtable}

\begin{itemize}
\item
\item
\item
\item
\item
  \begin{longtable}[]{@{}
    >{\raggedright\arraybackslash}p{(\columnwidth - 10\tabcolsep) * \real{0.1667}}
    >{\raggedright\arraybackslash}p{(\columnwidth - 10\tabcolsep) * \real{0.1667}}
    >{\raggedright\arraybackslash}p{(\columnwidth - 10\tabcolsep) * \real{0.1667}}
    >{\raggedright\arraybackslash}p{(\columnwidth - 10\tabcolsep) * \real{0.1667}}
    >{\raggedright\arraybackslash}p{(\columnwidth - 10\tabcolsep) * \real{0.1667}}
    >{\raggedright\arraybackslash}p{(\columnwidth - 10\tabcolsep) * \real{0.1667}}@{}}
  \toprule\noalign{}
  \endhead
  \bottomrule\noalign{}
  \endlastfoot
  Est-ce qu'il est légitime que le gouvernement impose à tous les
  travailleurs de payer 0.5\% de leur revenu (jusqu'à concurrence de
  430\$) pour s'assurer des deux parents d'un nouveau-né puisse recevoir
  70\% de son salaire pour rester à la maison avec son enfant pendant 50
  semaines. & & & & & \\
  \end{longtable}
\item
\item
\item
\item
\item
  \hfill\break
  Est-ce que vous accepteriez que le gouvernement diminue la Taxe de
  Vente du Québec pour qu'elle passe de 9.975\% à 9.575\%, mais que les
  frais de scolarité à l'université passent de 3000\$ à 7000\$ par
  année? \textbar{}
\item
\item
\item
\item
\item
\end{itemize}

\begin{enumerate}
\def\labelenumi{\arabic{enumi}.}
\setcounter{enumi}{24}
\tightlist
\item
  Si le gouvernement avait des surplus budgétaires, est-ce qu'il devrait
  surtout réduire les impôts ou surtout augmenter les dépenses en
  éducation ou en santé.
\end{enumerate}

\begin{itemize}
\tightlist
\item
  Surtout augmentant les dépenses en éducation ou en santé
\item
  Surtout réduisant les impôts
\end{itemize}

\begin{enumerate}
\def\labelenumi{\arabic{enumi}.}
\setcounter{enumi}{25}
\tightlist
\item
  Si le gouvernement devait réduire son déficit budgétaire, est-ce qu'il
  devrait le faire surtout en augmentant les impôts ou surtout en
  réduisant les dépenses en éducation ou en santé.
\end{enumerate}

\begin{itemize}
\tightlist
\item
  Surtout augmentant les impôts
\item
  Surtout réduisant les dépenses en éducation ou en santé
\end{itemize}

\begin{enumerate}
\def\labelenumi{\arabic{enumi}.}
\setcounter{enumi}{26}
\tightlist
\item
  Dans quelle mesure êtes-vous d'accord avec l'affirmation suivante :
  ``Les impôts sont déjà élevés. Le gouvernement ne devrait plus
  prélever davantage d'argent auprès des citoyens par le biais des
  impôts.''
\end{enumerate}

\begin{itemize}
\tightlist
\item
  Fortement en accord
\item
  Plutôt en accord
\item
  Plutôt en désaccord
\item
  Fortement en désaccord
\end{itemize}

\begin{enumerate}
\def\labelenumi{\arabic{enumi}.}
\setcounter{enumi}{27}
\tightlist
\item
  Les gens ont différentes façons de se définir. Comment diriez-vous que
  vous vous considérez?
\end{enumerate}

\begin{itemize}
\tightlist
\item
  Uniquement comme Canadien(ne)
\item
  D'abord comme Canadien(ne), puis comme Québécois(e)
\item
  Également comme Canadien(ne) puis comme Québécois(e)
\item
  D'abord comme Québécois(e)
\item
  Uniquement comme québécois(e)
\item
  Autre
\end{itemize}

\begin{enumerate}
\def\labelenumi{\arabic{enumi}.}
\setcounter{enumi}{28}
\tightlist
\item
  Le Québec devrait devenir un État indépendant.
\end{enumerate}

\begin{itemize}
\tightlist
\item
  Fortement en désaccord
\item
  Modérément en désaccord
\item
  Un peu en désaccord
\item
  Ni en accord ni en désaccord
\item
  Un peu en accord
\item
  Modérément en accord
\item
  Fortement en accord
\end{itemize}

\begin{enumerate}
\def\labelenumi{\arabic{enumi}.}
\setcounter{enumi}{29}
\tightlist
\item
  Est-ce que vous considérez que votre situation économique s'est
  améliorée, est restée la même, ou s'est détériorée pendant les 12
  derniers mois?
\end{enumerate}

\begin{itemize}
\tightlist
\item
  Améliorée
\item
  Restée la même
\item
  Détériorée
\end{itemize}

\begin{enumerate}
\def\labelenumi{\arabic{enumi}.}
\setcounter{enumi}{30}
\tightlist
\item
  D'une manière générale, diriez-vous que l'on peut faire confiance à la
  plupart des gens ou que l'on ne peut pas être trop prudent dans ses
  relations avec les autres? Veuillez utiliser cette échelle de 0 à 10,
  où 0 signifie que vous ne pouvez pas être trop prudent et 10 que vous
  pouvez faire confiance à la plupart des gens.
\end{enumerate}

\begin{longtable}[]{@{}llllllllllll@{}}
\toprule\noalign{}
0 & 1 & 2 & 3 & 4 & 5 & 6 & 7 & 8 & 9 & 10 & \\
\midrule\noalign{}
\endhead
\bottomrule\noalign{}
\endlastfoot
\end{longtable}

\begin{longtable}[]{@{}ll@{}}
\toprule\noalign{}
Déplacez le curseur & \\
\midrule\noalign{}
\endhead
\bottomrule\noalign{}
\endlastfoot
\end{longtable}

\begin{enumerate}
\def\labelenumi{\arabic{enumi}.}
\setcounter{enumi}{31}
\tightlist
\item
  Je considère que je reçois ma juste part des services publics,
  considérant les taxes et impôts que je paie.
\end{enumerate}

\begin{itemize}
\tightlist
\item
  Fortement en accord
\item
  Plutôt en accord
\item
  Plutôt en désaccord
\item
  Fortement en désaccord
\end{itemize}

\begin{enumerate}
\def\labelenumi{\arabic{enumi}.}
\setcounter{enumi}{32}
\tightlist
\item
  Qui est le ou la ministre des Finances du Canada?
\end{enumerate}

\begin{itemize}
\tightlist
\item
  Eric Girard
\item
  Chrystia Freeland
\item
  Steven Guilbault
\item
  Pierre Poilièvre
\item
  Mélanie Joly
\item
  Ne sait pas
\end{itemize}

\begin{enumerate}
\def\labelenumi{\arabic{enumi}.}
\setcounter{enumi}{33}
\tightlist
\item
  Pensez-vous que le Canada devrait admettre:
\end{enumerate}

\begin{itemize}
\tightlist
\item
  Plus d'immigrants
\item
  Moins d'immigrants
\item
  À peu près le même nombre d'immigrants
\end{itemize}

\begin{enumerate}
\def\labelenumi{\arabic{enumi}.}
\setcounter{enumi}{34}
\tightlist
\item
  Pensez-vous que le Canada devrait admettre:
\end{enumerate}

\begin{itemize}
\tightlist
\item
  Plus de réfugiés
\item
  Moins de réfugiés
\item
  À peu près le même nombre de réfugiés
\end{itemize}

\begin{enumerate}
\def\labelenumi{\arabic{enumi}.}
\setcounter{enumi}{35}
\tightlist
\item
  Le Canada devrait accepter les réfugiés provenant de pays aux prises
  avec des catastrophes écologiques.
\end{enumerate}

\begin{itemize}
\tightlist
\item
  Fortement en accord
\item
  Plutôt d'accord
\item
  Neutre
\item
  Plutôt en désaccord
\item
  Fortement en désaccord
\end{itemize}

\begin{enumerate}
\def\labelenumi{\arabic{enumi}.}
\setcounter{enumi}{36}
\tightlist
\item
  Nous devrions développer les pistes cyclables et le transport
  collectif, même si cela implique de réduire l'espace pour les
  voitures.
\end{enumerate}

\begin{itemize}
\tightlist
\item
  Fortement en accord
\item
  Plutôt d'accord
\item
  Neutre
\item
  Plutôt en désaccord
\item
  Fortement en désaccord
\end{itemize}

\begin{enumerate}
\def\labelenumi{\arabic{enumi}.}
\setcounter{enumi}{37}
\tightlist
\item
  Les changements climatiques constituent une menace pour moi au cours
  de ma vie.
\end{enumerate}

\begin{itemize}
\tightlist
\item
  Fortement en accord
\item
  Plutôt d'accord
\item
  Neutre
\item
  Plutôt en désaccord
\item
  Fortement en désaccord
\end{itemize}

\begin{enumerate}
\def\labelenumi{\arabic{enumi}.}
\setcounter{enumi}{38}
\tightlist
\item
  Si les choses continuent ainsi, nous connaîtrons bientôt une
  catastrophe écologique majeure.
\end{enumerate}

\begin{itemize}
\tightlist
\item
  Fortement en accord
\item
  Plutôt d'accord
\item
  Neutre
\item
  Plutôt en désaccord
\item
  Fortement en désaccord
\end{itemize}

\begin{enumerate}
\def\labelenumi{\arabic{enumi}.}
\setcounter{enumi}{39}
\tightlist
\item
  Les changements climatiques mèneront à la fin de l'humanité.
\end{enumerate}

\begin{itemize}
\tightlist
\item
  Fortement en accord
\item
  Plutôt d'accord
\item
  Neutre
\item
  Plutôt en désaccord
\item
  Fortement en désaccord
\end{itemize}

\begin{enumerate}
\def\labelenumi{\arabic{enumi}.}
\setcounter{enumi}{40}
\tightlist
\item
  Les citoyens mécontents du gouvernement ne doivent jamais recourir à
  la violence pour exprimer leurs sentiments
\end{enumerate}

\begin{itemize}
\tightlist
\item
  Fortement en accord
\item
  Plutôt d'accord
\item
  Neutre
\item
  Plutôt en désaccord
\item
  Fortement en désaccord
\end{itemize}

\begin{enumerate}
\def\labelenumi{\arabic{enumi}.}
\setcounter{enumi}{41}
\tightlist
\item
  À quel point seriez-vous prêt à tolérer les actions politiques
  suivantes pour l'avancement de la cause climatique?
\end{enumerate}

\begin{longtable}[]{@{}
  >{\raggedright\arraybackslash}p{(\columnwidth - 8\tabcolsep) * \real{0.2000}}
  >{\raggedright\arraybackslash}p{(\columnwidth - 8\tabcolsep) * \real{0.2000}}
  >{\raggedright\arraybackslash}p{(\columnwidth - 8\tabcolsep) * \real{0.2000}}
  >{\raggedright\arraybackslash}p{(\columnwidth - 8\tabcolsep) * \real{0.2000}}
  >{\raggedright\arraybackslash}p{(\columnwidth - 8\tabcolsep) * \real{0.2000}}@{}}
\toprule\noalign{}
\begin{minipage}[b]{\linewidth}\raggedright
Aucune tolérance
\end{minipage} & \begin{minipage}[b]{\linewidth}\raggedright
Tolérance faible
\end{minipage} & \begin{minipage}[b]{\linewidth}\raggedright
Tolérance moyenne
\end{minipage} & \begin{minipage}[b]{\linewidth}\raggedright
Tolérance élevée
\end{minipage} & \begin{minipage}[b]{\linewidth}\raggedright
\end{minipage} \\
\midrule\noalign{}
\endhead
\bottomrule\noalign{}
\endlastfoot
Signer une pétition & & & & \\
\end{longtable}

\begin{itemize}
\item
\item
\item
\item
  \begin{longtable}[]{@{}lllll@{}}
  \toprule\noalign{}
  \endhead
  \bottomrule\noalign{}
  \endlastfoot
  Boycotter des produits et compagnies & & & & \\
  \end{longtable}
\item
\item
\item
\item
  \hfill\break
  Désinvestir des placements \textbar{}
\item
\item
\item
\item
  \hfill\break
  Participer à une manifestation \textbar{}
\item
\item
\item
\item
  \hfill\break
  Occuper temporairement un espace public \textbar{}
\item
\item
\item
\item
  \hfill\break
  S'attacher à un arbre ou un véhicule \textbar{}
\item
\item
\item
\item
  \hfill\break
  Bloquer un pont ou une route \textbar{}
\item
\item
\item
\item
  \hfill\break
  Bloquer la construction d'un oléoduc (pipeline) \textbar{}
\item
\item
\item
\item
  \hfill\break
  Faire du vandalisme sur des objets \textbar{}
\item
\item
\item
\item
  \hfill\break
  Saboter des infrastructures, des véhicules, etc \textbar{}
\item
\item
\item
\item
  \hfill\break
  Tirer un objet sur des infrastructures, des véhicules, etc. \textbar{}
\item
\item
\item
\item
  \hfill\break
  Affronter des policiers dans une manifestation \textbar{}
\item
\item
\item
\item
  \hfill\break
  Violenter des individus en position de pouvoir \textbar{}
\item
\item
\item
\item
\end{itemize}

\begin{enumerate}
\def\labelenumi{\arabic{enumi}.}
\setcounter{enumi}{42}
\tightlist
\item
  Si vous ouvriez un site Web d'information et que vous voyiez les
  articles suivants, dans quelle mesure seriez-vous intéressé par la
  lecture de chaque article? Déplacez la barre vis-à-vis un nombre
  compris entre 0 et 10, où 0 signifie « Pas du tout intéressé, je ne le
  lirai pas » et 10 signifie « Très intéressé, je le lirai fort
  probablement »
\end{enumerate}

\begin{longtable}[]{@{}llllllllllll@{}}
\toprule\noalign{}
0 & 1 & 2 & 3 & 4 & 5 & 6 & 7 & 8 & 9 & 10 & \\
\midrule\noalign{}
\endhead
\bottomrule\noalign{}
\endlastfoot
\end{longtable}

\begin{longtable}[]{@{}
  >{\raggedright\arraybackslash}p{(\columnwidth - 2\tabcolsep) * \real{0.5000}}
  >{\raggedright\arraybackslash}p{(\columnwidth - 2\tabcolsep) * \real{0.5000}}@{}}
\toprule\noalign{}
\begin{minipage}[b]{\linewidth}\raggedright
Santé (ex.: restrictions en cas de pandémie, conditions de travail des
infirmiers(ères))
\end{minipage} & \begin{minipage}[b]{\linewidth}\raggedright
\end{minipage} \\
\midrule\noalign{}
\endhead
\bottomrule\noalign{}
\endlastfoot
Affaires internationales (ex.: conflits diplomatiques entre le Canada et
la Chine, guerre en Ukraine) & \\
Loi et crime (ex.: financement de la police, peines pour des crimes
violents) & \\
Éducation (ex.: frais de scolarité universitaires, financement des
écoles publiques et privées) & \\
Politique partisane (ex.: les élections fédérales, les partis
politiques) & \\
\end{longtable}

\begin{enumerate}
\def\labelenumi{\arabic{enumi}.}
\setcounter{enumi}{43}
\tightlist
\item
  Quel est votre intérêt pour la politique en général? Glissez la barre
  sur un chiffre de 0 à 10, où 0 indique aucun intérêt du tout et 10
  indique beaucoup d'intérêt.
\end{enumerate}

\begin{longtable}[]{@{}llllllllllll@{}}
\toprule\noalign{}
0 & 1 & 2 & 3 & 4 & 5 & 6 & 7 & 8 & 9 & 10 & \\
\midrule\noalign{}
\endhead
\bottomrule\noalign{}
\endlastfoot
\end{longtable}

\begin{longtable}[]{@{}ll@{}}
\toprule\noalign{}
Déplacez le curseur & \\
\midrule\noalign{}
\endhead
\bottomrule\noalign{}
\endlastfoot
\end{longtable}

\begin{enumerate}
\def\labelenumi{\arabic{enumi}.}
\setcounter{enumi}{44}
\tightlist
\item
  Certains pensent que le gouvernement à Ottawa devrait faire tout en
  son pouvoir pour améliorer la situation sociale et économique des
  personnes noires. D'autres estiment que le gouvernement ne devrait
  faire aucun effort particulier pour aider les personnes noires. Et,
  bien sûr, d'autres personnes ont des opinions quelque part entre les
  deux. Où vous situeriez-vous sur l'échelle suivante, où 1 signifie que
  le gouvernement devrait aider les personnes noires et 7 que le
  gouvernement ne devrait pas aider les personnes noires)?
\end{enumerate}

\begin{itemize}
\tightlist
\item
  1 (Le gouvernement devrait aider les personne noires)
\item
  2
\item
  3
\item
  4
\item
  5
\item
  6
\item
  7 (Le gouvernement ne devrait pas aider les personnes noires)
\end{itemize}

\begin{enumerate}
\def\labelenumi{\arabic{enumi}.}
\setcounter{enumi}{45}
\tightlist
\item
  Personnellement, êtes-vous en accord ou en désaccord avec l'imposition
  de quotas pour qu'il y ait plus de personnes noires en politique?
\end{enumerate}

\begin{itemize}
\tightlist
\item
  Fortement en désaccord
\item
  Plutôt en désaccord
\item
  Ni accord ni en désaccord
\item
  Plutôt d'accord
\item
  Fortement en accord
\end{itemize}

\begin{enumerate}
\def\labelenumi{\arabic{enumi}.}
\setcounter{enumi}{46}
\tightlist
\item
  Sur l'échelle suivante, où 1 correspond à « équitablement » et 7 à «
  inéquitablement », comment pensez-vous que la police traite les
  personnes noires par rapport aux personnes blanches au Canada?
\end{enumerate}

\begin{itemize}
\tightlist
\item
  1 (Équitablement)
\item
  2
\item
  3
\item
  4
\item
  5
\item
  6
\item
  7 (Inéquitablement)
\end{itemize}

\begin{enumerate}
\def\labelenumi{\arabic{enumi}.}
\setcounter{enumi}{47}
\tightlist
\item
  Selon vous, quel est le degré de discrimination au Canada aujourd'hui
  contre chacun des groupes suivants?
\end{enumerate}

\begin{longtable}[]{@{}llllll@{}}
\toprule\noalign{}
Aucune & Un peu & Une quantité modérée & Beaucoup & Énormément & \\
\midrule\noalign{}
\endhead
\bottomrule\noalign{}
\endlastfoot
Les noirs & & & & & \\
\end{longtable}

\begin{itemize}
\item
\item
\item
\item
\item
  \begin{longtable}[]{@{}llllll@{}}
  \toprule\noalign{}
  \endhead
  \bottomrule\noalign{}
  \endlastfoot
  Les asiatiques & & & & & \\
  \end{longtable}
\item
\item
\item
\item
\item
  \hfill\break
  Les blancs \textbar{}
\item
\item
\item
\item
\item
  \hfill\break
  Les autochtones \textbar{}
\item
\item
\item
\item
\item
  \hfill\break
  Les hispaniques \textbar{}
\item
\item
\item
\item
\item
  \hfill\break
  Les arabes \textbar{}
\item
\item
\item
\item
\item
\end{itemize}

\section{Perceptions}\label{perceptions}

\begin{enumerate}
\def\labelenumi{\arabic{enumi}.}
\setcounter{enumi}{48}
\tightlist
\item
  À votre avis, quel parti gérerait le mieux l'enjeu du coût de la vie?
\end{enumerate}

\begin{itemize}
\tightlist
\item
  CAQ
\item
  PLQ
\item
  PQ
\item
  QS
\item
  PCQ
\end{itemize}

\begin{enumerate}
\def\labelenumi{\arabic{enumi}.}
\setcounter{enumi}{49}
\tightlist
\item
  Que représente l'enjeu du coût de la vie pour vous, personnellement?
\end{enumerate}

\_\_\_\_\_\_\_\_\_\_\_\_\_\_\_\_\_\_\_\_\_\_\_\_\_\_\_\_\_\_\_\_\_\_\_\_\_\_\_\_\_\_\_\_\_\_\_\_\_\_\_\_\_\_\_\_\_\_\_\_\_\_\_\_

\begin{enumerate}
\def\labelenumi{\arabic{enumi}.}
\setcounter{enumi}{50}
\tightlist
\item
  Il serait possible d'accomplir mes activités quotidiennes sans
  voiture.
\end{enumerate}

\begin{itemize}
\tightlist
\item
  Fortement en accord
\item
  Plutôt d'accord
\item
  Neutre
\item
  Plutôt en désaccord
\item
  Fortement en désaccord
\end{itemize}

\section{Lifestyle}\label{lifestyle}

\begin{enumerate}
\def\labelenumi{\arabic{enumi}.}
\setcounter{enumi}{51}
\tightlist
\item
  En comparaison avec le reste de la société, comment évaluez-vous votre
  consommation d'alcool ?
\end{enumerate}

\begin{itemize}
\tightlist
\item
  Très en-dessous de la moyenne
\item
  En-dessous de la moyenne
\item
  Dans la moyenne
\item
  Au-dessus de la moyenne
\item
  Très au-dessus de la moyenne
\end{itemize}

\begin{enumerate}
\def\labelenumi{\arabic{enumi}.}
\setcounter{enumi}{52}
\tightlist
\item
  Comment percevez-vous les gens qui sont contre la consommation
  d'alcool? {[}Sur un thermomètre de sentiment allant de 0 (Très
  négatif) à 100 (Très positif){]}
\end{enumerate}

\begin{longtable}[]{@{}llllllllllll@{}}
\toprule\noalign{}
0 & 10 & 20 & 30 & 40 & 50 & 60 & 70 & 80 & 90 & 100 & \\
\midrule\noalign{}
\endhead
\bottomrule\noalign{}
\endlastfoot
\end{longtable}

\begin{longtable}[]{@{}ll@{}}
\toprule\noalign{}
Déplacez le curseur & \\
\midrule\noalign{}
\endhead
\bottomrule\noalign{}
\endlastfoot
\end{longtable}

\begin{longtable}[]{@{}l@{}}
\toprule\noalign{}
\endhead
\bottomrule\noalign{}
\endlastfoot
\end{longtable}

\begin{enumerate}
\def\labelenumi{\arabic{enumi}.}
\setcounter{enumi}{53}
\tightlist
\item
  Au cours des 12 derniers mois, à quelle fréquence avez-vous consommé
  des boissons alcoolisées ?
\end{enumerate}

\begin{itemize}
\tightlist
\item
  Moins d'une fois par mois
\item
  Une fois par mois
\item
  2 à 3 fois par mois
\item
  Une fois par semaine
\item
  2 à 3 fois par semaine
\item
  4 à 6 fois par semaine
\item
  Tous les jours
\item
  Préfère ne pas répondre
\item
  Ne sais pas
\end{itemize}

\begin{enumerate}
\def\labelenumi{\arabic{enumi}.}
\setcounter{enumi}{54}
\tightlist
\item
  Au cours du dernier mois, combien de fois avez-vous bu {[}5 pour les
  hommes /4 pour les femmes{]} verres ou plus d'alcool à une même
  occasion ?
\end{enumerate}

\begin{itemize}
\tightlist
\item
  Jamais
\item
  Moins d'une fois
\item
  Une fois
\item
  2 à 3 fois
\item
  Une fois par semaine
\item
  Plus d'une fois par semaine
\item
  Tous les jours
\item
  Préfère ne pas répondre
\item
  Ne sais pas
\end{itemize}

\begin{enumerate}
\def\labelenumi{\arabic{enumi}.}
\setcounter{enumi}{55}
\tightlist
\item
  À quelle fréquence lisez-vous des livres?
\end{enumerate}

\begin{itemize}
\tightlist
\item
  Jamais
\item
  Presque jamais
\item
  Parfois
\item
  Souvent
\item
  Très souvent
\end{itemize}

\begin{enumerate}
\def\labelenumi{\arabic{enumi}.}
\setcounter{enumi}{56}
\tightlist
\item
  Quel est votre livre préféré?
\end{enumerate}

\_\_\_\_\_\_\_\_\_\_\_\_\_\_\_\_\_\_\_\_\_\_\_\_\_\_\_\_\_\_\_\_\_\_\_\_\_\_\_\_\_\_\_\_\_\_\_\_\_\_\_\_\_\_\_\_\_\_\_\_\_\_\_\_

\begin{enumerate}
\def\labelenumi{\arabic{enumi}.}
\setcounter{enumi}{57}
\tightlist
\item
  Quel est votre genre de roman préféré?
\end{enumerate}

\begin{itemize}
\tightlist
\item
  Roman policier
\item
  Roman historique
\item
  Roman picaresque
\item
  Roman d'amour
\item
  Roman philosophique
\item
  Roman d'aventure
\item
  Autre (veuillez préciser)
  \_\_\_\_\_\_\_\_\_\_\_\_\_\_\_\_\_\_\_\_\_\_\_\_\_\_\_\_\_\_\_\_\_\_\_\_\_\_\_\_\_\_\_\_\_\_\_\_\_\_
\end{itemize}

\section{Other Opinions}\label{other-opinions}

\begin{enumerate}
\def\labelenumi{\arabic{enumi}.}
\setcounter{enumi}{58}
\tightlist
\item
  Le gouvernement doit réduire les inégalités de revenus entre les
  riches et les pauvres.
\end{enumerate}

\begin{itemize}
\tightlist
\item
  Fortement en accord
\item
  Plutôt en accord
\item
  Ni en accord ni en désaccord
\item
  Plutôt en désaccord
\item
  Fortement en désaccord
\end{itemize}

\begin{enumerate}
\def\labelenumi{\arabic{enumi}.}
\setcounter{enumi}{59}
\tightlist
\item
  Voici une liste de domaines où l'État intervient. Spécifiez si vous
  souhaitez que le gouvernement dépense plus ou moins dans chaque
  domaine. Soyez conscients que dépenser plus ou beaucoup plus peut
  occasionner une augmentation des taxes et impôts.
\end{enumerate}

\begin{longtable}[]{@{}
  >{\raggedright\arraybackslash}p{(\columnwidth - 10\tabcolsep) * \real{0.1667}}
  >{\raggedright\arraybackslash}p{(\columnwidth - 10\tabcolsep) * \real{0.1667}}
  >{\raggedright\arraybackslash}p{(\columnwidth - 10\tabcolsep) * \real{0.1667}}
  >{\raggedright\arraybackslash}p{(\columnwidth - 10\tabcolsep) * \real{0.1667}}
  >{\raggedright\arraybackslash}p{(\columnwidth - 10\tabcolsep) * \real{0.1667}}
  >{\raggedright\arraybackslash}p{(\columnwidth - 10\tabcolsep) * \real{0.1667}}@{}}
\toprule\noalign{}
\begin{minipage}[b]{\linewidth}\raggedright
Dépenser beaucoup plus
\end{minipage} & \begin{minipage}[b]{\linewidth}\raggedright
Dépenser plus
\end{minipage} & \begin{minipage}[b]{\linewidth}\raggedright
Dépenser le même montant qu'actuellement
\end{minipage} & \begin{minipage}[b]{\linewidth}\raggedright
Dépenser moins
\end{minipage} & \begin{minipage}[b]{\linewidth}\raggedright
Dépenser beaucoup moins
\end{minipage} & \begin{minipage}[b]{\linewidth}\raggedright
\end{minipage} \\
\midrule\noalign{}
\endhead
\bottomrule\noalign{}
\endlastfoot
Soins de santé & & & & & \\
\end{longtable}

\begin{itemize}
\item
\item
\item
\item
\item
  \begin{longtable}[]{@{}llllll@{}}
  \toprule\noalign{}
  \endhead
  \bottomrule\noalign{}
  \endlastfoot
  Soins à domicile pour les personnes âgées & & & & & \\
  \end{longtable}
\item
\item
\item
\item
\item
  \hfill\break
  Services de garde \textbar{}
\item
\item
\item
\item
\item
  \hfill\break
  Éducation \textbar{}
\item
\item
\item
\item
\item
  \hfill\break
  Les prestations aux personnes à faible revenu \textbar{}
\item
\item
\item
\item
\item
  \hfill\break
  Les prestations aux familles \textbar{}
\item
\item
\item
\item
\item
\end{itemize}

\begin{enumerate}
\def\labelenumi{\arabic{enumi}.}
\setcounter{enumi}{60}
\tightlist
\item
  Imaginez que le gouvernement ait les moyens d'augmenter certains
  programmes sociaux, mais pas tous. Parmi les améliorations suivantes
  des programmes sociaux, lesquels vous semblent les plus importantes ?
  Vous devez attribuer 100 points. Donnez plus de points aux
  améliorations que vous considérez comme plus importantes et moins de
  points à celles que vous considérez comme moins importantes. Le
  gouvernement devrait\ldots{}
\end{enumerate}

\begin{longtable}[]{@{}llllllllllll@{}}
\toprule\noalign{}
0 & 10 & 20 & 30 & 40 & 50 & 60 & 70 & 80 & 90 & 100 & \\
\midrule\noalign{}
\endhead
\bottomrule\noalign{}
\endlastfoot
\end{longtable}

\begin{longtable}[]{@{}ll@{}}
\toprule\noalign{}
Améliorer les soins à domicile pour les personnes âgées & \\
\midrule\noalign{}
\endhead
\bottomrule\noalign{}
\endlastfoot
Augmenter les prestations offertes aux familles & \\
Augmenter le nombre de places dans les services de garde & \\
Améliorer la qualité du réseau d'éducation & \\
Améliorer l'accès au réseau de la santé & \\
Améliorer la lutte à la pauvreté & \\
\end{longtable}

\begin{enumerate}
\def\labelenumi{\arabic{enumi}.}
\setcounter{enumi}{61}
\tightlist
\item
  Selon vous, quel est le meilleur\ldots{}
\end{enumerate}

\begin{longtable}[]{@{}
  >{\raggedright\arraybackslash}p{(\columnwidth - 4\tabcolsep) * \real{0.3333}}
  >{\raggedright\arraybackslash}p{(\columnwidth - 4\tabcolsep) * \real{0.3333}}
  >{\raggedright\arraybackslash}p{(\columnwidth - 4\tabcolsep) * \real{0.3333}}@{}}
\toprule\noalign{}
\begin{minipage}[b]{\linewidth}\raggedright
Veuillez inscrire votre réponse dans l'encadré.
\end{minipage} & \begin{minipage}[b]{\linewidth}\raggedright
Cochez la case si vous n'en connaissez aucun.
\end{minipage} & \begin{minipage}[b]{\linewidth}\raggedright
\end{minipage} \\
\midrule\noalign{}
\endhead
\bottomrule\noalign{}
\endlastfoot
& & \\
Question ouverte & Je n'en connais aucun & \\
Livre canadien de tous les temps? & & \\
& & \\
\end{longtable}

\begin{itemize}
\item
  \hfill\break
  Album de musique canadien de tous les temps? \textbar{}\\
\item
  \hfill\break
  Film canadien de tous les temps? \textbar{}\\
\item
  \hfill\break
  Journal quotidien canadien de tous les temps? \textbar{}\\
\item
\end{itemize}

\begin{enumerate}
\def\labelenumi{\arabic{enumi}.}
\setcounter{enumi}{62}
\tightlist
\item
  Nous souhaitons savoir si les gens connaissent mieux certains types de
  personnalités publiques que d'autres au Canada. Sans consulter
  d'autres sources, veuillez nommer une personnalité publique
  francophone canadienne travaillant:
\end{enumerate}

\begin{longtable}[]{@{}
  >{\raggedright\arraybackslash}p{(\columnwidth - 4\tabcolsep) * \real{0.3333}}
  >{\raggedright\arraybackslash}p{(\columnwidth - 4\tabcolsep) * \real{0.3333}}
  >{\raggedright\arraybackslash}p{(\columnwidth - 4\tabcolsep) * \real{0.3333}}@{}}
\toprule\noalign{}
\begin{minipage}[b]{\linewidth}\raggedright
Veuillez inscrire votre réponse dans l'encadré.
\end{minipage} & \begin{minipage}[b]{\linewidth}\raggedright
Cochez la case si vous n'en connaissez aucun.
\end{minipage} & \begin{minipage}[b]{\linewidth}\raggedright
\end{minipage} \\
\midrule\noalign{}
\endhead
\bottomrule\noalign{}
\endlastfoot
& & \\
Question ouverte & Je n'en connais aucun & \\
Dans les médias (ex.: animateur/animatrice, journaliste, \ldots) & & \\
& & \\
\end{longtable}

\begin{itemize}
\item
  \hfill\break
  Comme chanteur/chanteuse \textbar{}\\
\item
  \hfill\break
  Comme acteur/actrice \textbar{}\\
\item
  \hfill\break
  Comme écrivain/écrivaine \textbar{}\\
\item
\end{itemize}

\begin{enumerate}
\def\labelenumi{\arabic{enumi}.}
\setcounter{enumi}{63}
\tightlist
\item
  Nous souhaitons savoir si les gens connaissent mieux certains types de
  personnalités publiques que d'autres au Canada. Sans consulter
  d'autres sources, veuillez nommer une personnalité publique anglophone
  canadienne travaillant:
\end{enumerate}

\begin{longtable}[]{@{}
  >{\raggedright\arraybackslash}p{(\columnwidth - 4\tabcolsep) * \real{0.3333}}
  >{\raggedright\arraybackslash}p{(\columnwidth - 4\tabcolsep) * \real{0.3333}}
  >{\raggedright\arraybackslash}p{(\columnwidth - 4\tabcolsep) * \real{0.3333}}@{}}
\toprule\noalign{}
\begin{minipage}[b]{\linewidth}\raggedright
Veuillez inscrire votre réponse dans l'encadré.
\end{minipage} & \begin{minipage}[b]{\linewidth}\raggedright
Cochez la case si vous n'en connaissez aucun.
\end{minipage} & \begin{minipage}[b]{\linewidth}\raggedright
\end{minipage} \\
\midrule\noalign{}
\endhead
\bottomrule\noalign{}
\endlastfoot
& & \\
Question ouverte & Je n'en connais aucun & \\
Dans les médias (ex.: animateur/animatrice, journaliste, \ldots) & & \\
& & \\
\end{longtable}

\begin{itemize}
\item
  \hfill\break
  Comme chanteur/chanteuse \textbar{}\\
\item
  \hfill\break
  Comme acteur/actrice \textbar{}\\
\item
  \hfill\break
  Comme écrivain/écrivaine \textbar{}\\
\item
\end{itemize}

\begin{enumerate}
\def\labelenumi{\arabic{enumi}.}
\setcounter{enumi}{64}
\tightlist
\item
  Combien d'amis avez-vous de chacun des groupes suivants? Par ``amis'',
  nous référons à des personnes que vous appréciez et connaissez bien,
  mais qui ne sont pas des membres de votre famille.
\end{enumerate}

\begin{longtable}[]{@{}lllll@{}}
\toprule\noalign{}
1-2 & 3-4 & 5-6 & 7 ou plus & \\
\midrule\noalign{}
\endhead
\bottomrule\noalign{}
\endlastfoot
Des gens de langue maternelle francophone & & & & \\
\end{longtable}

\begin{itemize}
\item
\item
\item
\item
  \begin{longtable}[]{@{}lllll@{}}
  \toprule\noalign{}
  \endhead
  \bottomrule\noalign{}
  \endlastfoot
  Des gens de langue maternelle anglophone & & & & \\
  \end{longtable}
\item
\item
\item
\item
  \hfill\break
  Des gens d'une autre langue maternelle \textbar{}
\item
\item
\item
\item
\end{itemize}

\begin{enumerate}
\def\labelenumi{\arabic{enumi}.}
\setcounter{enumi}{65}
\tightlist
\item
  Sur une échelle de 0 à 10, comment évaluez-vous vos compétences dans
  les langues suivantes ?
\end{enumerate}

\begin{longtable}[]{@{}llllllllllll@{}}
\toprule\noalign{}
0 & 1 & 2 & 3 & 4 & 5 & 6 & 7 & 8 & 9 & 10 & \\
\midrule\noalign{}
\endhead
\bottomrule\noalign{}
\endlastfoot
\end{longtable}

\begin{longtable}[]{@{}ll@{}}
\toprule\noalign{}
Le français & \\
\midrule\noalign{}
\endhead
\bottomrule\noalign{}
\endlastfoot
L'anglais & \\
\end{longtable}

\chapter{Political Interest by Gender}\label{sec-appendix4}

\begin{figure}

\centering{

\includegraphics{_graphs/GenderDG.pdf}

}

\caption{\label{fig-interestdg}Gender Differences in Interest for
Specific Political Topics Among Canadian Adults, 2023 Datagotchi PES}

\end{figure}%

\begin{figure}

\centering{

\includegraphics{_graphs/GenderCCPIS.pdf}

}

\caption{\label{fig-interestccpis}Gender Differences in Interest for
Specific Political Topics Among Canadian Children, 2022 CCPIS}

\end{figure}%

\begin{table}
\centering\centering
\caption{Interest in Topic by Gender and Age Group, CCPIS \label{tab:lmeInterestYoungOldCCPIS}}
\centering
\fontsize{6}{8}\selectfont
\begin{tabular}[t]{lcccccc}
\toprule
  & Politics (general) & Health care & International affairs & Law and crime & Education & Partisan politics\\
\midrule
\addlinespace[0.5em]
\multicolumn{7}{l}{\textit{Ages 9--15}}\\
\midrule \hspace{1em}(Intercept) & 3.990*** & 3.638*** & 5.356*** & 4.519*** & 3.569*** & 3.854***\\
\hspace{1em} & (0.294) & (0.232) & (0.310) & (0.293) & (0.257) & (0.321)\\
\hspace{1em}Gender (1 = girl) & -0.113 & 0.077 & -0.891* & 0.630+ & -0.073 & -0.755*\\
\hspace{1em} & (0.319) & (0.309) & (0.375) & (0.379) & (0.357) & (0.364)\\
\hspace{1em}SD (Intercept Class) & 0.703 & 0.205 & 0.561 & 0.392 & 0.000 & 0.707\\
\hspace{1em}SD (Observations) & 2.527 & 2.467 & 2.965 & 3.022 & 2.847 & 2.860\\
\hspace{1em}Num.Obs. & 256 & 256 & 253 & 256 & 254 & 251\\
\hspace{1em}R2 Marg. & 0.000 & 0.000 & 0.021 & 0.011 & 0.000 & 0.016\\
\addlinespace[0.5em]
\multicolumn{7}{l}{\textit{Ages 16--18}}\\
\midrule \hspace{1em}(Intercept) & 4.997*** & 4.365*** & 5.903*** & 5.291*** & 4.621*** & 4.097***\\
\hspace{1em} & (0.184) & (0.202) & (0.209) & (0.208) & (0.247) & (0.202)\\
\hspace{1em}Gender (1 = girl) & -0.546* & 0.213 & -0.932** & 0.473 & -0.013 & -0.886**\\
\hspace{1em} & (0.271) & (0.261) & (0.293) & (0.293) & (0.293) & (0.302)\\
\hspace{1em}SD (Intercept Class) & 0.174 & 0.458 & 0.318 & 0.321 & 0.674 & 0.000\\
\hspace{1em}SD (Observations) & 2.484 & 2.354 & 2.682 & 2.659 & 2.621 & 2.815\\
\hspace{1em}Num.Obs. & 345 & 349 & 349 & 345 & 351 & 351\\
\hspace{1em}R2 Marg. & 0.012 & 0.002 & 0.029 & 0.008 & 0.000 & 0.024\\
\bottomrule
\multicolumn{7}{l}{\rule{0pt}{1em}+ p $<$ 0.1, * p $<$ 0.05, ** p $<$ 0.01, *** p $<$ 0.001}\\
\multicolumn{7}{l}{\rule{0pt}{1em}Method: Multilevel linear regression}\\
\multicolumn{7}{l}{\rule{0pt}{1em}Fixed Effects: Classroom}\\
\multicolumn{7}{l}{\rule{0pt}{1em}Controls: None}\\
\end{tabular}
\end{table}

\chapter{Political Interest Transmission (With Interaction Terms, Boys
and Girls Together)}\label{sec-appendix5}

\begin{table}
\centering\centering
\caption{Interest in Topic by Gender of Parent who Discusses that Topic the Most (With Interactions) \label{tab:lmeParentCtrlInterac}}
\centering
\fontsize{6}{8}\selectfont
\begin{tabular}[t]{lcccccc}
\toprule
 & All & Health care & International affairs & Law and crime & Education & Partisan politics\\
\midrule
(Intercept) & 7.161 & 10.839 & 11.161 & 3.305 & 19.727* & 24.890\\
\hspace{1em} & (5.906) & (9.560) & (11.141) & (10.783) & (10.011) & (15.098)\\
Mother discusses topic more than father & -0.417*** & -0.144 & -0.096 & -0.114 & 0.032 & 0.117\\
\hspace{1em} & (0.121) & (0.299) & (0.294) & (0.290) & (0.293) & (0.340)\\
Gender (1 = girl) & 0.007 & 0.169 & -3.255 & 2.377 & -0.679 & -0.157\\
\hspace{1em} & (1.290) & (2.397) & (2.809) & (2.879) & (2.586) & (3.354)\\
Age & -0.616 & -1.336 & -0.909 & 0.034 & -2.600+ & -3.335\\
\hspace{1em} & (0.798) & (1.305) & (1.506) & (1.473) & (1.362) & (2.048)\\
Age squared & 0.024 & 0.052 & 0.027 & 0.006 & 0.096* & 0.114\\
\hspace{1em} & (0.027) & (0.045) & (0.051) & (0.050) & (0.046) & (0.070)\\
Ethnicity (1 = white) & 0.364+ & 0.156 & 1.034* & 0.046 & -0.073 & 0.663\\
\hspace{1em} & (0.189) & (0.365) & (0.400) & (0.415) & (0.387) & (0.474)\\
Immigrant & -0.225 & 0.165 & -0.745+ & -0.510 & 0.048 & -0.255\\
\hspace{1em} & (0.194) & (0.367) & (0.416) & (0.429) & (0.392) & (0.492)\\
English spoken at home & -0.395 & -0.198 & -0.741 & -0.793 & -0.041 & 0.308\\
\hspace{1em} & (0.277) & (0.490) & (0.580) & (0.555) & (0.548) & (0.688)\\
French spoken at home & 0.056 & 0.193 & -0.475 & -0.410 & 0.226 & 0.650\\
\hspace{1em} & (0.166) & (0.315) & (0.348) & (0.360) & (0.335) & (0.425)\\
Agency & 1.696*** & 1.495* & 1.492+ & 1.541+ & 1.125 & 3.676***\\
\hspace{1em} & (0.387) & (0.719) & (0.833) & (0.858) & (0.779) & (1.006)\\
Communality & 0.887* & 0.901 & 2.239* & -0.702 & 1.519+ & 0.702\\
\hspace{1em} & (0.411) & (0.759) & (0.877) & (0.897) & (0.847) & (1.088)\\
Gender (1 = girl):Age & 0.012 & 0.014 & 0.165 & -0.114 & 0.049 & -0.009\\
\hspace{1em} & (0.082) & (0.153) & (0.178) & (0.183) & (0.164) & (0.212)\\
Gender (1 = girl):Ethnicity (1 = white) & -0.712** & -0.441 & -0.828 & -0.357 & -0.396 & -0.895\\
\hspace{1em} & (0.254) & (0.474) & (0.544) & (0.557) & (0.521) & (0.653)\\
\hspace{1em}SD (Intercept Class) & 0.595 & 0.407 & 0.474 & 0.352 & 0.535 & 0.529\\
\hspace{1em}SD (Observations) & 2.671 & 2.384 & 2.567 & 2.723 & 2.636 & 2.732\\
\midrule
\hspace{1em}Num.Obs. & 2036 & 445 & 400 & 412 & 454 & 325\\
\hspace{1em}R2 Marg. & 0.035 & 0.045 & 0.110 & 0.029 & 0.052 & 0.108\\
\bottomrule
\multicolumn{7}{l}{\rule{0pt}{1em}+ p $<$ 0.1, * p $<$ 0.05, ** p $<$ 0.01, *** p $<$ 0.001}\\
\multicolumn{7}{l}{\rule{0pt}{1em}Method: Multilevel linear regression}\\
\multicolumn{7}{l}{\rule{0pt}{1em}Fixed Effects: Classroom}\\
\multicolumn{7}{l}{\rule{0pt}{1em}Reference Category for Language: Other languages spoken at home}\\
\end{tabular}
\end{table}

\begin{table}
\centering\centering
\caption{Interest in Topic Most Often Discussed with Role Models (With Interactions) \label{tab:lmeAgentsCtrlInterac}}
\centering
\fontsize{6}{8}\selectfont
\begin{tabular}[t]{lcccccc}
\toprule
 & All & Health care & International affairs & Law and crime & Education & Partisan politics\\
\midrule
(Intercept) & -13.304 & -12.115 & -23.329 & 7.461 & 4.325 & 0.921\\
\hspace{1em} & (9.414) & (17.197) & (16.302) & (17.065) & (17.518) & (18.381)\\
Gender (1 = girl) & -2.392 & -6.739 & -3.920 & 7.129 & -5.966 & 0.149\\
\hspace{1em} & (2.147) & (4.489) & (4.433) & (4.517) & (5.097) & (4.611)\\
Topic most discussed with mother? & -0.218 & 0.990+ & 0.166 & -0.203 & -0.597 & 0.395\\
\hspace{1em} & (0.299) & (0.555) & (0.938) & (1.334) & (0.557) & (1.516)\\
Topic most discussed with father? & 0.810** & 1.197 & 0.099 & 0.998 & 0.105 & 2.149*\\
\hspace{1em} & (0.301) & (1.239) & (0.525) & (0.627) & (0.698) & (0.986)\\
Topic most discussed with female friends? & 0.695* & -0.220 & -0.696 & 2.627** & 1.850** & 1.758\\
\hspace{1em} & (0.308) & (0.749) & (0.577) & (0.885) & (0.563) & (2.020)\\
Topic most discussed with male friends? & 0.614+ & -2.922* & 0.606 & 0.490 & 0.150 & 2.794*\\
\hspace{1em} & (0.323) & (1.169) & (0.544) & (0.697) & (0.717) & (1.363)\\
Topic most discussed by teacher? & 0.357 & 1.292 & 1.500** & -1.045 & 0.022 & 0.160\\
\hspace{1em} & (0.310) & (0.991) & (0.554) & (1.265) & (0.633) & (1.267)\\
Topic most discussed by social media influencer? & 0.752* & 1.061 & 0.273 & 0.345 & 1.395 & -1.777\\
\hspace{1em} & (0.304) & (0.660) & (0.521) & (0.711) & (1.095) & (1.996)\\
Age & 2.047 & 2.055 & 3.808+ & -1.112 & -0.673 & 0.119\\
\hspace{1em} & (1.299) & (2.398) & (2.272) & (2.378) & (2.402) & (2.563)\\
Age squared & -0.064 & -0.067 & -0.133+ & 0.050 & 0.032 & 0.000\\
\hspace{1em} & (0.044) & (0.083) & (0.079) & (0.082) & (0.083) & (0.089)\\
Ethnicity (1 = white) & 0.134 & -0.708 & 0.544 & 0.371 & -0.326 & 0.758\\
\hspace{1em} & (0.278) & (0.595) & (0.593) & (0.580) & (0.620) & (0.625)\\
Immigrant & -0.635* & -0.111 & -1.200* & -0.796 & -0.203 & -0.887\\
\hspace{1em} & (0.286) & (0.595) & (0.592) & (0.614) & (0.614) & (0.630)\\
English spoken at home & 0.022 & 0.018 & -0.965 & -0.255 & 0.242 & 0.055\\
\hspace{1em} & (0.381) & (0.755) & (0.711) & (0.734) & (0.776) & (0.780)\\
French spoken at home & 0.409 & 0.200 & -0.405 & -0.054 & 0.051 & 0.416\\
\hspace{1em} & (0.267) & (0.558) & (0.521) & (0.540) & (0.559) & (0.579)\\
Agency & 1.266* & 0.237 & 2.510* & 2.235+ & 0.790 & 2.457+\\
\hspace{1em} & (0.608) & (1.295) & (1.244) & (1.240) & (1.332) & (1.286)\\
Communality & 1.530* & 1.449 & 1.395 & 1.552 & 4.254** & -0.512\\
\hspace{1em} & (0.642) & (1.350) & (1.256) & (1.301) & (1.403) & (1.356)\\
Gender (1 = girl):Topic\\
most discussed with mother? & 0.941* & -0.730 & 0.518 & 1.814 & 2.596** & \\
\hspace{1em} & (0.440) & (0.808) & (1.780) & (1.642) & (0.853) & \\
Gender (1 = girl):Topic\\
most discussed with father? & -0.349 & -0.844 & -0.201 & 1.097 & 0.538 & -2.226\\
\hspace{1em} & (0.437) & (1.518) & (0.798) & (0.960) & (1.014) & (1.439)\\
Gender (1 = girl):Topic\\
most discussed with female friends? & -0.228 & 0.082 & 1.677+ & -2.825* & -1.482+ & 0.162\\
\hspace{1em} & (0.448) & (1.024) & (0.922) & (1.163) & (0.840) & (3.374)\\
Gender (1 = girl):Topic\\
most discussed with male friends? & 0.026 & 2.112 & -0.685 & -1.074 & 0.072 & 0.433\\
\hspace{1em} & (0.449) & (1.464) & (0.830) & (0.893) & (1.071) & (1.787)\\
Gender (1 = girl):Topic\\
most discussed by teacher? & -0.270 & -2.473 & -1.354 & 0.547 & 0.384 & 2.754+\\
\hspace{1em} & (0.440) & (1.619) & (0.864) & (1.465) & (0.875) & (1.487)\\
Gender (1 = girl):Topic\\
most discussed by social media influencer? & 0.058 & 0.015 & -0.636 & 1.427 & -3.440* & 3.256\\
\hspace{1em} & (0.437) & (0.965) & (0.799) & (1.000) & (1.475) & (2.538)\\
Gender (1 = girl):Age & 0.109 & 0.392 & 0.235 & -0.429 & 0.251 & -0.073\\
\hspace{1em} & (0.133) & (0.280) & (0.276) & (0.281) & (0.311) & (0.289)\\
Gender (1 = girl):Ethnicity (1 = white) & -0.351 & 1.209 & -0.753 & -0.520 & 0.560 & -1.583+\\
\hspace{1em} & (0.387) & (0.825) & (0.788) & (0.820) & (0.846) & (0.833)\\
\hspace{1em}SD (Intercept Class) & 0.746 & 0.676 & 0.000 & 0.315 & 0.486 & 0.675\\
\hspace{1em}SD (Observations) & 2.415 & 2.294 & 2.253 & 2.301 & 2.419 & 2.368\\
\midrule
\hspace{1em}Num.Obs. & 831 & 166 & 166 & 165 & 167 & 167\\
\hspace{1em}R2 Marg. & 0.151 & 0.173 & 0.284 & 0.250 & 0.250 & 0.307\\
\bottomrule
\multicolumn{7}{l}{\rule{0pt}{1em}+ p $<$ 0.1, * p $<$ 0.05, ** p $<$ 0.01, *** p $<$ 0.001}\\
\multicolumn{7}{l}{\rule{0pt}{1em}Method: Multilevel linear regression}\\
\multicolumn{7}{l}{\rule{0pt}{1em}Fixed Effects: Classroom}\\
\multicolumn{7}{l}{\rule{0pt}{1em}Reference Category for Language: Other languages spoken at home}\\
\end{tabular}
\end{table}

\begin{figure}

\centering{

\includegraphics{_graphs/MotherDiscuss.pdf}

}

\caption{\label{fig-mother}Interest in Topic Most Often Discussed with
One's Mother}

\end{figure}%

\begin{figure}

\centering{

\includegraphics{_graphs/FatherDiscuss.pdf}

}

\caption{\label{fig-father}Interest in Topic Most Often Discussed with
One's Father, 2022 CCPIS}

\end{figure}%

\chapter{Political Interest Transmission by Parents (Without Control
Variables)}\label{sec-appendix6}

\begin{table}
\centering\centering
\caption{Interest in Topic by Gender of Parent who Discusses that Topic the Most \label{tab:lmeParent}}
\centering
\fontsize{6}{8}\selectfont
\begin{tabular}[t]{lcccccc}
\toprule
  & All & Health care & International affairs & Law and crime & Education & Partisan politics\\
\midrule
\addlinespace[0.5em]
\multicolumn{7}{l}{\textit{Boys}}\\
\midrule \hspace{1em}(Intercept) & 5.289*** & 4.631*** & 6.333*** & 5.504*** & 4.260*** & 4.601***\\
\hspace{1em} & (0.180) & (0.379) & (0.251) & (0.222) & (0.340) & (0.286)\\
\hspace{1em}Mother discusses topic more than father & -0.665*** & -0.411 & -0.332 & -0.242 & 0.286 & -0.182\\
\hspace{1em} & (0.161) & (0.401) & (0.401) & (0.381) & (0.379) & (0.480)\\
\hspace{1em}SD (Intercept Class) & 0.763 & 0.705 & 0.820 & 0.349 & 0.649 & 0.582\\
\hspace{1em}SD (Observations) & 2.671 & 2.361 & 2.483 & 2.638 & 2.694 & 2.973\\
\hspace{1em}Num.Obs. & 1138 & 241 & 225 & 228 & 252 & 192\\
\hspace{1em}R2 Marg. & 0.014 & 0.004 & 0.003 & 0.002 & 0.002 & 0.001\\
\addlinespace[0.5em]
\multicolumn{7}{l}{\textit{Girls}}\\
\midrule \hspace{1em}(Intercept) & 4.749*** & 4.138*** & 5.057*** & 5.652*** & 4.602*** & 3.449***\\
\hspace{1em} & (0.174) & (0.424) & (0.240) & (0.265) & (0.423) & (0.279)\\
\hspace{1em}Mother discusses topic more than father & -0.254 & 0.158 & -0.118 & 0.047 & -0.434 & 0.110\\
\hspace{1em} & (0.173) & (0.442) & (0.415) & (0.397) & (0.445) & (0.475)\\
\hspace{1em}SD (Intercept Class) & 0.645 & 0.665 & 0.345 & 0.492 & 0.849 & 0.360\\
\hspace{1em}SD (Observations) & 2.703 & 2.376 & 2.687 & 2.777 & 2.599 & 2.748\\
\hspace{1em}Num.Obs. & 1032 & 237 & 199 & 212 & 228 & 156\\
\hspace{1em}R2 Marg. & 0.002 & 0.001 & 0.000 & 0.000 & 0.004 & 0.000\\
\bottomrule
\multicolumn{7}{l}{\rule{0pt}{1em}+ p $<$ 0.1, * p $<$ 0.05, ** p $<$ 0.01, *** p $<$ 0.001}\\
\multicolumn{7}{l}{\rule{0pt}{1em}Method: Multilevel linear regression}\\
\multicolumn{7}{l}{\rule{0pt}{1em}Fixed Effects: Classroom}\\
\multicolumn{7}{l}{\rule{0pt}{1em}Controls: None}\\
\end{tabular}
\end{table}

\begin{table}
\centering\centering
\caption{Interest in Topic Most Often Discussed with One's Mother \label{tab:lmeMother}}
\centering
\fontsize{6}{8}\selectfont
\begin{tabular}[t]{lcccccc}
\toprule
  & All & Health care & International affairs & Law and crime & Education & Partisan politics\\
\midrule
\addlinespace[0.5em]
\multicolumn{7}{l}{\textit{Boys}}\\
\midrule \hspace{1em}(Intercept) & 4.810*** & 4.059*** & 5.981*** & 5.091*** & 4.372*** & 4.139***\\
\hspace{1em} & (0.154) & (0.241) & (0.211) & (0.173) & (0.261) & (0.199)\\
\hspace{1em}Topic most discussed with mother? & -0.009 & 0.391 & 1.087 & 1.731* & 0.223 & 1.548\\
\hspace{1em} & (0.190) & (0.310) & (0.726) & (0.675) & (0.338) & (1.231)\\
\hspace{1em}SD (Intercept Class) & 0.693 & 0.757 & 0.642 & 0.149 & 0.614 & 0.388\\
\hspace{1em}SD (Observations) & 2.767 & 2.398 & 2.690 & 2.689 & 2.717 & 2.948\\
\hspace{1em}Num.Obs. & 1320 & 262 & 265 & 265 & 264 & 264\\
\hspace{1em}R2 Marg. & 0.000 & 0.006 & 0.008 & 0.024 & 0.002 & 0.006\\
\addlinespace[0.5em]
\multicolumn{7}{l}{\textit{Girls}}\\
\midrule \hspace{1em}(Intercept) & 4.215*** & 4.014*** & 4.735*** & 5.160*** & 3.802*** & 3.099***\\
\hspace{1em} & (0.139) & (0.221) & (0.215) & (0.189) & (0.273) & (0.168)\\
\hspace{1em}Topic most discussed with mother? & 0.813*** & 0.336 & 1.215+ & 3.145*** & 0.893** & 2.901+\\
\hspace{1em} & (0.191) & (0.314) & (0.636) & (0.616) & (0.336) & (1.553)\\
\hspace{1em}SD (Intercept Class) & 0.605 & 0.586 & 0.704 & 0.284 & 0.915 & 0.000\\
\hspace{1em}SD (Observations) & 2.721 & 2.384 & 2.597 & 2.749 & 2.524 & 2.674\\
\hspace{1em}Num.Obs. & 1277 & 258 & 255 & 254 & 255 & 255\\
\hspace{1em}R2 Marg. & 0.013 & 0.004 & 0.014 & 0.093 & 0.027 & 0.014\\
\bottomrule
\multicolumn{7}{l}{\rule{0pt}{1em}+ p $<$ 0.1, * p $<$ 0.05, ** p $<$ 0.01, *** p $<$ 0.001}\\
\multicolumn{7}{l}{\rule{0pt}{1em}Method: Multilevel linear regression}\\
\multicolumn{7}{l}{\rule{0pt}{1em}Fixed Effects: Classroom}\\
\multicolumn{7}{l}{\rule{0pt}{1em}Controls: None}\\
\end{tabular}
\end{table}

\begin{table}
\centering\centering
\caption{Interest in Topic Most Often Discussed with One's Father \label{tab:lmeFather}}
\centering
\fontsize{6}{8}\selectfont
\begin{tabular}[t]{lcccccc}
\toprule
  & All & Health care & International affairs & Law and crime & Education & Partisan politics\\
\midrule
\addlinespace[0.5em]
\multicolumn{7}{l}{\textit{Boys}}\\
\midrule \hspace{1em}(Intercept) & 4.678*** & 4.294*** & 5.883*** & 5.078*** & 4.488*** & 4.056***\\
\hspace{1em} & (0.150) & (0.212) & (0.228) & (0.185) & (0.230) & (0.219)\\
\hspace{1em}Topic most discussed with father? & 0.929*** & -0.227 & 0.504 & 0.922* & 0.025 & 1.960**\\
\hspace{1em} & (0.190) & (0.512) & (0.346) & (0.425) & (0.409) & (0.645)\\
\hspace{1em}SD (Intercept Class) & 0.659 & 0.722 & 0.379 & 0.000 & 0.613 & 0.568\\
\hspace{1em}SD (Observations) & 2.704 & 2.376 & 2.672 & 2.652 & 2.698 & 2.841\\
\hspace{1em}Num.Obs. & 1262 & 250 & 254 & 253 & 253 & 252\\
\hspace{1em}R2 Marg. & 0.018 & 0.001 & 0.008 & 0.018 & 0.000 & 0.035\\
\addlinespace[0.5em]
\multicolumn{7}{l}{\textit{Girls}}\\
\midrule \hspace{1em}(Intercept) & 4.188*** & 4.109*** & 4.796*** & 5.124*** & 4.030*** & 3.137***\\
\hspace{1em} & (0.161) & (0.216) & (0.248) & (0.203) & (0.266) & (0.206)\\
\hspace{1em}Topic most discussed with father? & 0.616** & 0.341 & 0.028 & 1.652** & 0.060 & 0.291\\
\hspace{1em} & (0.199) & (0.449) & (0.377) & (0.507) & (0.397) & (0.644)\\
\hspace{1em}SD (Intercept Class) & 0.753 & 0.733 & 0.634 & 0.177 & 0.959 & 0.502\\
\hspace{1em}SD (Observations) & 2.690 & 2.335 & 2.660 & 2.790 & 2.536 & 2.650\\
\hspace{1em}Num.Obs. & 1154 & 233 & 230 & 230 & 231 & 230\\
\hspace{1em}R2 Marg. & 0.008 & 0.002 & 0.000 & 0.044 & 0.000 & 0.001\\
\bottomrule
\multicolumn{7}{l}{\rule{0pt}{1em}+ p $<$ 0.1, * p $<$ 0.05, ** p $<$ 0.01, *** p $<$ 0.001}\\
\multicolumn{7}{l}{\rule{0pt}{1em}Method: Multilevel linear regression}\\
\multicolumn{7}{l}{\rule{0pt}{1em}Fixed Effects: Classroom}\\
\multicolumn{7}{l}{\rule{0pt}{1em}Controls: None}\\
\end{tabular}
\end{table}

\begin{figure}

\centering{

\includegraphics{_graphs/DiscussParentYO.pdf}

}

\caption{\label{fig-discussparentyo}Interest in Topics by Gender, Age
and Discussion with Parents, 2022 CCPIS}

\end{figure}%

\chapter{Political Interest Transmission by Peers (Without Control
Variables)}\label{sec-appendix7}

\begin{table}
\centering\centering
\caption{Interest in Topic Most Often Discussed with one's Female Friends \label{tab:lmeFemaleFriends}}
\centering
\fontsize{6}{8}\selectfont
\begin{tabular}[t]{lcccccc}
\toprule
  & All & Health care & International affairs & Law and crime & Education & Partisan politics\\
\midrule
\addlinespace[0.5em]
\multicolumn{7}{l}{\textit{Boys}}\\
\midrule \hspace{1em}(Intercept) & 4.929*** & 4.474*** & 6.297*** & 5.068*** & 4.459*** & 4.576***\\
\hspace{1em} & (0.163) & (0.273) & (0.242) & (0.250) & (0.316) & (0.250)\\
\hspace{1em}Topic most discussed with female friends? & 0.541* & -0.078 & 0.267 & 1.605* & 0.752 & -1.114\\
\hspace{1em} & (0.245) & (0.547) & (0.486) & (0.659) & (0.458) & (1.127)\\
\hspace{1em}SD (Intercept Class) & 0.643 & 0.745 & 0.034 & 0.625 & 0.638 & 0.400\\
\hspace{1em}SD (Observations) & 2.750 & 2.568 & 2.630 & 2.564 & 2.756 & 2.896\\
\hspace{1em}Num.Obs. & 783 & 155 & 157 & 157 & 157 & 157\\
\hspace{1em}R2 Marg. & 0.006 & 0.000 & 0.002 & 0.036 & 0.017 & 0.006\\
\addlinespace[0.5em]
\multicolumn{7}{l}{\textit{Girls}}\\
\midrule \hspace{1em}(Intercept) & 4.370*** & 4.246*** & 4.687*** & 5.477*** & 4.249*** & 3.376***\\
\hspace{1em} & (0.155) & (0.268) & (0.238) & (0.251) & (0.296) & (0.201)\\
\hspace{1em}Topic most discussed with female friends? & 0.902*** & 0.151 & 1.195* & 1.085* & 0.274 & 4.624**\\
\hspace{1em} & (0.223) & (0.423) & (0.533) & (0.511) & (0.400) & (1.570)\\
\hspace{1em}SD (Intercept Class) & 0.647 & 1.010 & 0.567 & 0.478 & 0.946 & 0.000\\
\hspace{1em}SD (Observations) & 2.687 & 2.214 & 2.616 & 2.788 & 2.506 & 2.698\\
\hspace{1em}Num.Obs. & 914 & 183 & 182 & 181 & 184 & 184\\
\hspace{1em}R2 Marg. & 0.017 & 0.001 & 0.027 & 0.024 & 0.002 & 0.045\\
\bottomrule
\multicolumn{7}{l}{\rule{0pt}{1em}+ p $<$ 0.1, * p $<$ 0.05, ** p $<$ 0.01, *** p $<$ 0.001}\\
\multicolumn{7}{l}{\rule{0pt}{1em}Method: Multilevel linear regression}\\
\multicolumn{7}{l}{\rule{0pt}{1em}Fixed Effects: Classroom}\\
\multicolumn{7}{l}{\rule{0pt}{1em}Controls: None}\\
\end{tabular}
\end{table}

\begin{table}
\centering\centering
\caption{Interest in Topic Most Often Discussed with One's Male Friends \label{tab:lmeMaleFriends}}
\centering
\fontsize{6}{8}\selectfont
\begin{tabular}[t]{lcccccc}
\toprule
  & All & Health care & International affairs & Law and crime & Education & Partisan politics\\
\midrule
\addlinespace[0.5em]
\multicolumn{7}{l}{\textit{Boys}}\\
\midrule \hspace{1em}(Intercept) & 4.615*** & 4.432*** & 5.474*** & 5.023*** & 4.449*** & 4.175***\\
\hspace{1em} & (0.155) & (0.232) & (0.239) & (0.197) & (0.230) & (0.239)\\
\hspace{1em}Topic most discussed with male friends? & 1.541*** & -1.111+ & 1.586*** & 1.503*** & 0.626 & 1.509+\\
\hspace{1em} & (0.198) & (0.669) & (0.356) & (0.397) & (0.443) & (0.810)\\
\hspace{1em}SD (Intercept Class) & 0.680 & 0.860 & 0.301 & 0.000 & 0.528 & 0.685\\
\hspace{1em}SD (Observations) & 2.701 & 2.365 & 2.673 & 2.604 & 2.743 & 2.900\\
\hspace{1em}Num.Obs. & 1158 & 230 & 233 & 232 & 232 & 231\\
\hspace{1em}R2 Marg. & 0.047 & 0.011 & 0.079 & 0.058 & 0.008 & 0.014\\
\addlinespace[0.5em]
\multicolumn{7}{l}{\textit{Girls}}\\
\midrule \hspace{1em}(Intercept) & 4.516*** & 4.387*** & 5.133*** & 5.876*** & 4.313*** & 3.592***\\
\hspace{1em} & (0.167) & (0.211) & (0.298) & (0.293) & (0.289) & (0.231)\\
\hspace{1em}Topic most discussed with male friends? & 0.944*** & 0.070 & 0.871+ & -0.343 & 0.638 & 2.033*\\
\hspace{1em} & (0.242) & (0.656) & (0.512) & (0.462) & (0.515) & (1.000)\\
\hspace{1em}SD (Intercept Class) & 0.691 & 0.410 & 0.858 & 0.000 & 0.961 & 0.000\\
\hspace{1em}SD (Observations) & 2.646 & 2.297 & 2.595 & 2.768 & 2.444 & 2.751\\
\hspace{1em}Num.Obs. & 753 & 153 & 149 & 149 & 152 & 150\\
\hspace{1em}R2 Marg. & 0.019 & 0.000 & 0.019 & 0.004 & 0.010 & 0.027\\
\bottomrule
\multicolumn{7}{l}{\rule{0pt}{1em}+ p $<$ 0.1, * p $<$ 0.05, ** p $<$ 0.01, *** p $<$ 0.001}\\
\multicolumn{7}{l}{\rule{0pt}{1em}Method: Multilevel linear regression}\\
\multicolumn{7}{l}{\rule{0pt}{1em}Fixed Effects: Classroom}\\
\multicolumn{7}{l}{\rule{0pt}{1em}Controls: None}\\
\end{tabular}
\end{table}

\begin{figure}

\centering{

\includegraphics{_graphs/DiscussPeersYO.pdf}

}

\caption{\label{fig-discusspeersyo}Interest in Topics by Gender, Age and
Discussion with Peers, 2022 CCPIS}

\end{figure}%

\chapter{Correlation Matrixes}\label{sec-appendix8}

\begin{figure}

\centering{

\includegraphics{_graphs/CorMatrixDG.pdf}

}

\caption{\label{fig-corccpis}Correlation Matrix for Interest in Topics,
2022 CCPIS}

\end{figure}%

\begin{figure}

\centering{

\includegraphics{_graphs/CorMatrixDG.pdf}

}

\caption{\label{fig-cordg}Correlation Matrix for Interest in Topics,
2023 Datagotchi PES}

\end{figure}%

\chapter{Political Interest Transmission by Influencers (Without Control
Variables)}\label{sec-appendix9}

\begin{table}
\centering\centering
\caption{Interest in Topic by Gender Congruence of Influencer who Discusses that Topic \label{tab:lmeInfluencer}}
\centering
\fontsize{6}{8}\selectfont
\begin{tabular}[t]{lcccccc}
\toprule
  & All & Health care & International affairs & Law and crime & Education & Partisan politics\\
\midrule
\addlinespace[0.5em]
\multicolumn{7}{l}{\textit{Same-Gender Influencers}}\\
\midrule \hspace{1em}(Intercept) & 4.429*** & 4.101*** & 5.335*** & 4.981*** & 4.460*** & 3.832***\\
\hspace{1em} & (0.153) & (0.203) & (0.235) & (0.189) & (0.219) & (0.217)\\
\hspace{1em}Topic most discussed with influencer? & 1.331*** & 0.780* & 0.939** & 1.507*** & 0.038 & 1.551+\\
\hspace{1em} & (0.165) & (0.319) & (0.298) & (0.359) & (0.487) & (0.881)\\
\hspace{1em}SD (Intercept Class) & 0.747 & 0.757 & 0.739 & 0.450 & 0.836 & 0.801\\
\hspace{1em}SD (Observations) & 2.704 & 2.369 & 2.654 & 2.705 & 2.668 & 2.831\\
\hspace{1em}Num.Obs. & 1678 & 336 & 338 & 335 & 335 & 334\\
\hspace{1em}R2 Marg. & 0.035 & 0.017 & 0.027 & 0.050 & 0.000 & 0.009\\
\addlinespace[0.5em]
\multicolumn{7}{l}{\textit{Other-Gender Influencers}}\\
\midrule \hspace{1em}(Intercept) & 4.107*** & 3.795*** & 5.146*** & 5.349*** & 3.828*** & 3.167***\\
\hspace{1em} & (0.201) & (0.240) & (0.346) & (0.354) & (0.328) & (0.277)\\
\hspace{1em}Topic most discussed with influencer? & 1.136*** & 0.569 & -0.217 & 1.241* & 0.932 & 0.583\\
\hspace{1em} & (0.283) & (0.580) & (0.523) & (0.542) & (0.817) & (1.464)\\
\hspace{1em}SD (Intercept Class) & 0.858 & 0.150 & 0.725 & 1.073 & 1.023 & 0.000\\
\hspace{1em}SD (Observations) & 2.688 & 2.299 & 2.611 & 2.522 & 2.650 & 2.876\\
\hspace{1em}Num.Obs. & 560 & 112 & 112 & 111 & 113 & 112\\
\hspace{1em}R2 Marg. & 0.025 & 0.009 & 0.002 & 0.042 & 0.011 & 0.001\\
\bottomrule
\multicolumn{7}{l}{\rule{0pt}{1em}+ p $<$ 0.1, * p $<$ 0.05, ** p $<$ 0.01, *** p $<$ 0.001}\\
\multicolumn{7}{l}{\rule{0pt}{1em}Method: Multilevel linear regression}\\
\multicolumn{7}{l}{\rule{0pt}{1em}Fixed Effects: Classroom}\\
\multicolumn{7}{l}{\rule{0pt}{1em}Controls: None}\\
\end{tabular}
\end{table}



\end{document}
